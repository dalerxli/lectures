%% -*- coding:utf-8 -*-
\documentclass[10pt,pdf,hyperref={unicode}]{beamer}
\input ./preamble.tex
\usetheme{Warsaw}
\title[Криптография и квантовые вычисления]{Классическая
  криптография\\Квантовые вычисления}
\author{Мурашко И. В.}
\institute{Санкт Петербургский Государственный Политехнический Университет}
\date{}
\begin{document}

\section{Введение}

\begin{frame}{Введение}
\begin{itemize}
\item Квантовая механика
\item Квантовые вычисления
\item Методы симметричного шифрования и алгоритм Гровера
\item Методы несимметричного шифрования (RSA, Diffie-Hellman, Elliptic
curve) и алгоритм Шора.
\end{itemize}
\end{frame}

\section{Квантовая механика}
\begin{frame}{Двухуровневый атом}
\begin{figure}
\centering

\input ../add/quant/picmeasurex.tex

\caption{Процесс измерения энергии двухуровневого атома находящегося в
чистом состоянии $\left|\psi\right> = 
\frac{1}{\sqrt{2}}\left|a\right> + \frac{1}{\sqrt{2}}\left|b\right>$.
Прибором регистрируется значение энергии $E_a$ или $E_b$.
}
\label{fig:add:mesure_ex}
\end{figure}
\end{frame}

\begin{frame}{Двухуровневый атом. Измерение $E_a$}
\begin{figure}
\centering

\input ../add/quant/picmeasurex_a.tex

\caption{Процесс измерения энергии двухуровневого атома находящегося в
чистом состоянии $\left|\psi\right> = 
\frac{1}{\sqrt{2}}\left|a\right> + \frac{1}{\sqrt{2}}\left|b\right>$.
Прибором регистрируется значение энергии $E_a$. При измерении
происходит следующая редукция $\left|\psi\right> \to \left|a\right>$
}
\label{fig:add:mesure_ex_a}
\end{figure}
\end{frame}

\begin{frame}{Двухуровневый атом. Измерение $E_b$}
\begin{figure}
\centering

\input ../add/quant/picmeasurex_b.tex

\caption{Процесс измерения энергии двухуровневого атома находящегося в
чистом состоянии $\left|\psi\right> = 
\frac{1}{\sqrt{2}}\left|a\right> + \frac{1}{\sqrt{2}}\left|b\right>$.
Прибором регистрируется значение энергии $E_b$. При измерении
происходит следующая редукция $\left|\psi\right> \to \left|b\right>$
}
\label{fig:add:mesure_ex_b}
\end{figure}
\end{frame}

\begin{frame}{Кот Шредингера}
TBD
\end{frame}

\begin{frame}{Отрицательные вероятности}
TBD
\end{frame}

\section{Квантовые вычисления}
\begin{frame}{Базовые блоки квантового компьютера}
TBD
\end{frame}

\section{Алгоритм Гровера}
\begin{frame}{Задача о поиске иголки в стоге сена}
TBD
\end{frame}

\begin{frame}{Алгоритм Гровера}
TBD
\end{frame}

\begin{frame}{Влияние на рекомендации к использованию}
$AES_{128} \rightarrow AES_{256}$
\end{frame}

\section{Алгоритм Шора}
\begin{frame}{RSA и задача факторизации чисел}
TBD
\end{frame}

\begin{frame}{Diffie-Hellman, Elliptic
curve и дискретный логарифм}
TBD
\end{frame}

\begin{frame}{Задача о нахождении периода функций и алгоритм Шора}
TBD
\end{frame}

\begin{frame}{Влияние на рекомендации к использованию}
NSA не рекомендует использование алгоритмов на элиптических кривых для
внутреннего использования.
\end{frame}

\section{Заключение}
\begin{frame}{Что дальше?}
TBD
\end{frame}

\begin{frame}{Вопросы}
TBD
\end{frame}

\end{document}
