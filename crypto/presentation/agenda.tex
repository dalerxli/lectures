%% -*- coding:utf-8 -*-
\section{Введение}
Лекция посвящена современным алгоритмам классического шифрования и новым
методам вскрытия данных алгоритмов посредством квантовых вычислений. 

В первой части лекции предполагается дать краткое (научно популярное)
введение в квантовую механику. Предполагается дать ответ на вопрос,
что такого уникального есть в квантовой механике, что дает возможность
построения таких удивительных приборов как квантовые компьютеры. 
В частности предполагается рассказать о связи такого мема как ``Кот
Шредингера'' с квантовыми вычислениями, об ``отрицательных
вероятностях'' которые проявляются при рассмотрении отдельных
квантовомеханических экспериментов, и каким образом это помогает построить
системы защищенной связи с помощью квантовой криптографии.

Во второй части будет дано базовое определения квантового
компьютера и, в частности, будет предоставлен ответ на вопрос какие
задачи могут быть решены с его помощью.

В третьей части предполагается рассмотреть конкретные примеры
классических алгоритмов шифрования и методы их вскрытия с помощью квантовых
компьютеров. В частности:
- Методы симметричного шифрования и алгоритм Гровера
- Методы несимметричного шифрования (RSA, Diffie-Hellman, Elliptic
curve) и алгоритм Шора.

Формат лекции будет предполагать базовое знакомство с предметом, т.е.
не предполагать специальных знаний. По результатам этой лекции будет
принято решении о целесообразности дальнейших, более глубоких лекций.
