%% -*- coding:utf-8 -*- 
\chapter*{Введение}

Квантовая оптика изучает оптические явления, в которых проявляется
квантовая сущность света. Можно сказать, что квантовая оптика
рассматривает оптические явления, при которых свет и взаимодействующую
с ним систему необходимо описывать квантовыми уравнениями. 

В квантовой оптике рассматривается область частот примерно от 
\(f_1 \simeq 10^{13} \mbox{Гц}\) до \(f_2 \simeq 10^{18}
\mbox{Гц}\), т. е. от инфракрасного диапазона до
рентгеновского. Нижний предел определяется условием превышения
энергией кванта энергии теплового движения: 
\(\omega_1 \hbar > k T\). Верхний предел
устанавливается исходя из того, что в квантовой оптике
рассматриваются, как правило, нерелятивистские энергии электронов и,
следовательно, энергия кванта должна быть заметно меньше, чем энергия
покоя электрона: \(\omega_2 \hbar \ll m c^2\).

В настоящее время существует мало учебных материалов посвященных
квантовой оптике. Из отечественных стоит отметить 
\cite{bTarasovQuantumOpticsIntro2008}.  Предлагаемый материал отражает
ряд вопросов квантовой теории света, излагаемых в курсе ``Квантовая
оптика''. Пособие состоит из трех частей. 

В первой части пособия дается введение в квантовую
электродинамику. Рассматриваются простейшие задачи, связанные с
взаимодействием светового поля и вещества. На простейшей модели
строится квантовая теория лазера. Рассматривается квантовое описание
когерентности света и ее связь с классическим описанием. 
В главе 1 дается краткое введение в квантовую
электродинамику в объеме, необходимом для изложения курса квантовой
оптики. Рассматриваются различные квантовые состояния светового поля,
применяемые в квантовой оптике. Особое внимание уделяется когерентным
состояниям, позволяющим максимально приблизить квантовое описание
оптических явлений к классическому. 
В главе 2  рассматриваются простейшие задачи для светового поля,
связанного как с отдельными атомами, так и с большим числом атомов,
находящихся в тепловом равновесии. Особое внимание уделяется
связи поля с термостатом, приводящей к релаксационным процессам,
играющим важную роль в квантовой оптике и квантовой
электронике.

Во второй части пособия рассмотрены некоторые прикладные вопросы
квантовой оптики. В 3 главе на простой модели рассматривается квантовая теория
лазера. Получено выражение для статистики лазерных фотонов и формула,
оценивающая ``естественную'' ширину линии генерации лазера. 
В 4 главе рассматривается квантовая теория лазера в представлении
Гейзенберга. В 5 главе дается квантовое описание
интерферометрических экспериментов.
6 глава посвящена оптике фотонов. Излагается квантовое описание
когерентности света и его связь с классическим описанием, функции
когерентности различных порядков, проблема статистики фотонов и ее
связь со статистикой фото-отсчетов (формула Манделя). Приводятся
многочисленные примеры. Далее рассматривается связь статистики фотонов
со спектральными свойствами световых пучков и ее применение для
спектральных измерений.

В третьей части пособия обсуждаются неклассические
состояния света. Дано определение неклассического состояния. Описаны
сжатые и перепутанные неклассические состояния. 

В четвертой части пособия рассмотрена квантовая теория информации. В
частности в главе 8 дано введение в квантовую теорию
информации. Определено понятие количества 
информации в классическом и квантовом случае. Рассмотрена классическая
теорема кодирования Шенона и ее квантовый аналог. Глава 9 посвящена
теории защиты информации (криптографии): описана классическая
теория криптографии и ее основные недостатки. Приведено описание
квантовой криптографии. Глава 10 описывает основные квантовые
алгоритмы: алгоритм Шора для факторизации целых числел и алгоритм
Гровера для поиска в неструктурированном массиве данных.

В приложениях собран материал который дополняет основной курс и носит
в основном справочный характер, позволяющий не прибегать к специальной
литературе. 

Пособие предназначено для студентов старших курсов, специализирующихся
в области квантовой электроники.  
