%% -*- coding:utf-8 -*- 
\input preamble.tex

\begin{document}
\Russian
\input title.tex

\subsection*{Билет 1} 
\begin{enumerate}
\item Разложение электромагнитного поля по модам (типам колебаний).
Гамильтонова форма уравнений электромагнитного поля.
Квантование электромагнитного поля. 
\item Излучение и поглощение атомом света.
Гамильтониан системы атом-поле.
\end{enumerate}

\subsection*{Билет 2} 
\begin{enumerate}
\item Свойства операторов $ \hat a $ и $ \hat a ^\dag $ 
Квантовое состояние электромагнитного поля  с определенной
  энергией. 
\item Взаимодействие электромагнитного поля резонатора
(гармонического осциллятора) с резервуаром атомов, находящихся при
температуре $T$.
Уравнение для матрицы плотности поля в представлении чисел
заполнения.
\end{enumerate}

\subsection*{Билет 3} 
\begin{enumerate}
\item Когерентные состояния. 
\item Взаимодействие атома с модой электромагнитного поля.
Релаксация динамической системы. Метод матрицы плотности.
\end{enumerate}

\subsection*{Билет 4} 
\begin{enumerate}
\item Смешанные состояния электромагнитного поля.
Представление оператора плотности через когерентные
состояния. 
\item Модель лазера. Теория лазерной генерации. Статистика лазерных
фотонов. 
\end{enumerate}

\subsection*{Билет 5} 
\begin{enumerate}
\item Смешанные состояния электромагнитного поля.
Представление оператора плотности через когерентные
состояния. 
\item Поляризационные свойства света. Параметры Стокса. Парадокс ЭПР
  для параметров Стокса и перепутанные состояния.  
\end{enumerate}

\subsection*{Билет 6} 
\begin{enumerate}
\item Разложение электромагнитного поля по модам (типам колебаний).
Гамильтонова форма уравнений электромагнитного поля.
Квантование электромагнитного поля. 
\item Криптография. Проблемы классической криптографии. Квантовая
  криптография. 
\end{enumerate}

\end{document}


