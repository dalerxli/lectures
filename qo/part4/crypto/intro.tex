%% -*- coding:utf-8 -*- 
С того момента, как была осознана важность информации, стали появляться
средства ее защиты. 
%% Одними из первых кто стал применять методы
%% защиты информации были древние греки. Для этих целей они использовали
%% первую шифровальную машину - скитал. Скитал представлял собой
%% конусообразную дубинку на которую наматывалась
%% полоска кожи. В дальнейшем на коже писалось сообщение. Когда полоску кожи
%% снимали, то записи на ней представляли собой некоторую перестановку
%% символов исходного сообщения - шифротекст. 

%% С тех пор наука о защите информации, которая называется
%% криптографией проделала очень большой путь. 
Изобретались новые методы
шифрования, такие например, как шифр Цезаря, в котором каждая буква
алфавита заменялась на другую (например, следующую через три позиции в
алфавите после нее). Наряду с новыми методами шифрования появлялись
способы вскрытия этих шифров, например для шифра Цезаря можно
воспользоваться статистическими свойствами языка, на котором
писалось исходное сообщение.

Очень часто безопасность шифра обеспечивалась тем, что алгоритм, по
которому обеспечивалось шифрование, держался в секрете, как например в
рассмотренном выше шифре Цезаря. В современной классической
криптографии чаще всего алгоритмы
публикуются и доступны для изучения каждому. Секретность
обеспечивается тем, что само сообщение смешивается с секретным ключом
по некоторому открытому алгоритму. 

Допустим нам надо передать некоторое сообщение от Алисы к Бобу по
некоторому защищенному каналу связи. Сообщение должно быть
представлено в некоторой цифровой форме.
Протокол, описывающий такую
передачу, состоит из нескольких этапов. На первом Алиса и Боб должны
получить некоторую общую случайную последовательность чисел, которая
будет называться ключом. Эта процедура называется распределением
ключа. 

На следующем этапе Алиса должна с помощью
некоторого алгоритма $E$ получить из исходного сообщения $P$ и ключа $K$
зашифрованное сообщение $C$. Данная процедура может быть описана следующим
соотношением: 
\begin{equation}
E_{K}\left(P\right) = C.
\label{eqPart3CryptoEncryptClass}
\end{equation}

На третьем этапе полученное зашифрованное сообщение должно быть
передано Бобу.

На последнем этапе Боб с помощью известного алгоритма $D$ и полученного на
первом этапе ключа $K$ должен восстановить исходное сообщение $P$ из
полученного зашифрованного $C$. Данная процедура может быть описана
следующим соотношением
\begin{equation}
D_{K}\left(C\right) = P.
\label{eqPart3CryptoDeEncryptClass}
\end{equation}

При анализе данного протокола возникают следующие вопросы. Как
реализовать безопасное распределение ключа. Второй - существует ли
абсолютно стойкий алгоритм. И наконец последний - возможна ли
безопасная передача зашифрованного сообщения, когда оно не может быть
прослушано или подменено. 

Классическая криптография дает однозначный ответ только на второй
вопрос. Абсолютно криптостойкий алгоритм существует - он носит название
одноразового блокнота. Ниже представлено детальное описание этого алгоритма.
