%% -*- coding:utf-8 -*- 
\section{Квантовая криптография}
\label{subsecPart3QuantInfoQuantCrypto}
Случайность, присущая квантовым объектам, наталкивает на мысль
использовать их для распределения ключа. Существует много различных
безопасных схем распределения ключа, основанных на использовании
квантовых объектов. Мы исследуем схему, базирующуюся на проверке
неравенств Белла (см. \ref{pPart3EntangleBell}). Рассматриваемая схема
изображена на рис. \ref{figPart3QuantInfoCryptoBell}.

\input ./part4/crypto/figquantcrypto.tex

Источник перепутанных фотонов $S$ создает пары фотонов,
один из которых отсылается Алисе, а второй Бобу, которые выполняют
измерения параметров Стокса своих фотонов. 

Алиса в случайном порядке измеряет либо $\hat{A} = \hat{S}_1^{(1)}$
либо $\hat{A}' = \hat{S}_2^{(1)}$. Боб в случайном порядке производит
измерения следующих величин 
\begin{eqnarray}
\hat{B} = \frac{1}{\sqrt{2}}\left(\hat{S}_1^{(2)} +
  \hat{S}_2^{(2)}\right), 
\nonumber \\
\hat{B}' = \frac{1}{\sqrt{2}}\left(\hat{S}_1^{(2)} - \hat{S}_2^{(2)}\right)
\nonumber \\
\hat{C} = \hat{S}_1^{(2)},
\nonumber \\
\hat{C}' = \hat{S}_2^{(2)}.
\nonumber
\end{eqnarray}
Таким образом в результате эксперимента мы получим 8 пар значений,
которые могут быть объединены в три группы. 

В первой находятся
кореллирующие комбинации операторов $\hat{A}$, $\hat{A}'$, $\hat{C}$ и
$\hat{A}'$: $\left(a, c\right)$ и $\left(a', c'\right)$. Для этих
комбинаций мы можем сказать, что если одно из чисел $a$ или $a'$
равно $\pm 1$, то второе ($c$ или $c'$) равно $\mp 1$. Таким образом
эти числа могут быть использованы для получения случайной
последовательности чисел, которая будет в дальнейшем использоваться в
качестве ключа. 

Во второй группе находятся 4 пары значений, которые будут
использоваться для проверки неравенств Белла: $\left(a, b\right)$,
$\left(a', b\right)$, $\left(a, b'\right)$ и $\left(a', b'\right)$.

В последней группе находятся пары значений $\left(a, c'\right)$ и
$\left(a', c\right)$. В дальнейшем эта группа значений отбрасывается.

На начальном этапе Боб и Алиса в случайном порядке (независимо друг от
друга) производят измерения. По завершении измерений они говорят друг
другу (по обычному открытому каналу связи) какие величины они измеряли
в каждом конкретном испытании, при этом результат самих измерений не
сообщается. В дальнейшем отбрасываются те испытания, в которых кто-либо
из них не смог зарегистрировать фотон, и результаты, относящиеся к
третьей группе измерений. Результаты второй группы измерений открыто
публикуются и по ним считается среднее значение $\left<F\right>$
(\ref{eqEntangFmain}). 

Если полученное значение по модулю близко к предсказаниям квантовой
механики (\ref{eqEntangQuant})
\[
\left<F\right>_{quant} = - \sqrt{2},
\]
то результат первой группы измерений может быть
интерпретирован в качестве ключа. 

Если злоумышленник Ева хочет попытаться узнать результаты первой
группы измерений, которые составляют распределяемый ключ, то
один из способов сделать это - подменить собой источник
перепутанных пар фотонов и отсылать Алисе и Бобу по паре фотонов с
определенными поляризационными свойствами, тогда результаты
опытов Алисы и Боба будут предопределены заранее. Но в этом случае
$\left<F\right>$, согласно (\ref{eqEntangClass}), будет лежать в
интервале:
\[
-1 \le \left<F\right>_{class} \le 1.
\]
Таким образом Алиса и Боб по результатам проверки неравенств Белла
могут определить факт вмешательства Евы и считать полученный
ключ скомпрометированным, если проверка установит факт присутствия Евы.

% FIXME add подмена Алиса Евой для Боба и наоборот
