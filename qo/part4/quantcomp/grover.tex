%% -*- coding:utf-8 -*- 
\section{Алгоритм Гровера}
\index{aлгоритм Гровера}
Рассмотрим следующую задачу. Допустим имеется большой набор данных
состоящий из $N$ элементов в котором необходимо найти элемент
удовлетворяющий некоторым условиям (см. \autoref{figQuantCompSearch}).
Если данные отсортированны, то с  
помощью алгоритмов типа ``разделяй и властвуй'' искомый элемент может
быть найдет за время порядка $O\left(log N\right)$
(см. разд. \ref{addDivideAndConquer}). В ряде случаев исходный набор данных
не может быть подготовлен для быстрого поиска, в этом случае
классический поиск осуществляется за время порядка $O\left(N\right)$.

\input ./part4/quantcomp/figsearch.tex

Одним из примеров являются алгоритмы симметричного шифрования в
которых стоит задача определения ключа по известному шифрованному
тексту и соотвествующему ему оригинальному тексту. В этом случае
предварительная обработка данных представляется невозможной и решением
задачи ``в лоб'' является простой перебор всех возможных значений.

Алгоритм Гровера \cite{Grover96afast} решает задачу
неструктурированного поиска за время порядка $O\left(\sqrt{N}\right)$.

\subsection{Описание алгоритма}

Допустим у нас имеется квантовый контур который вычисляет значение
функции $f\left(x\right)$ которая может принимать только два значения:
$0$ и $1$. При этом значение $1$ справедливо только для искомого
элемента: 
\begin{eqnarray}
\left.f\left(x\right)\right|_{x = x^{\ast}} = 1,
\nonumber \\
\left.f\left(x\right)\right|_{x \ne x^{\ast}} = 0.
\label{eqQuantCompGroverF}
\end{eqnarray}
На \autoref{figQuantCompGrover1} изображена схема для вычесления
искомой функции. На выходе мы имеем сотояние вида
\begin{equation}
\left|out\right> = \frac{1}{\sqrt{N}}\left(
 \sum_{x \ne x^{\ast}} \left|x\right>\otimes\left|0\right>
+ \left|x^{\ast}\right>\otimes\left|1\right>
\right),
\label{eqQuantCompGroverFake}
\end{equation}
где $N$ - общее число элементов в последовательности в которой
производится поиск.

\input ./part4/quantcomp/figgroverfake.tex

Если посмотреть на выражение (\ref{eqQuantCompGroverFake}), то можно
заметить, что предложенная схема, несмотря на то что она производит
вычисление функции в искомой точке, не позволяет выбрать искомый
элемент, потому что все элементы результирующей последовательности
равновероятны, т. е. каждый элемент может быть выбран (в результате
измерения) с одинаковой вероятностью: $\frac{1}{N}$.

Гровером был предложен алгоритм, который позволил бы повысить
вероятность обнаружения искомого элемента в результирующей
суперпозиции (\ref{eqQuantCompGroverFake}).

\input ./part4/quantcomp/figgrover.tex

\input ./part4/quantcomp/figgroverbase.tex

Схема, реализующая алгоритм Гровера представляет собой некоторый блок,
описываемый оператором $\hat{U}_G$, который повторяется некоторое число
раз (см. \autoref{figQuantCompGrover}). При этом на каждом шаге
итерации вероятность обнаружения искомого элемента повышается. 

Базовый элемент $\hat{U}_G$ представляет собой последовательное действие
двух операторов (см. \autoref{figQuantCompGroverBase}):
\begin{equation}
\hat{U}_G=\hat{U}_s\hat{U}_{x^{\ast}},
\nonumber
\end{equation}
где $\hat{U}_{x^{\ast}}$ - оператор инверсии фазы, $\hat{U}_s$
- оператор обращения относительно среднего.

\input ./part4/quantcomp/figgroverinv.tex

Действие оператора $\hat{U}_{x^{\ast}}$ описывается следующим соотношением
(см. \autoref{figQuantCompGroverInv}):
\begin{equation}
\hat{U}_{x^{\ast}}\left(\sum_x \alpha_x \left|x\right>\right) = 
\sum_x \alpha_x \left(-1\right)^{f\left(x\right)}\left|x\right>.
\label{eqQuantCompGroverUxast}
\end{equation} 
Оператор $\hat{U}_{x^{\ast}}$ может быть переписан в виде
\begin{equation}
\hat{U}_{x^{\ast}} = \hat{I} - 2 \left|x^{\ast}\right>\left<x^{\ast}\right|.
\nonumber
\end{equation} 
Действительно
\begin{eqnarray}
\left(\hat{I} - 2 \left|x^{\ast}\right>\left<x^{\ast}\right|\right)
\left(\sum_x \alpha_x \left|x\right>\right) =
\nonumber \\
= \sum_x \alpha_x \left|x\right> - 2 \alpha_{x^{\ast}}
\left|x^{\ast}\right> = 
\sum_{x\ne x^{\ast}} \alpha_x \left|x\right> -  \alpha_{x^{\ast}}
\left|x^{\ast}\right> =
\nonumber \\
=
\sum_x \alpha_x \left(-1\right)^{f\left(x\right)}\left|x\right>,
\nonumber
\end{eqnarray}
что совпадает с (\ref{eqQuantCompGroverUxast}).

\input ./part4/quantcomp/figgroverinvmiddle.tex

Действие оператора $\hat{U}_s$ описывается следующим соотношением
(см. \autoref{figQuantCompGroverInvMiddle}):
\begin{equation}
\hat{U}_s\left(\sum_x \alpha_x \left|x\right>\right) = 
\sum_x \left(2 \mathcal{M} - \alpha_x \right)\left|x\right>,
\label{eqQuantCompGroverUs}
\end{equation} 
где $\mathcal{M} = \sum_x \frac{\alpha_x}{N}$.

Оператор $\hat{U}_s$ может быть переписан в следующем виде
\begin{equation}
\hat{U}_s = 
2 \left|s\right>\left<s\right| - \hat{I},
\nonumber
\end{equation}
где $\left|s\right>=\frac{1}{\sqrt{N}}\sum_x \left|x\right>$ -
начальное состояние в алгоритме Гровера.
Действительно
\begin{eqnarray}
\left(2 \left|s\right>\left<s\right| - \hat{I}\right)
\left(\sum_x \alpha_x \left|x\right>\right) =
\nonumber \\
=  2 \sum_x \alpha_x \left<s\right.\left|x\right> \left|s\right> 
- \sum_x \alpha_x \left|x\right> = 
\nonumber \\
=
\frac{2}{N} \sum_x \alpha_x \sum_x \left|x\right> -
\sum_x \alpha_x \left|x\right> = 
\nonumber \\
= \sum_x \left( 2 \mathcal{M} -\alpha_x \right) \left|x\right>,
\nonumber
\end{eqnarray}
что совпадает с (\ref{eqQuantCompGroverUs}).

\input part4/quantcomp/groveranalyze.tex

\input part4/quantcomp/groverrealize.tex
