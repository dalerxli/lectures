%% -*- coding:utf-8 -*- 
\section{Теория сложности алгоритмов и квантовые вычисления}

При решении той или иной задачи возникают прежде всего два
вопроса. Имеет ли поставленная задача решение в принципе, т. е. может ли быть
построен алгоритм решающий поставленную задачу. Как только найден
положительный ответ на первый вопрос, т. е. построен алгоритм решающий
задачу, то следующей проблемой встает вопрос о практической
реализуемости предложенного решения. Здесь нам поможет теория
сложности алгоритмов которая решает вопрос о реализуемости тех или
иных алгоритмов. 

\subsection{Большие числа}
В следующей таблице \cite{bSchneier} 
% стр. 34
приведены некоторые большие числа
которые полезны при анализе сложности алгоритмов
\begin{table}
\centering
\begin{tabular}{|p {10cm}|p {5cm}|}
\hline
Физический аналог & Число \\ \hline
Вероятность быть убитым молнией (в течение дня) &
1 из 9 миллиардов ($2^{33}$) \\ 
Вероятность выиграть главный приз в государственной лотерее США &
1 из 4000000 ($2^{22}$) \\
Вероятность выиграть главный приз в государственной лотерее США и быть
убитым молнией в течение того же дня &
1 из $2^{61}$ \\
Вероятность утонуть (в США в течение года) &
1 из 59000 ($2^{16}$) \\
Вероятность погибнуть в автокатастрофе (в США в году) &
1 из 6100 ($2^{13}$) \\
Вероятность погибнуть в автокатастрофе (в США в течение времени жизни) &
1 из 88 ($2^7$) \\
Время до следующего оледенения &
14000 ($2^{14}$) лет \\
Время до превращения Солнца в сверхновую звезду &
$10^9$ ($2^{30}$) лет \\
Возраст планеты &
$10^9$ ($2^{30}$) лет \\
Возраст Вселенной & 
$10^{10}$ ($2^{34}$) лет \\
Число атомов планеты &
$10^{51}$ ($2^{170}$) \\
Число атомов Солнца &
$10^{57}$ ($2^{190}$) \\
Число атомов галактики & 
$10^{67}$ ($2^{223}$) \\
Число атомов Вселенной &
$10^{77}$ ($2^{265}$) \\
Объем Вселенной &
$10^{84}$ ($2^{280}$) $\mbox{см}^3$ \\
\hline
\multicolumn{2}{|l|}{Если Вселенная конечна:} \\
\hline
Полное время жизни вселенной &
$10^{11}$ ($2^{37}$) лет \\
& $10^{18}$ ($2^{61}$) секунд \\
\hline
\multicolumn{2}{|l|}{Если Вселенная бесконечна:} \\
\hline
Время до остывания легких звезд &
$10^{14}$ ($2^{47}$) лет \\
Время до отрыва планет от звезд &
$10^{15}$ ($2^{50}$) лет \\
Время до отрыва звезд от галактик & 
$10^{19}$ ($2^{64}$) лет \\
Время до разрушения орбит гравитационной радиацией &
$10^{20}$ ($2^{67}$) лет \\
Время до разрушения черных дыр процессами Хокинга &
$10^{64}$ ($2^{213}$) лет \\
Время до превращения материи в жидкость при нулевой температуре &
$10^{65}$ ($2^{216}$) лет \\
\hline
\end{tabular}
\caption{Большие числа \cite{bSchneier}}
\label{tblBigNumber}
\end{table}

\subsection{Классы сложности алгоритмов $P$, $NP$, $BQP$ }

\begin{definition}
Алгоритм принадлежит к классу $P$ (Polynomial) если он может быть решен за
 $O\left(N^k\right)$ шагов на детерминированной машине Тьюринга
(см. \ref{addTuring}).  Здесь $N$ - размер исходной задачи, $k$ -
произвольное целое число, которое не зависит от $N$. 
\end{definition}

\begin{definition}
Алгоритм принадлежит к классу $NP$ (Nondetermenistic Polynomial )если
он может быть решен за  $O\left(N^k\right)$ шагов на
недетерминированной машине Тьюринга.  Здесь $N$ - размер исходной
задачи, $k$ - произвольное целое число, которое не зависит от $N$. 
\end{definition}

