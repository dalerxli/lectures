%% -*- coding:utf-8 -*- 
\section{Алгоритм Шора}
\label{Part4QuantCompShor}
Один из наиболее популярных алгоритмов шифрования RSA (см. 
\autoref{AddRSA}) \index{алгоритм RSA}
построен на предположении о сложности факторизации
(разложимости на простые множители) больших чисел. Соответственно
алгоритмы позволяющие осуществлять разложение на простые множители
представляют особый интерес. Ниже представлено описание такого
алгоритма предложенное Шором \cite{bShor94}.

\subsection{Факторизация чисел и нахождение периода функций}
Задача факторизации некоторого числа $N$ тесно связана с нахождением периода
функций. Рассмотрим следующую, которая называется функцией возведения
в степень по модулю
\begin{equation}
f\left(x, a\right) = a^x mod N.
\label{eqPart4QuantCompShorClassPart}
\end{equation}
Функция (\ref{eqPart4QuantCompShorClassPart}) зависит от
анализируемого числа $N$ и двух аргументов $x$ и $a$. Аргумент $a$
выбирается из следующих условий
\begin{eqnarray}
0 < a < N,
\nonumber \\
\mbox{НОД}\left(N, a\right) = 1.
\label{eqPart4QuantCompShorAConditions}
\end{eqnarray}
Типичный график функции (\ref{eqPart4QuantCompShorClassPart}) представлен на
\autoref{picPart4QuantCompShorClassPart}.

% maxima a = 2 N = 21
%draw2d(yrange=[0, 18], points_joined = true, point_type = 6,
%point_size = 2, points(makelist([x, power_mod(2, x, 21)], x, 0, 30)),
%user_preamble="set output 'picshorclass.tex'; set terminal latex; set
%xlabel '$x$'; set ylabel '$f(x)$'");

\input ./part4/quantcomp/figshorclass.tex

Условия выбора коэффициента $a$
(\ref{eqPart4QuantCompShorAConditions}) такие что $a$ и $N$ не имеют
общих делителей. Если же такие делители существуют, то они являются
искомым решением задачи факторизации и легко находятся с помощью
алгоритма Евклида (см. \autoref{AddEuclidean}).

Функция (\ref{eqPart4QuantCompShorClassPart}) периодическая,
т. е. существует такое число $r$, что $f\left(x + r, a\right) = 
f\left(x, a\right)$. Минимальное из возможных чисел $r$ называется
периодом функции (\ref{eqPart4QuantCompShorClassPart}). 

Для доказательства периодичности отметим, что $f\left(x, a\right)$ не
может быть равной нулю. Действительно если выполнено условие
$f\left(x, a\right) = 0$, то 
\[
\exists x \in \left\{0, 1, \dots\right\}:
a^x = k \cdot N,
\]
где $k$ - целое число, что не возможно в силу
взаимной простоты $a$ и $N$ (\ref{eqPart4QuantCompShorAConditions})
\footnote{При это предполагается очевидно, что $N > 1$}.

Таким образом область значений функции
(\ref{eqPart4QuantCompShorClassPart}) ограничена множеством 
\begin{equation}
f\left(x,
a\right) \in \left\{1, \dots, N - 1\right\},
\nonumber
\end{equation}
откуда 
\begin{equation}
\exists k,j: k > j, k,j \in \left\{0, 1, \dots, N\right\},
f\left(k,a\right) = f\left(j,a\right),
\nonumber
\end{equation}
что и доказывает периодичность функции (\ref{eqPart4QuantCompShorClassPart}).

Пусть $k = j + r$, тогда
\[
a^k \mod{N} = a^{j + r} \mod{N} = a^j a^r \mod{N}= a^j \mod{N},
\]
т. к. $a$ и $N$ взаимно просты то мы можем записать
\begin{equation}
a^r \equiv 1 \mod{N}.
\label{eqPart4QuantCompShorPeriodConditions}
\end{equation}


Период функции (\ref{eqPart4QuantCompShorClassPart}) может быть как
четным так и нечетным. В алгоритме Шора нам интересен первый вариант:
период - четное число. В противном случае выбирают новое число $a$ и
повторяют нахождение периода. Таким образом с учетом $r= 2\cdot l$ мы
можем переписать (\ref{eqPart4QuantCompShorPeriodConditions}) в виде
\begin{equation}
a^{2 \cdot l} \equiv 1 \mod{N},
\nonumber
\end{equation}
при этом в силу того $r$ - минимальное число удовлетворяющее условию
периодичности, то
\[
a^{l} \not\equiv 1 \mod{N}.
\]
Если при этом подобрать число $a$ таким образом, что 
\[
a^{l} \not\equiv -1 \mod{N},
\]
то имеем 
\begin{equation}
\left(a^l - 1\right)\left(a^l + 1\right) = k \cdot N,
\label{eqPart4QuantCompShorPeriodFinal}
\end{equation}
где $k$ - некоторое целое число. Из
(\ref{eqPart4QuantCompShorPeriodFinal}) получаем что $a^l \pm 1$ имеют
общие нетривиальные (отличные от 1) делители с $N$.

\begin{example}
\emph{Нахождение делителей числа $N=21$}
\label{exPart4QuantCompShorGCD}
В качестве примера рассмотрим задачу о нахождении делителей числа $N =
21$. Выбрав $a=2$ мы получим период функции
(\ref{eqPart4QuantCompShorClassPart}) $r = 6$ (см. рис.
\ref{picPart4QuantCompShorClassPart}). 
Очевидно что 
\[
2^3 \equiv 8 \mod{21} \not\equiv -1 \mod{21}.
\]
Таким образом находя соответствующие наибольшие общие делители решаем
задачу
\begin{eqnarray}
\mbox{НОД}\left( 2^3 - 1, 21 \right) = \mbox{НОД}\left( 7, 21 \right)
= 7,
\nonumber \\
\mbox{НОД}\left( 2^3 + 1, 21 \right) = \mbox{НОД}\left( 9, 21 \right)
= 3,
\nonumber \\
21 = 7 \cdot 3.
\nonumber
\end{eqnarray}
\end{example}

Таким образом задача факторизации числа $N$ может быть сведена к
задаче о нахождении периода некоторой функции посредством следующего
алгоритма:
\index{aлгоритм Шора}
\begin{algorithm}
\caption{Алгоритм Шора}
\begin{algorithmic}
    \STATE $a \Leftarrow 0$
    \REPEAT
        \STATE Выбрать новое число $a$ такое, что $0 < a < N$
        \IF{$\mbox{НОД}\left(a, N\right) \ne 1$} 
            \RETURN $a$
        \ENDIF
        \STATE Найти период $r$ функции $f\left(x, a\right) = a^x \mod{N}$
    \UNTIL{ ($r \not\equiv 0 \mod{2}$) \OR ($a^{\frac{r}{2}} \equiv -1
      \mod{N}$)}
    \RETURN $\mbox{НОД}\left(a^{\frac{r}{2}} \pm 1, N\right)$
\end{algorithmic}
\end{algorithm}

\input part4/quantcomp/fourier.tex


\input part4/quantcomp/periodfinding.tex
