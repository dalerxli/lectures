%% -*- coding:utf-8 -*- 
\section{Квантовое преобразование Фурье и дискретное логарифмирование}
Дискретный логарифм (см. \autoref{AddDiscretLog}) является основой для
большого числа современных криптографических алгоритмов (см.
\autoref{sec:add:dm:elgamal}, \autoref{sec:add:dm:dh}). Вместе с тем 
метод предложеный Шором для факторизации целых чисел может быть также
применен для вычисления дискретных логарифмов, что делает возможным
взлом соотвествующих криптографических алгоритмов.

Поставим задачу следующим образом: имеется выражение 
\[
b = a^x \mod p,
\]
в котором числа $a, b$ и $p$ заданны, а число $x$ является
неизвестным, которое необходимо определить. По аналогии с применением
квантового преобразования Фурье для факторизации чисел (см.
\autoref{sec:part4:algoshor:periodfind}), мы должны
построить некоторую периодическую функцию, период которой даст нам
возможность определить искомое число $x$.

Выберем в качестве анализируемой функции
\begin{equation}
f\left(x_1, x_2\right) = b^{x_1}a^{x_2} = a^{x \cdot x_1} a^{x_2} \mod p
\label{eq:part4:quantcomp:discretlogfunc}
\end{equation}
Функция двух аргументов \eqref{eq:part4:quantcomp:discretlogfunc}
будет периодической, при этом период может быть записан в виде пары
чисел $(t_1, t_2)$ так что
\begin{eqnarray}
f\left(x_1 + t_1, x_2\right) = f\left(x_1, x_2\right),
f\left(x_1, x_2 + t_2\right) = f\left(x_1, x_2\right),
\end{eqnarray}
т. о. 
\[
a^{x \cdot \left( x_1 + t_1 \right)} a^{x_2} = a^{x \cdot x_1} a^{x_2}
\mod p,
\]
т.е. 
\[
a^{x t_1} = 1 \mod p.
\]
С другой стороны
\[
a^{x \cdot t_1} a^{x_2 + t_2} = a^{x \cdot x_1} a^{x_2}
\mod p,
\]
т.е.
\[
a^{t_2} = 1 \mod p.
\]
Следовательно если обозначить $t_1 = t$, то период может быть
переписан в виде
\[
(t_1, t_2) = (t, x t).
\]


TBD

