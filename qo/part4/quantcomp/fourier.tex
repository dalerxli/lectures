%% -*- coding:utf-8 -*- 
\section{Квантовое преобразование Фурье}
Для анализа периодических последовательностей (функций) может быть
использовано дискретное преобразование Фурье
(см. \ref{AddFourier}), которое определяется следующим соотношением
(\ref{eqAddFourierDiscretFourier}):
\begin{equation}
\tilde{X}_k = \sum^{M - 1}_{m = 0} x_m e^{-\frac{2 \pi}{M} k\cdot m},
\label{eqPart4QuantCompShorFourierDiscretFourier}
\end{equation}
где исходная последовательность чисел $\left\{x_m\right\}$ имеет $M$
членов. 

\subsection{Схема квантового преобразования Фурье}
Квантовое преобразование Фурье имеет дело с состояниями вида
\begin{equation}
\left|x\right> = \sum_{k = 0}^{M - 1}x_k \left|k\right>,
\label{eqPart4QuantCompShorQuantFourierSeries}
\end{equation}
где имеется последовательнеость амплитуд $\left\{x_k\right\}$, которая
задает исходную последовательность для преобразования Фурье 
(\ref{eqPart4QuantCompShorFourierDiscretFourier}). В базисном векторе
$\left|k\right>$ записан номер члена этой последовательности.   

Очевидно, что члены последовательности
(\ref{eqPart4QuantCompShorQuantFourierSeries})  должны удовлетворять
условию нормировки 
\[
\sum_k\left|x_k\right|^2 = 1.
\]

Допустим что некоторый оператор $\hat{F}^{M}$ (оператор квантового
преобразования Фурье) преобразует базисный вектор $\left|k\right>$ по
правилу задаваемому соотношением (\ref{eqAddFourierDiscretFourier}):
\begin{equation}
\hat{F}^{M}\left|k\right> = \frac{1}{\sqrt{M}}\sum_{j = 0}^{M -1}
e^{-i \omega k j}\left|j\right>_{inv} 
\label{eqPart4QuantCompShorQuantFourierBasis}
\end{equation}
Системы базисных векторов $\left\{\left|k\right>\right\}$ и 
$\left\{\left|k\right>_{inv}\right\}$ представляют собой один и тот же
набор векторов которые пронумерованны различным способом.

Из (\ref{eqPart4QuantCompShorQuantFourierSeries}) и
(\ref{eqPart4QuantCompShorQuantFourierBasis}) получим
\begin{eqnarray}
\hat{F}^{M}\left|x\right> = \sum_{j = 0}^{M - 1}x_k \hat{F}^{M}
\left|k\right> = 
\nonumber \\
= \frac{1}{\sqrt{M}}\sum_{k = 0}^{M -1}\sum_{j = 0}^{M - 1}
e^{-i \omega k j}x_k\left|j\right>_{inv} = 
\nonumber \\
= \sum_{j = 0}^{M - 1} \left\{\frac{1}{\sqrt{M}}\left(
\sum_{k = 0}^{M - 1}e^{-i \omega k j} x_k
\right)\right\}\left|j\right>_{inv} = 
\nonumber \\
= \sum_{j = 0}^{M - 1}\tilde{X}_j\left|j\right>_{inv} = \left|\tilde{X}\right>_{inv},
\nonumber
\end{eqnarray}
где 
\begin{equation}
\tilde{X}_j = \tilde{X}_j^{M} = 
\frac{1}{\sqrt{M}}\sum_{k = 0}^{M - 1}e^{-i \omega k j} x_k.
\label{eqPart4QuantCompShorQuantFourierEnd}
\end{equation}
Выражение (\ref{eqPart4QuantCompShorQuantFourierEnd}) повторяет
классический аналог (\ref{eqAddFourierDiscretFourier}), т. е. можно записать
\[
 \left|x\right> \longleftrightarrow \left|\tilde{X}\right>_{inv}.
\]

\input ./part4/quantcomp/figquantfourier0.tex

Допустим теперь, что на вход нашей системы подается состояние вида
(\ref{eqPart4QuantCompShorQuantFourierSeries}) которое представляет
собой суперпозицию $M$ базисных состояний
$\left\{\left|k\right>\right\}$ 
(см. рис. \ref{figQuantCompQuantFourier0}). 
Предположим что число базисных
состояний является степенью двойки, т. е. базисное состояние
представимо в виде тензорного произведения $n = \log_2{M}$ кубитов:
\begin{equation}
\left|k\right> = \left|a^{(k)}_0\right> \otimes  \left|a^{(k)}_1\right>
\otimes \cdots \otimes \left|a^{(k)}_{n-1}\right>, 
\nonumber
\end{equation}
где
\begin{eqnarray}
k = a^{(k)}_0 + 2^1 a^{(k)}_1 + \dots + 2^{n-1} a^{(k)}_{n-1},
\nonumber \\
a^{(k)}_i \in \left\{0, 1\right\}.
\nonumber
\end{eqnarray}

На выходе
(см. рис. \ref{figQuantCompQuantFourier0})
мы имеем суперпозицию $M$ базисных состояний
$\left\{\left|j\right>_{inv}\right\}$, где для состояния
$\left|j\right>_{inv}$ получим 
\begin{equation}
\left|j\right>_{inv} = \left|b^{(j)}_{n-1}\right> \otimes
\left|b^{(j)}_{n-2}\right> 
\otimes \cdots \otimes \left|b^{(j)}_{0}\right>, 
\nonumber
\end{equation}
где
\begin{eqnarray}
j = b^{(j)}_0 + 2^1 b^{(j)}_1 + \dots + 2^{n-1} b^{(j)}_{n-1},
\nonumber \\
b^{(j)}_i \in \left\{0, 1\right\}.
\nonumber
\end{eqnarray}

Из формулы (\ref{eqAddFourierDiscretFourierFFT}) можно
заметить, что если у нас имеется входной сигнал $x$ состоящий из $n =
\log_2{M}$ 
битов, то бит $a^{(k)}_0$ может быть использован для выбора четных
(первого члена суммы (\ref{eqAddFourierDiscretFourierFFT}))
или нечетных 
(второго члена суммы (\ref{eqAddFourierDiscretFourierFFT})).

Действительно, исключая $a^{(k)}_0$, состояние
(\ref{eqPart4QuantCompShorQuantFourierSeries}) можно представить в
виде суммы четных и нечетных компонент: 
\begin{eqnarray}
\left|x\right> = \sum_{k = 0}^{M - 1}x_k \left|k\right> = 
\sum_{k = 0}^{M - 1}x_k \left|a^{(k)}_0\right> \otimes  \left|a^{(k)}_1\right>
\otimes \cdots \otimes \left|a^{(k)}_{n-1}\right> = 
\nonumber \\
 = \sum_{m = 0}^{\frac{M}{2} - 1}x_{k=2m} \left|0\right> \otimes  \left|a^{(k)}_1\right>
\otimes \cdots \otimes \left|a^{(k)}_{n-1}\right> +
\nonumber \\
+
\sum_{m = 0}^{\frac{M}{2} - 1}x_{k=2m + 1} \left|1\right> \otimes  \left|a^{(k)}_1\right>
\otimes \cdots \otimes \left|a^{(k)}_{n-1}\right> = 
\nonumber \\
 = \sum_{m = 0}^{\frac{M}{2} - 1}x_{k=2m} \left|0\right> \otimes  \left|m\right> +
\sum_{m = 0}^{\frac{M}{2} - 1}x_{k=2m + 1} \left|1\right> \otimes  \left|m\right> = 
\nonumber \\
= \sum_{m = 0}^{\frac{M}{2} - 1}x_{2m} \left|2m\right> +
\sum_{m = 0}^{\frac{M}{2} - 1}x_{2m + 1} \left|2m+1\right>,
\nonumber
\end{eqnarray}
где
\begin{equation}
m = a^{(k)}_1 + 2^1 a^{(k)}_2 + \dots + 2^{n-2} a^{(k)}_{n-1}.
\nonumber
\end{equation}

\input ./part4/quantcomp/figquantfourier1.tex

Применяя преобразование Фурье только для старших бит $\hat{F}^{\frac{M}{2}}$,
т. е. исключая $a^{(k)}_0$, получим (см. рис. \ref{figQuantCompQuantFourier1}):
\begin{eqnarray}
\left|x\right> \rightarrow
\hat{F}^{\frac{M}{2}} \sum_{m = 0}^{\frac{M}{2} - 1}x_{2m} \left|2m\right> +
\hat{F}^{\frac{M}{2}} \sum_{m = 0}^{\frac{M}{2} - 1}x_{2m + 1}
\left|2m+1\right> = 
\nonumber \\
=
\hat{F}^{\frac{M}{2}} \sum_{m = 0}^{\frac{M}{2} - 1}x_{2m} 
\left|0\right> \otimes  \left|m\right> +
\hat{F}^{\frac{M}{2}} \sum_{m = 0}^{\frac{M}{2} - 1}x_{2m + 1}
\left|1\right> \otimes  \left|m\right>
=
\nonumber \\
=
\sum_{m = 0}^{\frac{M}{2} - 1}x_{2m} 
\left|0\right> \otimes \hat{F}^{\frac{M}{2}} \left|m\right> +
\sum_{m = 0}^{\frac{M}{2} - 1}x_{2m + 1}
\left|1\right> \otimes \hat{F}^{\frac{M}{2}} \left|m\right>.
\label{eqPart4QuantCompShorQuantFourierStep1}
\end{eqnarray}
С учетом выражения (\ref{eqPart4QuantCompShorQuantFourierBasis}) получим
\begin{equation}
\hat{F}^{\frac{M}{2}} \left|m\right> = \sqrt{\frac{2}{M}}
\sum_{j= 0}^{\frac{M}{2} - 1} e^{-i \frac{4 \pi}{M} m j}\left|j\right>_{inv}.
\nonumber
\end{equation}
Таким образом для (\ref{eqPart4QuantCompShorQuantFourierStep1}) имеем
\begin{eqnarray}
\left|x\right> \rightarrow
\sum_{m = 0}^{\frac{M}{2} - 1}x_{2m} 
\left|0\right> \otimes \hat{F}^{\frac{M}{2}} \left|m\right> +
\sum_{m = 0}^{\frac{M}{2} - 1}x_{2m + 1}
\left|1\right> \otimes \hat{F}^{\frac{M}{2}} \left|m\right> = 
\nonumber \\
=
\sqrt{\frac{2}{M}} \sum_{j = 0}^{\frac{M}{2} - 1} e^{-i \frac{4 \pi}{M} m j} 
\sum_{m = 0}^{\frac{M}{2} - 1}x_{2m} \left|0\right> \otimes
\left|j\right>_{inv}
+
\nonumber \\
+
\sqrt{\frac{2}{M}} \sum_{j = 0}^{\frac{M}{2} - 1} e^{-i \frac{4 \pi}{M} m j} 
\sum_{m = 0}^{\frac{M}{2} - 1}x_{2m+1} \left|1\right> \otimes
\left|j\right>_{inv}
=
\nonumber \\
=
\sum_{j = 0}^{\frac{M}{2} - 1}  
\left( \sqrt{\frac{2}{M}} 
\sum_{m = 0}^{\frac{M}{2} - 1} e^{-i \frac{4 \pi}{M} m j} x_{2m} 
\right) \left|j\right>_{inv}
+
\nonumber \\
+
\sum_{j = 0}^{\frac{M}{2} - 1}
\left( \sqrt{\frac{2}{M}}  
\sum_{m = 0}^{\frac{M}{2} - 1}e^{-i \frac{4 \pi}{M} m j} x_{2m+1} 
\right)
\left|\frac{M}{2} + j\right>_{inv}
=
\nonumber \\
= \sum^{\frac{M}{2} - 1}_{j = 0}  \tilde{A}_{j} \left|j\right>_{inv} +
\sum^{\frac{M}{2} - 1}_{j = 0}  \tilde{B}_{j} \left|\frac{M}{2} + j\right>_{inv},
\nonumber
\end{eqnarray}
где
\begin{eqnarray}
\tilde{A}_{j} = 
\sqrt{\frac{2}{M}} 
\sum_{m = 0}^{\frac{M}{2} - 1} e^{-i \frac{4 \pi}{M} m j} x_{2m} 
\nonumber \\
\tilde{B}_{j} =
\sqrt{\frac{2}{M}} 
\sum_{m = 0}^{\frac{M}{2} - 1} e^{-i \frac{4 \pi}{M} m j} x_{2m+1} 
\label{eqPart4QuantCompShorAB}
\end{eqnarray}

\input ./part4/quantcomp/figquantfourier2.tex

Если добавить теперь фазовый сдвиг для нечетных элементов, т. е. для
тех у которых $a_0^k = 1$ то получим схему изображенную на
рис. \ref{figQuantCompQuantFourier2}: 
\begin{eqnarray}
\left|x\right> \rightarrow
\hat{F}^{\frac{M}{2}} \sum_{m = 0}^{\frac{M}{2} - 1}x_{2m} \left|2m\right> +
\hat{R}\hat{F}^{\frac{M}{2}} \sum_{m = 0}^{\frac{M}{2} - 1} x_{2m + 1}
\left|2m+1\right> =
\nonumber \\
= 
\sum^{\frac{M}{2} - 1}_{j = 0} \tilde{A}_{j} \left|j\right>_{inv} +
\sum^{\frac{M}{2} - 1}_{j = 0}  
\tilde{B}_{j} \hat{R}\left|\frac{M}{2} + j\right>_{inv},
\nonumber \\
= 
\sum^{\frac{M}{2} - 1}_{j = 0}  \tilde{A}_{j} \left|j\right>_{inv} +
\sum^{\frac{M}{2} - 1}_{j = 0}  
\tilde{C}_{j} \left|\frac{M}{2} + j\right>_{inv}.
\label{eqPart4QuantCompShorFourierStep2}
\end{eqnarray}
Воспользовавшись выражением
\begin{equation}
\hat{R}_l \left|b^{(j)}_l\right> = 
exp{\left(- 2 \pi i \frac{b^{(j)}_l}{2^{n-l}}\right)}
\left|b^{(j)}_l\right>
\nonumber
\end{equation}
получим, что оператор $\hat{R}$ действует на состояние 
$\left|\frac{M}{2} + j\right>_{inv}$ следующим образом:
\begin{eqnarray}
\hat{R}\left|\frac{M}{2} + j\right>_{inv} = 
\hat{R}\left|1\right> \otimes  \left|j\right>_{inv} = 
\nonumber \\
= 
\left|1\right> \otimes \hat{R}_0 \left|b^{(j)}_0\right>
\otimes \cdots \otimes \hat{R}_{n-2} \left|b^{(j)}_{n-2}\right> = 
\nonumber \\
= 
\prod_{l = 0}^{n-2}exp{\left(- 2 \pi i \frac{2^lb^{(j)}_l}{2^n}\right)}
\left|1\right> \otimes \left|j\right>_{inv} = 
\nonumber \\
=
exp{\left(-2 \pi i \frac{j}{M}\right)}
\left|\frac{M}{2} + j\right>_{inv} 
\label{eqPart4QuantCompShorFourierStep2Pre}
\end{eqnarray}
При выводе (\ref{eqPart4QuantCompShorFourierStep2Pre}) было учтено,
что $j = b^{(j)}_0 + 2^1 b^{(j)}_1 + \dots + 2^{n-2} b^{(j)}_{n-2}$. 

Таким образом для $\tilde{C}_{j}$ в 
(\ref{eqPart4QuantCompShorFourierStep2}) имеем
\begin{eqnarray}
\tilde{C}_{j} = 
\sqrt{\frac{2}{M}} 
\sum_{m = 0}^{\frac{M}{2} - 1} 
e^{- 2 \pi i \frac{j}{M}}
e^{-i \frac{4 \pi}{M} m j} x_{2m+1} =
\nonumber \\
=
\sqrt{\frac{2}{M}} 
\sum_{m = 0}^{\frac{M}{2} - 1} 
e^{-i \frac{2 \pi}{M} \left(2m+1\right) j} x_{2m+1}
\label{eqPart4QuantCompShorC}
\end{eqnarray}

\input ./part4/quantcomp/figquantfourier.tex

Если теперь применить преобразование Адамара для кубита
$\left|a_0\right>$, то получим схему изображенную на
рис. \ref{figQuantCompQuantFourier}. При этом исходное состояние
преобразуется по следующему закону:
\begin{eqnarray}
\left|x\right> \rightarrow
\hat{H}_0\hat{F}^{\frac{M}{2}} \sum_{m = 0}^{\frac{M}{2} - 1}x_{2m} \left|2m\right> +
\hat{H}_0\hat{R}\hat{F}^{\frac{M}{2}}\sum_{m = 0}^{\frac{M}{2} - 1} x_{2m + 1} =
\nonumber \\
=
\sum_{j = 0}^{\frac{M}{2} - 1}
\tilde{A}_{j}
\hat{H}\left|0\right> \otimes \left|j\right>_{inv}
+
\sum_{j = 0}^{\frac{M}{2} - 1} 
\tilde{C}_{j}
\hat{H}\left|1\right> \otimes \left|j\right>_{inv} 
=
\nonumber \\
= 
\frac{1}{\sqrt{2}}\sum_{j = 0}^{\frac{M}{2} - 1}
\tilde{A}_{j} 
\left(\left|0\right> + \left|1\right> \right) \otimes  
\left|j\right>_{inv}
+
\frac{1}{\sqrt{2}}\sum_{j = 0}^{\frac{M}{2} - 1}
\tilde{C}_{j} 
\left(\left|0\right> - \left|1\right> \right) \otimes  
\left|j\right>_{inv}
=
\nonumber \\
=
\sum_{j = 0}^{\frac{M}{2} - 1}
\frac{\tilde{A}_{j} + \tilde{C}_{j} }{\sqrt{2}} 
\left|0\right> \otimes \left|j\right>_{inv} +
\sum_{j = 0}^{\frac{M}{2} - 1}
\frac{ \tilde{A}_{j} - \tilde{C}_{j}}{\sqrt{2}} 
\left|1\right> \otimes \left|j\right>_{inv}
=
\nonumber \\
=
\sum_{j = 0}^{\frac{M}{2} - 1}
\frac{\tilde{A}_{j} + \tilde{C}_{j} }{\sqrt{2}} \left|j\right>_{inv} +
\sum_{j = 0}^{\frac{M}{2} - 1}
\frac{ \tilde{A}_{j} - \tilde{C}_{j}}{\sqrt{2}} 
\left|\frac{M}{2} + j \right>_{inv}.
\label{eqPart4QuantCompShorFourierStep3}
\end{eqnarray}
Для членов (\ref{eqPart4QuantCompShorFourierStep3}) с учетом равенств
(\ref{eqPart4QuantCompShorAB}) и (\ref{eqPart4QuantCompShorC}) имеем:
\begin{eqnarray}
\frac{\tilde{A}_{j} + \tilde{C}_{j} }{\sqrt{2}} = 
\sqrt{\frac{1}{M}} 
\sum_{m = 0}^{\frac{M}{2} - 1} e^{-i \frac{4 \pi}{M} m j} x_{2m}  +
\sqrt{\frac{1}{M}} 
\sum_{m = 0}^{\frac{M}{2} - 1} 
e^{-i \frac{2 \pi}{M} \left(2m+1\right) j} x_{2m+1} = 
\nonumber \\
=
\sqrt{\frac{1}{M}} \sum_{m = 0}^{M - 1}
e^{-i \frac{2 \pi}{M} m j} x_{m}
\label{eqPart4QuantCompShorFourierStep3_1}
\end{eqnarray}
и
\begin{eqnarray}
\frac{\tilde{A}_{j} - \tilde{C}_{j} }{\sqrt{2}} = 
\sqrt{\frac{1}{M}} 
\sum_{m = 0}^{\frac{M}{2} - 1} e^{-i \frac{4 \pi}{M} m j} x_{2m}  -
\sqrt{\frac{1}{M}} 
\sum_{m = 0}^{\frac{M}{2} - 1} 
e^{-i \frac{2 \pi}{M} \left(2m+1\right) j} x_{2m+1}
= 
\nonumber \\
=
\sqrt{\frac{1}{M}} \sum_{m = 0}^{M - 1}
e^{-i \frac{2 \pi}{M} m j} x_{m} \frac{1 + e^{-i \pi m}}{2}
-
\sqrt{\frac{1}{M}} \sum_{m = 0}^{M - 1}
e^{-i \frac{2 \pi}{M} m j} x_{m} \frac{1 - e^{-i \pi m}}{2} 
=
\nonumber \\
=
\sqrt{\frac{1}{M}} \sum_{m = 0}^{M - 1}
e^{-i \frac{2 \pi}{M} m j} e^{-i \pi m } x_{m} 
=
\sqrt{\frac{1}{M}} \sum_{m = 0}^{M - 1}
e^{-i \frac{2 \pi}{M} m j} e^{-i \frac{2 \pi}{M} m \frac{M}{2} } x_{m} 
=
\nonumber \\
=
\sqrt{\frac{1}{M}} \sum_{m = 0}^{M - 1}
e^{-i \frac{2 \pi}{M} m \left(\frac{M}{2} + j\right)} x_{m}
\label{eqPart4QuantCompShorFourierStep3_2}
\end{eqnarray}

Объединяя (\ref{eqPart4QuantCompShorFourierStep3}), 
(\ref{eqPart4QuantCompShorFourierStep3_1}) и
(\ref{eqPart4QuantCompShorFourierStep3_2}) окончательно получим 
\begin{eqnarray}
\left|x\right> \rightarrow
\sum_{j = 0}^{\frac{M}{2} - 1} \sqrt{\frac{1}{M}} \sum_{m = 0}^{M - 1}
e^{-i \frac{2 \pi}{M} m j} x_{m} \left|j\right>_{inv} +
\nonumber \\
+
\sum_{j = 0}^{\frac{M}{2} - 1} \sqrt{\frac{1}{M}} \sum_{m = 0}^{M - 1}
e^{-i \frac{2 \pi}{M} m \left(\frac{M}{2} + j\right)} x_{m} 
\left|\frac{M}{2} + j\right>_{inv} =
\nonumber \\
= \sum_{j = 0}^{M - 1} \tilde{X}_j^{M} \left|j\right>_{inv}
\nonumber
\end{eqnarray}
