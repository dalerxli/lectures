%% -*- coding:utf-8 -*- 
\section{Квантовые логические элементы}
Каким образом может быть сконструирован элемент осуществляющий
преобразование $\hat{U}_f$ \eqref{eqQuantCompQuantComp}. Существует
набор элементов из которых можно построить, с заданной точностью,
элемент осуществляющий необходимое преобразование $\hat{U}_f$. Такие
наборы называются универсальными. 

\subsection{Универсальный набор квантовых вентилей}

\subsubsection{Тождественное преобразование}

\[
\hat{\sigma}_0 = \begin{pmatrix}
1 & 0 \\
0 & 1
\end{pmatrix}
\]

\subsubsection{Отрицание}

\[
\hat{\sigma}_1 = \begin{pmatrix}
0 & 1 \\
1 & 0
\end{pmatrix}
\]

\subsubsection{Фазовый сдвиг}

\[
\hat{\sigma}_2 = \begin{pmatrix}
1 & 0 \\
0 & -1
\end{pmatrix}
\]

\subsubsection{Преобразование Адамара}
\index{Преобразование Адамара!определение}
Одним из базовых квантовых логических элементов является
преобразование Адамара (см. \autoref{figQuantCompHadamar1}), которое
определяется следующими соотношениями
\begin{eqnarray}
\hat{H} \ket{0} = \ket{+} =  
\frac{\ket{0} + \ket{1} }{\sqrt{2}},
\nonumber \\
\hat{H} \ket{1} = \ket{-} = 
\frac{\ket{0} - \ket{1} }{\sqrt{2}},
\nonumber
\end{eqnarray}

\input ./part4/quantcomp/fighadamar1.tex

Это преобразование используется для получения суперпозиции состояний
содержащие все возможные значения аргумента вычисляемой функции
(см. \autoref{figQuantCompHadamar2}). 

\input ./part4/quantcomp/fighadamar2.tex

\subsubsection{CNOT}

Матрица преобразования имеет вид
\[
CNOT=\begin{pmatrix}
1 & 0 & 0 & 0 \\
0 & 1 & 0 & 0 \\
0 & 0 & 0 & 1 \\
0 & 0 & 1 & 0 
\end{pmatrix}
\]

Данный вентиль (см. \autoref{figQuantCompCNOT}) применяется для двух
кубит и инвертирует состояние 
второго кубита только если первый кубит равен единице.

\input ./part4/quantcomp/figcnot.tex

Таким образом если наше исходное состояние двух кубит было 
\[
\ket{\psi_i} = a \ket{00} + b \ket{01} + c \ket{10} + d \ket{11}
\]
то оно преобразуется в 
\[
CNOT \ket{\psi_i} = \ket{\psi_f} = 
a \ket{00} + b \ket{01} + c \ket{11} + d \ket{10}
\]

\subsubsection{Универсальный набор}

\begin{definition}[Универсальный набор квантовых вентилей]
Набор квантовых вентилей называют универсальным, если любое унитарное
преобразование можно аппроксимировать с заданной точностью конечной
последовательностью вентилей из этого набора. 
\end{definition}

\begin{theorem}[Китаев]
Набор $\hat{\sigma}_0, \hat{\sigma}_1,
\hat{\sigma}_2, \hat{H}, CNOT$ является универсальным.
\begin{proof}
TBD
\end{proof}
\end{theorem}


\subsection{Управляющие элементы}


\input ./part4/quantcomp/figcelem.tex

\input ./part4/quantcomp/figphase.tex
 


