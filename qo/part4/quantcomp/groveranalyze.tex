%% -*- coding:utf-8 -*- 
\subsection{Анализ алгоритма Гровера}

Нас будет интересовать два вопроса: какова алгоритмическая сложность
алгоритма Гровера и существуют ли алгоритмы которые могут выполнять
задачу поиска в неструктурированном объеме данных более эффективно чем
алгоритм Гровера.

Критерием эффективности алгоритма служит следующий факт: хороший
алгоритм должен находить искомое значение с минимальным числом вызовов
функции \eqref{eqQuantCompGroverF}.

Рассмотрим самую первую итерацию. Начальное состояние
$\left|\psi\right>_0$ имеет следующий вид
\begin{equation}
\left|\psi\right>_0 =
\sum_x \alpha_x \left|x\right> =  
\left|s\right> = 
\frac{1}{\sqrt{N}}\sum_x \left|x\right> = 
\frac{1}{\sqrt{N}}\sum_{x\ne x^{\ast}} \left|x\right> +
\frac{1}{\sqrt{N}} \left|x^{\ast}\right>.
\nonumber
\end{equation}
Таким образом коэффициент перед искомым элементом имеет вид
$\alpha_x^{\ast} = \frac{1}{\sqrt{N}}$. 

После применения оператора инверсии
фазы $U_{x^{\ast}}$ из \eqref{eqQuantCompGroverUxast} получим
\begin{equation}
\hat{U}_{x^{\ast}} \left|\psi\right>_0 =
\frac{1}{\sqrt{N}}\sum_{x\ne x^{\ast}} \left|x\right> - 
\frac{1}{\sqrt{N}} \left|x^{\ast}\right>.
\nonumber
\end{equation}

После применения оператора обращения относительно среднего $\hat{U}_s$ 
из \eqref{eqQuantCompGroverUs} получим
\begin{eqnarray}
\hat{U}_s \hat{U}_{x^{\ast}} \left|\psi\right>_0 = 
\hat{U}_G \left|\psi\right>_0 \approx 
\sum_{x\ne x^{\ast}} \left( 2 \frac{1}{\sqrt{N}} - \frac{1}{\sqrt{N}}
\right) \left|x\right> + 
\nonumber \\
+ \left( 2 \frac{1}{\sqrt{N}} +
\frac{1}{\sqrt{N}} \right) \left|x^{\ast}\right> = 
\nonumber \\
= 
\frac{1}{\sqrt{N}}\sum_{x\ne x^{\ast}} \left|x\right> + 
\frac{3}{\sqrt{N}} \left|x^{\ast}\right>.
\label{eqQuantCompGroverAnalyzeRought}
\end{eqnarray}
При выводе \eqref{eqQuantCompGroverAnalyzeRought} было принятно 
\[
\mathcal{M} = \frac{\sum_x \alpha_x}{N} \approx
\frac{N}{N \sqrt{N}} = \frac{1}{\sqrt{N}}.
\]
Таким образом после первой итерации алгоритма Гровера амплитуда
$\alpha_{x^{\ast}}$ возросла на $\frac{2}{\sqrt{N}}$. Если
апроксимировать этот результат на произвольную итерацию, то можно
получить, что $50\%$ вероятность обнаружить $\left|x^{\ast}\right>$
будет достижима за следующее число итераций:
\[
\frac{1}{\sqrt{2}}/\frac{2}{\sqrt{N}} =
\frac{\sqrt{N}}{2 \sqrt{2}} = O\left(\sqrt{N}\right).
\]

Более точные расчеты \cite{nielsen2000quantum} дают для числа итераций
$\frac{\pi}{4}\sqrt{N}$. Таким образом алгоритм Гровера может быть
записан следующим образом  
\begin{algorithm}
\caption{Алгоритм Гровера}
\label{algQuantCompGrover}
\begin{algorithmic}
    \STATE $\left|\psi\right>_0 \Leftarrow \frac{1}{\sqrt{N}}\sum_x 
    \left|x\right>$
    \STATE $t \Leftarrow 1$
    \REPEAT
    \STATE $\left|\psi\right>_t \Leftarrow \hat{U}_s\hat{U}_{x^{\ast}}
    \left|\psi\right>_{t-1}$
    \STATE $t \Leftarrow t + 1$
    \UNTIL{ ($t < \frac{\pi}{4}\sqrt{N}$)}
    \RETURN результат измерения состояния $\left|\psi\right>_t$
\end{algorithmic}
\end{algorithm}

Можно задать вопрос об оптимальности алгоритма Гровера: существует ли
квантовый алгоритм, который выполняет поиск в неструктурированном
объеме данных быстрее чем за $O\left(\sqrt{N}\right)$ обращений к
функции \eqref{eqQuantCompGroverF}. В статье
\cite{bBennettGroverOptimal} показано, что такого алгоритма
не существует.
