%% -*- coding:utf-8 -*- 
\input preamble.tex

\begin{document}
\Russian
\input title.tex

\section{Цели изучения дисциплины}
Целью данного курса является ознакомление студентов с теоретическими и
экспериментальными исследованиями в ряде областей современной
квантовой оптики.  На основе полученных сведений студенты смогут более
углубленно  изучать интересующие их разделы квантовой оптики и
заниматься исследовательской работой. 

\section{Место дисциплины в учебном плане}
Квантовая оптика изучает оптические явления, в которых проявляется
квантовая сущность света. Можно сказать, что квантовая оптика
рассматривает оптические явления, при которых свет и взаимодействующую
с ним систему необходимо описывать квантовыми уравнениями. 

В квантовой оптике рассматривается область частот примерно от 
\(f_1 \simeq 10^{13} \mbox{Гц}\) до \(f_2 \simeq 10^{18}
\mbox{Гц}\), т. е. от инфракрасного диапазона до
рентгеновского. Нижний предел определяется условием превышения
энергией кванта энергии теплового движения: 
\(\omega_1 \hbar > k T\). Верхний предел
устанавливается исходя из того, что в квантовой оптике
рассматриваются, как правило, нерелятивистские энергии электронов и,
следовательно, энергия кванта должна быть заметно меньше, чем энергия
покоя электрона: \(\omega_2 \hbar \ll m c^2\).

\section{Объем дисциплины по видам учебной работы и формы контроля}
\subsection{Распределение времени по видам занятий}
\begin{longtable}{|l|c|c|}
\hline
Разделы программы & Семестр & Лекции (часы) \\ \hline
Квантовая электродинамика & 1 & 16 \\ 
Взаимодействие света с атомом & 1 & 16\\ 
Квантовая теория лазера & 2 & 8 \\ 
Оптика фотонов & 2  & 24 \\ \hline
Итого & & 64 \\ \hline
\end{longtable}

\subsection{Виды занятий и форма контроля}
\begin{longtable}{|l|c|}
\hline
Лекции (кол./нед.) & 2 \\ 
Экзамены (кол./сем.)& 1 \\
Зачет (кол./сем.)& нет \\ \hline
\end{longtable}

\section{Содержание дисциплины}
Курс состоит из четырех частей
\begin{itemize}
\item Квантовая электродинамика
\item Взаимодействие света с атомом
\item Квантовая теория лазера
\item Оптика фотонов (квантовые явления в оптике)
\end{itemize}

Ниже представленны краткие аннотации и содержание для каждой из
четырех частей курса.

\subsection{Квантовая электродинамика}
Квантовая электродинамика служит основой квантовой оптики. В первой
части курса излагаются положения квантовой теории электромагнитного поля,
необходимые для изучения и понимания квантовой оптики. Подробно
освещаются следующие вопросы
\begin{itemize}
\item Разложение электромагнитного поля по модам (типам колебаний)
\item Гамильтонова форма уравнений электромагнитного поля 
\item Квантование электромагнитного поля 
\item Разложение поля по плоским волнам в свободном пространстве 
\item Плотность состояний 
\item Гамильтонова форма уравнений поля при разложении по плоским
  волнам 
\item Квантование электромагнитного поля при разложении его по
  плоским волнам 
\item Свойства операторов $ \hat a $ и $ \hat a ^+ $ 
\item Квантовое состояние электромагнитного поля  с определенной
  энергией 
\item Многомодовые состояния 
\item Когерентные состояния 
\item Смешанные состояния электромагнитного поля 
\item Представление оператора плотности через когерентные
  состояния 
\end{itemize}

\subsection{Взаимодействие света с атомом}
Рассматриваются вопросы, связанные с взаимодействием квантованного
электромагнитного поля с атомом. Используется упрощенная модель атома 
двухуровневый атом. Такое упрощение оправдано при резонансном
взаимодействии и широко используется в задачах квантовой электроники и
квантовой оптики. Большое внимание уделено рассмотрению
взаимодействия атома, моды резонатора (динамической системы) с
термостатом (диссипативной системой), которое ответственно за
релаксацию динамической системы.  

В данной части курса подробно раскрываются следующие вопросы
\begin{itemize}
\item Излучение и поглощение атомом света
\item Гамильтониан системы атом-поле
\item Взаимодействие атома с модой электромагнитного поля
\item Взаимодействие атома с многомодовым полем. Вынужденные и
  спонтанные переходы
\item Релаксация динамической системы. Метод матрицы плотности
\item Взаимодействие электромагнитного поля резонатора
  (гармонического осциллятора) с резервуаром атомов, находящихся при
  температуре $T$
\item Уравнение для матрицы плотности поля в представлении чисел
  заполнения
\item Уравнение движения статистического оператора поля моды в
  представлении когерентных состояний
\item Общая теория взаимодействия динамической системы с
  термостатом (диссипативной системой, резервуаром)
\item Затухание (релаксация) поля и атома в случае простейшего
  резервуара, состоящего из гармонических осцилляторов
\end{itemize}

\subsection{Квантовая теория лазера}
Полуклассическая теория лазера не может ответить на все вопросы,
возникающие в связи с его работой. По этой теории лазер до достижения
порога  не генерирует вообще, а при превышении порога начинает
генерировать классическое электромагнитное поле (свет). 

В действительности же заметно ниже порога лазер генерирует хаотический
свет, а значительно выше порога его излучение близко к
классическому. На пороге и вблизи него находится переходная область от
хаотического света к упорядоченному излучению. Адекватно описать это
может только полностью квантовая теория. 

Другая задача, также требующая квантования электромагнитного поля, -
это определение предельной (естественной) ширины линии излучения
лазера, когда ширина линии определяется квантовыми флуктуациями поля,
а различные внешние воздействия, принципиально устранимые, не
принимаются во внимание. 

В данной части курса подробно раскрываются следующие вопросы
\begin{itemize}
\item Модель лазера
\item Теория лазерной генерации
\item Статистика лазерных фотонов
\item Теория лазера. Представление когерентных состояний
\end{itemize}

\subsection{Оптика фотонов (квантовые явления в оптике)}
В данной части курса рассматриваются те оптические явления, в которых
в той или иной степени проявляются квантовые свойства света. 

Несмотря на то, что большое число оптических явлений можно
рассматривать с классических позиций, многие явления могут быть до
конца поняты и описаны только в рамках полностью квантового описания. 

Квантовое рассмотрение позволяет более полно понять суть
интерференционных опытов и на этой основе понять связь между
классическим и квантовым описаниями. Кроме того, квантовый подход
позволяет рассматривать эксперименты нового типа, в которых изучается
статистика фотонов в световых пучках и ее связь со спектральными
свойствами света.

Подробно освещаются следующие вопросы
\begin{itemize}
\item Фотоэффект
\item Когерентные свойства света
\item Когерентность второго порядка
\item Когерентность высших порядков
\item Счет и статистика фотонов
\item Связь статистики фотонов со статистикой фотоотсчетов
\item Распределение фотоотсчетов для когерентного и хаотического
  света
\item Определение статистики фотонов через распределение
  фотоотсчетов
\item Квантовое выражение для распределения фотоотсчетов
\item Эксперименты по счету фотонов. Применение техники счета
  фотонов для спектральных измерений
\end{itemize} 

\section{Учебно-методическое обеспечение дисциплины}
\begin{enumerate}
\item В. Ю. Петрунькин, О. И. Котов Квантовая оптика -
  С. Петербург: Изд. СПбГПУ, 2003.  
\item Л.Мандель, Э.Вольф. Оптическая когерентность и
  квантовая оптика. Пер. с англ./Под ред. В.В.Самарцева - М.:
  Наука. ФИЗМАТЛИТ, 2000.- 896с. 
\item В.В.Белокуров, О.Д.Тимофеевская,
  О.А.Хрусталев. Квантовая телепортация - обыкновенное чудо. - Ижевск:
  НИЦ ``Регулярная и хаотическая динамика''. 2000. - 256с. 
\itemС.Я.Килин. Квантовая информация. - УФН, 1999, т.169,
  - N 5, с.507-526. 
\item Д.Н.Клышко. Квантовая оптика: квантовые,
  классические и метафизические аспекты. - УФН, 1994, т.164, N 11,
  с.1187-1214. 
\item Д.Н.Клышко. Неклассический свет. - УФН, 1996,
  т.166, N 6, с.613-638. 
\item Ю.И.Воронцов. Фаза осциллятора в квантовой
  теории. Что это такое на самом деле? - УФН, 2002, т.172, N 8,
  с.907-929. 
\item M.O.Scully, M.S.Zubuiry. Quantum Optics. 1997,
  Cambridge University Press, UK, 635p. 
\item R.Loudon. The Quantum Theory of Light. Third
  Edition. Oxford University Press, 2000, 438. 
\item Y.Yamamoto, A.Imamoglu.  Mesoscopic quantum
  optics. 1999, USA, J.Wiley \& Sons, 301p. 
\end{enumerate}

Программу разработали: профессор, д. т. н. Петрунькин В. Ю. и
к.ф.м. Мурашко И. В.

\end{document}
