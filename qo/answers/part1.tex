\chapter{Квантовая электродинамика}

\section{Квантование электромагнитного поля} 

\subsection{\ref{qQuantelNumberMods}: Число мод}
Сколько мод электромагнитного поля с длиной волны
  $\lambda \ge 500 \mbox{нм}$ находятся в кубе квантования со стороной
$L=1 \mbox{мм}$?

Диапазону длин волн $\lambda \ge \lambda_0 = 500 \mbox{нм}$
соответствует диапазон частот
$\omega \le \omega_0 = \frac{2 \pi c}{\lambda_0}$.

Воспользуемся формулой \ref{eqCh1_modenumber_1}
\[
d N = 2 \left(\frac{L}{2 \pi c} \right)^3 \omega^2 d \omega d \Omega
\]
откуда
\begin{eqnarray}
  N = \int_0^{\omega_0} 2 \left(\frac{L}{2 \pi c} \right)^3 \omega^2 d
  \omega d \int_{4 \pi} d \Omega =
  \nonumber \\
  = 8 \pi \int_0^{\omega_0} \left(\frac{L}{2 \pi c} \right)^3 d \omega
  = 8 \pi \left(\frac{L}{2 \pi c}\right)^3 \frac{\omega_0^3}{3} =
  \nonumber \\
  = \frac{8 \pi}{3} \left(\frac{L}{\lambda_0}\right)^3 \approx 67
  \cdot 10^9
  \nonumber
\end{eqnarray}


\section{Взаимодействие света с атомом}
