\chapter{\nameref{part1}}

\section{\nameref{chQuantel}} 

\subsection{\ref{qQuantelNumberMods}: Число мод}
Сколько мод электромагнитного поля с длиной волны
  $\lambda \ge 500 \mbox{нм}$ находятся в кубе квантования со стороной
$L=1 \mbox{мм}$?

Диапазону длин волн $\lambda \ge \lambda_0 = 500 \mbox{нм}$
соответствует диапазон частот
$\omega \le \omega_0 = \frac{2 \pi c}{\lambda_0}$.

Воспользуемся формулой \ref{eqCh1_modenumber_1}
\[
d N = 2 \left(\frac{L}{2 \pi c} \right)^3 \omega^2 d \omega d \Omega
\]
откуда
\begin{eqnarray}
  N = \int_0^{\omega_0} 2 \left(\frac{L}{2 \pi c} \right)^3 \omega^2 d
  \omega d \int_{4 \pi} d \Omega =
  \nonumber \\
  = 8 \pi \int_0^{\omega_0} \left(\frac{L}{2 \pi c} \right)^3 d \omega
  = 8 \pi \left(\frac{L}{2 \pi c}\right)^3 \frac{\omega_0^3}{3} =
  \nonumber \\
  = \frac{8 \pi}{3} \left(\frac{L}{\lambda_0}\right)^3 \approx 67
  \cdot 10^9
  \nonumber
\end{eqnarray}


\section{\nameref{chInteraction}}

\subsection{\ref{qInteractionFreq}: Определение частоты перехода атома
лития} 

%\input ./part1/interaction/figfreq.tex

Обозначим через $\Omega_R^{(1,2)}$ эффективные частоты Раби для
измерения 1 и 2 соответственно. Расстройки частот для двух
экспериментов $\delta_{1,2}$. Таким образом имеем
\begin{eqnarray}
  \Omega_R^{(1)} = \sqrt{\omega_R^2 + \delta_1^2},
  \nonumber \\
  \Omega_R^{(2)} = \sqrt{\omega_R^2 + \delta_2^2}
  \label{eqAnswersInteraction40}
\end{eqnarray}

С другой стороны из графиков
\autoref{figPart1InteractionQuestionFreq} видно что максимальная
вероятность обнаружить атомы лития в состоянии $\left|2\right>$ для
частоты $\omega_2 = 2 \pi \cdot 228.4 \mbox{МГц}$ будет $P_2 = 0.49$,
с другой стороны \eqref{eqPart1InteractionRabiProbability} имеем
\begin{equation}
  P_2 = \left(\frac{\omega_R}{\Omega_R^{(2)}}\right)^2 =
  \frac{1}{1 + \frac{\delta_2^2}{\omega_R^2}},
  \nonumber
\end{equation}
откуда
\begin{equation}
  \omega_R = 0.98 \cdot \delta_2 .
  \label{eqAnswersInteraction41}
\end{equation}

С другой стороны можно заключить, что
\[
\frac{\Omega_R^{(1)} \cdot 6 \mbox{мкс}}{2} = \pi
\]
и
\[
\frac{\Omega_R^{(2)} \cdot 12.5 \mbox{мкс}}{2} = \pi,
\]
т. о.
\[
\Omega_R^{(1)} = 2 \pi \cdot 0.167 \mbox{МГц},
\]
и
\[
\Omega_R^{(2)} = 2 \pi \cdot 0.08 \mbox{МГц},
\]

Из \eqref{eqAnswersInteraction40} и  \eqref{eqAnswersInteraction41}
можно получить
\[
\delta_2 = 2 \pi \cdot 0.057 \mbox{МГц},
\]
т. е. два кандидата
\begin{equation}
  f_0 = 228.4 \pm 0.057 \mbox{МГц} = 228.343 \mbox{МГц},\,
  228.457 \mbox{МГц} 
\label{eqAnswersInteraction4Res1}
\end{equation}

Для $\delta_1$ имеем
\[
\delta_1 = \sqrt{\left(\Omega_R^{(1)}\right)^2 - \omega_R^2} =
2 \pi \sqrt{0.167^2 - (0.98 \cdot 0.057)^2} \mbox{МГц} =
2 \pi \cdot 0.157 \mbox{МГц}
\]
т. о. получаем очередные два кандидата
\begin{equation}
  f_0 = 228.5 \pm 0.157 \mbox{МГц} = 228.343 \mbox{МГц},\,
  228.657 \mbox{МГц} 
\label{eqAnswersInteraction4Res2}
\end{equation}

Из \eqref{eqAnswersInteraction4Res1} и
\eqref{eqAnswersInteraction4Res2} окончательно имеем
\[
f_0 = 228.343 \mbox{МГц}.
\]

Стоит отметить, что полученное значение отличается от реального $f_0 =
228.205 \mbox{МГц}$. Отличие вызвано влиянием эффекта Зеемана от
магнитного поля, которое обычно присутствует в таких экспериментах.



