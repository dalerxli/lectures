%% -*- coding:utf-8 -*- 
\chapter{Неклассический свет}
\label{chNonClass}
Квантовые состояния света можно разбить на две группы: состояния,
которые при росте интенсивности света асимптотически стремятся к
классическому состоянию (классическому свету) и состояния, которые не
обладают таким свойством. Например, из рассмотренных нами ранее
состояний когерентное состояние имеет классический предел, а
энергетическое (фоковское) состояние его не имеет.

Ранее мы определили понятие функции когерентности второго порядка 
$G^{(2)}$. Для одномодового света (\ref{eqCh4_28})
\begin{eqnarray}
G^{(2)} = \frac{\left<\hat{a}^{+}\hat{a}^{+}\hat{a}\hat{a}\right>}
{\left<\hat{a}^{+}\hat{a}\right>^2} = 
\nonumber \\
= 
\frac{\left<\hat{a}^{+}\hat{a}\left(\hat{a}^{+}\hat{a} - 1\right)\right>}
{\left<\hat{a}^{+}\hat{a}\right>^2} = 
\frac{\left<n^2\right> - \left<n\right>}{\left<n\right>^2},
\label{eqPart3_Nonclass_Nonclass1}
\end{eqnarray}
где использованы коммутационные соотношения
$\left[\hat{a}\hat{a}^{+}\right] = 1$, 
$\hat{n} = \hat{a}^{+}\hat{a}$ - оператор числа частиц.

В классическом случае $\left<\hat{a}^{+}\hat{a}\right> =
\left|\alpha\right|^2$, где $\left|\alpha\right|^2 = I$ интенсивность
поля (модуль квадрата амплитуды поля). Поскольку классические величины
коммутируют, формулу (\ref{eqPart3_Nonclass_Nonclass1}) в классическом
случае можно заменить
\begin{equation}
G^{(2)}_{\mbox{кл.}} = 
\frac{\left<\left|\alpha\right|^4\right>}{\left<\left|\alpha\right|^2\right>^2}
= 
\frac{\left<I^2\right>}{\left<I\right>^2},
\label{eqPart3_Nonclass_Nonclass2}
\end{equation}
где усреднение осуществляется при помощи положительно
определенной классической функции распределения $P\left(\alpha\right)$
\begin{equation}
\left<\left|\alpha\right|^{2n}\right> = 
\int_0^{\infty}P\left(\left|\alpha\right|^2\right)
\left|\alpha\right|^{2n}d^2\alpha, \, n=1,2.
\nonumber
\end{equation}
Рассмотрим разность 
$G^{(2)}_{\mbox{кл.}} - G^{(1)}_{\mbox{кл.}} = G^{(2)}_{\mbox{кл.}} -
1$, т. к. для одномодового поля, как мы знаем, $G^{(1)}_{\mbox{кл.}} =
1$ всегда. Из (\ref{eqPart3_Nonclass_Nonclass2}) следует
\begin{eqnarray}
G^{(2)}_{\mbox{кл.}} - G^{(1)}_{\mbox{кл.}} = G^{(2)}_{\mbox{кл.}} -
1 = 
\nonumber \\
=
\frac{\left<\left|\alpha\right|^4\right>}{\left<\left|\alpha\right|^2\right>^2}
- 1 = 
\frac{\left<\left|\alpha\right|^4\right> -
  \left<\left|\alpha\right|^2\right>^2}{\left<\left|\alpha\right|^2\right>^2}
= 
\nonumber \\
= \frac{\int_0^{\infty} P \left|\alpha\right|^4 d^2\alpha - 2
  \left<\left|\alpha\right|^2\right>^2 +
  \left<\left|\alpha\right|^2\right>^2}{\left<\left|\alpha\right|^2\right>^2}
= 
\nonumber \\
= 
\frac{\int_0^{\infty} P \left(\left|\alpha\right|^4  - 2
  \left|\alpha\right|^2
  \left<\left|\alpha\right|^2\right> +
  \left<\left|\alpha\right|^2\right>^2\right)d^2\alpha}
     {\left<\left|\alpha\right|^2\right>^2}
=
\nonumber \\
=
\frac{\int_0^{\infty} P \left(\left|\alpha\right|^2  - 
  \left<\left|\alpha\right|^2\right>\right)^2d^2\alpha}{\left<\left|\alpha\right|^2\right>^2}
\ge 0.
\label{eqPart3_Nonclass_Nonclass4}
\end{eqnarray}
Из неравенства (\ref{eqPart3_Nonclass_Nonclass4}) следует, что
\begin{equation}
G^{(2)}_{\mbox{кл.}} \ge 1.
\nonumber
\end{equation}
В квантовом случае из-за некоммутативности операторов ситуация будет
иная. Формулу (\ref{eqPart3_Nonclass_Nonclass1}) перепишем в следующем
виде
\begin{eqnarray}
G^{(2)} = \frac{\left<\hat{n}^2\right> - \left<\hat{n}\right>}{\left<\hat{n}\right>^2} = 
\nonumber \\
=
1 + \frac{\left(\left<\hat{n}^2\right> - \left<\hat{n}\right>^2\right) -
  \left<\hat{n}\right>}{\left<\hat{n}\right>^2} = 1 + \frac{\sigma^2 -
  \left<\hat{n}\right>}{\left<\hat{n}\right>^2}, 
\label{eqPart3_Nonclass_Nonclass6}
\end{eqnarray}
где $\sigma^2$ - среднеквадратичное отклонение от среднего,
т. е. $\sigma$ - дисперсия. 

Из (\ref{eqPart3_Nonclass_Nonclass6}) следует, что $G^{(2)}$ может
быть как больше, так и меньше 1, смотря что больше: $\sigma^2$ или
$\left<\hat{n}\right>$. В классическом случае $G^{(2)} \ge 1$, поэтому
критерием неклассичности можно принять 
\begin{equation}
G^{(2)} < 1
\label{eqPart3_Nonclass_Nonclass7}
\end{equation}

Из (\ref{eqPart3_Nonclass_Nonclass6},
\ref{eqPart3_Nonclass_Nonclass7}) следует, что условием неклассичности
будет 
\(
\sigma^2 - \left<\bar{n}\right> < 0,
\) 
т. е. квадрат дисперсии должен быть меньше среднего числа
фотонов. 
Известно, что поток фотонов, для которого
\(
\sigma^2 - \left<\bar{n}\right> = 0,
\)
имеет пуассоновскую статистику (примером служит когерентное
состояние). Случай 
\(
\sigma^2 - \left<\bar{n}\right> < 0
\)
соответствует более регулярному потоку, который называется
субпуассоновским. Это случай антигруппировки фотонов. Примером
состояния, для которого это справедливо, является например состояние,
возникающее в результате параметрического рассеяния, которое мы
рассмотрим позднее. Случай
\(
\sigma^2 - \left<\bar{n}\right> > 0
\)
соответствует менее регулярному процессу, когда фотоны группируются
(примером является тепловое возбуждение света). Такое состояние
называется суперпуассоновским. Все три случая изображены на
рис. \ref{figPart3Nonclass1}, где показано расположение фотоотсчетов
во времени.

\input ./part3/nonclass/fignonclass1.tex

Мы приходим к заключению, что неклассическим состояниям соответствует
субпуассоновский порог. Число $G^{(2)}$ просто связано с другими
параметрами, характеризующими флуктуации числа фотонов: с дисперсией и
фактором Фоко, которые часто встречаются в литературе.
\begin{eqnarray}
\Phi  = \frac{\sigma^2}{\left<\bar{n}\right>},
\nonumber \\
G^{(2)} - 1 = \frac{\Phi - 1}{\left<\bar{n}\right>}
\nonumber
\end{eqnarray}
Экспериментальный характер статистики можно установить из эксперимента,
в котором определяется корреляционная функция второго порядка
(эксперимент Брауна - Твисса). Схема эксперимента изображена на
рис. \ref{figPart3Nonclass2} 

\input ./part3/nonclass/fignonclass2.tex

Регулируемая задержка $\tau$ осуществляется изменением длины одного из
плеч.

\input ./part3/nonclass/fignonclass3.tex

На рис. \ref{figPart3Nonclass3} изображены три кривые, которые
получаются в результате эксперимента. Кривая (б) соответствует
когерентному состоянию. Кривая (а) соответствует группировке фотонов и
сверхпуассоновской статистике. Кривая (в) соответствует
антигруппировке и субпуассоновской статистике. Такая кривая характерна
для неклассического света. Объяснение поведения кривых просто. Если
$\tau$ велико, фотоотсчеты фотоприемников случайны и не зависят друг
от друга. Поэтому поведение всех трех кривых определяется случайными
совпадениями и для всех трех будет одинаково. При малых $\tau$
поведение кривых различно. В случае группировки фотонов мы будем иметь
максимум при $\tau = 0$, а при антигруппировке - минимум.

Еще одним критерием неклассичности является отсутствие положительной
определенности квазивероятности $P\left(\alpha\right)$, если
используется представление когерентных состояний.

В классическом случае функция $P\left(\alpha\right)$ должна быть всюду
положительно определена, т. е. $P\left(\alpha\right) \ge 0$ при всех
значения $\alpha$. Рассмотрим этот вопрос подробнее. Условие
неклассичности 
\(
G^{(2)} < 1, 
\)
с помощью (\ref{eqPart3_Nonclass_Nonclass1}) это условие
можно представить в виде
\begin{equation}
\left<\hat{a}^{+}\hat{a}^{+}\hat{a}\hat{a}\right> -
\left<\hat{a}^{+}\hat{a}\right>^2 < 0.
\label{eqPart3_Nonclass_Nonclass9}
\end{equation}
Рассмотрим выражение
\begin{equation}
\int P\left(\alpha\right)\left(\left|\alpha\right|^2 -
\left<\hat{n}\right>\right)^2 d^2\alpha < 0
\label{eqPart3_Nonclass_Nonclass10}
\end{equation}
\begin{eqnarray}
\int P\left(\alpha\right)\left(\left|\alpha\right|^2 -
\left<\hat{n}\right>\right)^2 d^2\alpha = 
\int P\left(\alpha\right)\left(\left|\alpha\right|^4 -
2\left|\alpha\right|^2 \left<\hat{n}\right> +
\left<\hat{n}\right>^2\right) d^2\alpha = 
\nonumber \\
=
\left<\hat{a}^{+}\hat{a}^{+}\hat{a}\hat{a}\right> -
\left<\hat{a}^{+}\hat{a}\right>^2 < 0,
\nonumber
\end{eqnarray}
т. е. выражения (\ref{eqPart3_Nonclass_Nonclass9}) и
(\ref{eqPart3_Nonclass_Nonclass10}) эквивалентны. Это следует из
следующих соотношений
\[
\int P\left(\alpha\right)\hat{a}^{+}\hat{a}^{+}\hat{a}\hat{a} d^2
\alpha = 
\int P\left(\alpha\right)\left|\alpha\right|^4 d^2\alpha = 
\left<\hat{a}^{+}\hat{a}^{+}\hat{a}\hat{a}\right>,
\]
\[
\int P\left(\alpha\right)\left|\alpha\right|^2 d^2\alpha = 
\left<\hat{a}^{+}\hat{a}\right> = \left<\hat{n}\right>.
\]
Мы получили, что выражение (\ref{eqPart3_Nonclass_Nonclass10}) должно
быть отрицательным. Поскольку 
\(
\left(\left|\alpha\right|^2 -
\left<\hat{n}\right>\right)^2
\)
всегда положительно, то отрицательный результат будет получен, если 
$P\left(\alpha\right)$ по крайней мере в части области $(\alpha)$
будет отрицательно. Таким образом, для неклассического света
квазивероятность $P\left(\alpha\right)$ не является положительно
определенной функцией.

