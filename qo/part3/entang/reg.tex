%% -*- coding:utf-8 -*- 
\section{Регистрация Белловских состояний}
\label{pPart3EntangleBellReg}
Из четырех базисных состояний (\ref{eqEntangBellBase}) лишь 
$\left|\psi^{-}\right>_{12}$ является антисимметричным, т. е. меняет
свой знак при перестановке частиц (фотонов). Это позволяет выделить
это состояние среди всех остальных. Для этого может быть использована
схема, изображенная на рис. \ref{figBellReg}, в которой два фотона
будут зарегистрированы в разных фотодетекторах только для
антисимметричного состояния, для всех остальных состояний оба фотона
регистрируются либо в $D^{(1)}$, либо в $D^{(2)}$.

\input ./part3/entang/figreg.tex
 
Для доказательства рассмотрим, как действует светоделитель на однофотонное
состояние. Фотон может падать на светоделитель из полуплоскости
$\left(1\right)$ (рис. \ref{figBellLS}), в этом случае с вероятностью
$0.5$ фотон окажется в полуплоскости $\left(1\right)$, и c вероятностью
$0.5$ в $\left(2\right)$. Если фотон падает из полуплоскости
$\left(2\right)$, то он опять же с равной вероятностью может оказаться
в любой полуплоскости. 
Таким образом, для описания светоделителя мы будем пользоваться выражением
для матрицы рассеяния в виде (\ref{eqPart2Interfero4b}), в котором
$r=t=\frac{1}{\sqrt{2}}$, т. е. действие светоделителя может быть
выражено следующим оператором (Адамара\cite{bPhisQuantInfo}):
\begin{eqnarray}
\hat{H} \left|1\right> = \frac{1}{\sqrt{2}}
\left(\left|1'\right> +
\left|2'\right>\right),
\nonumber \\
\hat{H} \left|2\right> = \frac{1}{\sqrt{2}}\left(\left|1'\right> -
\left|2'\right>\right).
\label{eqEntangBellHadamar}
\end{eqnarray}

В нашем случае два фотона падают с разных сторон светоделителя,
следовательно возможны только две волновые функции, описывающие
пространственную часть нашего начального состояния:
\begin{eqnarray}
\left|S\right>_{12} = \frac{1}{\sqrt{2}}\left(
\left|1\right>_1\left|2\right>_2 +
\left|2\right>_1\left|1\right>_2\right),
\nonumber \\
\left|A\right>_{12} = \frac{1}{\sqrt{2}}\left(
\left|1\right>_1\left|2\right>_2 -
\left|2\right>_1\left|1\right>_2\right).
\label{eqEntangBellSpaceInit}
\end{eqnarray}
Как видно из (\ref{eqEntangBellSpaceInit}), состояние $\left|S\right>_{12}$
является симметричным, т. е. сохраняет свой вид при перестановке
фотонов, а состояние $\left|A\right>_{12}$ - антисимметричным (меняет свой
знак при перестановке фотонов).

\input ./part3/entang/figls.tex


Рассмотрим, как преобразуются состояния (\ref{eqEntangBellSpaceInit})
при прохождении через светоделитель. В силу того, что оператор
$\hat{H}$ действует на каждый из фотонов независимо, для
$\left|S\right>_{12}$ имеем 
\begin{equation}
\hat{H}\left|S\right>_{12} = \left|S'\right>_{12} = 
\frac{1}{\sqrt{2}}
\left(
\hat{H}\left|1\right>_1\hat{H}\left|2\right>_2 +
\hat{H}\left|2\right>_1\hat{H}\left|1\right>_2
\right),
\nonumber 
\end{equation}
откуда с помощью (\ref{eqEntangBellHadamar}) получаем
\begin{eqnarray}
\hat{H}\left|S\right>_{12} =
\frac{1}{\sqrt{2}}
\left(
\frac{1}{2}
\left(\left|1'\right>_1 +
\left|2'\right>_1\right)
\left(\left|1'\right>_2 -
\left|2'\right>_2\right) +
\right.
\nonumber \\
+ \left.
\frac{1}{2}
\left(\left|1'\right>_1 -
\left|2'\right>_1\right)
\left(\left|1'\right>_2 +
\left|2'\right>_2\right)
\right) = 
\nonumber \\
=
\frac{1}{2 \sqrt{2}}
\left(
\left|1'\right>_1 \left|1'\right>_2 +
\left|2'\right>_1 \left|1'\right>_2 -
\left|1'\right>_1 \left|2'\right>_2 -
\left|2'\right>_1 \left|2'\right>_2 +
\right. 
\nonumber \\
+ \left.
\left|1'\right>_1 \left|1'\right>_2 -
\left|2'\right>_1 \left|1'\right>_2 +
\left|1'\right>_1 \left|2'\right>_2 -
\left|2'\right>_1 \left|2'\right>_2
\right) =
\nonumber \\
=
\frac{1}{2 \sqrt{2}}
\left(
\left|1'\right>_1 \left|1'\right>_2 
- \left|2'\right>_1 \left|2'\right>_2 
+ \left|1'\right>_1 \left|1'\right>_2 
- \left|2'\right>_1 \left|2'\right>_2
\right) = 
\nonumber \\
\frac{1}{\sqrt{2}}
\left(
\left|1'\right>_1 \left|1'\right>_2 
- \left|2'\right>_1 \left|2'\right>_2 
\right).
\nonumber
\end{eqnarray}
Таким образом для состояния $\left|S'\right>_{12}$ оба фотона оказываются
либо в полуплоскости $\left(1\right)$, либо в полуплоскости
$\left(2\right)$. 

Для антисимметричного состояния имеем
\begin{eqnarray}
\hat{H}\left|A\right>_{12} = \left|A'\right>_{12} = 
\frac{1}{\sqrt{2}}
\left(
\hat{H}\left|1\right>_1\hat{H}\left|2\right>_2 -
\hat{H}\left|2\right>_1\hat{H}\left|1\right>_2
\right) = 
\nonumber \\
=
\frac{1}{\sqrt{2}}
\left(
\frac{1}{2}
\left(\left|1'\right>_1 +
\left|2'\right>_1\right)
\left(\left|1'\right>_2 -
\left|2'\right>_2\right) -
\right.
\nonumber \\
- \left.
\frac{1}{2}
\left(\left|1'\right>_1 -
\left|2'\right>_1\right)
\left(\left|1'\right>_2 +
\left|2'\right>_2\right)
\right) = 
\nonumber \\
=
\frac{1}{2 \sqrt{2}}
\left(
\left|1'\right>_1 \left|1'\right>_2 +
\left|2'\right>_1 \left|1'\right>_2 -
\left|1'\right>_1 \left|2'\right>_2 -
\left|2'\right>_1 \left|2'\right>_2 -
\right. 
\nonumber \\
- \left.
\left|1'\right>_1 \left|1'\right>_2 +
\left|2'\right>_1 \left|1'\right>_2 -
\left|1'\right>_1 \left|2'\right>_2 +
\left|2'\right>_1 \left|2'\right>_2
\right) =
\nonumber \\
=
\frac{1}{2 \sqrt{2}}
\left(
\left|2'\right>_1 \left|1'\right>_2 
- \left|1'\right>_1 \left|2'\right>_2 
+ \left|2'\right>_1 \left|1'\right>_2 
- \left|1'\right>_1 \left|2'\right>_2
\right) = 
\nonumber \\
- \frac{1}{\sqrt{2}}
\left(
\left|1'\right>_1 \left|2'\right>_2 
- \left|2'\right>_1 \left|1'\right>_2 
\right).
\nonumber
\end{eqnarray}
Таким образом в случае антисимметричного состояния после прохождения
светоделителя фотоны окажутся в разных полуплоскостях.

В силу того, что фотоны являются бозонами, общая волновая функция,
описывающая и поляризационные и пространственные свойства,
должна быть симметричной \cite{bFeinman}
(см. прил. \ref{AddFermionBoson}). Таким образом из объединения 
(\ref{eqEntangBellBase}) и (\ref{eqEntangBellSpaceInit})возможны
только следующие комбинации:
\begin{eqnarray}
\left|\psi^{+}\right>_{12}\left|S\right>_{12},
\nonumber \\ 
\left|\psi^{-}\right>_{12}\left|A\right>_{12}, 
\nonumber \\ 
\left|\phi^{+}\right>_{12}\left|S\right>_{12}, 
\nonumber \\ 
\left|\phi^{-}\right>_{12}\left|S\right>_{12}.
\nonumber
\end{eqnarray}
Следовательно в случае белловских состояний 
$\left|\psi^{+}\right>_{12}$, $\left|\phi^{+}\right>_{12}$, и
$\left|\phi^{-}\right>_{12}$ оба фотона после прохождения
светоделителя попадут в один фотодетектор, а в случае
$\left|\psi^{-}\right>_{12}$ - в разные, что и позволяет нам
однозначно отделить состояние $\left|\psi^{-}\right>_{12}$ от всех
остальных. 

\input ./part3/entang/figreg2.tex

Схема на рис. \ref{figBellReg} может быть изменена так, чтобы она
смогла регистрировать также и состояние
$\left|\psi^{+}\right>_{12}$. Для этого заметим, что из всех
симметричных белловских состояний только в этом фотоны имеют разную
поляризацию. Таким образом это состояние может быть выделено путем
измерения поляризации фотонов.

На рис. \ref{figBellReg2} приведена схема, которая позволяет
регистрировать $\left|\psi^{-}\right>_{12}$ и
$\left|\psi^{+}\right>_{12}$. Для состояния
$\left|\psi^{+}\right>_{12}$ одновременно сработают фотодекторы 
$D^{(1)}_x$ и $D^{(1)}_y$ или $D^{(2)}_x$ и $D^{(2)}_y$. Для
$\left|\psi^{-}\right>_{12}$ - $D^{(1)}_x$ и $D^{(2)}_y$ или
$D^{(2)}_x$ и $D^{(1)}_y$. Для $\left|\phi^{+}\right>_{12}$ и
$\left|\phi^{-}\right>_{12}$ оба фотона будут зарегистрированы
одновременно на одном из 4-х фотодетекторов. В последнее время
появились работы \cite{bKulik} в которых была проведена регистрация
всех четырех Белловских состояний.
