%% -*- coding:utf-8 -*- 
\section{Регистрация Белловских состояний}
\label{pPart3EntangleBellReg}
\index{Базисные состояния Белла!регистрация}
Из четырех базисных состояний \eqref{eqEntangBellBase} лишь 
$\left|\psi^{-}\right>_{12}$ является антисимметричным, т. е. меняет
свой знак при перестановке частиц (фотонов). Это позволяет выделить
это состояние среди всех остальных. Для этого может быть использована
схема, изображенная на \autoref{figBellReg}, в которой два фотона
будут зарегистрированы в разных фотодетекторах только для
антисимметричного состояния, для всех остальных состояний оба фотона
регистрируются либо в $D^{(1)}$, либо в $D^{(2)}$.

\input ./part3/entang/figreg.tex
 
Для доказательства рассмотрим, как действует светоделитель на однофотонное
состояние. Фотон может падать на светоделитель из полуплоскости
$\left(1\right)$ (\autoref{figBellLS}), в этом случае с вероятностью
$0.5$ фотон окажется в полуплоскости $\left(1\right)$, и c вероятностью
$0.5$ в $\left(2\right)$. Если фотон падает из полуплоскости
$\left(2\right)$, то он опять же с равной вероятностью может оказаться
в любой полуплоскости. 
Таким образом, для описания светоделителя мы будем пользоваться выражением
для матрицы рассеяния в виде \eqref{eqPart2Interfero4b}, в котором
$r=t=\frac{1}{\sqrt{2}}$, т. е. действие светоделителя может быть
выражено следующим оператором (Адамара\cite{bPhisQuantInfo}):
\begin{eqnarray}
\hat{H} \ket{1} = \frac{1}{\sqrt{2}}
\left(\ket{1'} +
\ket{2'}\right),
\nonumber \\
\hat{H} \ket{2} = \frac{1}{\sqrt{2}}\left(\ket{1'} -
\ket{2'}\right).
\label{eqEntangBellHadamar}
\end{eqnarray}

В нашем случае два фотона падают с разных сторон светоделителя,
следовательно возможны только две волновые функции, описывающие
пространственную часть нашего начального состояния:
\begin{eqnarray}
\ket{S}_{12} = \frac{1}{\sqrt{2}}\left(
\ket{1}_1\ket{2}_2 +
\ket{2}_1\ket{1}_2\right),
\nonumber \\
\ket{A}_{12} = \frac{1}{\sqrt{2}}\left(
\ket{1}_1\ket{2}_2 -
\ket{2}_1\ket{1}_2\right).
\label{eqEntangBellSpaceInit}
\end{eqnarray}
Как видно из \eqref{eqEntangBellSpaceInit}, состояние $\ket{S}_{12}$
является симметричным, т. е. сохраняет свой вид при перестановке
фотонов, а состояние $\ket{A}_{12}$ - антисимметричным (меняет свой
знак при перестановке фотонов).

\input ./part3/entang/figls.tex


Рассмотрим, как преобразуются состояния \eqref{eqEntangBellSpaceInit}
при прохождении через светоделитель. В силу того, что оператор
$\hat{H}$ действует на каждый из фотонов независимо, для
$\ket{S}_{12}$ имеем 
\begin{equation}
\hat{H}\ket{S}_{12} = \ket{S'}_{12} = 
\frac{1}{\sqrt{2}}
\left(
\hat{H}\ket{1}_1\hat{H}\ket{2}_2 +
\hat{H}\ket{2}_1\hat{H}\ket{1}_2
\right),
\nonumber 
\end{equation}
откуда с помощью \eqref{eqEntangBellHadamar} получаем
\begin{eqnarray}
\hat{H}\ket{S}_{12} =
\frac{1}{\sqrt{2}}
\left(
\frac{1}{2}
\left(\ket{1'}_1 +
\ket{2'}_1\right)
\left(\ket{1'}_2 -
\ket{2'}_2\right) +
\right.
\nonumber \\
+ \left.
\frac{1}{2}
\left(\ket{1'}_1 -
\ket{2'}_1\right)
\left(\ket{1'}_2 +
\ket{2'}_2\right)
\right) = 
\nonumber \\
=
\frac{1}{2 \sqrt{2}}
\left(
\ket{1'}_1 \ket{1'}_2 +
\ket{2'}_1 \ket{1'}_2 -
\ket{1'}_1 \ket{2'}_2 -
\ket{2'}_1 \ket{2'}_2 +
\right. 
\nonumber \\
+ \left.
\ket{1'}_1 \ket{1'}_2 -
\ket{2'}_1 \ket{1'}_2 +
\ket{1'}_1 \ket{2'}_2 -
\ket{2'}_1 \ket{2'}_2
\right) =
\nonumber \\
=
\frac{1}{2 \sqrt{2}}
\left(
\ket{1'}_1 \ket{1'}_2 
- \ket{2'}_1 \ket{2'}_2 
+ \ket{1'}_1 \ket{1'}_2 
- \ket{2'}_1 \ket{2'}_2
\right) = 
\nonumber \\
\frac{1}{\sqrt{2}}
\left(
\ket{1'}_1 \ket{1'}_2 
- \ket{2'}_1 \ket{2'}_2 
\right).
\nonumber
\end{eqnarray}
Таким образом для состояния $\ket{S'}_{12}$ оба фотона оказываются
либо в полуплоскости $\left(1\right)$, либо в полуплоскости
$\left(2\right)$. 

Для антисимметричного состояния имеем
\begin{eqnarray}
\hat{H}\ket{A}_{12} = \ket{A'}_{12} = 
\frac{1}{\sqrt{2}}
\left(
\hat{H}\ket{1}_1\hat{H}\ket{2}_2 -
\hat{H}\ket{2}_1\hat{H}\ket{1}_2
\right) = 
\nonumber \\
=
\frac{1}{\sqrt{2}}
\left(
\frac{1}{2}
\left(\ket{1'}_1 +
\ket{2'}_1\right)
\left(\ket{1'}_2 -
\ket{2'}_2\right) -
\right.
\nonumber \\
- \left.
\frac{1}{2}
\left(\ket{1'}_1 -
\ket{2'}_1\right)
\left(\ket{1'}_2 +
\ket{2'}_2\right)
\right) = 
\nonumber \\
=
\frac{1}{2 \sqrt{2}}
\left(
\ket{1'}_1 \ket{1'}_2 +
\ket{2'}_1 \ket{1'}_2 -
\ket{1'}_1 \ket{2'}_2 -
\ket{2'}_1 \ket{2'}_2 -
\right. 
\nonumber \\
- \left.
\ket{1'}_1 \ket{1'}_2 +
\ket{2'}_1 \ket{1'}_2 -
\ket{1'}_1 \ket{2'}_2 +
\ket{2'}_1 \ket{2'}_2
\right) =
\nonumber \\
=
\frac{1}{2 \sqrt{2}}
\left(
\ket{2'}_1 \ket{1'}_2 
- \ket{1'}_1 \ket{2'}_2 
+ \ket{2'}_1 \ket{1'}_2 
- \ket{1'}_1 \ket{2'}_2
\right) = 
\nonumber \\
- \frac{1}{\sqrt{2}}
\left(
\ket{1'}_1 \ket{2'}_2 
- \ket{2'}_1 \ket{1'}_2 
\right).
\nonumber
\end{eqnarray}
Таким образом в случае антисимметричного состояния после прохождения
светоделителя фотоны окажутся в разных полуплоскостях.

В силу того, что фотоны являются бозонами, 
\index{бозон}
\index{фотон}
общая волновая функция,
описывающая и поляризационные и пространственные свойства,
должна быть симметричной \cite{bFeinman}
(см. \autoref{AddFermionBoson}). Таким образом из объединения 
\eqref{eqEntangBellBase} и \eqref{eqEntangBellSpaceInit}возможны
только следующие комбинации:
\begin{eqnarray}
\left|\psi^{\dag}\right>_{12}\ket{S}_{12},
\nonumber \\ 
\left|\psi^{-}\right>_{12}\ket{A}_{12}, 
\nonumber \\ 
\left|\phi^{\dag}\right>_{12}\ket{S}_{12}, 
\nonumber \\ 
\left|\phi^{-}\right>_{12}\ket{S}_{12}.
\nonumber
\end{eqnarray}
Следовательно в случае белловских состояний 
$\left|\psi^{\dag}\right>_{12}$, $\left|\phi^{+}\right>_{12}$, и
$\left|\phi^{-}\right>_{12}$ оба фотона после прохождения
светоделителя попадут в один фотодетектор, а в случае
$\left|\psi^{-}\right>_{12}$ - в разные, что и позволяет нам
однозначно отделить состояние $\left|\psi^{-}\right>_{12}$ от всех
остальных. 

\input ./part3/entang/figreg2.tex

Схема на \autoref{figBellReg} может быть изменена так, чтобы она
смогла регистрировать также и состояние
$\left|\psi^{\dag}\right>_{12}$. Для этого заметим, что из всех
симметричных белловских состояний только в этом фотоны имеют разную
поляризацию. Таким образом это состояние может быть выделено путем
измерения поляризации фотонов.

На \autoref{figBellReg2} приведена схема, которая позволяет
регистрировать $\left|\psi^{-}\right>_{12}$ и
$\left|\psi^{\dag}\right>_{12}$. Для состояния
$\left|\psi^{\dag}\right>_{12}$ одновременно сработают фотодекторы 
$D^{(1)}_x$ и $D^{(1)}_y$ или $D^{(2)}_x$ и $D^{(2)}_y$. Для
$\left|\psi^{-}\right>_{12}$ - $D^{(1)}_x$ и $D^{(2)}_y$ или
$D^{(2)}_x$ и $D^{(1)}_y$. Для $\left|\phi^{\dag}\right>_{12}$ и
$\left|\phi^{-}\right>_{12}$ оба фотона будут зарегистрированы
одновременно на одном из 4-х фотодетекторов. В последнее время
появились работы \cite{bKulik} в которых была проведена регистрация
всех четырех Белловских состояний.
