%% -*- coding:utf-8 -*- 
\chapter{Перепутанные состояния}
\label{chEntangl}

С момента появления квантовой механики возник вопрос о полноте
этой теории. Точность предсказаний квантовой механики очень велика, но
при этом 
предсказываются только вероятности тех или иных событий. В частности,
нет возможности измерить с произвольной точностью координату и
импульс некоторой частицы. Кажется, что
вероятности, присущие квантовомеханическому описанию, отражают его
неполноту и, возможно существует другая теория, которая будет обладать
точностью квантовой механики и при этом не будет использовать
вероятностный подход.

Оказалось, что вероятности, лежащие в основе квантовой механики, имеют
глубокий физический смысл и не существует теорий, в которых 
можно было бы от них отказаться, и в которых, например, было бы
возможно измерять координату и импульс с произвольной точностью. 
Особую роль при этом приобретают так называемые перепутанные
состояния, которые описывают системы, состоящие из нескольких частиц,
при этом поведение такой составной системы описывается общей волновой
функцией. 

В силу того, что перепутанные состояния являются чисто квантовыми,
т. е. не имеют классических аналогов, с их помощью имеется
возможность наблюдать явления, которые кажутся совершенно
невозможными с классической точки зрения, такие например, как квантовая
телепортация. Кроме этого, в последнее время появились практические
приложения перепутанных состояний, такие как квантовая плотная
кодировка 
(см. \ref{subsecPart3QuantInfoBigCoding})
и квантовая криптография
(см. \ref{subsecPart3QuantInfoQuantCrypto}).

Существует несколько способов получения перепутанных фотонов
\cite{bPhisQuantInfo}, т. е. имеющих отношение к квантовой оптике,
среди которых следует выделить перепутанные по поляризации
состояния. Связано это с тем, что существует большое количество
способов управления поляризационными характеристиками, а также
методов измерения поляризационных свойств света.

% Так сложилось, что наиболее простые способы получения перепутанных
% состояний имеют отношение к квантовой оптике. Одним из таких способов
% является состояние перепутанное по поляризации, которое и будет в
% дальнейшем рассматриваться.

%\input ./part3/entang/epr.tex

\input ./part3/entang/entang.tex

\input ./part3/entang/bell.tex

\input ./part3/entang/bellbase.tex

\input ./part3/entang/gener.tex

\input ./part3/entang/reg.tex

\input ./part3/entang/teleport.tex

\section{Упражнения}
\begin{enumerate}
\item Доказать ортонормированность базисных Белловских состояний
  \eqref{eqEntangBellBase}. 
\item Какое Белловское состояние должна регистрировать Алиса, в схеме
  изображенной на \autoref{figTeleport}, для
  подтверждения факта телепортации в случае когда источник
  перепутанных пар фотонов $S$ производит фотоны в состоянии 
\begin{equation}
  \left|\psi\right>_{23} = \left|\psi^{+}\right>_{23} = \frac{1}{\sqrt{2}}\left(
  \left|x\right>_2\left|y\right>_3 +
  \left|y\right>_2\left|x\right>_3
  \right).
  \nonumber
\end{equation}
\item Получить выражения для коэффициентов 
$c_{\left|\psi^{+}\right>_{12}}$, 
$c_{\left|\phi^{+}\right>_{12}}$ и 
$c_{\left|\phi^{-}\right>_{12}}$
в разложении \eqref{eqPart3EntangTeleportsepar}
\end{enumerate}
