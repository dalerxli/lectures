%% -*- coding:utf-8 -*- 
\section{Парадокс ЭПР для параметров
  Стокса и перепутанные состояния}

Рассмотрим однофотонное состояние (\ref{eqEntangSimpleState}). 
Для того чтобы объяснить
физический смысл параметров $\alpha$ и $\beta$, надо обратиться к схеме
экспериментальной установки, изображенной на
рис.\ref{figPart3EntangJones}. Предположим, что источник $S$ излучает
фотоны в состоянии (\ref{eqEntangSimpleState}). Тогда среднее значение
тока фотодетектора $D_x$ будет пропорционально
$\left|\alpha\right|^2$, а фотодетектора $D_y$ -
$\left|\beta\right|^2$, т. е. $P_x = \left|\alpha\right|^2$ и $P_y =
\left|\beta\right|^2$ описывают вероятности обнаружить фотон в
состоянии поляризованном по $x$ - $\left|x\right>$ или по $y$ -
$\left|y\right>$. Возникает следующий вопрос - ``что кроется за этими
вероятностями?''. Возможны два варианта ответа. В первом из них
предполагается, что на самом деле мы обладаем неполным знанием об источнике $S$,
т. е. если бы мы знали об этом источнике все, то смогли бы
предсказать, какой фотон, поляризованный по $x$ или по $y$, создается в
некоторый произвольный момент времени и, соответственно, показания
фотодетекторов $D_x$ и $D_y$. С этой точки зрения
квантовая механика представляет собой некоторую промежуточную
теорию, которая в дальнейшем может быть заменена на более точную,
описывающую квантовые свойства объектов с абсолютной точностью. 

Во втором варианте ответа предполагается, что невозможно опеределить
все параметры, описывающие некоторую квантовую систему с произвольной
точностью, поскольку результат измерения не предопределен заранее (как
в нашем случае свойствами источника $S$), а определяется в момент
измерения по результатам взаимодействия квантовой системы и
измерительного макроприбора. 

Первая попытка дать ответ на этот вопрос была предпринята 
А. Эйнштейном, Б. Подольским и Н. Розеном в 1935 г. в статье которую
часто называют ЭПР\cite{bEPR}. В ЭПР физическая теория предполагается
полной, если каждый элемент физической реальности имеет отражение в
физической теории. В качестве опеределения элемента физической
реальности принималось следующее: ``если мы можем достоверно (с
вероятностью равной единице) предсказать значение некоторой физической
величины, то существует элемент реальности соответствующий этой
величине''\cite{bBelokTimHrus}. Особый интерес при этом представлют
некоммутирующие операторы. В нашем случае это могут быть операторы
$\hat{S}_1$ и $\hat{S}_2$. Так как эти операторы не коммутируют
(\ref{eqEntangStokesOperS12Comm}), то
значения соответствующих величин не могут быть измерены с произвольной
точностью. Исходя из этого можно сделать два предположения:
\begin{enumerate}
\item Квантово-механическое описание реальности не полно
\item Величины, определяемые операторами $\hat{S}_1$ и $\hat{S}_2$, не
  могут быть одновременно реальными
\end{enumerate}

В качестве тестовой системы мы, следуя ЭПР, будем рассматривать сложную
систему, состоящую из нескольких частиц. При этом для описания всей
системы используется некоторая общая волновая функция.

Для описания поляризационных свойств сложной системы, состоящей из двух
фотонов необходимо использовать волновую функцию следующего вида:
\begin{eqnarray}
\left|\psi\right> = \sum_{i,j = x,y} 
c_{ij}\left|i\right>_1\left|j\right>_2 = 
\nonumber \\
= c_{xx} \left|x\right>_1\left|x\right>_2 +
c_{xy} \left|x\right>_1\left|y\right>_2 +
c_{yx} \left|y\right>_1\left|x\right>_2 +
c_{yy} \left|y\right>_1\left|y\right>_2.
\label{eqEntang2Common}
\end{eqnarray}
Если два и более коэффициентов $c_{ij}$ в (\ref{eqEntang2Common})
отличны от нуля, то волновая функция не факторизуется, т. е. 
\[
\left|\psi\right> \ne \left|\psi\right>_1 \left|\psi\right>_2.
\]
Таким образом, каждому отдельному фотону не может быть приписана
некоторая волновая функция. Такие состояния называются перепутанными.

В качестве примера мы будем исследовать следующую волновую функцию:
\begin{equation}
  \left|\psi\right> = \frac{
    \left| + \right>_1\left| - \right>_2 -
    \left| - \right>_1\left| + \right>_2
  }{\sqrt{2}},
\label{eqEntang2Test}
\end{equation}
т. е. все, что нам известно о фотонах - это то что они имеют разную
круговую поляризацию. С вероятностью 50\% первый фотон имеет левую
круговую поляризацию, и в этом случае, второй фотон имеет правую
круговую поляризацию. И наоборот -  с вероятностью 50\% первый фотон
имеет правую круговую поляризацию, и в этом случае, второй фотон имеет
левую круговую поляризацию. 

\input ./part3/entang/figent.tex

Для измерения поляризационных свойств состояния (\ref{eqEntang2Test})
может быть использована схема, приведенная на рис. \ref{figEntangMes}. 
В этой схеме перепутанные фотоны с источника $S$ подаются на два детектора
параметров Стокса $D^{(1,2)}$. Детектор $D^{(1)}$ измеряет средние
значения операторов Стокса для первого фотона - $\hat{S}_k^{(1)}$, а
$D^{(2)}$ - измеряет средние значения операторов Стокса
$\hat{S}_k^{(2)}$ для второго фотона. Схема каждого детектора
идентична схеме, изображенной на рис. \ref{figPart3EntangStokes}.

Так как собственные значения операторов Стокса
(\ref{eqEntangStokesOper}) равны $s = \pm 1$, на выходе детекторов мы
будем получать либо $+1$, либо $-1$. Допустим, что мы измеряем параметр
Стокса $\hat{S}_1^{(1)}$ для первого фотона. Предположим что
результат измерения $+1$. При этом происходит редукция волновой функции
(\ref{eqEntang2Test}). Редукция может быть описана оператором
проецирования (см. \ref{AddDiracProjector}) на состояние
$\left|x\right>_1$:
\[
\hat{P}_{\left|x\right>_1} = \left|x\right>_1\left<x\right|_1,
\]
при этом волновая функция
(\ref{eqEntang2Test}) преобразуется следующим образом:
\begin{eqnarray}
  \hat{P}_{\left|x\right>_1}\left|\psi\right> =
  \left|x\right>_1\left<x\right|_1 \left|\psi\right> =
  \nonumber \\
    =
  \left|x\right>_1\left<x\right|_1
  \frac{
    \left( \left|x\right>_1 + i \left|y\right>_1 \right)\left| - \right>_2 -
    \left( \left|x\right>_1 - i \left|y\right>_1 \right)\left| + \right>_2
  }{2} =
  \nonumber \\
  =
  \left|x\right>_1
  \frac{\left| - \right>_2 - \left| + \right>_2}{2} =
  \frac{1}{2\sqrt{2}}\left|x\right>_1 \left(2 i\right)
  \left|y\right>_2 =
  \frac{i}{\sqrt{2}}\left|x\right>_1\left|y\right>_2.
\nonumber
\end{eqnarray}
Следовательно после измерения $\hat{S}_1^{(1)}$ волновая функция всей
системы будет записана так
\[
\left|\psi\right>_{red} = \left|x\right>_1\left|y\right>_2.
\]
Это значит, что второй фотон будет в состоянии $\left|y\right>_2$,
т. е.  показание детектора $D^{(2)}$ будет равно $-1$.

Аналогично, если показание первого детектора было равно $-1$, то
волновая функция после измерения будет равна
\begin{eqnarray}
  P_{\left|y\right>_1}\left|\psi\right> =
  \left|y\right>_1\left<y\right|_1 \left|\psi\right> =
  \nonumber \\
  =
  \left|y\right>_1\left<y\right|_1
  \frac{
    \left( \left|x\right>_1 + i \left|y\right>_1 \right)\left| - \right>_2 -
    \left( \left|x\right>_1 - i \left|y\right>_1 \right)\left| + \right>_2
  }{2} =
  \nonumber \\
  =
  \left|y\right>_1 i 
  \frac{\left| - \right>_2 + \left| + \right>_2}{2} =
  \frac{i}{2\sqrt{2}}\left|y\right>_1 \left(2\right)
  \left|x\right>_2 =
  \frac{i}{\sqrt{2}}\left|y\right>_1\left|x\right>_2,
  \nonumber
\end{eqnarray}
т. е.
\[
\left|\psi\right>_{red} = \left|x\right>_2\left|y\right>_1,
\]
состояние второго фотона будет описываться состоянием
$\left|x\right>_2$ а показание детектора $D^{(2)}$ будет равно $+1$.
Таким образом, при любом измерении $\hat{S}_1^{(1)}$ мы можем
достоверно (с вероятностью, равной 1) предсказать результат измерения
$\hat{S}_1^{(2)}$, при этом прямо не воздействуя на второй фотон, т. е.
величина $\hat{S}_1^{(2)}$ должна рассматриваться как элемент
физической реальности.

При этом в рассматриваемом состоянии среднее значение параметра Стокса
$\left<\hat{S}_3^{(2)}\right>$ имеет вид
\begin{eqnarray}
  \left<\hat{S}_3^{(2)}\right> =
  \left<\psi\right|\hat{S}_3^{(2)}\left|\psi\right> =
  \left<\psi\right|\hat{S}_3^{(2)}\frac{
    \left| + \right>_1\left| - \right>_2 -
    \left| - \right>_1\left| + \right>_2
  }{\sqrt{2}} =
  \nonumber \\
  = \left<\psi\right|\frac{
    \left| + \right>_1\left| - \right>_2 +
    \left| - \right>_1\left| + \right>_2
  }{\sqrt{2}} =
  \nonumber \\
  = \frac{1}{2} \left(
  \left< + \right|_1\left< - \right|_2 -
  \left< - \right|_1\left< + \right|_2
  \right)
  \left(
  \left| + \right>_1\left| - \right>_2 -
  \left| - \right>_1\left| + \right>_2
  \right) = 1.
  \label{eqEntangS3Mean}
\end{eqnarray}

При выводе (\ref{eqEntangS3Mean}) использовались выражения
(\ref{eqEntangS3Eigenvec}). Таким образом из неравенства Гейзенберга
(\ref{eqAddHeisenbergUncertaintyPrinciple}) можно получить что
\[
\Delta s_1^{(2)} \Delta s_2^{(2)} \ge 1,
\]
т. е. из квантовой механики следует, что первый и второй параметры
Стокса для второй частицы не могут быть измерены одновременно. 


%% С другой стороны, мы могли бы в детекторе $D^{(1)}$ произвести
%% измерение $\hat{S}_2^{(1)}$. В этом случае для показаний детектора мы
%% имели бы два значения: $+1$ или $-1$, для первого из которых волновая
%% функция (\ref{eqEntang2Test})
%% преобразовывалась бы следующим образом:
%% \begin{eqnarray}
%% \hat{P}_{\frac{\left|x\right>_1 +
%%     \left|y\right>_1}{\sqrt{2}}}\left|\psi\right> = 
%% \nonumber \\
%% =
%% \frac{1}{2}\left[
%% \left(\left|x\right>_1 + \left|y\right>_1\right)
%% \left(\left<x\right|_1 + \left<y\right|_1\right)
%% \left(
%% \left|x\right>_1\left|y\right>_2 -
%% \left|y\right>_1\left|x\right>_2
%% \right)
%% \right] = 
%% \nonumber \\
%% = \frac{1}{2}\left[
%% \left(\left|x\right>_1 + \left|y\right>_1\right)
%% \left(
%% \left|y\right>_2 -
%% \left|x\right>_2
%% \right)
%% \right] =
%% \nonumber \\
%% = \frac{\left|x\right>_1 + \left|y\right>_1}{\sqrt{2}}
%%  \frac{\left|y\right>_2 - \left|x\right>_2}{\sqrt{2}}.
%% \nonumber
%% \end{eqnarray}
%% Таким образом при измерении $\hat{S}_2^{(2)}$ мы получим $-1$ в
%% качестве результата детектора $D^{(2)}$. Если же в качестве
%% показаний детектора $D^{(1)}$ мы получим $-1$, то состояние системы
%% преобразуется к 
%% \begin{eqnarray}
%% \hat{P}_{\frac{\left|x\right>_1 -
%%     \left|y\right>_1}{\sqrt{2}}}\left|\psi\right> = 
%% \nonumber \\
%% =
%% \frac{1}{2}\left[
%% \left(\left|x\right>_1 - \left|y\right>_1\right)
%% \left(\left<x\right|_1 - \left<y\right|_1\right)
%% \left(
%% \left|x\right>_1\left|y\right>_2 -
%% \left|y\right>_1\left|x\right>_2
%% \right)
%% \right] = 
%% \nonumber \\
%% = \frac{1}{2}\left[
%% \left(\left|x\right>_1 - \left|y\right>_1\right)
%% \left(
%% \left|y\right>_2 +
%% \left|x\right>_2
%% \right)
%% \right] =
%% \nonumber \\
%% = \frac{\left|x\right>_1 - \left|y\right>_1}{\sqrt{2}}
%%  \frac{\left|x\right>_2 + \left|y\right>_2}{\sqrt{2}}
%% \nonumber
%% \end{eqnarray}
%% и для показаний детектора $D^{(2)}$ получим $+1$. Следовательно, при
%% любом измерении $\hat{S}_2^{(1)}$ мы можем 
%% достоверно (с вероятностью равной 1) предсказать результат измерения
%% $\hat{S}_2^{(2)}$, при этом прямо не воздействуя на второй фотон, т. е.
%% $\hat{S}_2^{(2)}$ должна рассматриваться как элемент физической реальности.

Теперь если мы предполагаем, что результаты экспериментов
предопределены заранее, т. е. не зависят от того измерение какой
величины ($S_1$ или $S_2$) было выполнено, и так как измерения над
первым фотоном производятся в произвольном порядке, то
получается, что для второго фотона $\hat{S}^{(2)}_1$ и
$\hat{S}^{(2)}_2$ должны быть одновременно элементами физической
реальности. Это влечет за собой тезис о неполноте квантовой механики.

Мы имеем иную ситуацию, когда принимаем предположение о том, что
результат некоторого измерения определяется в момент проведения этого
самого эксперимента. В этом случае мы уже не можем говорить, что 
$\hat{S}_1^{(2)}$ и $\hat{S}_2^{(2)}$ должны {\bf одновременно}
рассматриваться как элементы физической реальности и соответственно нет
никаких противоречий с основными принципами квантовой механики.

% Если измерить усредненные значения
% операторов Стокса, т. е. усреднить показания детекторов за некоторый
% промежуток времени, то мы получим
% \begin{eqnarray}
% \left<\hat{S}_k^{(1)}\right> =
% \left<\psi\right|\hat{S}_k^{(1)}\left|\psi\right> = 0,
% \nonumber \\
% \left<\hat{S}_k^{(2)}\right> =
% \left<\psi\right|\hat{S}_k^{(2)}\left|\psi\right> = 0.
% \nonumber
% \end{eqnarray}

% С другой стороны волновая функция (\ref{eqEntang2Test}) является
% собственной для оператора $\hat{S}_k^{(1)}\hat{S}_k^{(2)}$:
% \begin{equation}
% \hat{S}_k^{(1)}\hat{S}_k^{(2)}\left|\psi\right> = -\left|\psi\right>
% \label{eqEntangCorrel}
% \end{equation}

% Из выражения (\ref{eqEntangCorrel}) следует, что если измерять среднее
% значение оператора $\hat{S}_k^{(1)}\hat{S}_k^{(2)}$ то мы получим
% \begin{equation}
% \left<\hat{S}_k^{(1)}\hat{S}_k^{(2)}\right> =
% \left<\psi\right|\hat{S}_k^{(1)}\hat{S}_k^{(2)}\left|\psi\right> = 
% - \left<\psi\right|\left.\psi\right> = -1,
% \nonumber
% \end{equation}
% т. е. имеется некоторая корреляция между показаниями детекторов
% $D^{(1)}$ и $D^{(2)}$: когда на выходе первого детектора мы имеем $+1$
% - на выходе второго всегда $-1$. Подобные корреляции возможны и в
% классическом случае. Различие между квантовым и классическим
% рассмотрением в количественной величине этих корреляций.
% Для демонстрации этого различия используют проверку так называемых
% неравенств Белла.


