%% -*- coding:utf-8 -*- 
\section{Генерация квадратурно сжатых состояний}
\label{pNonClassGenerSqueezed}
До сих пор рассматривалась теоретическая возможность существования и
генерации сжатых состояний. Возникает вопрос - в каких реальных
процессах можно получить свет в сжатом состоянии? Здесь рассмотрим
возможность получения квадратурно сжатого состояния при
параметрическом взаимодействии, возникающем при прохождении сильного
поля накачки через среду с квадратичной нелинейностью. При таком
взаимодействии возникают две волны, связанные с частотой накачки
соотношениями:
\begin{equation}
\omega_p = \omega_s + \omega_i,
\nonumber
\end{equation}
где $\omega_p$ - частота накачки, $\omega_s$ - сигнальная частота,
$\omega_i$ - холостая частота. Условно это изображено на рис. 
\ref{figPart3Squeezed_10}.

\input ./part3/squeezed/fig10.tex

Гамильтониан взаимодействия между накачкой, сигнальной и холостой
волнами в представлении взаимодействия имеет вид:
\begin{equation}
\hat{V} = \hbar \kappa \left(
\hat{a}^{+}_s \hat{a}^{+}_i \hat{b} + 
\hat{a}_s \hat{a}_i \hat{b}^{+}
\right),
\label{eqPart3Squeezed30}
\end{equation}
где $\hat{a}^{+}_s$, $\hat{a}^{+}_i$, 
$\hat{a}_s$ и $\hat{a}_i$ - операторы рождения и уничтожения для
сигнальной и холостой волны, $\kappa$ - постоянная взаимодействия.

\input ./part3/squeezed/fig11.tex

В вырожденном режиме сигнальная и холостая волна имеют одну и ту же
частоту накачки (см. \autoref{figPart3Squeezed_11}):
\[
\omega_s = \omega_i = \frac{\omega_p}{2} = \omega.
\]

Для вырожденного параметрического процесса гамильтониан взаимодействия
(\ref{eqPart3Squeezed30}) упрощается:
\begin{equation}
\hat{V} = \hbar \kappa \left(
\left(\hat{a}^{+}\right)^2 \hat{b} + 
\hat{a}^2 \hat{b}^{+}
\right).
\nonumber
\end{equation}
Если поле накачки находится в когерентном состоянии с большой
амплитудой $\alpha_p = A_p e^{i \varphi}$, $\left|\alpha_p\right| \gg
1$, накачку можно заменить классическим полем. При этом гамильтониан
взаимодействия еще больше упрощается
\begin{equation}
\hat{V} = \hbar \kappa A_p\left(
\left(\hat{a}^{+}\right)^2 e^{- i \varphi} + 
\hat{a}^2 e^{i \varphi}
\right),
\nonumber
\end{equation}
где $A_p$ и $\varphi$ - действительная амплитуда и фаза
накачки. Очевидно, что в этом приближении мы пренебрегаем истощением
накачки. Это будет справедливо, пока амплитуда сигнальной (холостой)
волны будет мала по сравнению с амплитудой поля накачки. Уравнение
Гейзенберга для оператора $\hat{a}$ будет иметь вид:
\begin{eqnarray}
\frac{d\hat{a}}{dt} = \frac{i}{\hbar}
\left[\hat{V}, \hat{a}\right] = i \kappa A_p
\left[
\left(\hat{a}^{+}\right)^2 e^{- i \varphi} + 
\hat{a}^2 e^{i \varphi},
\hat{a}
\right] = 
\nonumber \\
=
i \kappa A_p e^{- i \varphi}
\left(
\left(
\hat{a}^{+}\right)^2  \hat{a} -
\hat{a}
\left(
\hat{a}^{+}\right)^2
\right) = 
\nonumber \\
=
i \kappa A_p e^{- i \varphi}
\left(
\left(
\hat{a}^{+}\right)^2  \hat{a} -
\left(\hat{a}^{+}\hat{a} + 1\right)
\hat{a}^{+}
\right) = 
\nonumber \\
=
i \kappa A_p e^{- i \varphi}
\left(
\hat{a}^{+}\hat{a}^{+}  \hat{a} -
\hat{a}^{+}\hat{a}\hat{a}^{+} -
\hat{a}^{+}
\right) = 
\nonumber \\
=
i \kappa A_p e^{- i \varphi}
\hat{a}^{+}
\left(
\hat{a}^{+}  \hat{a} -
\hat{a}\hat{a}^{+} -
1
\right) = 
i \kappa A_p e^{- i \varphi}
\hat{a}^{+}
\left(
\left(-1\right) -
1
\right) =
\nonumber \\
= 
- 2 i \kappa A_p e^{- i \varphi}
\hat{a}^{+}.
\nonumber
\end{eqnarray}
Таким образом имеем
\begin{equation}
\frac{d\hat{a}}{dt} = 
- i \Omega_p e^{- i \varphi}
\hat{a}^{+},
\label{eqPart3Squeezed34a}
\end{equation}
где 
\[
\Omega_p = 2 \kappa A_p
\]
эквивалент частоты Рабби (энергия, выраженная через частоту).
Аналогично
\begin{equation}
\frac{d\hat{a}^{+}}{dt} = 
i \Omega_p e^{i \varphi}
\hat{a}.
\label{eqPart3Squeezed34b}
\end{equation}

Исключая из системы уравнений (\ref{eqPart3Squeezed34a}) и
(\ref{eqPart3Squeezed34b}) $\hat{a}^{+}$, получим
\begin{equation}
\frac{d^2\hat{a}\left(t\right)}{dt^2} = 
- i \Omega_p e^{- i \varphi}
\frac{d\hat{a}^{+}\left(t\right)}{dt} = 
\Omega_p^2\hat{a}\left(t\right).
\label{eqPart3Squeezed35}
\end{equation}
Начальные условия для этого уравнения имеют вид
\begin{eqnarray}
\hat{a}\left(0\right) = \hat{a}_0, 
\nonumber \\
\hat{a}^{+}\left(0\right) = \hat{a}^{+}_0, 
\nonumber
\end{eqnarray}
откуда
\begin{eqnarray}
\left.\hat{a}\right|_{t=0} = \hat{a}_0, 
\nonumber \\
\left.\frac{d\hat{a}}{dt}\right|_{t=0} = 
- i \Omega_p e^{- i \varphi} \hat{a}^{+}_0,
\label{eqPart3Squeezed36}
\end{eqnarray}
где $\hat{a}_0$ и $\hat{a}^{+}_0$ начальные значения операторов при $t
= 0$. Решение уравнения (\ref{eqPart3Squeezed35}) с учетом начальных
условий (\ref{eqPart3Squeezed36}) имеет вид:
\begin{equation}
\hat{a}\left(t\right) = \hat{a}_0 ch \left(\Omega_p t \right) - 
i \hat{a}^{+}_0 sh \left(\Omega_p t\right) e^{-i \varphi}.
\label{eqPart3Squeezed37a}
\end{equation}
Аналогично
\begin{equation}
\hat{a}^{+}\left(t\right) = \hat{a}^{+}_0 ch \left(\Omega_p t \right) +
i \hat{a}_0 sh \left(\Omega_p t\right) e^{i \varphi}.
\label{eqPart3Squeezed37b}
\end{equation}

%% Проверка в maxima
%% (%i10) e1: diff(x(t),t) = - %i * Omega * exp(-%i * Phi) * y(t);
%%                     d                        - %i Phi
%% (%o10)              -- (x(t)) = - %i Omega %e         y(t)
%%                     dt
%% (%i11) e2: diff(y(t),t) =  %i * Omega * exp(%i * Phi) * x(t);
%%                       d                      %i Phi
%% (%o11)                -- (y(t)) = %i Omega %e       x(t)
%%                       dt
%% (%i12) desolve ([e1, e2], [x(t), y(t)]);
%%                        %i Phi              Omega t - %i Phi
%%                (x(0) %e       - %i y(0)) %e
%% (%o12) [x(t) = --------------------------------------------
%%                                     2
%%            %i Phi              - Omega t - %i Phi
%%    (x(0) %e       + %i y(0)) %e
%%  + ----------------------------------------------, 
%%                          2
%%                   %i Phi           Omega t
%%        (%i x(0) %e       + y(0)) %e
%% y(t) = -----------------------------------
%%                         2
%%               %i Phi           - Omega t
%%    (%i x(0) %e       - y(0)) %e
%%  - -------------------------------------]
%%                      2
%% (%i13)


Заметим, что при $\varphi = \frac{\pi}{2}$ выражения
(\ref{eqPart3Squeezed37a}) и (\ref{eqPart3Squeezed37b}) приводят к
соотношениям
(\ref{eqPart3Squeezed16a}):
\begin{eqnarray}
\hat{a}\left(t\right) = \hat{a}_0 ch \left(\Omega_p t \right) - 
\hat{a}^{+}_0 sh \left(\Omega_p t\right),
\nonumber \\
\hat{a}^{+}\left(t\right) = \hat{a}^{+}_0 ch \left(\Omega_p t \right) -
\hat{a}_0 sh \left(\Omega_p t\right).
\nonumber
\end{eqnarray}

Как мы видим, оператор $\hat{a}\left(t\right)$ получается унитарным
преобразованием (\ref{eqPart3Squeezed16a}) которое в нашем случае
имеет вид:
\begin{equation}
\hat{a}\left(t\right) =
\hat{S}^{+}\left(t\right)
\hat{a}\left(0\right)
\hat{S}\left(t\right) =
\hat{a}_0 ch \left(\Omega_p t \right) - 
\hat{a}^{+}_0 sh \left(\Omega_p t\right).
\label{eqPart3SqueezedGenerHeizenberg}
\end{equation}
Таким образом, переходя от представления Гейзенберга
(\ref{eqAddWaveFunc_HeizenbergU})
к представлению Шредингера (\ref{eqAddWaveFunc_ShredingerU}), можно
заключить что исходное когерентное состояние подвергается тому же
унитарному преобразованию что и оператор уничтожения
(\ref{eqPart3SqueezedGenerHeizenberg}):
\begin{equation}
\hat{S}\left(t\right)
\left|\alpha\right> =
\left|t, \alpha\right>,
\nonumber
\end{equation}
т. е. параметрическое взаимодействие эквивалентно действию оператора
сжатия. В этом случае для
квадратурных компонент 
\[
\hat{X}_1 = \frac{\hat{a} + \hat{a}^{+}}{2}
\]
и
\[
\hat{X}_2 = \frac{\hat{a} - \hat{a}^{+}}{2i}
\]
будет осуществлено сжатое состояние с неопределенностями 
\begin{eqnarray}
\left(\Delta X_1\right)^2 = 
\frac{1}{4}e^{-2 \Omega_p t},
\nonumber \\
\left(\Delta X_2\right)^2 = 
\frac{1}{4}e^{2 \Omega_p t},
\nonumber \\
\left(\Delta X_1 \Delta X_2\right) = 
\frac{1}{4}.
\nonumber
\end{eqnarray}
Параметр сжатия $r = \Omega_p t$ неограниченно растет во времени и,
следовательно, степень сжатия неограниченно увеличивается. В
действительности этого не происходит. Неограниченное сжатие есть
следствие излишней идеализации задачи: не учитывалось истощение
накачки, отличие реальной накачки от монохроматического классического
света и ряд других факторов. Все это приводит к ограничению степени
сжатия. Чем ближе реальные условия к идеальным, тем большую степень
сжатия можно получить. 

Рассмотренный выше случай получения сжатого состояния за счет
параметрического процесса относится к получению сжатого вакуума,
т. к. генерация сжатого состояния возникает от всегда присутствующей
затравки вакуумного поля. Ее иногда называют параметрическим
рассеянием поля накачки. Как мы выяснили, ``сжатый вакуум'' не является
в полном смысле вакуумным полем. Среднее число фотонов в этом поле не
равно 0, а зависит от степени сжатия по формуле
(\ref{eqPart3Squeezed24}) и может быть при 
высокой степени сжатия достаточно велико.

