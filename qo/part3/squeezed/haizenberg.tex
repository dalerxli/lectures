%% -*- coding:utf-8 -*- 
\section{Соотношение неопределенности Гайзенберга}
В квантовой механике двум сопряженным наблюдаемым $A$ и $B$
соответствуют некоммутирующие операторы $\hat{A}$ и $\hat{B}$,
удовлетворяющие коммутационному соотношению
\begin{equation}
\left[
\hat{A}, \hat{B}
\right] = 
\hat{A}\hat{B} - \hat{B}\hat{A} = i \hat{C},
\nonumber
\end{equation}
где $\hat{C}$ некоторый эрмитов оператор. В этом случае наблюдаемые
$A$ и $B$ могут быть измерены только с некоторой неопределенностью,
выражаемой соотношением неопределенности Гайзенберга
(более подробно см. \ref{AddHeisenbergUncertaintyPrinciple})
\begin{equation}
\left(
\Delta A \Delta B
\right) \ge \frac{1}{2} \left|\left<\hat{C}\right>\right|,
\label{eqPart3Squeezed2}
\end{equation}
где
\[
\left(\Delta A\right)^2 = \left<\hat{A}^2\right> - \left<\hat{A}\right>^2
\]
и
\[
\left(\Delta B\right)^2 = \left<\hat{B}^2\right> - \left<\hat{B}\right>^2
\]
средний квадрат отклонения от среднего значения. 
Среднее $\left<\dots\right>$ соответствует квантовому состоянию, в
котором находится квантовый объект.

Соотношение (\ref{eqPart3Squeezed2}) ограничивает только произведение
неопределенностей, допуская состояния, для которых неопределенность
одной из наблюдаемых существенно меньше неопределенности другой.
Если 
\begin{equation}
\left(\Delta A\right)^2 < \frac{1}{2} \left|\left<\hat{C}\right>\right|,
\nonumber
\end{equation}
то такое состояние называется сжатым состоянием для наблюдаемой
$A$. Если при этом наблюдается условие минимальной неопределенности
\begin{equation}
\left(
\Delta A \Delta B
\right) = \frac{1}{2} \left|\left<\hat{C}\right>\right|,
\nonumber
\end{equation}
то такое состояние называют идеально сжатым.
