%% -*- coding:utf-8 -*- 
\section{Операторы квадратурных составляющих электромагнитного
  поля}
Рассмотрим оператор одномодового электромагнитного поля 
\begin{equation}
\hat{\bar{E}} = 
E_1 \vec{e}\left[
\hat{a} e^{-i\omega t} +
\hat{a}^{+} e^{i\omega t}
\right]
\nonumber
\end{equation}
и введем два новых оператора, связанных с ним
\begin{eqnarray}
\hat{X}_1 = \frac{1}{2}\left(\hat{a} + \hat{a}^{+}\right),
\nonumber \\
\hat{X}_2 = \frac{1}{2i}\left(\hat{a} - \hat{a}^{+}\right),
\nonumber
\end{eqnarray}
которые пропорциональны операторам $\hat{q}$ и $\hat{p}$ (координата,
импульс) с которыми мы познакомились раньше.

Операторы $\hat{X}_1$ и $\hat{X}_2$ являются эрмитовыми
операторами. Коммутационные соотношения для них имеют вид.
\begin{eqnarray}
\left[\hat{X}_1, \hat{X}_2\right] = 
\left(\hat{X}_1 \hat{X}_2 - \hat{X}_2 \hat{X}_1 \right) = 
\nonumber \\
=
\frac{1}{4i}
\left\{
\left(\hat{a} + \hat{a}^{+}\right)
\left(\hat{a} - \hat{a}^{+}\right)
-
\left(\hat{a} - \hat{a}^{+}\right)
\left(\hat{a} + \hat{a}^{+}\right)
\right\} = 
\nonumber \\
\frac{1}{4i}
\left\{
\hat{a}^{+}\hat{a}
- \hat{a}\hat{a}^{+}
- \hat{a}\hat{a}^{+}
+
\hat{a}^{+}\hat{a}
\right\} = 
\frac{1}{2i}
\left\{
\hat{a}^{+}\hat{a}
- \hat{a}\hat{a}^{+}
\right\} =
\nonumber \\
= 
\frac{1}{2i}
\left[\hat{a}^{+}, \hat{a}\right] = 
\frac{i}{2}
\nonumber
\end{eqnarray}

При помощи операторов $\hat{X}_1$ и $\hat{X}_2$ оператор
электрического поля можно представить в виде
\begin{equation}
E = 2 E_1 \vec{e}\left(
\hat{X}_1 \cos \, \omega t +
\hat{X}_2 \sin \, \omega t,
\right)
\nonumber
\end{equation}
т. е. эти операторы можно рассматривать, как операторы квадратурных
компонент электромагнитного поля. Из соотношения неопределенности
(\ref{eqPart3Squeezed2}) имеем
\begin{equation}
\left(
\Delta X_1 \Delta X_2
\right) \ge \frac{1}{4}.
\nonumber
\end{equation}
Условием сжатого состояния будет
\begin{equation}
\left(
\Delta X_i 
\right) < \frac{1}{4}, \, i = 1,2.
\nonumber
\end{equation}
Если при этом выполняется соотношение минимальной неопределенности
\begin{equation}
\left(
\Delta X_1 \Delta X_2
\right) = \frac{1}{4},
\nonumber
\end{equation}
то сжатие будет идеальным.

Для примера рассмотрим неопределенность $\hat{X}_1$, 
$\hat{X}_2$ в случае, когда поле находится в когерентном состоянии:
\begin{eqnarray}
\left(\Delta X_1\right)^2 = 
\left<\alpha\right|\hat{X}_1^2\left|\alpha\right> - 
\left<\alpha\right|\hat{X}_1\left|\alpha\right>^2 = 
\nonumber \\
=\frac{1}{4}
\left\{
\left<\alpha\right|\left(\hat{a} +
\hat{a}^{+}\right)^2\left|\alpha\right> -  
\left<\alpha\right|\left(\hat{a} +
\hat{a}^{+}\right)\left|\alpha\right>^2 
\right\} = 
\nonumber \\
=\frac{1}{4}
\left\{
\alpha^{*2} + \alpha^2 + 2 \left|\alpha\right|^2 + 1 - 
\left(\alpha^{*2} + \alpha^2 + 2 \left|\alpha\right|^2\right)
\right\} 
=\frac{1}{4}.
\nonumber
\end{eqnarray}
Аналогично можно получить
\begin{equation}
\left(\Delta X_2\right)^2 = \frac{1}{4}
\label{eqPart3SqueezedTaskX2Alpha_1}
\end{equation}
и
\begin{equation}
\left(\Delta X_1 \Delta X_2\right) = \frac{1}{4}.
\label{eqPart3SqueezedTaskX2Alpha_2}
\end{equation}
Таким образом мы получили, что рассматриваемый случай является
состоянием с минимальной неопределенностью, но сжатия нет.

Рассмотрим теперь, что дает энергетическое состояние
$\left|n\right>$. Имеем
\begin{eqnarray}
\left(\Delta X_1\right)^2 
=\frac{1}{4}
\left\{
\left<n\right|\left(\hat{a} +
\hat{a}^{+}\right)^2\left|n\right> -  
\left<n\right|\left(\hat{a} +
\hat{a}^{+}\right)\left|n\right>^2 
\right\} = 
\nonumber \\
=
\frac{1}{4}
\left<n\right|\left(\hat{a} +
\hat{a}^{+}\right)^2\left|n\right>
=
\frac{1}{4}
\left<n\right|\left(\hat{a}\hat{a}^{+} +
\hat{a}^{+}\hat{a}\right)\left|n\right> =
\nonumber \\
=
\frac{1}{4}
\left<n\right|\left(1 + 2
\hat{a}^{+}\hat{a}\right)\left|n\right>
=\frac{1}{4}\left(2 n + 1\right).
\nonumber
\end{eqnarray}
Аналогично
\begin{equation}
\left(\Delta X_2\right)^2 = \frac{1}{4}\left(2 n + 1\right)
\label{eqPart3SqueezedTaskX2N_1}
\end{equation}
и
\begin{equation}
\left(\Delta X_1 \Delta X_2\right) = \frac{1}{4}\left(2 n + 1\right).
\label{eqPart3SqueezedTaskX2N_2}
\end{equation}
Следовательно, энергетическое состояние не является состоянием с
минимальной неопределенностью и сжатым состоянием для наблюдаемых
$X_1$ или $X_2$.
