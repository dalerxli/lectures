%% -*- coding:utf-8 -*- 
\section{Неклассичность сжатого состояния}
В заключение покажем, что сжатое состояние неклассично. Условием
сжатого состояния является 
\begin{equation}
\left(\Delta X_1\right)^2 < \frac{1}{4},
\nonumber
\end{equation}
где 
\[
\left(\Delta X_1\right)^2 =
\left<\hat{X}_1^2\right> - 
\left<\hat{X}_1\right>^2.
\]
Оператор
\[
\hat{X}_1 = \frac{\hat{a} + \hat{a}^{+}}{2},
\]
тогда 
\begin{eqnarray}
\hat{X}_1^2 = \frac{\hat{a}^2 + \left(\hat{a}^{+}\right)^2 +
  \hat{a}^{+}\hat{a} + \hat{a} \hat{a}^{+}}{4} = 
\nonumber \\
=
\frac{\hat{a}^2 + \left(\hat{a}^{+}\right)^2 +
  2 \hat{a}^{+}\hat{a} + 1}{4}.
\nonumber
\end{eqnarray}

В то же время 
\[
\left(\hat{N}\hat{X}_1^2\right) = \frac{
\hat{a}^2 + \left(\hat{a}^{+}\right)^2 +
2 \hat{a}^{+}\hat{a}}{4},
\]
где $\hat{N}\hat{X}_1^2$ - оператор, подвергнутый действию оператора
$\hat{N}$, действие которого состоит в том, что устанавливается
нормальная последовательность операторов $\hat{a}^{+}$ и $\hat{a}$,
независимо от коммутационого соотношения.

Отсюда получим
\begin{equation}
\left(\Delta N X_1\right)^2 = 
\left(\Delta X_1\right)^2 - \frac{1}{4},
\nonumber
\end{equation}
или
\begin{equation}
\left(\Delta N X_1\right)^2 < 0,
\nonumber
\end{equation}
т. к. 
\[
\left(\Delta X_1\right)^2 < \frac{1}{4}.
\]

Положим, что сжатое состояние описывается оператором плотности,
который записан в представлении когерентных состояний
\[
\hat{\rho} = \int P\left(\alpha\right)
\left|\alpha\right>
\left<\alpha\right| d^2 \alpha,
\]
тогда для среднего значения 
\begin{equation}
\left(\Delta N X_1\right)^2 = 
\left<\hat{N} \hat{X_1}^2\right> - \left<\hat{N} \hat{X_1}\right>^2 
\label{eqPart3SqueezedNonclass3}
\end{equation}
можем написать
\begin{eqnarray}
  \left(\Delta N X_1\right)^2 =
  \int P\left(\alpha\right)
  \left<\alpha\right|
  \hat{N} \hat{X_1}^2
  \left|\alpha\right> d^2 \alpha -
  \int P\left(\alpha\right)
  \left< X_1 \right>^2
  d^2 \alpha
  =
  \nonumber \\
  =
  \int  
  P\left(\alpha\right)
  \left(
  \frac{
    \alpha^2 + {\alpha^\ast}^2 +
    2 \alpha \alpha^\ast}{4}
  - \left< X_1 \right>^2
  \right)
   d^2 \alpha
  =
  \nonumber \\
  =
 \int 
P\left(\alpha\right)
\left(\Delta X_1\right)^2
d^2 \alpha
\end{eqnarray}
где $\left(\Delta X_1\right)^2$ получается из
(\ref{eqPart3SqueezedNonclass3}), если заменить все $\hat{a}^{+}$ на
$\alpha^{*}$, а $\hat{a}$ на $\alpha$. Можно показать, что
\[
\left(\Delta X_1\right)^2 = \left(\Delta \alpha^{*} + \Delta \alpha\right)^2
\]
величина очевидно неотрицательная. По условию сжатия интеграл должен
быть меньше нуля:
\begin{equation}
\left(\Delta N X_1\right)^2 = 
 \int 
P\left(\alpha\right)
\left(\Delta X_1\left(\alpha^{*}\alpha\right)\right)^2
d^2 \alpha < 0.
\nonumber
\end{equation}
Это очевидно возможно, если хотя бы на части плоскости
$\left(\alpha\right)$ $P\left(\alpha\right) < 0$. Это, как мы знаем,
является условием неклассичности. 

\section{Когерентность второго порядка для сжатого вакуума}

Посчитаем чему равна когерентность второго порядка для сжатого вакуума
$\left|z, 0\right>$. Из (\ref{eqPart3_Nonclass_Nonclass1}) имеем
\[
G^{(2)} = \frac{\left<z,0\right|\hat{a}^{+}\hat{a}\hat{a}^{+}\hat{a}\left|z,0\right>
- \left<n\right>}{\left<n\right>^2}.
\]

Выражение
$\left<z,0\right|\hat{a}^{+}\hat{a}\hat{a}^{+}\hat{a}\left|z,0\right>$
можно перезаписать в виде
\begin{eqnarray}
  \left<z,0\right|\hat{a}^{+}\hat{a}\hat{a}^{+}\hat{a}\left|z,0\right>
  = \nonumber \\
  =
  \left<0\right|\hat{a}^{+}\hat{S}\hat{S}^{+}\hat{a}\hat{S}\hat{S}^{+}\hat{a}^{+}\hat{S}\hat{S}^{+}\hat{a}\hat{S}\left|0\right>
  = \left<\phi\right.\left|\phi\right>,
  \nonumber
\end{eqnarray}
где
\[
\left|\phi\right> = \hat{S}^{+}\hat{a}^{+}\hat{S}\hat{S}^{+}\hat{a}\hat{S}\left|0\right>.
\]
С помощью (\ref{eqPart3Squeezed16a}) можно получить
\begin{eqnarray}
  \left|\phi\right> =  
  \left(\hat{a}^{+} ch\,r - \hat{a} e^{-i\theta} sh \, r\right)
  \left(\hat{a} ch\,r - \hat{a}^{+} e^{i\theta} sh \, r\right)
  \left|0\right> =
  \nonumber \\
  =
   e^{i\theta} sh \, r \left(\hat{a}^{+} ch\,r - \hat{a} e^{-i\theta} sh \, r\right)
   \hat{a}^{+}\left|0\right> =
   sh^2 \, r \left|0\right> + \sqrt{2} e^{i\theta}  sh \, r ch \, r\left|2\right>.
  \nonumber
\end{eqnarray}
Т. о.
\begin{eqnarray}
  \left<\phi\right.\left|\phi\right> =
  sh^4 \, r + 2 sh^2 \, r ch^2 \, r.
  \nonumber
\end{eqnarray}
Следовательно, с учетом (\ref{eqPart3Squeezed24}),
\begin{eqnarray}
G^{(2)} = \frac{sh^4 \, r + 2 sh^2 \, r ch^2 \, r - sh^2 \, r}{sh^4 \,
  r} = 1 + 2 \frac{ch^2 \, r}{sh^2 \, r} - \frac{1}{sh^2 \, r} =
\nonumber \\
=
1 + \frac{ch^2 \, r}{sh^2 \, r} + \frac{ch^2 \, r - 1}{sh^2 \, r} =
1 + \frac{ch^2 \, r}{sh^2 \, r} + \frac{sh^2 \, r }{sh^2 \, r} =
2 + \frac{ch^2 \, r}{sh^2 \, r} =
\nonumber \\
= 2 + \frac{1 + sh^2 \, r}{sh^2 \, r} = 3 + \frac{1}{\left<n\right>}
\nonumber
\end{eqnarray}

