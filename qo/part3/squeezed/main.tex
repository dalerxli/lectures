%% -*- coding:utf-8 -*- 
\chapter{Сжатые состояния}
\label{ChSqueezed}
В сжатых состояниях реализуется возможность, допускаемая соотношением
неопределенности Гайзенберга, уменьшить неопределенность одной из
сопряженных наблюдаемых за счет увеличения неопределенности другой.
Такие состояния могут оказаться полезными в квантовой оптике для
практических нужд. Например, для повышения точности интерференционных
экспериментов. 

Далее рассматривается теория сжатых состояний в параметрических
процессах и применения сжатых состояний в интерференционных
экспериментах. 

\input ./part3/squeezed/haizenberg.tex
\input ./part3/squeezed/quad1.tex
\input ./part3/squeezed/quad2.tex
\input ./part3/squeezed/gener.tex
\input ./part3/squeezed/observ.tex
\input ./part3/squeezed/interfero.tex
\input ./part3/squeezed/nonclass.tex

\section{Упражнения}
\begin{enumerate}
\item Вывести соотношения для неопределенностей
  (\ref{eqPart3SqueezedTaskX2Alpha_1}) и
  (\ref{eqPart3SqueezedTaskX2Alpha_2}).
\item Вывести соотношения для неопределенностей
  (\ref{eqPart3SqueezedTaskX2N_1}) и
  (\ref{eqPart3SqueezedTaskX2N_2}).
\item Доказать (\ref{eqPart3Squeezed16}) и
  (\ref{eqPart3Squeezed16a}). 
\item Доказать соотношения неопределенностей
  (\ref{eqPart3Squeezed21}). 
\item Доказать выражения (\ref{eqPart3SqueezedTaskOffset}) для
  операторов смещения $\hat{D}\left(\alpha\right)$.
\item Доказать, что (\ref{eqPart3Squeezed37a}) и
  (\ref{eqPart3Squeezed37b}) являются решением 
  уравнения (\ref{eqPart3Squeezed35}) с учетом начальных
  условий (\ref{eqPart3Squeezed36}).
\end{enumerate}
