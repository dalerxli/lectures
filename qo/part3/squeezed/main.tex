%% -*- coding:utf-8 -*- 
\chapter{Сжатые состояния}
\label{chSqueezed}
В сжатых состояниях реализуется возможность, допускаемая соотношением
неопределенности Гайзенберга, уменьшить неопределенность одной из
сопряженных наблюдаемых за счет увеличения неопределенности другой.
Такие состояния могут оказаться полезными в квантовой оптике для
практических нужд. Например, для повышения точности интерференционных
экспериментов. 

Далее рассматривается теория сжатых состояний в параметрических
процессах и применения сжатых состояний в интерференционных
экспериментах. 

\input ./part3/squeezed/haizenberg.tex
\input ./part3/squeezed/quad1.tex
\input ./part3/squeezed/quad2.tex
\input ./part3/squeezed/gener.tex
\input ./part3/squeezed/observ.tex
\input ./part3/squeezed/interfero.tex
\input ./part3/squeezed/nonclass.tex
\input ./part3/squeezed/examples.tex

