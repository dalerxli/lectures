%% -*- coding:utf-8 -*- 
\section{Интерференционные измерения с применение сжатого света}
Рассмотрим схему интерферометра Маха-Цендера, изображенную на
рис. \ref{figPart3Squeezed_14}.
Такая схема уже рассматривалась нами ранее, при исследовании
погрешности интерференционных измерений.Различие в том, что на нулевой
вход подается не вакуумное излучение, а излучение сжатого вакуума.
Попрежнему считаем, что входное и выходное зеркала полупрозрачные ($t
=r =\frac{1}{\sqrt{2}}$), а угловые зеркала - глухие ($r = 1$, $t =
0$). Матрица рассеяния зеркал - 
\begin{equation}
\hat{S} = \frac{1}{\sqrt{2}}\left(
\begin{array}{cc}
1 & i \\
i & 1 \\
\end{array}
\right).
\nonumber
\end{equation}
Интерферометр рассматриваем как датчик какой-либо физической величины,
воздействующей на оптическую длину одного из плечей. На схеме
включенный в верхнее плечо фазовращатель может реагировать на
действие этой величины.

\input ./part3/squeezed/fig14.tex

Уравнения интерферометра имеют вид (\ref{eqPart2Interfero11}):
\begin{eqnarray}
\hat{a}_2 = \frac{1}{\sqrt{2}} \left(\hat{a}_0 + i \hat{a}_1\right),
\,
\hat{a}_3 = \frac{1}{\sqrt{2}} \left(i \hat{a}_0 + \hat{a}_1\right),
\nonumber \\
\hat{a}_4 = \frac{1}{\sqrt{2}} \left(i \hat{a}_2 + e^{i \varphi}
\hat{a}_3\right) = 
\frac{1}{2}\left[
i \left(1 + e^{i \varphi}\right)\hat{a}_0 -
\left(1 - e^{i \varphi}\right)\hat{a}_1
\right],
\nonumber \\
\hat{a}_5 = \frac{1}{\sqrt{2}} \left(\hat{a}_2 + i e^{i \varphi}
\hat{a}_3\right) = 
\frac{1}{2}\left[
\left(1 - e^{i \varphi}\right)\hat{a}_0 +
i \left(1 + e^{i \varphi}\right)\hat{a}_1
\right].
\label{eqPart3SqueezedAddAddAdd2}
\end{eqnarray}
Положим, что входное поле сигнала (вход 0, оператор $\hat{a}_0$)
находится в состоянии сжатого вакуума. Второе входное поле 
(вход 1, оператор $\hat{a}_1$) находится в когерентном состоянии с
большой амплитудой. Настолько большой, что входное поле является
двухмодовым 
\[
\left|\psi\right>_{\mbox{вх}} =
\left|\alpha\right>
\left|re^{i \theta}, 0\right>,
\]
где $\left|\alpha\right>$ - когерентное состояние,
$\left|re^{i \theta}, 0\right>$ - состояние сжатого вакуума. Каналы 4
и 5 детектируются каждый своим фотодетектором. Сигналы, идущие с
фотодетекторов, вычитаются и фиксируются.

Таким образом, выходной сигнал определяется средним значением оператора
\begin{eqnarray}
\hat{n}_{54} = \hat{a}_5^{+}\hat{a}_5 - 
\hat{a}_4^{+}\hat{a}_4 = 
\nonumber \\
=
\left(
\hat{a}_1^{+}\hat{a}_1 - 
\hat{a}_0^{+}\hat{a}_0
\right) \cos\,\varphi
- 
\left(
\hat{a}_0^{+}\hat{a}_1 - 
\hat{a}_1^{+}\hat{a}_0
\right) \sin\,\varphi
\end{eqnarray}
Здесь использовалось равенство (\ref{eqPart3SqueezedAddAddAdd2}).

Найдем среднее значение оператора
$\left<\psi\right|\hat{n}_{54}\left|\psi\right>$. При этом учтем, что
каждый оператор действует только на свою моду. Имеем:
\begin{eqnarray}
\left<\psi\right|\hat{n}_{54}\left|\psi\right> = 
\left(
\left<\alpha\right|\hat{a}_1^{+}\hat{a}_1\left|\alpha\right>
- 
\left<r e^{i\theta}, 0\right|\hat{a}_0^{+}\hat{a}_0\left|r
e^{i\theta}, 0\right> 
\right) \cos\,\varphi -
\nonumber \\
-
\left(
\left<r e^{i\theta}, 0\right|\hat{a}_0^{+}\left|r
e^{i\theta}, 0\right>\alpha +
\alpha 
\left<r e^{i\theta}, 0\right|\hat{a}_0\left|r
e^{i\theta}, 0\right>
\right) \sin\,\varphi.
\label{eqPart3SqueezedAddAddAdd4}
\end{eqnarray}
Первый член в первой скобке равен среднему числу фотонов накачки
$\left|\alpha\right|^2$, а второй член, как мы знаем, равен числу
фотонов в сжатом состоянииб, равном $sh^2 r$. Вторая скобка справа, как
мы получили при рассмотрении балансного детектора, равна нулю. Таким
образом получим
\begin{equation}
\left<\psi\right|\hat{n}_{54}\left|\psi\right> = 
\left(
\left|\alpha\right|^2 - sh^2 r
\right)
\cos\,\varphi.
\nonumber
\end{equation}
Если установить $\varphi=\frac{\pi}{2}$, то среднее значение выходного
сигнала равно нулю:
\[
\left<\psi\right|\hat{n}_{54}\left|\psi\right> = 0.
\]

Среднее значение второго члена в (\ref{eqPart3SqueezedAddAddAdd4})
равно нулю, но его средний квадрат нулю не равен. Это означает, что он
является источником шумов, которые ограничивают точность измерений.

Найдем средний квадрат $\hat{n}_{54}$:
\begin{equation}
\left(\Delta n_{54}\right)^2 = 
\left<\psi\right|\hat{n}_{54}^2\left|\psi\right> -
\left<\psi\right|\hat{n}_{54}\left|\psi\right>^2.
\nonumber
\end{equation}
Среднее в нашем случае равно нулю, следовательно имеем
\begin{eqnarray}
\left(\Delta n_{54}\right)^2 = 
\left<\psi\right|\hat{n}_{54}^2\left|\psi\right> =
\nonumber \\
=
\left|\alpha\right|^2
\left<r e^{i\theta}, 0\right|
\left(\hat{a}_0^{+} + \hat{a}_0\right)^2
\left|r e^{i\theta}, 0\right> =
\nonumber \\
=
\left|\alpha\right|^2
\left<r e^{i\theta}, 0\right|
\left(\hat{a}_0^{+}\right)^2 + 
\left(\hat{a}_0\right)^2 + 
\hat{a}_0^{+}\hat{a}_0 +
\hat{a}_0\hat{a}_0^{+}
\left|r e^{i\theta}, 0\right>.
\nonumber
\end{eqnarray}

Ранее мы имели (\ref{eqPart3SqueezedAddAdd7}):
\begin{eqnarray}
\left<\hat{a}_0\right> = 
\left<0\right|
\hat{S}^{+}\left(r, 0\right)
\hat{a}_0
\hat{S}\left(r, 0\right)
\left|0\right> = 
\left<0\right|
\hat{a}_0 ch\,r - \hat{a}_0^{+} sh\,r
\left|0\right> = 0,
\nonumber \\
\left<\hat{a}_0^{+}\right> = 
\left<0\right|
\hat{a}_0^{+} ch\,r - \hat{a}_0 sh\,r
\left|0\right> = 0,
\nonumber
\end{eqnarray}
откуда получим
\begin{equation}
\left(\Delta n_{54}\right)^2 = 
\left<\psi\right|\hat{n}_{54}^2\left|\psi\right> =
\left|\alpha\right|^2\left(
ch\,2r - sh\,2r
\right) = 
\left|\alpha\right|^2 e^{-2 r}.
\nonumber
\end{equation}
Среднеквадратичная величина шумов будет 
\[
\left|\Delta n_{54}\right| = 
\left|\alpha\right| e^{- r}.
\]
Примем эту величину за пороговое значение, ниже которого мы не сможем
ни обнаружить, ни измерить сигнал. Следовательно, сигнал должен быть
выше порога.

Если начальное значение $\varphi=\frac{\pi}{2}$ изменится на величину $\Delta
\varphi$, то появится сигнал равный
\[
\Delta n_{54} = 
\left|\alpha\right|^2
\Delta \varphi.
\]
Необходимо чтобы 
\[
\left|\alpha\right|^2
\Delta \varphi > 
\left|\alpha\right|
e^{-r},
\]
откуда получаем
\begin{equation}
\Delta \varphi >
\frac{1}{\left|\alpha\right|} e^{-r},
\label{eqPart3SqueezedAddAddAdd7}
\end{equation}
$\left|\alpha\right| = \sqrt{\bar{n}}$, где $\bar{n}$ - среднее число
фотонов поля гетеродина.

Выражение (\ref{eqPart3SqueezedAddAddAdd7}) дает меньшее значение, чем
то, которое мы получим ранее, рассматривая точность интерференционных
измерений. Применение сжатого состояния способно увеличить точность
измерений. 
