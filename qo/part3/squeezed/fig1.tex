%% -*- coding:utf-8 -*- 
\begin{figure}
\centering

\input ./part3/squeezed/pic1.tex

\caption{Область неопределенности и соответствующее им зависимость
  электрического поля от времени для когерентного
  состояния.
  Комплексное число $\alpha$,
  соответствующее рассматриваемому состоянию $\left|\alpha\right>$ на
  комплексной плоскости, задаваемой $X_1$ и $X_2$ лежит вдоль оси
  $X_1$.
  В этом случае, число $\alpha$ - вещественное, среднее значение
  $\left<\alpha\right|\hat{X_1}\left|\alpha\right> =
  \frac{\alpha + \alpha^\ast}{2} = 
  Re \alpha = \alpha$.
  Таким образом неопределенность $\Delta X_1$ может
  трактоваться как неопределенность амплитуды, а $\Delta X_2$ как
  неопределенность фазы. В предложенном на рис. случае когерентного
  (не сжатого) состояния, неопределенность по фазе и амплитуде одинаковая.}
\label{figPart3Squeezed_1}
\end{figure}
