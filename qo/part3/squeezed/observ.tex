%% -*- coding:utf-8 -*- 
\section{Наблюдение сжатого состояния. Измерение степени сжатия}

При квадратурном сжатии света флуктуации одной из квадратурных
компонент могут быть существенно меньше другой. Если выделить эту
компоненту и использовать ее в измерительных целях, можно ожидать
сильного уменьшения шумов, ограничивающих точность измерений. 

\input ./part3/squeezed/fig12.tex

Выделение можно осуществить, используя гомодинный (синхронный)
детектор. Простейшая схема гомодинного детектора изображена на
рис. \ref{figPart3Squeezed_12}. Делительное зеркало имеет следующие
параметры: коэффициент пропускания близок к 1 ($t \approx 1$) и,
следовательно, $r \ll 1$, т. к. $\left|t\right|^2 +\left|r\right|^2 =
1$. Это необходимо для того чтобы сигнал $\hat{a}_0$ не сильно
ослаблялся. При этом амплитуда гомодина, поле которого находится в
когерентном состоянии $\left|\alpha_{H}\right>$ так велика, чтобы поле
гомодина по интенсивности существенно превышало бы поле сигнала. Для
работы гомодинного детектора необходимо, чтобы поле гомодина и сигнала
имели бы одинаковую частоту. Это достигается тем, что поле сигнала и
поле гомодина формируются от одного заданного лазера. Например если
$\hat{a}_0$ соответствует сжатому вакууму, накачка параметрического
генератора осуществляется второй гармоникой задающего лазера $2
\omega_p$. При вырожденном параметрическом взаимодействии генерируется
сигнал с частотой $\omega_p$, что совпадает с частотой гомодина
$\omega_p$. Таким образом условие равенства частот соблюдается.

Оператор поля, поступающего на фотодетектор $\hat{a}_2$, можно выразить
через операторы $\hat{a}_0$ и $\hat{a}_1$ - операторы поля сигнала и
поля гомодина
\begin{equation}
\hat{a}_2 = t \hat{a}_0 + r\hat{a}_1.
\nonumber
\end{equation}
Тогда оператор числа фотонов равен
\begin{eqnarray}
\hat{a}_2^{+}\hat{a}_2 = 
\left(t \hat{a}_0^{+} + r\hat{a}_1^{+}\right)
\left(t \hat{a}_0 + r\hat{a}_1\right) = 
\nonumber \\
=
t^2\hat{a}_0^{+}\hat{a}_0 + r t \left(
\hat{a}_0^{+}\hat{a}_1 + \hat{a}_1^{+}\hat{a}_0 
\right) +
r^2\hat{a}_1^{+}\hat{a}_1.
\nonumber
\end{eqnarray}
Положим, что гомодин находится в когерентном состоянии, а сигнал в
состоянии сжатого вакуума. Тогда входное состояние будет двухмодовым 
\[
\left|\psi\right> = 
\left|\alpha_p\right> \left|r, 0\right>.
\]
Среднее значение оператора числа фотонов $\hat{n}_2 =
\hat{a}_2^{+}\hat{a}_2$ в таком случае будет равно:
\begin{eqnarray}
\left<\hat{n}_2\right> = 
\left<\psi\right|\hat{a}_2^{+}\hat{a}_2\left|\psi\right> = 
\nonumber \\
=
t^2\left<r, 0\right|\hat{a}_0^{+}\hat{a}_0\left|r, 0\right> + 
r^2 \left|\alpha_p\right|^2 + 2 t r \left|\alpha_p\right|
\left<r, 0\right|\hat{X}\left(\varphi\right)\left|r, 0\right>,
\label{eqPart3SqueezedAdd3}
\end{eqnarray}
где
\[
\hat{X}\left(\varphi\right) = \frac{1}{2}\left(
\hat{a}_0 e^{-i \varphi} +
\hat{a}_0^{+} e^{i \varphi}
\right),
\]
\[
\alpha_p = 
\left|\alpha_p\right|
 e^{i \varphi}.
\]
Здесь $\varphi$ - фаза гомодина. Тогда при $\varphi = 0$ и при 
$\varphi = \frac{\pi}{2}$ имеем
\begin{eqnarray}
\hat{X}\left(0\right) = 
\frac{1}{2}\left(\hat{a}_0 + \hat{a}_0^{+}\right) = \hat{X}_1,
\nonumber \\
\hat{X}\left(\frac{\pi}{2}\right) = 
\frac{1}{2i}\left(\hat{a}_0 - \hat{a}_0^{+}\right) = \hat{X}_2,
\nonumber
\end{eqnarray}
т. е меняя фазовращателем фазу гомодина, можно выделить ту или иную
квадратурную составляющие.

Равенство (\ref{eqPart3SqueezedAdd3}) содержит три члена, содержащие 
$\left<\hat{a}_0^{+}\hat{a}_0\right>$ - среднее число фотонов в моде
сигнала, $r^2\left|\alpha_p\right|^2$ - среднее число фотонов в моде
гомодина и интерференционный член 
$\left<r, 0\right|\hat{X}\left(\varphi\right)\left|r, 0\right>$,
который выделяет ту или иную квадратурные составляющие. Положим, что
интенсивность гомодина настолько велика, что 
\[
r^2 \left|\alpha_p\right|^2 \gg 
\left<\hat{a}_0^{+}\hat{a}_0\right>
\]
тогда первым членом в (\ref{eqPart3SqueezedAdd3}) можно пренебречь. В
этом случае среднее число фотонов, поступающих на фотодетектор, равно
\begin{equation}
\left<\hat{n}_2\right> = 
r^2 \left|\alpha_p\right|^2 + 2 t r \left|\alpha_p\right|
\left<r, 0\right|\hat{X}\left(\varphi\right)\left|r, 0\right>,
\nonumber
\end{equation}
где первый член является известным числом и его можно
вычесть. Остается только член, содержащий квадратуру сигнала. Далее
можно определить флуктуации числа фотонов (неопределенность числа
фотонов), воспользовавшись формулой
\begin{equation}
\left(\Delta n_2\right)^2 = 
\left<\hat{n}_2^2\right>
-
\left<\hat{n}_2\right>^2.
\nonumber
\end{equation}
Вычисления приводят к формуле
%FIXME!!! calculate it
\begin{equation}
\left(\Delta n_2\right)^2 =
r^2 \left|\alpha_p\right|^2
\left\{
r^2 + 4 t^2 
\left(
\Delta X \left(\varphi\right)
\right)^2
\right\}.
\label{eqPart3SqueezedAdd7}
\end{equation}
Из (\ref{eqPart3SqueezedAdd7}) следует, что шумы содержат две
составляющих, одна из которых $r^2 \left|\alpha_p\right|^2 r^2$ -
\index{гомодин}
выражает шумы гомодина, вторая 
$r^2 \left|\alpha_p\right|^2 4 t^2 
\left(
\Delta X \left(\varphi\right)
\right)^2$ -  связана с шумами сигнала. Если сигналом является вакуумное
колебание 
\[
\left(\Delta X \left(\varphi\right)\right)^2  = \frac{1}{4},
\] 
то условием сжатия является 
\[
\left(\Delta X_1 \right)^2 < \frac{1}{4}
\]
для сжатой компоненты. 

Обнаружить эффект сжатия можно при помощи
гомодинного детектора в следующем эксперименте. Сперва на входе
присутствует только вакуумное состояние, при этом измеряется уровень
шумов. Затем на вход подается исследуемое колебание и измеряется
уровень шумов в зависимости от $\varphi$. Если на вход подано сжатое
состояние, при нектором значении $\varphi = \varphi^{(1)}$ шумы будут
минимальными (меньшими, чем в случае вакуумных колебаний на входе). При
другом значении $\varphi = \varphi^{(2)}$, отличающимся на $\pi$ шумы
будут максимальными (больше, чем в случае вакуумных
колебаний). Подобные эксперименты проводились неоднократно. При этом
обнаруживался эффект сжатия и измерялась его степень (см. например
\cite{bNonclassSqueezedStateDetection}). 

\input ./part3/squeezed/fig13.tex

\subsection{Балансная схема гомодинного детектора}
Более сложной схемой гомодинного приемника является балансная схема,
изображенная на рис. \ref{figPart3Squeezed_13}. Она позволяет
существенно уменьшить шумы гомодина. От уже рассмотренной схемы она
отличается наличием двух фотоприемников, по одному для каждого
выходного пучка. Сигналы, идущие с каждого фотоприемника, вычитаются, и
разность фиксируется. Делительное зеркало полупрозрачное $t = r =
\frac{1}{\sqrt{2}}$ (на этот раз мы приняли, что $t$ и $r$
вещественны, но коэффициент отражения по разные стороны зеркала
отличается знаком).

На схеме $\hat{a}_0$ - оператор сигнала, а $\hat{a}_1$ - оператор моды
гомодина. Предполагается, что гомодин находится в когерентном
состоянии с большой амплитудой. Поля мод 2 и 3 (операторы $\hat{a}_2$
и $\hat{a}_3$) детектируются отдельными фотодетекторами $D^{(2)}$ и
$D^{(3)}$, и сигналы, идущие с каждого фотоэлемента, вычитаются. Такими
образом, на выходе сигнал определяется средним значением оператора
\begin{equation}
\hat{n}_{23} = \hat{a}_2^{+}\hat{a}_2 - 
\hat{a}_3^{+}\hat{a}_3.
\nonumber
\end{equation}
Операторы $\hat{a}_2$ и $\hat{a}_3$ связаны  делительным зеркалом с
операторами входных полей $\hat{a}_0$ и $\hat{a}_1$ следующими
соотношениями: 
\begin{eqnarray}
\hat{a}_2 = \frac{1}{\sqrt{2}} \left(\hat{a}_0 + \hat{a}_1\right),
\nonumber \\
\hat{a}_3 = \frac{1}{\sqrt{2}} \left(- \hat{a}_0 + \hat{a}_1\right),
\nonumber
\end{eqnarray}
таким образом
\begin{eqnarray}
\hat{n}_{23} = \frac{1}{2}
\left(
\left(\hat{a}_0^{+} + \hat{a}_1^{+}\right)
\left(\hat{a}_0 + \hat{a}_1\right)
-
\left(-\hat{a}_0^{+} + \hat{a}_1^{+}\right)
\left(-\hat{a}_0 + \hat{a}_1\right)
\right) = 
\nonumber \\
=
\frac{1}{2}
\left(
\hat{a}_0^{+}\hat{a}_0 + \hat{a}_1^{+}\hat{a}_0
+
\hat{a}_0^{+}\hat{a}_1 + \hat{a}_1^{+}\hat{a}_1
-
\left(
\hat{a}_0^{+}\hat{a}_0 - \hat{a}_1^{+}\hat{a}_0
-\hat{a}_0^{+}\hat{a}_1 + \hat{a}_1^{+}\hat{a}_1
\right)
\right) = 
\nonumber \\
=\hat{a}_0^{+}\hat{a}_1 + \hat{a}_1^{+}\hat{a}_0 = 
\left(
\hat{a}_0^{+}e^{i\varphi} + \hat{a}_0 e^{- i\varphi}
\right)\left|\alpha_H\right|,
\nonumber
\end{eqnarray}
где использовано то обстоятельство, что гомодин находится в
когерентном состоянии с большой амплитудой 
\[
\alpha_H = \left|\alpha_H\right|e^{i\varphi},
\]
и $\hat{a}_1$ заменено на классическое поле
$\left|\alpha_H\right|e^{i\varphi}$ ($\left|\alpha_H\right| \gg 1$).  

Найдем среднее значение оператора $\hat{n}_{23}$ в случае сигнала в
сжатом вакуумном состоянии $\left|z, 0\right>$, где для простоты
положим $z = r$ - вещественным ($r = \left|z\right|$). Также для
простоты положим $\varphi = 0$. В этом случае 
\[
\hat{n}_{23} =
\left(\hat{a}_0 + \hat{a}_0^{+}\right) \left|\alpha_H\right| = 
2 \hat{X}_1 \left|\alpha_H\right|.
\]
Это означает, что мы рассматриваем случай сжатия квадратурной
компоненты $\hat{X}_1$. Среднее значение оператора $\hat{n}_{23}$
записывается в следующем виде
\begin{eqnarray}
\left<\hat{n}_{23}\right> = 
2 \left|\alpha_H\right|\left<\hat{X}_1\right> = 
\left|\alpha_H\right|\left<r, 0\right|\hat{a}_0 +
\hat{a}_0^{+}\left|r, 0\right> =
\nonumber \\
=
\left|\alpha_H\right|\left<0\right|
\hat{S}^{+}\left(r\right)
\left(\hat{a}_0 +
\hat{a}_0^{+}\right)
\hat{S}\left(r\right)
\left|0\right> =
\nonumber \\
=
\left|\alpha_H\right|
\left(
\left<0\right|
\hat{a}_0 ch\,r -
\hat{a}_0^{+} sh\,r
\left|0\right> 
+
\left<0\right|
\hat{a}_0^{+} ch\,r -
\hat{a}_0 sh\,r
\left|0\right> 
\right)
= 0,
\nonumber
\end{eqnarray}
т. к. $\hat{a}_0\left|0\right> = 0$, 
$\left<0\right|\hat{a}_0^{+} = 0$.
Таким образом, среднее значение $\left<\hat{n}_{23}\right> =
0$. Однако средний квадрат не равен 0. Он выражает шумы, ограничивающие
точность измерений.

Вычислим неопределенность $\left(\Delta X_{1,2}\right)^2$, которая
определяет уровень шумов рассматриваемой балансной схемы:
\begin{eqnarray}
\left(\Delta X_{1,2}\right)^2 = 
\left<r,
0\right|\frac{1}{4}\left(\hat{a}_0\pm\hat{a}_0^{+}\right)^2\left|r,
0\right> -
\nonumber \\
-
\left<r,
0\right|\frac{1}{2}\left(\hat{a}_0\pm\hat{a}_0^{+}\right)\left|r,
0\right>^2.
\label{eqPart3SqueezedAddAdd5}
\end{eqnarray}
Поскольку мы показали, что последний член в
(\ref{eqPart3SqueezedAddAdd5}) равен нулю, получаем:
\begin{eqnarray}
\left(\Delta X_{1,2}\right)^2 = 
\left<0\right|\hat{S}^{+}\left(r, 0\right)
\frac{1}{4}\left(\hat{a}_0\pm\hat{a}_0^{+}\right)^2
\hat{S}\left(r, 0\right)
\left|0\right> = 
\nonumber \\
=
\left<0\right|\hat{S}^{+}\left(r, 0\right)
\frac{1}{4}
\left(\hat{a}_0\pm\hat{a}_0^{+}\right)
\hat{S}^{+}\left(r, 0\right)
\hat{S}\left(r, 0\right)
\left(\hat{a}_0\pm\hat{a}_0^{+}\right)
\hat{S}\left(r, 0\right)
\left|0\right>.
\label{eqPart3SqueezedAddAdd6}
\end{eqnarray}
Чтобы найти 
$\left(\Delta X_{1,2}\right)^2$, нужно вычислить средние:
\begin{eqnarray}
\left<\hat{a}_0\right> = 
\left<0\right|
\hat{S}^{+}\left(r, 0\right)
\hat{a}_0
\hat{S}\left(r, 0\right)
\left|0\right> = 
\left<0\right|
\hat{a}_0 ch\,r - \hat{a}_0^{+} sh\,r
\left|0\right> = 0,
\nonumber \\
\left<\hat{a}_0^{+}\right> = 
\left<0\right|
\hat{a}_0^{+} ch\,r - \hat{a}_0 sh\,r
\left|0\right> = 0,
\label{eqPart3SqueezedAddAdd7}
\end{eqnarray}
т. к. $\hat{a}_0\left|0\right> = 0$, 
$\left<0\right|\hat{a}_0^{+} = 0$.

Таким же образом можно показать
\begin{eqnarray}
\left<\hat{a}_0^{+}\hat{a}_0\right> = sh^2 r,
\nonumber \\
\left<\hat{a}_0\hat{a}_0^{+}\right> = 
1 + \left<\hat{a}_0^{+}\hat{a}_0\right> =
1 + sh^2 r = ch^2 r,
\nonumber \\
\left<\hat{a}_0\hat{a}_0\right> = 
\left<\hat{a}_0^{+}\hat{a}_0^{+}\right>^{*} = - ch\,r sh\,r.
\label{eqPart3SqueezedAddAdd8}
\end{eqnarray}
Преобразуя (\ref{eqPart3SqueezedAddAdd6}) при помощи
(\ref{eqPart3SqueezedAddAdd7}) 
и
(\ref{eqPart3SqueezedAddAdd8}),
получим следующее выражение
\begin{equation}
\left(\Delta X_{1,2}\right)^2 = 
ch\,2 r \mp sh\, 2 r = 
e^{\mp 2 r},
\nonumber
\end{equation}
откуда
\begin{equation}
\left(\Delta n_{2,3}\right)^2 = 
n_H
e^{- 2 r}
\nonumber
\end{equation}
для компоненты $X_1$. Дисперсия равна 
\[
\Delta n_{2,3} = 
\sqrt{n_H}
e^{- r},
\]
где $n_H=\left|\alpha_H\right|^2$ - число фотонов в поле гомодина.
При помощи балансной схемы можно измерять степень сжатия таким же
образом, как и при помощи простейшей схемы гомодина. Преимущество
здесь в том, что шумы самого гомодина в этом случае компенсируются и
точность измерения возрастет.


