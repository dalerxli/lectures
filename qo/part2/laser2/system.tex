%% -*- coding:utf-8 -*- 
\section{Система уравнений Гейзенберга-Ланжевена, описывающая лазер}
%% Для получения системы уравнений, описывающих работу лазера надо
%% воспользоваться уравнением Гейзенберга \eqref{eqLaserHaizenberg1} подставив в него
%% гамильтониан лазерной системы \eqref{eqLaserHaizenberg2}. Поскольку
%% атомы и полевые операторы коммутируют, при выводе уравнений для
%% полевых операторов можно опустить атомные операторы, входящие в
%% гамильтониан, а при выводе уравнений для атомных операторов можно
%% опустить полевые операторы. Это несколько облегчает преобразования.

%% Вывод уравнений в идейном плане ничем не отличается от того, что мы
%% уже делали, когда выводили квантовое уравнение, описывающее затухание
%% моды резонатора. Однако лазерная система сложнее простого
%% гармонического осцилатора (моды резонатора). По этой причине вывод
%% уравнений более грузоемок, а выкладки связанные с этим, более
%% громозки. Чтобы не загромождать изложение мы напишем соотвествующие
%% выражения без выводов, ссылаясь на аналогию с задачей, рассмотренной
%% ранее. Необходимые дополнительные пояснения приведены в приложении
%% \ref{AddLanzheven}.

%% Начнем с уравнения для оператора моды поля (оператора уничтожения):
%% \begin{equation}
%% \frac{\hat{a}\left(t\right)}{dt} =
%% - 
%% \left\{
%% \frac{1}{2}\frac{\omega}{Q} + \left(\omega_r - \omega\right)
%% \right\}\hat{a} + i g N \hat{\sigma} + \hat{F}_f\left(t\right).
%% \label{eqLaserHaizenberg11}
%% \end{equation}
%% Уравнение выводится таким же образом, как уравнение описывающее
%% затухание моды. Первый член справа имеет такой же вид, как полученный
%% ранее. Появился новый (второй) член, описывающий усиление, связанное с
%% вынужденными переходами. Последний член - Ланжевеновский шумовой
%% оператор, описывающий влияние резервуара. Корреляионная функция,
%% характеризующая статистические свойства $\hat{F}_f\left(t\right)$ как
%% и раньше равна: 
%% \begin{eqnarray}
%% \left<\hat{F}^{+}_f\left(t\right)\hat{F}_f\left(t'\right)\right> = 
%% \frac{\omega}{Q} \bar{n}_{T} \delta\left(t - t'\right),
%% \nonumber \\
%% \left<\hat{F}_f\left(t\right)\hat{F}^{+}_f\left(t'\right)\right> = 
%% \frac{\omega}{Q}  \left(\bar{n}_{T} + 1\right) \delta\left(t - t'\right).
%% \label{eqLaserHaizenberg12}
%% \end{eqnarray}

%% В \eqref{eqLaserHaizenberg12} выражено через добротность резонатора
%% $Q$. $N$ - полное число активных атомов. $\bar{n}_T$ - среднее число
%% атомовв моде резервуара при температуре $T$. $\omega$ - частота
%% генерации, $\omega_r$ - собственная частота резонатора.

%% Для других операторов уравнения Гпйзенберга-Ланжевена будут иметь
%% следующий вид: для оператора перехода $\sigma =
%% \left|b\right>\left<a\right|$ с которым связан дипольный момент атома:
%% \begin{eqnarray}
%% \frac{d \hat{\sigma}}{d t} = - \left[\gamma+i\left(\omega_{ab} -
%%   \omega\right)\right]\hat{\sigma} + 
%% \nonumber \\
%% + i g \left(\hat{\sigma}_a - \hat{\sigma}_b \right) + 
%% \hat{F}_{\sigma}\left(t\right),
%% \label{eqLaserHaizenberg13}
%% \end{eqnarray}
%% где $\gamma$ - скорость релаксации дипольного момента, $\omega$ -
%% частота генерации, $\omega_{ab}$ - частота перехода между атомами $a$
%% и $b$, $g$ - констатнта взаимодействия,
%% $\hat{F}_{\sigma}\left(t\right)$ - Ланжевеновский шумовой оператор,
%% имющий следующие корреляционные функции:
%% \begin{eqnarray}
%% \left<\hat{F}^{+}_{\sigma}\left(t\right)\hat{F}_{\sigma}\left(t'\right)\right>
%% =  
%% \delta\left(t - t'\right) \left[\left(2\gamma -
%%   \gamma_z\right)\left<\hat{\sigma_a}\right> + r_a\right],
%% \nonumber \\
%% \left<\hat{F}_{\sigma}\left(t\right)\hat{F}^{+}_{\sigma}\left(t'\right)\right>
%% =  
%% \delta\left(t - t'\right) \left[\left(2\gamma -
%%   \gamma_z\right)\left<\hat{\sigma_b}\right>\right],
%% \label{eqLaserHaizenberg14}
%% \end{eqnarray}
%% где $\gamma_z$ - характеризует скорость редаксации населенности,
%% $\left<\hat{\sigma_a}\right>$ - среднее значение оператора
%% населенности верхнего уровня, $\left<\hat{\sigma_b}\right>$ - среднее
%% значение оператора населенности нижнего уровня. Для простоты считаем,
%% что $\gamma_a = \gamma_b = \gamma_z$.

%% Уравнения оператора разности населенностей уровней $a$ и $b$ 
%% \[
%% \left|a\right>\left<a\right| - \left|b\right>\left<b\right| = 
%% \hat{\sigma_a} - \hat{\sigma_b} = \hat{\sigma_z}
%% \]
%% будут иметь вид:
%% \begin{eqnarray}
%% \frac{d}{d t}\hat{\sigma_z} = r_a - \gamma_z \hat{\sigma_z} +
%% \nonumber \\
%% + 2 i g \left(\hat{a}^{+}\hat{\sigma} - \hat{a}\hat{\sigma}^{+}\right)
%% + \hat{F}_z,
%% \label{eqLaserHaizenberg15}
%% \end{eqnarray}
%% где $\hat{F}_z$ - ланжевеновский шумовой оператор для которого
%% корреляционная функция будет иметь вид 
%% \begin{equation}
%% \left<\hat{F}_z\left(t\right)\hat{F}_z\left(t'\right)\right>
%% =  
%% \delta\left(t - t'\right) \left[r_a +
%%   \gamma_z\left<\hat{\sigma}_a + \hat{\sigma}_b\right>\right],
%% \label{eqLaserHaizenberg16}
%% \end{equation}

Объединив полученные нами уравнения \eqref{eqLaserHaizenbergA}, 
\eqref{eqLaserHaizenberNB_AB} и
\eqref{eqLaserHaizenberNZ} мы можем записать следующую 
систему:
\begin{eqnarray}
\frac{d \hat{a}}{d t} = 
- \frac{1}{2}\frac{\omega}{Q}\hat{a}
-i g \hat{N}_{ab} + 
\hat{F}_F,
\nonumber \\
\frac{d \hat{N}_{ab}}{d t} = 
- \gamma \hat{N}_{ab} 
+ i g \hat{N}_{z} \hat{a} 
 + \hat{F}_{ab},
\nonumber \\
\frac{d \hat{N}_z}{d t} = r_a
- \gamma \hat{N}_{z} +
2 i g 
 \left(
\hat{a}^{+}\hat{N}_{ab} -
\hat{N}_{ab}^{+}\hat{a}
\right) + \hat{F}_{z},
\label{eqLaserHaizenberMain}
\end{eqnarray}
которую мы будем называть системой уравнений Гейзенберга-Ланжевена,
описывающих работу лазера.

По своему виду эти уравнения \eqref{eqLaserHaizenberMain} очень похожи
на уравнения лазера, полученные в полуклассическом
приближении. Различие заключается в том, 
что вместо классических величин в уравнении стоят операторы, которые
могу не коммутировать между собой, следовательно при преобразовании
нужно соблюдать порядок следования сомножителей. Кроме того, в
уравнениях появляются Ланжевеновские члены, описывающие квантовые
шумы, имеющиеся в лазерной системе.

К уравнениям \eqref{eqLaserHaizenberMain}
можно применить те же приближения, которые применяются в
полуклассическом случае.

Если, как это обычно бывает, 
%% скорость релаксации дипольного момента
%% $\gamma$ много больше скорости релаксации населенности $\gamma_z$ и
%% скорости релаксации моды $\frac{\omega}{Q}$, т. е. $\gamma \gg
%% \gamma_z,\frac{\omega}{Q}$ 
можно применить адиабатическое приближение,
пренебрегая во втором уравнении \eqref{eqLaserHaizenberMain} производной
$\frac{d\hat{N}_{ab}}{d t}$ по сравнению с членом $\gamma
\hat{N}_{ab}$ получим
\begin{equation}
\hat{N}_{ab} = \frac{1}{\gamma} \left(i g \hat{N}_z\hat{a} +
\hat{F}_{ab}\right).
\nonumber
\end{equation}
Подставляя это выражение для $\hat{N}_{ab}$ в оставшиеся два уравнения
системы \eqref{eqLaserHaizenberMain} получим новую
систему уравнений:
\begin{eqnarray}
\frac{d}{dt} \hat{a} = -
\frac{1}{2}\left(\frac{\omega}{Q}\right)\hat{a} + \frac{g^2}{\gamma}
\hat{N}_z\hat{a} + \hat{F}_F - i\frac{g}{\gamma}\hat{F}_{ab},
\nonumber \\
\frac{d}{dt}\hat{N}_z = r_a -
\gamma\hat{N}_z - \frac{4g^2}{\gamma}
\hat{N}_z\hat{a}^{+}\hat{a} +
\nonumber \\
+ \hat{F}_z + \frac{2ig}{\gamma} \left(\hat{a}^{+}\hat{F}_{ab} -
\hat{F}_{ab}^{+}\hat{a}\right).
\label{eqLaserHaizenberg16add}
\end{eqnarray}

Если ввести новое обозначение 
\begin{equation}
\hat{F}_{\sum} = - \left(\frac{g}{\gamma}\hat{F}_{ab} + i \hat{F}_F\right),
\label{eqLaserHaizenbergFSum}
\end{equation}
то систему 
\eqref{eqLaserHaizenberg16add} можно переписать в виде
\begin{eqnarray}
\frac{d}{dt} \hat{a} = -
\frac{1}{2}\left(\frac{\omega}{Q}\right)\hat{a} + \frac{g^2}{\gamma}
\hat{N}_z\hat{a} + i \hat{F}_{\sum},
\nonumber \\
\frac{d}{dt}\hat{N}_z = r_a -
\gamma\hat{N}_z - \frac{4g^2}{\gamma}
\hat{N}_z\hat{a}^{+}\hat{a} +
\nonumber \\
+ \hat{F}_z + \frac{2ig}{\gamma} \left(\hat{a}^{+}\hat{F}_{ab} -
\hat{F}_{ab}^{+}\hat{a}\right).
\label{eqLaserHaizenberg17}
\end{eqnarray}

Для шумового оператора \eqref{eqLaserHaizenbergFSum} при помощи
\eqref{eqLaserHaizenbergFABCorrel_1},
\eqref{eqLaserHaizenbergFABCorrel_2} и
\eqref{eqLaserHaizenbergFFCorrel} 
можно написать следующие корреляционные соотношения:
\begin{eqnarray}
\left<\hat{F}^{+}_{\sum}\left(t_1\right)\hat{F}_{\sum}\left(t_2\right)\right>
=
\frac{g^2}{\gamma^2}\left<\hat{F}^{+}_{ab}\left(t_1\right)\hat{F}_{ab}\left(t_2\right)\right>
+
\left<\hat{F}^{+}_{F}\left(t_1\right)\hat{F}_{F}\left(t_2\right)\right>
= 
\nonumber \\
= 
\frac{1}{2}
\left(
\frac{g^2}{\gamma}\bar{N}_a + 2 \frac{g^2}{\gamma^2}r_a + 
\frac{\omega}{Q}\bar{n}_T
\right)
\delta\left(t_1 - t_2\right),
\nonumber \\
\left<\hat{F}_{\sum}\left(t_1\right)\hat{F}^{+}_{\sum}\left(t_2\right)\right>
= \frac{1}{2}
\left(
\frac{g^2}{\gamma}\bar{N}_b + 
\frac{\omega}{Q}\left(\bar{n}_T + 1\right)
\right)
\delta\left(t_1 - t_2\right)
\label{eqLaserHaizenbergFSumCorrel}
\end{eqnarray}


Уравнения \eqref{eqLaserHaizenberg17} напоминают полуклассическое
уравнение лазера, и для их решения можно в значительной мере
воспользоваться такими же приближениями и приемами которые
используются при решении полуклассической задачи.





