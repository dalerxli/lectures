%% -*- coding:utf-8 -*- 
\section{Модель лазера}

Схема лазерных уровней изображена на
рис. \ref{figPart2Laser2_1}. Уровень $c$ - основной уровень, уровень $b$ -
нижний рабочий уровень, $a$ - верхний рабочий уровень. На самом деле
схема - четырехуровневая. Накачка ведется некогерентным светом через
достаточно широкую линию поглощения, изображенную на рисунке
пунктиром. Но так как предполагается высокая скорость
безизлучательного перехода на верхний рабочий уровень, можно считать,
что накачка происходит прямо на уровень $a$. Наличие четвертого уровня
позволяет сильно уменьшить обратный переход с уровня $a$ на основной
уровень $c$. На рис. \ref{figPart2Laser2_1} обозначены переходы,
которые будут учитываться в теории. $\gamma_a$ и $\gamma_b$
характеризуют релаксацию населенностей с уровней $a$ и $b$ за счет
связи с диссипативной системой. $r_a$ - скорость накачки верхнего
рабочего уровня $a$ за счет некогерентной оптической накачки. Переход
$a \rightarrow c$ вынужденный переход, вызываемый генерируемым лазером
полем.

\input ./part2/laser2/fig1.tex

Схема лазера представлена на рис. \ref{figPart2Laser2_2}. Схема
содержит резонатор $F$, в котором возбуждается генерируемая мода,
взаимодействующая с активными атомами рабочей среды, схема которой
представлена на рис. \ref{figPart2Laser2_1}. Кроме того, имеются два
резервуара при температуре $T$ $R_{a}$ (связан с
активными атомами) и $R_{F}$ (связан с модой), вызывающие релаксацию
атомов и поля моды.

\input ./part2/laser2/fig2.tex

