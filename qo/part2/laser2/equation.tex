%% -*- coding:utf-8 -*- 
\section{Уравнение Гейзенберга для лазерной системы}

Рассматриваемая система состоит из 4-х подсистем, которые связаны
между собой посредством 3-х гамильтонианов взаимодействия. Первая из
подсистем представляет собой моду электромагнитного поля с частотой
$\omega$ (на \autoref{figPart2Laser2_2} обозначена через $F$),
которая описывается операторами рождения и уничтожения моды: 
$\hat{a}^{\dag}$ и $\hat{a}$. 

Мода электромагнитного поля связана с
резервуаром $R_F$ (см. \autoref{figPart2Laser2_2}), который
представляет собой совокупность гармонических осцилляторов с частотой
$\omega_k^{(F)}$, описываемых операторами рождения и уничтожения
$\hat{b}_k^{\dag}$, $\hat{b}_k$. При этом справедливы следующие
коммутационные соотношения 
\begin{eqnarray}
\left[\hat{b}_k, \hat{b}^{\dag}_{k'}\right] = \delta_{kk'},
\nonumber \\
\left[\hat{b}_k, \hat{b}_{k'}\right] = 0,
\nonumber \\
\left[\hat{b}^{\dag}_k, \hat{b}^{\dag}_{k'}\right] = 0.
\nonumber
\end{eqnarray}

Атомная подсистема состоит из большого числа атомов. Атом $j$ может
находится в следующих основных состояниях:
$\ket{a}^j$, $\ket{b}^j$ и $\ket{c}^j$. Поведение
этого атома может быть описано с помощью операторов проектирования в
состояния $\ket{a}^j$ и $\ket{b}^j$:
\begin{equation}
\hat{\sigma}_a^j = \ket{a}^j\bra{a}^j,\,
\hat{\sigma}_b^j = \ket{b}^j\bra{b}^j,
\nonumber
\end{equation}
а также операторов перехода
\begin{eqnarray}
\hat{\sigma}_{a \rightarrow b}^j = \hat{\sigma}_{ab}^j =
\ket{b}^j\bra{a}^j,\,
\hat{\sigma}_{b \rightarrow a}^j = \hat{\sigma}_{ba}^j =
\left(\hat{\sigma}_{ab}^j\right)^{\dag},
\nonumber \\
\hat{\sigma}_{a \rightarrow c}^j = \hat{\sigma}_{ac}^j =
\ket{c}^j\bra{a}^j,\,
\hat{\sigma}_{c \rightarrow a}^j = \hat{\sigma}_{ca}^j =
\left(\hat{\sigma}_{ac}^j\right)^{\dag},
\nonumber \\
\hat{\sigma}_{b \rightarrow c}^j = \hat{\sigma}_{bc}^j =
\ket{c}^j\bra{b}^j,\,
\hat{\sigma}_{c \rightarrow b}^j = \hat{\sigma}_{cb}^j =
\left(\hat{\sigma}_{bc}^j\right)^{\dag}.
\nonumber 
\end{eqnarray}

Атомная подсистема связана с резервуаром $R_A$, который состоит из
большого числа гармонических осцилляторов каждый из которых имеет
частоту $\omega_k^{(A)}$ и описывается операторами рождения и
уничтожения $\hat{c}_k^{\dag}$, $\hat{c}_k$ удовлетворяющими следующим
коммутационным соотношениям:
\begin{eqnarray}
\left[\hat{c}_k, \hat{c}^{\dag}_{k'}\right] = \delta_{kk'},
\nonumber \\
\left[\hat{c}_k, \hat{c}_{k'}\right] = 0,
\nonumber \\
\left[\hat{c}^{\dag}_k, \hat{c}^{\dag}_{k'}\right] = 0.
\nonumber
\end{eqnarray}

В общем случае уравнение Гейзенберга имеет вид:
\begin{equation}
\frac{d \hat{O}}{d t} = \frac{i}{\hbar}\left[\hat{\mathcal{H}},
  \hat{O}\right], 
\label{eqLaserHaizenberg1}
\end{equation}
где $\hat{\mathcal{H}}$ - гамильтониан системы, $\hat{O}$ - оператор
некоторой наблюдаемой, относящейся к этой системе. В целях упрощения
мы будем использовать представление взаимодействия в котором уравнение
\eqref{eqLaserHaizenberg1} перепишется в виде:
\begin{equation}
\frac{d \hat{O}_I}{d t} = \frac{i}{\hbar}\left[\hat{V}_I,
  \hat{O}_I\right], 
\label{eqLaserHaizenbergInteraction1}
\end{equation}
где операторы взаимодействия $\hat{V}_I$ и наблюдаемой $\hat{O}_I$ в
представлении взаимодействия записываются следующим образом: 
\begin{eqnarray}
\hat{O}_I = 
e^{i\frac{\hat{\mathcal{H}}_0t}{\hbar}}
\hat{O}
e^{-i\frac{\hat{\mathcal{H}}_0t}{\hbar}},
\nonumber \\
\hat{V}_I = 
e^{i\frac{\hat{\mathcal{H}}_0t}{\hbar}}
\hat{V}
e^{-i\frac{\hat{\mathcal{H}}_0t}{\hbar}}.
\label{eqLaserHaizenbergInteraction2}
\end{eqnarray}
Через $\hat{\mathcal{H}}_0$ в \eqref{eqLaserHaizenbergInteraction2}
обозначен полный гамильтониан рассматриваемой системы без учета
взаимодействия между ее частями. 

В дальнейшем все операторы у нас будут предполагаться записанными в
представлении взаимодействия \eqref{eqLaserHaizenbergInteraction2} при
этом индекс $I$ будет опускаться.

Гамильтониан взаимодействия между $j$-м атомом и модой
электромагнитного поля в представлении
взаимодействия имеет вид (см. \ref{eqCh2_task22}):
\begin{equation}
\hat{V}_{AF}^j =
\Delta\left(t - t_j\right) 
g \hbar \left(
\hat{a}^{\dag}\hat{\sigma}^{j}_{ab} e^{i \left(\omega -
  \omega_{ab}\right)t} +
\left(\hat{\sigma}^{j}_{ab}\right)^{\dag} 
\hat{a} e^{-i \left(\omega - \omega_{ab}\right)t}
\right),
\label{eqLaserHaizenbergVAF_pre1}
\end{equation}
где $\Delta\left(t - t_j\right)$ ступенчатая функция описывающая
процесс инжекции атомов, т. е. учитывающая тот факт, что
взаимодействие между $j$-м атомом и полем начинается в момент времени $t_j$:
\begin{equation}
\Delta\left(t - t_j\right) =
\left\{
\begin{array}{c}
1;\, t \ge t_j \\
0;\, t < t_j 
\end{array}
\right.
\nonumber
\end{equation}
Полный гамильтониан взаимодействия будет суммой гамильтонианов
отдельных атомов
\eqref{eqLaserHaizenbergVAF_pre1}:
\begin{equation}
\hat{V}_{AF} =
g \hbar
\sum_j
\Delta\left(t - t_j\right) 
 \left(
\hat{a}^{\dag}\hat{\sigma}^{j}_{ab}  +
\left(\hat{\sigma}^{j}_{ab}\right)^{\dag} 
\hat{a}
\right).
\label{eqLaserHaizenbergVAF}
\end{equation}
В \eqref{eqLaserHaizenbergVAF} мы предполагаем что частота моды
электромагнитного поля совпадает с частотой перехода: $\omega =
\omega_{ab}$.

\index{Гамильтониан}Гамильтониан взаимодействия поля с резервуаром $R_F$ записывается
следующим образом: 
\begin{equation}
\hat{V}_{FR_F} =
\hbar
\sum_k
g_k^{(F)}
 \left(
\hat{b}_k^{\dag}\hat{a} e^{i\left(\omega_k^{(F)} - \omega\right)t} +
\hat{a}^{\dag}\hat{b}_k e^{-i\left(\omega_k^{(F)} - \omega\right)t}
\right).
\label{eqLaserHaizenbergVFRF}
\end{equation}


Сложнее дело обстоит с гамильтонианом взаимодействия атомной
подсистемы с резервуаром $R_A$. Исходя из рассматриваемой модели
(см. \autoref{figPart2Laser2_2}) мы имеем дело с двумя переходами:
$a \rightarrow c$ и $b \rightarrow c$. Соответственно полный
гамильтониан запишется в виде
\begin{equation}
\hat{V}_{AR_A} = \hat{V}_{AR_A}^{a \rightarrow c} + \hat{V}_{AR_A}^{b \rightarrow c}.
\nonumber
\end{equation}
Предполагая в дальнейшем, что константы взаимодействия для
рассматриваемых переходов одинаковыми и вещественными, получим
\begin{eqnarray}
\hat{V}_{AR_A}^{a \rightarrow c} = \hbar \sum_{k,j} 
g_k^{(A)}
 \left(
\hat{c}_k^{\dag}\hat{\sigma}_{ac}^{j} e^{i\left(\omega_k^{(A)} - \omega_{ac}\right)t} +
\left(\hat{\sigma}_{ac}^{j}\right)^{\dag}\hat{c}_k e^{-i\left(\omega_k^{(A)} - \omega_{ac}\right)t}
\right),
\nonumber \\
\hat{V}_{AR_A}^{b \rightarrow c} = \hbar \sum_{k,j} 
g_k^{(A)}
 \left(
\hat{c}_k^{\dag}\hat{\sigma}_{bc}^{j} e^{i\left(\omega_k^{(A)} - \omega_{bc}\right)t} +
\left(\hat{\sigma}_{bc}^{j}\right)^{\dag}\hat{c}_k e^{-i\left(\omega_k^{(A)} - \omega_{bc}\right)t}
\right).
\label{eqLaserHaizenbergVARA}
\end{eqnarray}
В \eqref{eqLaserHaizenbergVARA} суммирование предполагается по всем
атомам (индекс $j$) и по всем модам резервуара ($k$), через
$\omega_{ac}$ и $\omega_{bc}$ обозначены частоты переходов.

\subsection{Уравнение движения атомной подсистемы}

Прежде всего рассмотрим уравнения которым удовлетворяют операторы
$\hat{c}_k$ и $\hat{c}_k^{\dag}$. Поскольку эти операторы входят только в
гамильтониан $\hat{V}_{AR_A}$, то уравнение движения имеет вид
\begin{eqnarray}
\frac{d \hat{c}_k}{d t} = \frac{i}{\hbar}\left[\hat{V}_{AR_A}, \hat{c}_k
\right] = 
\nonumber \\
= \frac{i}{\hbar} \hbar 
\sum_{k',j} 
g_{k'}^{(A)}\left[\hat{c}_{k'}^{\dag}, \hat{c}_k\right]
\left(
\hat{\sigma}_{ac}^{j} e^{i\left(\omega_{k'}^{(A)} - \omega_{ac}\right)t}
+
\hat{\sigma}_{bc}^{j} e^{i\left(\omega_{k'}^{(A)} - \omega_{bc}\right)t}
\right) +
\nonumber \\
+
\frac{i}{\hbar} \hbar 
\sum_{k',j} 
g_{k'}^{(A)}
\left(
\left(\hat{\sigma}_{ac}^{j}\right)^{\dag} e^{- i\left(\omega_{k'}^{(A)} - \omega_{ac}\right)t}
+
\left(\hat{\sigma}_{bc}^{j}\right)^{\dag} e^{- i\left(\omega_{k'}^{(A)} - \omega_{bc}\right)t}
\right)
\left[\hat{c}_{k'}, \hat{c}_k\right] =
\nonumber \\
= 
 - i 
\sum_{k',j} 
g_{k'}^{(A)}\delta_{kk'}
\left(
\hat{\sigma}_{ac}^{j} e^{i\left(\omega_{k'}^{(A)} - \omega_{ac}\right)t}
+
\hat{\sigma}_{bc}^{j} e^{i\left(\omega_{k'}^{(A)} - \omega_{bc}\right)t}
\right) = 
\nonumber \\
=
 - i g_{k}^{(A)}
\sum_{j} 
\left(
\hat{\sigma}_{ac}^{j} e^{i\left(\omega_{k}^{(A)} - \omega_{ac}\right)t}
+
\hat{\sigma}_{bc}^{j} e^{i\left(\omega_{k}^{(A)} - \omega_{bc}\right)t}
\right).
\label{eqLaserHaizenbergCk_pre1}
\end{eqnarray} 
Формальное интегрирование \eqref{eqLaserHaizenbergCk_pre1} дает
\begin{eqnarray}
\hat{c}_k\left(t\right) = \hat{c}_k\left(0\right) -
\nonumber \\
 - i g_{k}^{(A)}
\sum_{j} 
\int_0^t d t'
\left(
\hat{\sigma}_{ac}^{j}\left(t'\right) e^{i\left(\omega_{k}^{(A)} - \omega_{ac}\right)t'}
+
\hat{\sigma}_{bc}^{j}\left(t'\right) e^{i\left(\omega_{k}^{(A)} - \omega_{bc}\right)t'}
\right),
\label{eqLaserHaizenbergCk}
\end{eqnarray}
откуда для эрмитово сопряженного оператора $\hat{c}_k^{\dag}$ имеем
\begin{eqnarray}
\hat{c}_k^{\dag}\left(t\right) = \hat{c}_k^{\dag}\left(0\right) +
\nonumber \\
 + i g_{k}^{(A)}
\sum_{j} 
\int_0^t d t'
\left(
\hat{\sigma}_{ca}^{j}\left(t'\right) e^{-i\left(\omega_{k}^{(A)} - \omega_{ac}\right)t'}
+
\hat{\sigma}_{cb}^{j}\left(t'\right) e^{-i\left(\omega_{k}^{(A)} - \omega_{bc}\right)t'}
\right),
\label{eqLaserHaizenbergCkPlus}
\end{eqnarray}
где введены следующие обозначения $\hat{\sigma}_{ca}^{j} =
\left(\hat{\sigma}_{ac}^{j}\right)^{\dag}$ и
$\hat{\sigma}_{cb}^{j} = \left(\hat{\sigma}_{bc}^{j}\right)^{\dag}$.

Следующим шагом будет получение уравнений которым удовлетворяют
операторы $\hat{\sigma}_a^j$ и $\hat{\sigma}_b^j$. Для
$\hat{\sigma}_a^j$ имеем
\begin{equation}
\frac{d \hat{\sigma}_a^j}{d t} = 
\frac{i}{\hbar}
\left[\hat{V}_{AF}, \hat{\sigma}_a^j\right] + 
\frac{i}{\hbar}
\left[\hat{V}_{AR_A}, \hat{\sigma}_a^j\right].
\label{eqLaserHaizenbergSigmaA_pre1}
\end{equation}
Для первого члена суммы \eqref{eqLaserHaizenbergSigmaA_pre1} справедливо
\begin{eqnarray}
\frac{i}{\hbar}
\left[\hat{V}_{AF}, \hat{\sigma}_a^j\right] = 
 \frac{i}{\hbar} g \hbar
\sum_i
\Delta\left(t - t_i\right) 
 \left(
\hat{a}^{\dag}\left[\hat{\sigma}^{i}_{ab},\hat{\sigma}_a^j\right]  +
\left[\left(\hat{\sigma}^{i}_{ab}\right)^{\dag},\hat{\sigma}_a^j\right]
\hat{a}
\right),
\nonumber
\end{eqnarray}
откуда с учетом коммутационных соотношений 
\begin{eqnarray}
\left[\hat{\sigma}^{i}_{ab},\hat{\sigma}_a^j\right] = 
\ket{b^i}\bra{a^i}\ket{a^j}\bra{a^j} -
\ket{a^j}\bra{a^j}\ket{b^i}\bra{a^i} = 
\delta_{ij}\hat{\sigma}^{j}_{ab},
\nonumber \\
\left[\left(\hat{\sigma}^{i}_{ab}\right)^{\dag},\hat{\sigma}_a^j\right] = 
\ket{a^i}\bra{b^i}\ket{a^j}\bra{a^j} -
\ket{a^j}\bra{a^j}\ket{a^i}\bra{b^i} = 
- \delta_{ij}\left(\hat{\sigma}^{j}_{ab}\right)^{\dag}
\nonumber
\end{eqnarray}
получим
\begin{eqnarray}
\frac{i}{\hbar}
\left[\hat{V}_{AF}, \hat{\sigma}_a^j\right] = 
i g 
\Delta\left(t - t_j\right) 
 \left(
\hat{a}^{\dag}\hat{\sigma}^{j}_{ab} -
\left(\hat{\sigma}^{j}_{ab}\right)^{\dag}\hat{a}
\right).
\label{eqLaserHaizenbergSigmaA_pre1_1}
\end{eqnarray}

Для второго члена суммы \eqref{eqLaserHaizenbergSigmaA_pre1} имеем
\begin{eqnarray}
\frac{i}{\hbar}
\left[\hat{V}_{AR_A}, \hat{\sigma}_a^j\right] = 
\nonumber \\
=
i
\sum_{k,i}
g_k^{(A)}
\hat{c}_k^{\dag}
 \left(
\left[\hat{\sigma}^{i}_{ac},\hat{\sigma}_a^j\right]  
e^{i\left(\omega_k^{(A)} - \omega_{ac}\right)t}
+ 
\left[\hat{\sigma}^{i}_{bc},\hat{\sigma}_a^j\right]  
e^{i\left(\omega_k^{(A)} - \omega_{bc}\right)t}
\right)
+
\nonumber \\
+
 i
\sum_{k,i}
g_k^{(A)}
 \left(
\left[\left(\hat{\sigma}^{i}_{ac}\right)^{\dag},\hat{\sigma}_a^j\right]  
e^{-i\left(\omega_k^{(A)} - \omega_{ac}\right)t}
+ 
\left[\left(\hat{\sigma}^{i}_{bc}\right)^{\dag},\hat{\sigma}_a^j\right]  
e^{-i\left(\omega_k^{(A)} - \omega_{bc}\right)t}
\right)\hat{c}_k,
\nonumber
\end{eqnarray}
которое с помощью коммутационных соотношений
\begin{eqnarray}
\left[\hat{\sigma}^{i}_{ac},\hat{\sigma}_a^j\right] = 
\delta_{ij}\hat{\sigma}^{j}_{ac},
\nonumber \\
\left[\left(\hat{\sigma}^{i}_{ac}\right)^{\dag},\hat{\sigma}_a^j\right] = 
- \delta_{ij}\left(\hat{\sigma}^{j}_{ac}\right)^{\dag},
\nonumber \\
\left[\hat{\sigma}^{i}_{bc},\hat{\sigma}_a^j\right] = 0,
\nonumber \\
\left[\left(\hat{\sigma}^{i}_{bc}\right)^{\dag},\hat{\sigma}_a^j\right] = 0
\label{eqLaserHaizenbergTaskKommutator}
\end{eqnarray}
переписывается в виде
\begin{eqnarray}
\frac{i}{\hbar}
\left[\hat{V}_{AR_A}, \hat{\sigma}_a^j\right] = 
i
\sum_{k}
g_k^{(A)}
\hat{c}_k^{\dag}\hat{\sigma}^{j}_{ac}  
e^{i\left(\omega_k^{(A)} - \omega_{ac}\right)t}
-
\nonumber \\
- i
\sum_{k}
g_k^{(A)}
\left(\hat{\sigma}^{i}_{ac}\right)^{\dag} 
\hat{c}_k
e^{-i\left(\omega_k^{(A)} - \omega_{ac}\right)t}.
\label{eqLaserHaizenbergSigmaA_pre1_2}
\end{eqnarray}
Объединяя вместе\eqref{eqLaserHaizenbergSigmaA_pre1_1} и 
\eqref{eqLaserHaizenbergSigmaA_pre1_2} получим 
\begin{equation}
\frac{d \hat{\sigma}_a^j}{d t} = 
i g 
\Delta\left(t - t_j\right) 
\hat{a}^{\dag}\hat{\sigma}^{j}_{ab}  +
i
\sum_{k}
g_k^{(A)}
\hat{c}_k^{\dag}\hat{\sigma}^{i}_{ac}  
e^{i\left(\omega_k^{(A)} - \omega_{ac}\right)t} + \mbox{э. с.}
\label{eqLaserHaizenbergSigmaA_pre2}
\end{equation}

При помощи \eqref{eqLaserHaizenbergCkPlus} можно вычислить второе
слагаемое \eqref{eqLaserHaizenbergSigmaA_pre2}:
\begin{eqnarray}
i
\sum_{k}
g_k^{(A)}
\hat{c}_k^{\dag}\hat{\sigma}^{i}_{ac}  
e^{i\left(\omega_k^{(A)} - \omega_{ac}\right)t} = 
i
\sum_{k}
g_k^{(A)}
\hat{c}_k^{\dag}\left(0\right)\hat{\sigma}^{i}_{ac}  
e^{i\left(\omega_k^{(A)} - \omega_{ac}\right)t} -
\nonumber \\
-
\sum_{k,i} 
\left(g_{k}^{(A)}\right)^2
\int_0^t d t'
\hat{\sigma}_{ca}^{i}\left(t'\right)\hat{\sigma}^{j}_{ac}\left(t\right) 
e^{i\left(\omega_{k}^{(A)} - \omega_{ac}\right)\left(t-t'\right)}
-
\nonumber \\
-
\sum_{k,i} 
\left(g_{k}^{(A)}\right)^2
\int_0^t d t'
\hat{\sigma}_{cb}^{i}\left(t'\right)\hat{\sigma}^{j}_{ac}\left(t\right) 
e^{i\left(\omega_{k}^{(A)} - \omega_{bc}\right)\left(t-t'\right)}
e^{i\left(\omega_{bc} - \omega_{ac}\right)t}.
\nonumber
\end{eqnarray}
Обозначив
\begin{equation}
\hat{f}_{a}^{j}\left(t\right) = i
\sum_{k}
g_k^{(A)}
\hat{c}_k^{\dag}\left(0\right)\hat{\sigma}^{i}_{ac}  
e^{i\left(\omega_k^{(A)} - \omega_{ac}\right)t}
\label{eqLaserHaizenbergFAJ}
\end{equation}
получим
\begin{eqnarray}
i
\sum_{k}
g_k^{(A)}
\hat{c}_k^{\dag}\hat{\sigma}^{i}_{ac}  
e^{i\left(\omega_k^{(A)} - \omega_{ac}\right)t} = \hat{f}_{a}^{j} -
\nonumber \\
-
\sum_i
\int_0^t d t'
\sum_k 
\left(g_{k}^{(A)}\right)^2
\hat{\sigma}_{ca}^{i}\left(t'\right)\hat{\sigma}^{j}_{ac}\left(t\right) 
e^{i\left(\omega_{k}^{(A)} - \omega_{ac}\right)\left(t-t'\right)}
-
\nonumber \\
-
e^{i\left(\omega_{bc} - \omega_{ac}\right)t}
\sum_i 
\int_0^t d t'
\sum_k
\left(g_{k}^{(A)}\right)^2
\hat{\sigma}_{cb}^{i}\left(t'\right)\hat{\sigma}^{j}_{ac}\left(t\right) 
e^{i\left(\omega_{k}^{(A)} - \omega_{bc}\right)\left(t-t'\right)}
 = 
\nonumber \\
=
\hat{f}_{a}^{j}
-
D\left(\omega_{ac}\right)
\left(g^{(A)}\left(\omega_{ac}\right)\right)^2
\sum_i
\int_0^t d t'
\int_0^{\infty}d \omega' 
\hat{\sigma}_{ca}^{i}\left(t'\right)\hat{\sigma}^{j}_{ac}\left(t\right) 
e^{i\left(\omega' - \omega_{ac}\right)\left(t-t'\right)}
-
\nonumber \\
-
D\left(\omega_{bc}\right)
\left(g^{(A)}\left(\omega_{bc}\right)\right)^2
e^{i\left(\omega_{bc} - \omega_{ac}\right)t} \cdot
\nonumber \\ 
\cdot
\sum_i 
\int_0^t d t'
\int_0^{\infty}d \omega' 
\hat{\sigma}_{cb}^{i}\left(t'\right)\hat{\sigma}^{j}_{ac}\left(t\right) 
e^{i\left(\omega' - \omega_{bc}\right)\left(t-t'\right)}
\approx
\nonumber \\
\approx
\hat{f}_{a}^{j}
- 
\frac{\gamma_{a}}{2}
\sum_i
\int_0^t d t'
\hat{\sigma}_{ca}^{i}\left(t'\right)\hat{\sigma}^{j}_{ac}\left(t\right) 
\delta\left(t-t'\right)
-
\nonumber \\
-
\frac{\gamma_{b}}{2}
e^{i\left(\omega_{bc} - \omega_{ac}\right)t}
\sum_i 
\int_0^t d t'
\hat{\sigma}_{cb}^{i}\left(t'\right)\hat{\sigma}^{j}_{ac}\left(t\right) 
\delta\left(t-t'\right) =
\nonumber \\
= \hat{f}_{a}^{j}
- 
\frac{\gamma_{a}}{2}
\hat{\sigma}_{a}^{j}\left(t\right) 
-
\frac{\gamma_{b}}{2}
e^{i\left(\omega_{bc} - \omega_{ac}\right)t}
\hat{\sigma}_{ba}^{j}\left(t\right) \approx
\nonumber \\
\approx
\hat{f}_{a}^{j}
- 
\frac{\gamma_{a}}{2}
\hat{\sigma}_{a}^{j}\left(t\right),
\label{eqLaserHaizenbergSigmaA_pre3}
\end{eqnarray}
где 
\[
\gamma_{a} = 4 \pi D\left(\omega_{ac}\right)
\left(g^{(A)}\left(\omega_{ac}\right)\right)^2
\]
и
\[
\gamma_{b} = 4 \pi D\left(\omega_{bc}\right)
\left(g^{(A)}\left(\omega_{bc}\right)\right)^2.
\]
При выводе \eqref{eqLaserHaizenbergSigmaA_pre3} мы также пренебрегли
быстро осциллирующим членом 
$\frac{\gamma_{b}}{2}
e^{i\left(\omega_{bc} - \omega_{ac}\right)t}
\hat{\sigma}_{ba}^{j}\left(t\right)$ 
по сравнению с 
$\frac{\gamma_{a}}{2}
\hat{\sigma}_{a}^{j}\left(t\right)$.
Таким образом в результате получаем
\begin{equation}
\frac{d \hat{\sigma}_a^j}{d t} = 
- \gamma_{a} \hat{\sigma}_{a}^{j} +
i g 
\Delta\left(t - t_j\right) 
 \left(
\hat{a}^{\dag}\hat{\sigma}^{j}_{ab} -
\left(\hat{\sigma}^{j}_{ab}\right)^{\dag}\hat{a}
\right) + \hat{f}_{a}^{j}.
\label{eqLaserHaizenbergSigmaAJ}
\end{equation}

По аналогии, можно получить выражение для $\hat{\sigma}_b^j$:
\begin{equation}
\frac{d \hat{\sigma}_b^j}{d t} = 
- \gamma_{b} \hat{\sigma}_{b}^{j} -
i g 
\Delta\left(t - t_j\right) 
 \left(
\hat{a}^{\dag}\hat{\sigma}^{j}_{ab} -
\left(\hat{\sigma}^{j}_{ab}\right)^{\dag}\hat{a}
\right) + \hat{f}_{b}^{j},
\label{eqLaserHaizenbergSigmaBJ}
\end{equation}
где
\begin{equation}
\hat{f}_{b}^{j}\left(t\right) = i
\sum_{k}
g_k^{(A)}
\hat{c}_k^{\dag}\left(0\right)\hat{\sigma}^{i}_{bc}  
e^{i\left(\omega_k^{(A)} - \omega_{bc}\right)t} + \mbox{э. с.}
\label{eqLaserHaizenbergFBJ}
\end{equation}

Для $\hat{\sigma}^{j}_{ab}$ имеем
\begin{equation}
\frac{d \hat{\sigma}_{ab}^j}{d t} = 
\frac{i}{\hbar}
\left[\hat{V}_{AF}, \hat{\sigma}_{ab}^j\right] + 
\frac{i}{\hbar}
\left[\hat{V}_{AR_A}, \hat{\sigma}_{ab}^j\right].
\label{eqLaserHaizenbergSigmaAB_pre1}
\end{equation}
Для первого члена суммы \eqref{eqLaserHaizenbergSigmaAB_pre1} справедливо
\begin{eqnarray}
\frac{i}{\hbar}
\left[\hat{V}_{AF}, \hat{\sigma}_{ab}^j\right] = 
 \frac{i}{\hbar} g \hbar
\sum_i
\Delta\left(t - t_i\right) 
 \left(
\hat{a}^{\dag}\left[\hat{\sigma}^{i}_{ab},\hat{\sigma}_{ab}^j\right]  +
\left[\left(\hat{\sigma}^{i}_{ab}\right)^{\dag},\hat{\sigma}_{ab}^j\right]
\hat{a}
\right),
\nonumber
\end{eqnarray}
Откуда с учетом 
\begin{eqnarray}
\left[\left(\hat{\sigma}^{i}_{ab}\right)^{\dag},\hat{\sigma}_{ab}^j\right] = 
\ket{a^i}\bra{b^i}\ket{b^j}\bra{a^j} -
\ket{b^j}\bra{a^j}\ket{a^i}\bra{b^i} = 
- \delta_{ij}\left(\hat{\sigma}^{j}_{a} - \hat{\sigma}^{j}_{b}\right)
\nonumber
\end{eqnarray}
получим
\begin{eqnarray}
\frac{i}{\hbar}
\left[\hat{V}_{AF}, \hat{\sigma}_{ab}^j\right] = 
i g 
\Delta\left(t - t_j\right) 
\left(\hat{\sigma}^{j}_{a} -
\hat{\sigma}^{j}_{b}\right)\hat{a}.
\label{eqLaserHaizenbergSigmaAB_pre1_1}
\end{eqnarray}

Для второго члена суммы \eqref{eqLaserHaizenbergSigmaAB_pre1} имеем
\begin{eqnarray}
\frac{i}{\hbar}
\left[\hat{V}_{AR_A}, \hat{\sigma}_{ab}^j\right] = 
\nonumber \\
=
i
\sum_{k,i}
g_k^{(A)}
\hat{c}_k^{\dag}
 \left(
\left[\hat{\sigma}^{i}_{ac},\hat{\sigma}_{ab}^j\right]  
e^{i\left(\omega_k^{(A)} - \omega_{ac}\right)t}
+ 
\left[\hat{\sigma}^{i}_{bc},\hat{\sigma}_{ab}^j\right]  
e^{i\left(\omega_k^{(A)} - \omega_{bc}\right)t}
\right)
+
\nonumber \\
+
 i
\sum_{k,i}
g_k^{(A)}
 \left(
\left[\left(\hat{\sigma}^{i}_{ac}\right)^{\dag},\hat{\sigma}_{ab}^j\right]  
e^{-i\left(\omega_k^{(A)} - \omega_{ac}\right)t}
+ 
\left[\left(\hat{\sigma}^{i}_{bc}\right)^{\dag},\hat{\sigma}_{ab}^j\right]  
e^{-i\left(\omega_k^{(A)} - \omega_{bc}\right)t}
\right)\hat{c}_k,
\nonumber
\end{eqnarray}
которое с помощью коммутационных соотношений
\begin{eqnarray}
\left[\hat{\sigma}^{i}_{ac},\hat{\sigma}_{ab}^j\right] = 0, 
\nonumber \\
\left[\left(\hat{\sigma}^{i}_{ac}\right)^{\dag},\hat{\sigma}_{ab}^j\right] = 
- \delta_{ij}\left(\hat{\sigma}^{j}_{bc}\right)^{\dag},
\nonumber \\
\left[\hat{\sigma}^{i}_{bc},\hat{\sigma}_{ab}^j\right] = \delta_{ij}\hat{\sigma}^{j}_{ac},
\nonumber \\
\left[\left(\hat{\sigma}^{i}_{bc}\right)^{\dag},\hat{\sigma}_{ab}^j\right] = 0
\label{eqLaserHaizenbergTaskKommutator2}
\end{eqnarray}
переписывается в виде
\begin{eqnarray}
\frac{i}{\hbar}
\left[\hat{V}_{AR_A}, \hat{\sigma}_{ab}^j\right] = 
i
\sum_{k}
g_k^{(A)}
\hat{c}_k^{\dag}
\hat{\sigma}^{j}_{ac}  
e^{i\left(\omega_k^{(A)} - \omega_{bc}\right)t} -
\nonumber \\
-
 i
\sum_{k}
g_k^{(A)}
\left(\hat{\sigma}^{j}_{bc}\right)^{\dag}\hat{c}_k
e^{-i\left(\omega_k^{(A)} - \omega_{ac}\right)t},
\label{eqLaserHaizenbergSigmaAB_pre1_2}
\end{eqnarray}
Объединяя вместе\eqref{eqLaserHaizenbergSigmaAB_pre1_1} и 
\eqref{eqLaserHaizenbergSigmaAB_pre1_2} получим 
\begin{eqnarray}
\frac{d \hat{\sigma}_{ab}^j}{d t} = 
i g 
\Delta\left(t - t_j\right) 
\left(\hat{\sigma}^{j}_{a} -
\hat{\sigma}^{j}_{b}\right)\hat{a} +
\nonumber \\
+ i \sum_{k}
g_k^{(A)}
\hat{c}_k^{\dag}
\hat{\sigma}^{j}_{ac}  
e^{i\left(\omega_k^{(A)} - \omega_{bc}\right)t}
-
 i
\sum_{k}
g_k^{(A)}
\left(\hat{\sigma}^{j}_{bc}\right)^{\dag}\hat{c}_k
e^{-i\left(\omega_k^{(A)} - \omega_{ac}\right)t}
\label{eqLaserHaizenbergSigmaAB_pre2}
\end{eqnarray}
Подставив в \eqref{eqLaserHaizenbergSigmaAB_pre2} выражения для
$\hat{c_k}$ \eqref{eqLaserHaizenbergCk} и $\hat{c_k}^{\dag}$
\eqref{eqLaserHaizenbergCkPlus} и пренебрегая быстро осциллирующими
членами имеем:
\begin{eqnarray}
\frac{d \hat{\sigma}_{ab}^j}{d t} = 
i g 
\Delta\left(t - t_j\right) 
\left(\hat{\sigma}^{j}_{a} -
\hat{\sigma}^{j}_{b}\right)\hat{a} -
\nonumber \\
- \frac{\gamma_{b}}{2}
\int_0^{\infty}dt'
\hat{\sigma}^{j}_{cb}\left(t'\right)  
\hat{\sigma}^{j}_{ac}\left(t\right) 
\delta\left(t - t'\right) -
\nonumber \\ 
- \frac{\gamma_{a}}{2}
\int_0^{\infty}dt'
\hat{\sigma}^{j}_{cb}\left(t\right)  
\hat{\sigma}^{j}_{ac}\left(t'\right) 
\delta\left(t - t'\right) + \hat{f}_{ab}^{j},
\label{eqLaserHaizenbergSigmaAB_pre3}
\end{eqnarray}
где
\begin{eqnarray}
\hat{f}_{ab}^{j}\left(t\right) = 
i
\sum_{k}
g_k^{(A)}
\hat{c}_k^{\dag}\left(0\right)
\hat{\sigma}^{j}_{ac}  
e^{i\left(\omega_k^{(A)} - \omega_{bc}\right)t} -
\nonumber \\
-
 i
\sum_{k}
g_k^{(A)}
\left(\hat{\sigma}^{j}_{bc}\right)^{\dag}\hat{c}_k\left(0\right)
e^{-i\left(\omega_k^{(A)} - \omega_{ac}\right)t}.
\label{eqLaserHaizenbergFABJ}
\end{eqnarray}

В дальнейшем мы будем пользоваться предположением о том, что
\begin{equation}
\gamma_{a} =
\gamma_{b} = \gamma.
\label{eqLaserHaizenbergGamma}
\end{equation}
Приняв во внимание
\[
\hat{\sigma}^{j}_{cb}\hat{\sigma}^{j}_{ac} = 
\hat{\sigma}^{j}_{ab},
\]
в результате из \eqref{eqLaserHaizenbergSigmaAB_pre3} получим 
\begin{eqnarray}
\frac{d \hat{\sigma}_{ab}^j}{d t} = 
i g 
\Delta\left(t - t_j\right) 
\left(\hat{\sigma}^{j}_{a} -
\hat{\sigma}^{j}_{b}\right)\hat{a} 
- \gamma \hat{\sigma}^{j}_{cb}\hat{\sigma}^{j}_{ac} 
 + \hat{f}_{ab}^{j} = 
\nonumber \\
= 
i g 
\Delta\left(t - t_j\right) 
\left(\hat{\sigma}^{j}_{a} -
\hat{\sigma}^{j}_{b}\right)\hat{a} 
- \gamma \hat{\sigma}^{j}_{ab} 
 + \hat{f}_{ab}^{j}
\label{eqLaserHaizenbergSigmaABJ}
\end{eqnarray}

Для шумового оператора $\hat{f}_{ab}^{j}$ можно записать следующие
корреляционные функции:
\begin{eqnarray}
\left<\left(\hat{f}_{ab}^{i}\left(t_1\right)\right)^{\dag}\hat{f}_{ab}^{j}\left(t_2\right)\right>
= \delta_{ij}
\frac{\gamma}{2}\left<\hat{\sigma}^{j}_{a}\right>\delta\left(t_1 -
t_2\right),
\nonumber \\
\left<\hat{f}_{ab}^{j}\left(t_1\right)\left(\hat{f}_{ab}^{i}\left(t_2\right)\right)^{\dag}\right>
= \delta_{ij}
\frac{\gamma}{2}\left<\hat{\sigma}^{j}_{b}\right>\delta\left(t_1 -
t_2\right).
\label{eqLaserHaizenbergFABJCorrel}
\end{eqnarray}
Действительно для первого соотношения
\eqref{eqLaserHaizenbergFABJCorrel} предполагая среднее число фононов
в моде резервуара равным $0$ в начальный момент времени:
\begin{equation}
\left<\hat{c}_{k_1}^{\dag}\left(0\right)\hat{c}_{k_2}\left(0\right)\right>
= 0,
\nonumber
\end{equation}
так что
\begin{equation}
\left<\hat{c}_{k_1}\left(0\right)\hat{c}_{k_2}^{\dag}\left(0\right)\right>
= \delta_{k_1,k_2},
\nonumber
\end{equation}
получим
\begin{eqnarray}
\left<\left(\hat{f}_{ab}^{i}\left(t_1\right)\right)^{\dag}\hat{f}_{ab}^{j}\left(t_2\right)\right>
= 
\nonumber \\
=
\sum_{k_1,k_2}
g_{k_1}^{(A)}g_{k_2}^{(A)}
\left<\hat{c}_{k_1}\left(0\right)\hat{c}_{k_2}^{\dag}\left(0\right)\right>
\left<\left(\hat{\sigma}^{i}_{ac}\right)^{\dag}\hat{\sigma}^{j}_{ac}\right>  
e^{- i\left(\omega_{k_1}^{(A)} - \omega_{bc}\right)t_1}
e^{i\left(\omega_{k_2}^{(A)} - \omega_{bc}\right)t_2} +
\nonumber \\
+
\sum_{k_1,k_2}
g_{k_1}^{(A)}g_{k_2}^{(A)}
\left<\hat{c}_{k_1}^{\dag}\left(0\right)\hat{c}_{k_2}\left(0\right)\right>
\left<\hat{\sigma}^{i}_{bc}\left(\hat{\sigma}^{j}_{bc}\right)^{\dag}\right>
e^{i\left(\omega_{k_1}^{(A)} -
  \omega_{ac}\right)t_1}e^{-i\left(\omega_{k_2}^{(A)} -
  \omega_{ac}\right)t_2} = 
\nonumber \\
= 
\sum_{k}
\left(g_{k}^{(A)}\right)^2
\left<\left(\hat{\sigma}^{i}_{ac}\right)^{\dag}\hat{\sigma}^{j}_{ac}\right>  
e^{- i\left(\omega_{k}^{(A)} - \omega_{bc}\right)\left(t_1 -
  t_2\right)} = 
\nonumber \\
=
\delta_{ij}
\sum_{k}
\left(g_{k}^{(A)}\right)^2 
\left<\hat{\sigma}^{j}_{a}\right>  
e^{- i\left(\omega_{k}^{(A)} - \omega_{bc}\right)\left(t_1 -
  t_2\right)} = 
\delta_{ij}
\frac{\gamma}{2}\left<\hat{\sigma}^{j}_{a}\right>\delta\left(t_1 - t_2\right).
\nonumber
\end{eqnarray}


\subsection{Уравнения движения для макро-величин}
От уравнений движения, описывающих поведение отдельного атома
(\ref{eqLaserHaizenbergSigmaAJ}, \ref{eqLaserHaizenbergSigmaBJ},
\ref{eqLaserHaizenbergSigmaABJ}), следует перейти к уравнениям,
описывающим поведение всей атомной подсистемы в целом. Для этого от
операторов $\hat{\sigma}^{j}_{a}$, $\hat{\sigma}^{j}_{b}$ и
$\hat{\sigma}^{j}_{ab}$ мы перейдем к новым операторам которые
определяют совокупное воздействие всех атомов:
\begin{eqnarray}
\hat{N}_{a} = \sum_j \Delta\left(t - t_j\right)\hat{\sigma}^{j}_{a}, 
\nonumber \\
\hat{N}_{b} = \sum_j \Delta\left(t - t_j\right)\hat{\sigma}^{j}_{b}, 
\nonumber \\
\hat{N}_{ab} = \sum_j \Delta\left(t - t_j\right)\hat{\sigma}^{j}_{ab}.
\label{eqLaserHaizenbergSigmaMacroDef}
\end{eqnarray}

Рассмотрим далее $\frac{d \hat{N}_{a}}{d t}$:
\begin{eqnarray}
\frac{d \hat{N}_{a}}{d t} = 
\sum_j \frac{d \Delta\left(t - t_j\right)}{d t}\hat{\sigma}^{j}_{a} +
\sum_j \Delta\left(t - t_j\right)\frac{d \hat{\sigma}^{j}_{a}}{d t} = 
\nonumber \\
= \sum_j \delta\left(t - t_j\right)\hat{\sigma}^{j}_{a} + \sum_j
\Delta\left(t - t_j\right)\frac{d \hat{\sigma}^{j}_{a}}{d t}. 
\label{eqLaserHaizenbergNA_pre1}
\end{eqnarray}
Подставляя далее \eqref{eqLaserHaizenbergSigmaAJ} в
\eqref{eqLaserHaizenbergNA_pre1} и принимая во внимание
\eqref{eqLaserHaizenbergGamma} получим:
\begin{eqnarray}
\frac{d \hat{N}_{a}}{d t} 
= \sum_j \delta\left(t - t_j\right)\hat{\sigma}^{j}_{a} - \gamma
\sum_j \Delta\left(t - t_j\right) \hat{\sigma}_{a}^{j} +
\nonumber \\
+
i g \sum_j
\Delta\left(t - t_j\right) 
 \left(
\hat{a}^{\dag}\hat{\sigma}^{j}_{ab} -
\left(\hat{\sigma}^{j}_{ab}\right)^{\dag}\hat{a}
\right) + \sum_j \Delta\left(t - t_j\right) \hat{f}_{a}^{j} = 
\nonumber \\
= 
\sum_j \delta\left(t - t_j\right)\hat{\sigma}^{j}_{a} - \gamma
\hat{N}_{a} + 
\nonumber \\ 
+ i g  \left(
\hat{a}^{\dag}\hat{N}_{ab} -
\hat{N}_{ab}^{\dag}\hat{a}
\right) + \sum_j \Delta\left(t - t_j\right) \hat{f}_{a}^{j}
\label{eqLaserHaizenbergNA_pre2}
\end{eqnarray}

Рассмотрим далее среднее значение от первого члена \eqref{eqLaserHaizenbergNA_pre2}:
\begin{eqnarray}
\left<\sum_j \delta\left(t -
t_j\right)\hat{\sigma}^{j}_{a}\left(t\right)\right> = 
\left<\sum_j \delta\left(t -
t_j\right)\hat{\sigma}^{j}_{a}\left(t_j\right)\right> =
\nonumber \\
\left<\sum_j \delta\left(t -
t_j\right)\left<\hat{\sigma}^{j}_{a}\left(t_j\right)\right>\right>_T,
\label{eqLaserHaizenbergMiddNAFirst_pre1}
\end{eqnarray}
где $\left<\dots\right>_T$ означает усреднение по моментам инжекции
атомов:
\begin{equation}
\left<\dots\right>_T =
r_a\int_{-\infty}^{\infty}\left(\dots\right)d t_j,
\label{eqLaserHaizenbergMiddNAFirst_pre2}
\end{equation}
где $r_a$ скорость инжекции атомов. 
Для среднего значения оператора $\hat{\sigma}^{j}_{a}$
предполагая, что в момент времени $t_j$ $j$-й атом инжектируется в
возбужденном состоянии, имеем
\begin{equation}
\left<\hat{\sigma}^{j}_{a}\left(t_j\right)\right> = 
\bra{a}\hat{\sigma}^{j}_{a}\ket{a} = 1.
\label{eqLaserHaizenbergMiddNAFirst_pre3}
\end{equation}  
Подставив \eqref{eqLaserHaizenbergMiddNAFirst_pre2} и
\eqref{eqLaserHaizenbergMiddNAFirst_pre3} в
\eqref{eqLaserHaizenbergMiddNAFirst_pre1} получим
\begin{equation}
\left<\sum_j \delta\left(t -
t_j\right)\hat{\sigma}^{j}_{a}\left(t\right)\right> = r_a.
\label{eqLaserHaizenbergMiddNAFirst}
\end{equation}

Далее добавим и вычтем среднее \eqref{eqLaserHaizenbergMiddNAFirst}
из \eqref{eqLaserHaizenbergNA_pre2}, в результате получим:
\begin{eqnarray}
\frac{d \hat{N}_{a}}{d t} 
= r_a  - \gamma
\hat{N}_{a} 
+ i g  \left(
\hat{a}^{\dag}\hat{N}_{ab} -
\hat{N}_{ab}^{\dag}\hat{a}
\right) + \hat{F}_{a},
\label{eqLaserHaizenbergNA}
\end{eqnarray}
где введен новый шумовой оператор $\hat{F}_{a}$, который учитывает не только
квантовые флуктуации, но флуктуации моментов инжекции активных атомов:
\begin{equation}
\hat{F}_{a} = \sum_j \delta\left(t - t_j\right)\hat{\sigma}^{j}_{a} -
r_a + \sum_j \Delta\left(t - t_j\right) \hat{f}_{a}^{j}, 
\label{eqLaserHaizenbergFNA}
\end{equation}
при этом очевидно что среднее значение данного оператора равно 0:
\begin{eqnarray}
\left<\hat{F}_{a}\right> = \left<\sum_j \delta\left(t - t_j\right)\hat{\sigma}^{j}_{a}\right> -
r_a + \left<\sum_j \Delta\left(t - t_j\right)
\hat{f}_{a}^{j}\right> = 0. 
\nonumber
\end{eqnarray}

Из (\ref{eqLaserHaizenbergSigmaBJ},
\ref{eqLaserHaizenbergSigmaABJ} ) можно получить следующие соотношения для
$\hat{N}_b$ и $\hat{N}_{ab}$:
\begin{eqnarray}
\frac{d \hat{N}_b}{d t} = 
- \gamma \hat{N}_{b} -
i g 
 \left(
\hat{a}^{\dag}\hat{N}_{ab} -
\hat{N}_{ab}^{\dag}\hat{a}
\right) + \hat{F}_{b},
\nonumber \\
\frac{d \hat{N}_{ab}}{d t} = 
- \gamma \hat{N}_{ab} 
+ i g 
\left(\hat{N}_{a} -
\hat{N}_{b}\right)\hat{a} 
 + \hat{F}_{ab},
\label{eqLaserHaizenberNB_AB}
\end{eqnarray}
где введены следующие шумовые операторы
\begin{eqnarray}
\hat{F}_{b} = \sum_j\Delta\left(t - t_j\right)\hat{f}^{j}_{b} + \sum_j
\delta\left(t - t_j\right)\hat{\sigma}^{j}_{b}, 
\nonumber \\
\hat{F}_{ab} = \sum_j\Delta\left(t - t_j\right)\hat{f}^{j}_{ab} +
\sum_j \delta\left(t - t_j\right)\hat{\sigma}^{j}_{ab}. 
\label{eqLaserHaizenbergFB_AB}
\end{eqnarray}
С учетом того, что
\[
\left<\hat{\sigma}^{j}_{b}\left(t_j\right)\right> = 
\left<\hat{\sigma}^{j}_{ab}\left(t_j\right)\right> = 0,
\]
эти операторы обладают нулевым средним. 

Для коррелятора 
$\left<\hat{F}_{ab}^{\dag}\left(t_1\right)\hat{F}_{ab}\left(t_2\right)\right>$
из \eqref{eqLaserHaizenbergFABJCorrel} можно получить следующее
соотношение
\begin{eqnarray}
\left<\hat{F}_{ab}^{\dag}\left(t_1\right)\hat{F}_{ab}\left(t_2\right)\right>
= \sum_{ij}\Delta\left(t_1 - t_i\right)\Delta\left(t_2 -
t_j\right)\left<\left(\hat{f}^{i}_{ab}\right)^{\dag}\hat{f}^{j}_{ab}\right>
+
\nonumber \\
+ 
\sum_{ij} \delta\left(t_1 - t_i\right)\delta\left(t_2 - t_j\right)
\left<\left(\hat{\sigma}^{i}_{ab}\right)^{\dag}\hat{\sigma}^{j}_{ab}\right>
= 
\nonumber \\
=
\frac{\gamma}{2}\delta\left(t_1 - t_2\right)\sum_{j}\Delta\left(t_1 -
t_j\right)\left<\hat{\sigma}^{j}_{a}\right> +
\delta\left(t_1 - t_2\right) \sum_{j} \delta\left(t_1 - t_j\right)
\left<\hat{\sigma}^{j}_{a}\right> 
=
\nonumber \\
= 
\left(
\frac{\gamma}{2}\bar{N}_a + r_a
\right)\delta\left(t_1 - t_2\right),
\label{eqLaserHaizenbergFABCorrel_1}
\end{eqnarray}
где через $\bar{N}_a$ обозначено среднее число атомов в возбужденном
состоянии в рассматриваемый момент времени $t_1$
\begin{equation}
\bar{N}_a\left(t_1\right) = \sum_{j} \delta\left(t_1 - t_j\right)
\left<\hat{\sigma}^{j}_{a}\right>
\nonumber 
\end{equation}
Аналогично для коррелятора 
$\left<\hat{F}_{ab}\left(t_1\right)\hat{F}_{ab}^{\dag}\left(t_2\right)\right>$
из \eqref{eqLaserHaizenbergFABJCorrel} можно получить следующее
соотношение
\begin{eqnarray}
\left<\hat{F}_{ab}\left(t_1\right)\hat{F}_{ab}^{\dag}\left(t_2\right)\right>
= \sum_{ij}\Delta\left(t_1 - t_i\right)\Delta\left(t_2 -
t_j\right)\left<\hat{f}^{i}_{ab}\left(\hat{f}^{j}_{ab}\right)^{\dag}\right>
+
\nonumber \\
+ 
\sum_{ij} \delta\left(t_1 - t_i\right)\delta\left(t_2 - t_j\right)
\left<\hat{\sigma}^{j}_{ab}\left(\hat{\sigma}^{i}_{ab}\right)^{\dag}\right>
= 
\nonumber \\
=
\frac{\gamma}{2}\delta\left(t_1 - t_2\right)\sum_{j}\Delta\left(t_1 -
t_j\right)\left<\hat{\sigma}^{j}_{b}\right> +
\delta\left(t_1 - t_2\right) \sum_{j} \delta\left(t_1 - t_j\right)
\left<\hat{\sigma}^{j}_{b}\right> 
=
\nonumber \\
= 
\frac{\gamma}{2}\bar{N}_b\delta\left(t_1 - t_2\right),
\label{eqLaserHaizenbergFABCorrel_2}
\end{eqnarray}
где опять же введено среднее число атомов в невозбужденном состоянии:
\begin{equation}
\bar{N}_b\left(t_1\right) = \sum_{j} \delta\left(t_1 - t_j\right)
\left<\hat{\sigma}^{j}_{b}\right>
\nonumber 
\end{equation}

В дальнейшем будет удобно использовать оператор разности населенностей
$\hat{N}_z = \hat{N}_a - \hat{N}_b$, для которого из
\eqref{eqLaserHaizenbergNA} и 
\eqref{eqLaserHaizenberNB_AB} получим:
\begin{equation}
\frac{d \hat{N}_z}{d t} = r_a
- \gamma \hat{N}_{z} +
2 i g 
 \left(
\hat{a}^{\dag}\hat{N}_{ab} -
\hat{N}_{ab}^{\dag}\hat{a}
\right) + \hat{F}_{z},
\label{eqLaserHaizenberNZ}
\end{equation}
где
\begin{equation}
\hat{F}_{z} = \hat{F}_{a} - \hat{F}_{b}.
\label{eqLaserHaizenberFZ}
\end{equation}

\subsection{Уравнения движения электромагнитного поля}

Так же как и в случае атомной подсистемы рассмотрение начнем с
уравнений которым удовлетворяют операторы резервуара $R_A$:
$\hat{b}_k$. Так как исследуемые операторы входят только в
гамильтониан $\hat{V}_{FR_F}$ то уравнение движения имеет вид
\begin{equation}
\frac{d \hat{b}_k}{d t} = \frac{i}{\hbar}\left[\hat{V}_{FR_F}, \hat{b}_k
\right].
\label{eqLaserHaizenbergBk_pre1}
\end{equation}
Подставляя в \eqref{eqLaserHaizenbergBk_pre1} выражение для
\index{Гамильтониан}
гамильтониана \eqref{eqLaserHaizenbergVFRF} получим
\begin{eqnarray}
\frac{d \hat{b}_k}{d t} = \frac{i}{\hbar}\left[\hat{V}_{FR_F}, \hat{b}_k
\right] = 
\nonumber \\
= \frac{i}{\hbar} \hbar 
\sum_{k',j} 
g_{k'}^{(F)}\left[\hat{b}_{k'}^{\dag},
  \hat{b}_{k'}\right]\hat{a}e^{i\left(\omega_{k'}^{(F)} - \omega\right)t} = 
-i g_{k}^{(F)}\hat{a}e^{i\left(\omega_{k}^{(F)} - \omega\right)t}.
\label{eqLaserHaizenbergBk_pre2}
\end{eqnarray}
Формальное интегрирование \eqref{eqLaserHaizenbergBk_pre2} дает нам
\begin{equation}
\hat{b}_k\left(t\right) = 
\hat{b}_k\left(0\right) 
-i g_{k}^{(F)}\int_0^t d
t'\hat{a}\left(t'\right)e^{i\left(\omega_{k}^{(F)} - \omega\right)t'}.
\label{eqLaserHaizenbergBk}
\end{equation}

Оператор моды электромагнитного поля $\hat{a}$ присутствует в
гамильтонианах $\hat{V}_{AF}$ и $\hat{V}_{FR_F}$, т. о.  с учетом
выражений для этих гамильтонианов \eqref{eqLaserHaizenbergVAF} и 
\eqref{eqLaserHaizenbergVFRF}и принимая во внимание определение
\eqref{eqLaserHaizenbergSigmaMacroDef} имеем 
\begin{eqnarray}
\frac{d \hat{a}}{d t} = \frac{i}{\hbar}
\left[\hat{V}_{AF}, \hat{a}\right] + 
\frac{i}{\hbar}
\left[\hat{V}_{FR_F}, \hat{a}\right] = 
\nonumber \\
= \frac{i}{\hbar}\hbar g
\sum_j
\Delta\left(t - t_j\right) 
\left[\hat{a}^{\dag},\hat{a}\right]\hat{\sigma}^{j}_{ab} +
\nonumber \\
+ 
\frac{i}{\hbar}\hbar
\sum_k
g_k^{(F)}
\left[\hat{a}^{\dag},\hat{a}\right]\hat{b}_k e^{-i\left(\omega_k^{(F)} -
  \omega\right)t} = 
\nonumber \\
= -i g 
\sum_j
\Delta\left(t - t_j\right) \hat{\sigma}^{j}_{ab} -
i \sum_k
g_k^{(F)}\hat{b}_k e^{-i\left(\omega_k^{(F)} -
  \omega\right)t} =
\nonumber \\ 
= -i g \hat{N}_{ab} -
i \sum_k
g_k^{(F)}\hat{b}_k e^{-i\left(\omega_k^{(F)} -
  \omega\right)t}.
\label{eqLaserHaizenbergA_pre1}
\end{eqnarray}

Подставив \eqref{eqLaserHaizenbergBk} в
\eqref{eqLaserHaizenbergA_pre1} получим
\begin{eqnarray}
\frac{d \hat{a}}{d t} = 
-i g \hat{N}_{ab}
- 2 \pi
D\left(\omega\right) \left(g^{(F)}\left(\omega\right)\right)^2
\int_0^td t' \hat{a}\left(t'\right)\delta\left(t - t'\right) - 
\nonumber \\
-
i
\sum_k
g_k^{(F)}\hat{b}_k \left(0\right)e^{-i\left(\omega_k^{(F)} -
  \omega\right)t}.
\nonumber
\end{eqnarray}
Обозначая через 
\[
\gamma^{(F)} = 4 \pi
D\left(\omega\right) \left(g^{(F)}\left(\omega\right)\right)^2 = \frac{\omega}{Q}
\]
и вводя шумовой оператор
\begin{equation}
\hat{F}_F\left(t\right) = - i
\sum_k
g_k^{(F)}\hat{b}_k \left(0\right)e^{-i\left(\omega_k^{(F)} -
  \omega\right)t}
\label{eqLaserHaizenbergFF}
\end{equation}
окончательно получим
\begin{equation}
\frac{d \hat{a}}{d t} = 
- \frac{1}{2}\frac{\omega}{Q}\hat{a}
-i g \hat{N}_{ab} + 
\hat{F}_F.
\label{eqLaserHaizenbergA}
\end{equation}

В силу того, что определение шумового оператора
\eqref{eqLaserHaizenbergFF} совпадает с
определением \eqref{eqPart1Ch2_LanzgevenDefenitionF}, то мы можем
написать корреляционные соотношения в виде
\eqref{eqPart1Ch2_LanzgevenCorrelations} и \eqref{eqPart1Ch2_Lanzgeven_Task1}:
\begin{eqnarray}
\left<\hat{F}_{F}^{\dag}\left(t_1\right)\hat{F}_{F}\left(t_2\right)\right> = 
\frac{\gamma \bar{n}_T}{2} \delta\left(t_1 - t_2\right) = 
\frac{1}{2}\frac{\omega}{Q}\bar{n}_T\delta\left(t_1 - t_2\right),
\nonumber \\
\left<\hat{F}_{F}\left(t_1\right)\hat{F}_{F}^{\dag}\left(t_2\right)\right> = 
\frac{1}{2}\frac{\omega}{Q}\left(\bar{n}_T + 1 \right)\delta\left(t_1 - t_2\right),
\label{eqLaserHaizenbergFFCorrel}
\end{eqnarray}
где $\bar{n}_T$ среднее число фононов в моде резервуара при
температуре $T$.
