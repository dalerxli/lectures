%% -*- coding:utf-8 -*- 
\section{Естественная ширина линии излучения}
Уравнения (\ref{eqLaserHaizenberg17}) пригодны для решения многих
задач, связанных с квантовой природой лазера: определение статистики
лазерных фотонов в различных режимах работы, определение ширины
линии излучения и с некоторыми добавлениями для решения более тонких
вопросов, например исследование вопроса непосредственной генерации
лазером сжатого состояния (подробнее о сжатых состояниях
см. гл. \ref{chSqueezed}). 

\input ./part2/laser2/fig3.tex

Здесь мы рассмотрим только задачу об естественной ширине линии
излучения лазера. Будем рассматривать стационарный режим работы лазера
который достаточно удален от порога генерации. В этом случае задачу
можно сильно упростить, т. к. ширина линии мало зависит от флуктуация
разности населенностей уровней и флуктуаций амплитуды генерируемого
излучения, как мы это установили рассматривая эту задачу ранее методом
матрицы плотности. Основное влияние на ширину линии оказывает
флуктуация (дрейф) фазы. Наглядно это можно представить на плоскости
комплексной амплитуды, как это изображено на
\autoref{figPart2Laser2_3}.

Если бы не было источников шумового воздействия (Ланжевеновских сил)
генерирующее поле имело бы определенную амплитуду и фазу. В
реальности, при их наличии неопределенность амплитуды будет занимать
узкую область вблизи окружности $\left|\alpha\right| = const$, а фаза
будет свободно диффундировать (блуждать) по этой окружности. 

В связи со сказанным, упрощение уравнений сведется к
следующему. Уравнение для населенности заменим усредненным уравнением 
\begin{equation}
r_a - \gamma \bar{N}_z - \frac{4 g^2}{\gamma}\bar{N}_z\bar{n} = 0,
\label{eqLaserHaizenberg19}
\end{equation}
где $\bar{N}_z$ - усредненное значение населенности, $\bar{n} =
\left<\hat{a}^{+}\hat{a}\right>$ - среднее число фотонов в
генерируемой моде. Ноль справа означает, что рассматривается
установившийся режим. 

Из (\ref{eqLaserHaizenberg19}) находим $\bar{N}_z$
\begin{equation}
\bar{N}_z = \frac{r_a}{\gamma\left(1+ \left(\frac{2
    g}{\gamma}\right)^2 \bar{n}\right)} 
\approx
\frac{r_a}{\gamma}
\left(1 - \left(\frac{2 g}{\gamma}\right)^2
\bar{n}\right).
\label{eqLaserHaizenberg20}
\end{equation}
Окончательное значение в (\ref{eqLaserHaizenberg20}) получено в
приближении, что второй член в скобке мал по сравнению с $1$ (лазер не
на пороге). Подставив этот результат в уравнение для поля получим
\begin{eqnarray}
\frac{d}{dt}\hat{a} = -\frac{1}{2}\left(\frac{\omega}{Q}\right)\hat{a}
+ r_a \frac{g^2}{\gamma^2}\hat{a} - 
\nonumber \\
- 4 r_a\frac{g^4}{\gamma^4} \bar{n}\hat{a} + i \hat{F}_{\sum},
\label{eqLaserHaizenberg21}
\end{eqnarray}
где первый член справа соответствует потерям в резонаторе, второй -
усилению активной средой, третий характеризует степень насыщенности
уровней. 

Корреляционная функция поля, которую нам надо определить выражается
следующим равенством
\begin{equation}
I\left(t, 0\right) = \left<\hat{a}^{+}\left(t\right)\hat{a}\left(0\right)\right>,
\nonumber
\end{equation}
где $\hat{a}\left(t\right)$ удовлетворяет уравнению
(\ref{eqLaserHaizenberg21}). 

В нашем приближении, когда мы пренебрегаем амплитудными флуктуациями,
но не пренебрегаем диффузией фазы, оператор $\hat{a}$ можно
представить в виде
\begin{eqnarray}
\hat{a}\left(t\right) = A e^{i\hat{\varphi}\left(t\right)},
\nonumber \\
\hat{a}^{+}\left(t\right) = A e^{-i\hat{\varphi}\left(t\right)},
\label{eqLaserHaizenbergAConstPhi}
\end{eqnarray}
где $A$ среднее значение амплитуды, а $\hat{\varphi}\left(t\right)$ -
оператор фазы. Тогда корреляционная функция может быть представлена в
виде 
\begin{eqnarray}
\left<\hat{a}^{+}\left(t\right)\hat{a}\left(0\right)\right> =
\left|A\right|^2 \left<e^{-i\left(\hat{\varphi}\left(t\right) -
  \hat{\varphi}\left(0\right)\right)}\right> = 
\nonumber \\
= 
\left|A\right|^2 \left<e^{-\frac{1}{2}\left(\hat{\varphi}\left(t\right) -
  \hat{\varphi}\left(0\right)\right)^2}\right>.
\label{eqLaserHaizenbergTaskMiddle}
\end{eqnarray}
Последнее следует из того, что 
\begin{equation}
\left<e^{-i \left(\Delta \varphi\right)}\right> = 
\left<e^{-\frac{1}{2} \left(\Delta \varphi\right)^2}\right>,
\nonumber
\end{equation}
где $\Delta \varphi = \hat{\varphi}\left(t\right) -
\hat{\varphi}\left(0\right)$, и что можно доказать, разложив
экспоненту в ряд и усреднив каждый член ряда, а затем просуммировав
результат.

Имеем
\[
\Delta \varphi = \int_0^t \frac{d \varphi}{d t}dt,
\]
тогда
\[
\left<\left(\Delta \varphi\right)^2\right> = \int_0^t d t_1 \left<\frac{d \varphi}{d t}\right>
\int_0^{t} d t_2 \left<\frac{d \varphi}{d t}\right>,
\]
где $\frac{d \varphi}{d t}$ можно получить из уравнения
(\ref{eqLaserHaizenberg21}) посредством следующих соотношений
\begin{eqnarray}
i \frac{d \varphi}{d t} = \frac{1}{\hat{a}}\frac{d
  \hat{a}}{d t} = 
-\frac{1}{2}\left(\frac{\omega}{Q}\right)
+ r_a \frac{g^2}{\gamma^2} - 
\nonumber \\
- 4 r_a\frac{g^4}{\gamma^4} \bar{n} + i \frac{\hat{F}_{\sum}}{\hat{a}},
\nonumber \\
- i \frac{d \varphi}{d t} = \frac{1}{\hat{a}^{+}}\frac{d
  \hat{a}^{+}}{d t} = 
-\frac{1}{2}\left(\frac{\omega}{Q}\right)
+ r_a \frac{g^2}{\gamma^2} - 
\nonumber \\
- 4 r_a\frac{g^4}{\gamma^4} \bar{n} - i \frac{\hat{F}^{+}_{\sum}}{\hat{a}^{+}},
\label{eqLaserHaizenberg25_pre1}
\end{eqnarray}
т. о. вычитая второе уравнение (\ref{eqLaserHaizenberg25_pre1}) из
первого получим
\begin{equation}
2 \frac{d \varphi}{d t} = 
\frac{\hat{F}_{\sum}}{\hat{a}} + \frac{\hat{F}^{+}_{\sum}}{\hat{a}^{+}}.
\label{eqLaserHaizenberg25}
\end{equation}
Подставив теперь (\ref{eqLaserHaizenbergAConstPhi}) в
(\ref{eqLaserHaizenberg25}) 
имеем:
\begin{equation}
\frac{d \varphi}{d t} = \frac{1}{2 A}
\left\{
e^{i\hat{\varphi}\left(t\right)}\hat{F}^{+}_{\sum}\left(t\right) +
e^{- i\hat{\varphi}\left(t\right)}\hat{F}_{\sum}\left(t\right)
\right\}.
\nonumber
\end{equation}
Остается вычислить  
\(
\left<\left(\Delta \varphi\right)^2\right>.
\)
Это приводит к двойному интегралу
\begin{eqnarray}
\left<\left(\Delta \varphi\right)^2\right> = 
\frac{1}{4 A^2}
\int_0^t d t_1 
\int_0^t d t_2
\left<
\left(
e^{i\hat{\varphi}\left(t_1\right)}\hat{F}^{+}_{\sum}\left(t_1\right) + 
e^{- i\hat{\varphi}\left(t_1\right)}\hat{F}_{\sum}\left(t_1\right)
\right)\right.
\nonumber \\
\left.
\left(
e^{i\hat{\varphi}\left(t_2\right)}\hat{F}^{+}_{\sum}\left(t_2\right) +
e^{- i\hat{\varphi}\left(t_2\right)}\hat{F}_{\sum}\left(t_2\right)
\right)
\right> = 
\nonumber \\
=
\frac{1}{4}\frac{1}{\bar{n}}
\int_0^t d t_1 
\int_0^t d t_2
\left<
\hat{F}^{+}_{\sum}\left(t_1\right)\hat{F}_{\sum}\left(t_2\right)e^{-i\left(
\hat{\varphi}\left(t_1\right) - \hat{\varphi}\left(t_2\right)
\right)} + \mbox{э. с.}\right>.
\nonumber
\end{eqnarray}
Откуда с помощью (\ref{eqLaserHaizenbergFSumCorrel}) можно получить
\begin{eqnarray}
\left<\left(\Delta \varphi\right)^2\right> = 
\frac{1}{4}\frac{1}{\bar{n}}
\left(
\frac{g^2}{2\gamma}\left(\bar{N}_a + \bar{N}_b\right) + \frac{g^2}{\gamma^2}r_a + 
\frac{\omega}{Q}\left(\bar{n}_T + \frac{1}{2}\right)
\right) t.
\label{eqLaserHaizenbergTaskDelta}
\end{eqnarray}

Из уравнений (\ref{eqLaserHaizenbergNA}) и 
(\ref{eqLaserHaizenberNB_AB}) можно записать уравнение для
$\bar{N}_a+\bar{N}_b$: 
\begin{equation}
\frac{d\left(\bar{N}_a+\bar{N}_b\right)}{d t} = r_a - \gamma
\left(\bar{N}_a+\bar{N}_b\right), 
\nonumber
\end{equation}
откуда в установившемся режиме:
\begin{equation}
r_a = \gamma
\left(\bar{N}_a+\bar{N}_b\right).
\nonumber
\end{equation}
Подставив теперь это соотношение в (\ref{eqLaserHaizenbergTaskDelta})
получим: 
\begin{eqnarray}
\left<\left(\Delta \varphi\right)^2\right> = 
\frac{1}{4}\frac{1}{\bar{n}}
\left(
\frac{3 g^2}{2\gamma}\left(\bar{N}_a + \bar{N}_b\right) + 
\frac{\omega}{Q}\left(\bar{n}_T + \frac{1}{2}\right)
\right) t.
\label{eqLaserHaizenbergTaskDelta2}
\end{eqnarray}

Рассмотрим лазер на пороге:
\begin{equation}
A=R_a\approx \frac{\omega}{Q},
\nonumber
\end{equation}
подставляя сюда выражение для $R_a$ (\ref{eqCh2_RaRbDefenition}) в
котором принимаем $\tau \approx \frac{1}{\gamma}$:
\begin{equation}
r_a \frac{g^2}{\gamma^2} \approx \frac{\omega}{Q}.
\nonumber 
\end{equation}
Используя это соотношение из (\ref{eqLaserHaizenberg20}) имеем:
\begin{equation}
\bar{N}_z \approx
\frac{r_a}{\gamma} \approx \frac{\omega}{Q} \frac{\gamma}{g^2},
\nonumber
\end{equation}
откуда
\begin{equation}
\frac{g^2}{\gamma} \approx
\frac{\omega}{Q}\frac{1}{\bar{N}_z}.
\nonumber
\end{equation}
Таким образом (\ref{eqLaserHaizenbergTaskDelta2}) перепишется в виде
\begin{eqnarray}
\left<\left(\Delta \varphi\right)^2\right> = 
\frac{1}{4}\frac{1}{\bar{n}}\frac{\omega}{Q}
\left(
\frac{3}{2}\frac{\bar{N}_a + \bar{N}_b}{\bar{N}_z} + 
\bar{n}_T + \frac{1}{2}
\right) t.
\label{eqLaserHaizenbergTaskDelta3}
\end{eqnarray}
Если теперь в (\ref{eqLaserHaizenbergTaskDelta3}) принять
предположение о полной инверсии: $\bar{N}_b = 0$, а также если
пренебречь $\bar{n}_T$ по сравнению с 1, то получим
\begin{eqnarray}
\left<\left(\Delta \varphi\right)^2\right> \approx
\frac{1}{2}\frac{1}{\bar{n}}\frac{\omega}{Q} t.
\label{eqLaserHaizenbergTaskDelta4}
\end{eqnarray}

Подставив это в (\ref{eqLaserHaizenbergTaskMiddle}) получим
\begin{eqnarray}
\left<\hat{a}^{+}\left(t\right)\hat{a}\left(0\right)\right> =
\left|A\right|^2 e^{-\frac{1}{4}\frac{1}{\bar{n}}\frac{\omega}{Q} t},
\label{eqLaserHaizenbergTaskMiddleFinal}
\end{eqnarray}
которое совпадает с выражением (\ref{eqPart2LaserCorr1}), полученным
нами ранее, если принять пороговый режим генерации: 
\[
A \approx \frac{\omega}{Q}.
\]
Таким образом выражение для естественной ширины линии будет тем же
самым:
\[
D \approx \frac{1}{2}\frac{1}{\bar{n}}\frac{\omega}{Q}.
\]

