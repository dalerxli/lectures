%% -*- coding:utf-8 -*- 
\chapter{Квантовая теория лазера в представлении Гейзенберга}

До сих пор мы рассматривали квантовую теорию лазера, используя
представление Шредингера, в котором от времени зависит матрица
плотности, а операторы от времени не зависят. Как мы видели на примере
задачи о затухании моды резонатора, возможен другой подход,
использующий представление Гейзенберга
\cite{bScullyQuantumOptics2003}, 
когда от времени зависят
операторы, а матрица плотности от времени не зависит. В ряде случаев
такой подход может оказаться более удобным для исследования тонких
вопросов квантовой теории лазеров, таких например, как генерация
лазером поля в сжатом состоянии. 

\input ./part2/laser2/model.tex
\input ./part2/laser2/equation.tex
\input ./part2/laser2/system.tex
\input ./part2/laser2/bandwidth.tex

\section{Упражнения}
\begin{enumerate}
\item Доказать коммутационные соотношения
  (\ref{eqLaserHaizenbergTaskKommutator}) и (\ref{eqLaserHaizenbergTaskKommutator2}).
\item Получить уравнения движения оператора $\hat{\sigma}_b^j$: (\ref{eqLaserHaizenbergSigmaBJ}) и 
(\ref{eqLaserHaizenbergFBJ}).
\item Вывести соотношения (\ref{eqLaserHaizenberNB_AB}) для
  операторов $\hat{N}_b$ и $\hat{N}_{ab}$.
\item Доказать второе соотношение (\ref{eqLaserHaizenbergFABJCorrel}).
\item Доказать (\ref{eqLaserHaizenbergTaskMiddle}).
\item Доказать (\ref{eqLaserHaizenbergTaskDelta}).
\end{enumerate}

%% \begin{thebibliography}{99}
%% \bibitem{bCh1LaserLangevinSkalliZubari} М. О. Скалли,
%%   М. С. Зубайри. Квантовая 
%%   оптика. М. Физматлит, 2003.
%% \bibitem{bCh1LaserLangevinYamamoto} Y.Yamamoto, A.Imamoglu.  Mesoscopic quantum
%%   optics. 1999, USA, J.Wiley \& Son.
%% \bibitem{bCh1LaserLangevinHaken} Г.Хакен. Лазерная светодинамика. М.: Мир,
%%   1988.
%% \end{thebibliography}
