%% -*- coding:utf-8 -*- 
\section{Интерферометр Маха-Цендера. Погрешность фазовых измерений.}
\subsection{Уравнение интерферометра}
\index{Интерферометр Маха-Цендера}
Рассмотрим теперь более подробно работу интерферометра Маха-Цендера,
схема которого изображена на \autoref{figPart2Interfero_4}.

\input ./part2/interfero/fig4.tex

Считаем, что входное $M_1$ и выходное $M_2$ зеркала полупрозрачные
(50:50), т. е. 
\[
t = r = \frac{1}{\sqrt{2}},
\]
а два других зеркала - глухие
\[
t = 0, \, r = 1.
\]
``Матрица рассеяния'' полузеркал имеет вид 
\begin{equation}
\hat{S} = \frac{1}{\sqrt{2}} 
\left(
\begin{array}{cc}
1 & i \\
i & 1 \\
\end{array}
\right).
\nonumber
\end{equation}
Мы здесь воспользовались записью \eqref{eqPart2Interfero4a}. Уравнение
интерферометра имеет вид:
\begin{eqnarray}
\hat{a}_2 = \frac{1}{\sqrt{2}} \left(\hat{a}_0 + i \hat{a}_1\right),
\,
\hat{a}_3 = \frac{1}{\sqrt{2}} \left(i \hat{a}_0 + \hat{a}_1\right),
\nonumber \\
\hat{a}_4 = \frac{1}{\sqrt{2}} \left(i \hat{a}_2 + e^{i \varphi}
\hat{a}_3\right) = 
\frac{1}{2}\left[
i \left(1 + e^{i \varphi}\right)\hat{a}_0 -
\left(1 - e^{i \varphi}\right)\hat{a}_1
\right],
\nonumber \\
\hat{a}_5 = \frac{1}{\sqrt{2}} \left(\hat{a}_2 + i e^{i \varphi}
\hat{a}_3\right) = 
\frac{1}{2}\left[
\left(1 - e^{i \varphi}\right)\hat{a}_0 +
i \left(1 + e^{i \varphi}\right)\hat{a}_1
\right],
\label{eqPart2Interfero11}
\end{eqnarray}
где $\varphi$ - разность оптических длин плеч интерферометра (разность
набега фаз в плечах).

Формулы \eqref{eqPart2Interfero11} связывают входящие поля с
выходящими. Положим, что входное поле нулевого входа находится в
вакуумном состоянии $\left|0\right>$, а входное поле первого входа - в
когерентном состоянии $\left|\alpha\right>$. Таким образом входное
поле находится в двухмодовом состоянии $\left|\psi\right> =
\left|0\right>_0\left|\alpha\right>_1$. Далее можно найти
среднее число фотонов на выходах 4 и 5 в зависимости от
$\varphi$. Результат получится неотличимый от классического.
Действительно
\begin{eqnarray}
  \hat{a}_5^{+}\hat{a}_5 =
  \frac{1}{4}
  \left[
    \left(1 - e^{-i \varphi}\right)\hat{a}_0^{+} -
    i \left(1 + e^{-i \varphi}\right)\hat{a}_1^{+}
    \right]
  \nonumber \\
  \left[
    \left(1 - e^{i \varphi}\right)\hat{a}_0 +
    i \left(1 + e^{i \varphi}\right)\hat{a}_1
    \right] = 
  \nonumber \\
  = \frac{1}{2}
  \left[
    \left(
    1 -  \cos \phi 
    \right)
    \hat{a}_0^{+}\hat{a}_0 +
    \left(
    1 +  \cos \phi 
    \right)
    \hat{a}_1^{+}\hat{a}_1
    \right] -
  \frac{\sin \phi}{2}
  \left[
    \hat{a}_0^{+}\hat{a}_1 +
    \hat{a}_1^{+}\hat{a}_0
    \right]
  \label{eqPart2InterferoA55}
\end{eqnarray}
Таким образом для состояния $\left|\psi\right> =
\left|0\right>_0\left|\alpha\right>_1$ получим на фотодетекторе 5
следующий сигнал
\[
\left<\psi\right|\hat{a}^{+}_5 \hat{a}_5\left|\psi\right> =
\frac{1}{2}\left(1+\cos \phi \right) \left|\alpha\right|^2.
\]

Для фотодетектора 4 аналогичным образом получаем
\begin{eqnarray}
  \hat{a}_4^{+}\hat{a}_4 =
  \frac{1}{4}
  \left[
    - i \left(1 + e^{-i \varphi}\right)\hat{a}_0^{+} -
    \left(1 - e^{-i \varphi}\right)\hat{a}_1^{+}
    \right]
  \nonumber \\
  \left[
    i \left(1 + e^{i \varphi}\right)\hat{a}_0 -
    \left(1 - e^{i \varphi}\right)\hat{a}_1
    \right] = 
  \nonumber \\
  = \frac{1}{2}
  \left[
    \left(
    1 +  \cos \phi 
    \right)
    \hat{a}_0^{+}\hat{a}_0 +
    \left(
    1 -  \cos \phi 
    \right)
    \hat{a}_1^{+}\hat{a}_1
    \right] +
  \frac{\sin \phi}{2}
  \left[
    \hat{a}_0^{+}\hat{a}_1 +
    \hat{a}_1^{+}\hat{a}_0
    \right]
  \label{eqPart2InterferoA44}
\end{eqnarray}
На фотодетекторе 4 имеем
следующий сигнал
\[
\left<\psi\right|\hat{a}^{+}_4 \hat{a}_4\left|\psi\right> =
\frac{1}{2}\left(1-\cos \phi \right) \left|\alpha\right|^2.
\]

\input ./part2/interfero/fig5.tex

Мы рассмотрим более сложную схему балансного детектирования,
изображенную на \autoref{figPart2Interfero_5}. Каждый канал
детектируется своим фотоприемником. Сигналы, идущие с фотоприемников, 
вычитаются и фиксируются. Такой способ регистрации по сути является
\index{гомодин!гомодинный детектор}
синхронным детектором (гомодинным детектором), где гомодином служит
поле когерентного состояния, а сигналом - вакуумные колебания.

Воспользовавшись (\ref{eqPart2InterferoA44},
\ref{eqPart2InterferoA55}) можно вычислить
сигнал на входе балансного детектора, который определяется средним значением
оператора  
\begin{equation}
\hat{R} = 
\hat{a}_5^{+} \hat{a}_5 - 
\hat{a}_4^{+} \hat{a}_4 =
\left(
\hat{a}_1^{+} \hat{a}_1 - 
\hat{a}_0^{+} \hat{a}_0
\right) \cos\,\varphi -
\left(
\hat{a}_0^{+} \hat{a}_1 + 
\hat{a}_1^{+} \hat{a}_0
\right) \sin\,\varphi.
\label{eqPart2Interfero12}
\end{equation}
При выводе \eqref{eqPart2Interfero12} были использованы соотношения
\eqref{eqPart2Interfero11}.

\index{гомодин}
Если мода гомодина (вход 1, мода $\hat{a}_1$) находится в
когерентном состоянии с большой амплитудой $\alpha$, а сигнальная мода
(вход 0, мода $\hat{a}_0$) в вакуумном состоянии, то выходной сигнал
равен
\begin{eqnarray}
\left<\hat{R}\right> = 
\left<\psi\right|\hat{R} \left|\psi\right> = 
\left<0\right|_0\left<\alpha\right|_1
\left(
\hat{a}_1^{+} \hat{a}_1 - 
\hat{a}_0^{+} \hat{a}_0
\right)
\left|0\right>_0\left|\alpha\right>_1
\cos\,\varphi
-
\nonumber \\
-
\left<0\right|_0\left<\alpha\right|_1
\left(
\hat{a}_0^{+} \hat{a}_1 + 
\hat{a}_1^{+} \hat{a}_0
\right) 
\left|0\right>_0\left|\alpha\right>_1
\sin\,\varphi = 
\nonumber \\
= \left|\alpha\right|^2 \cos\,\varphi.
\label{eqPart2Interfero13}
\end{eqnarray}
При выводе \eqref{eqPart2Interfero13} учитывалось, что операторы
первой и нулевой моды действуют только на состояние своей
моды. Поэтому имеем
\[
\hat{a}_0\left|0\right> = 0, \, 
\left<0\right|\hat{a}_0^{+} = 0,
\]
следовательно 
\[
\left<0\right|_0\left<\alpha\right|_1
\left(
\hat{a}_0^{+} \hat{a}_1 + 
\hat{a}_1^{+} \hat{a}_0
\right) 
\left|0\right>_0\left|\alpha\right>_1 = 0,
\]
а
\begin{eqnarray}
\left<\psi\right|\hat{R} \left|\psi\right>= 
\left<0\right|_0\left<\alpha\right|_1
\left(
\hat{a}_1^{+} \hat{a}_1 - 
\hat{a}_0^{+} \hat{a}_0
\right)
\left|0\right>_0\left|\alpha\right>_1
=
\nonumber \\
=
\left<0\right|_0\left<\alpha\right|_1
\left(
\hat{a}_1^{+} \hat{a}_1
\right)
\left|0\right>_0\left|\alpha\right>_1
= \left|\alpha\right|^2 = 
\left<\hat{n}\right>
\nonumber
\end{eqnarray}
\index{гомодин}
среднему числу фотонов в моде гомодина.

Если $\varphi = \frac{\pi}{2}$, то $\left<\hat{R}\right> = 0$ и
выходной сигнал отсутствует. При этом оператор $\hat{R} \ne 0$:
\[
\hat{R} = 
-
\left(
\hat{a}_0^{+} \hat{a}_1 + 
\hat{a}_1^{+} \hat{a}_0
\right).
\]
Хотя среднее этого оператора равно нулю, среднее его квадрата будет
отлично от нуля и будет описывать шумы, которые ограничивают точность
измерения $\varphi$, которая в свою очередь описывает различие
оптических длин плеч интерферометра.

\subsection{Точность измерения интерферометром}

Для начала найдем среднее квадрата шумового члена:
\begin{eqnarray}
\left<\psi\right|\hat{R}\hat{R}^{+}\left|\psi\right> = 
\left<\psi\right|
\left(
\hat{a}_0^{+} \hat{a}_1 + 
\hat{a}_1^{+} \hat{a}_0
\right)
\left(
\hat{a}_0 \hat{a}_1^{+} +
\hat{a}_1 \hat{a}_0^{+}
\right)
\left|\psi\right> = 
\nonumber \\
\left<\psi\right|
\left(
\hat{a}_0^{+} \hat{a}_1 
\hat{a}_0 \hat{a}_1^{+} 
+ 
\hat{a}_0^{+} \hat{a}_1 
\hat{a}_1 \hat{a}_0^{+}
+
\hat{a}_1^{+} \hat{a}_0
\hat{a}_0 \hat{a}_1^{+} 
+
\hat{a}_1^{+} \hat{a}_0
\hat{a}_1 \hat{a}_0^{+}
\right)
\left|\psi\right>.
\label{eqPart2Interfero15}
\end{eqnarray}
Из четырех членов, входящих в \eqref{eqPart2Interfero15} три первых
обращаются при усреднении в 0, а последний будет отличен от нуля.
Например, для первого члена имеем
\begin{eqnarray}
\left<\psi\right|
\hat{a}_0^{+} \hat{a}_1 
\hat{a}_0 \hat{a}_1^{+} 
\left|\psi\right> = 
\left<0\right|_0\left<\alpha\right|_1
\hat{a}_0^{+} \hat{a}_1 
\hat{a}_0 \hat{a}_1^{+} 
\left|0\right>_0\left|\alpha\right>_1 =
\nonumber \\
=
\left<0\right|_0\left<\alpha\right|_1
\hat{a}_0^{+} \hat{a}_0 
\hat{a}_1 \hat{a}_1^{+} 
\left|0\right>_0\left|\alpha\right>_1 =
\left<0\right|_0
\hat{a}_0^{+} \hat{a}_0 
\left|0\right>_0
\left<\alpha\right|_1
\hat{a}_1 \hat{a}_1^{+} 
\left|\alpha\right>_1 =
0.
\nonumber
\end{eqnarray}
Последний член равен
\begin{eqnarray}
\left<\psi\right|
\hat{a}_1^{+} \hat{a}_0
\hat{a}_1 \hat{a}_0^{+}
\left|\psi\right> = 
\left<0\right|_0\left<\alpha\right|_1
\hat{a}_1^{+} \hat{a}_0
\hat{a}_1 \hat{a}_0^{+}
\left|0\right>_0\left|\alpha\right>_1 =
\nonumber \\
=
\left<0\right|_0\left<\alpha\right|_1
\hat{a}_1^{+} \hat{a}_1 
\hat{a}_0 \hat{a}_0^{+}
\left|0\right>_0\left|\alpha\right>_1 =
\left<\alpha\right|_1
\hat{a}_1^{+} \hat{a}_1 
\left|\alpha\right>_1 
\left<0\right|_0
\hat{a}_0 \hat{a}_0^{+}
\left|0\right>_0 =
\nonumber \\
=
\left<\alpha\right|_1
\hat{a}_1^{+} \hat{a}_1 
\left|\alpha\right>_1 
\left<1\right.
\left|1\right>_0 =
\left<\alpha\right|_1
\hat{a}_1^{+} \hat{a}_1 
\left|\alpha\right>_1 
= \left|\alpha\right|^2.
\end{eqnarray}
Таким образом мы получили, что среднеквадратичная величина шумов равна
среднему числу фотонов в моде, находящейся в когерентном состоянии:
\begin{equation}
\bar{\left|R\right|^2} = 
\left<\hat{R}\hat{R}^{+}\right> = 
\left|\alpha\right|^2 =
\bar{n}.
\nonumber
\end{equation}
Дисперсия равна
\begin{equation}
\sqrt{\bar{\left|R\right|^2}} = 
\sqrt{\bar{n}}.
\nonumber
\end{equation}
Сигнал возникает из-за изменения 
\[
\varphi = \frac{\pi}{2} + \Delta \varphi.
\]
Величина его равна
\[
\bar{R} = \bar{n} \cos\left(\frac{\pi}{2} + \Delta \varphi\right) \approx
- \bar{n} \Delta \varphi.
\]
Максимально различимым сигналом, т. е. пороговым сигналом, будем
считать сигнал, равный дисперсии шумов:
\[
\left|\bar{n} \Delta \varphi\right| \ge
\sqrt{\bar{\left|R\right|^2}} = 
\sqrt{\bar{n}},
\]
откуда имеем
\[
\Delta \varphi  \ge \frac{1}{\sqrt{\bar{n}}}.
\]
Равенству соответствует пороговое значение $\Delta \varphi$.

Как мы увидим в дальнейшем (см. гл. \ref{chSqueezed}), точность измерения
можно заметно увеличить, если в сигнальное плечо интерферометра подать
``сжатый вакуум''.  


