%% -*- coding:utf-8 -*- 
\section{Общие соображения}
В квантовой оптике, как и в классической оптике, одним из основных
приборов, применяемых для исследования оптических полей (света),
является интерферометр.

В квантовой физике прибор описывается, исходя из представлений
классической физики, а квантовый объект, взаимодействующий с прибором,
описывают квантовыми уравнениями.

Граница между прибором и квантовым объектом несколько условна и может
быть выбрана в зависимости от задачи, которую предстоит решить. 

В квантовой оптике сам интерферометр описывается классическими
уравнениями, так же как в классической оптике. Световое поле
описывается квантовыми уравнениями для операторов поля, при этом
обычно используют представление Гейзенберга.

Важнейшим элементом любого интерферометра является зеркало. В
соответствии со сказанным выше, мы рассматриваем зеркало как часть
прибора и описываем его классически, введя в рассмотрение коэффициенты
отражения $r$, $r'$ и коэффициенты прохождения $t$, $t'$
см. рис. \ref{figPart2Interfero_1}. На рисунке показаны коэффициенты
$r$, $t$ и $r'$, $t'$ относящиеся к разным сторонам зеркала. Показаны
также операторы уничтожения двух входных мод поля $\hat{a}_0$ и
$\hat{a}_1$ и два оператора выходных мод $\hat{a}_2$ и $\hat{a}_3$

\input ./part2/interfero/fig1.tex

Если зеркало без потерь, то ``матрица рассеяния'' унитарная, как и в
классическом случае:
\[
\hat{S} = \left(
\begin{array}{cc}
t' & r \\
r' & t \\
\end{array}
\right),
\]
где $r$, $t$ и $r'$, $t'$ - коэффициенты прохождения и отражения с
разных сторон зеркала.

Условие унитарности записывается в следующем виде
\begin{eqnarray}
\hat{S}^{-1} = 
\hat{S}^{+} = 
\left(
\begin{array}{cc}
t'^{*} & r'^{*} \\
r^{*} & t^{*} \\
\end{array}
\right), 
\nonumber \\
\hat{S} \hat{S}^{+} = \hat{I}.
\label{eqPart2Interfero2}
\end{eqnarray}
Для того чтобы выполнялось условие унитарности
(\ref{eqPart2Interfero2}), необходимо выполнение закона сохранения
энергии и теоремы взаимности. Это приводит к некоторым условиям,
которым должны удовлетворять коэффициенты $r$, $r'$, $t$ и $t'$:
\begin{eqnarray}
\left|r\right| = \left|r'\right|, \, \left|t\right| = \left|t'\right|,
\nonumber \\
\left|r\right|^2 + \left|t\right|^2 = 1, 
\nonumber \\
r^{*}t' + r't^{*} = 0, \,
r^{*}t + r't'^{*} = 0.
\label{eqPart2Interfero3}
\end{eqnarray}

Плоскость отсчета фазы можно в небольших пределах смещать относительно
плоскости зеркала, тем более что реальное зеркало не является
бесконечно тонким. Плоскость отсчета в ряде случаев можно перемещать
на расстояние порядка длины волны. Выбором плоскости отсчета можно
перевести матрицу рассеяния к более простому виду, при этом условия 
(\ref{eqPart2Interfero3}) автоматически выполняются: 
\begin{equation}
\hat{S} = \left(
\begin{array}{cc}
t & i r \\
i r & t \\
\end{array}
\right)
\label{eqPart2Interfero4a}
\end{equation}
или
\begin{equation}
\hat{S} = \left(
\begin{array}{cc}
t & r \\
-r & t \\
\end{array}
\right).
\label{eqPart2Interfero4b}
\end{equation}
В уравнениях (\ref{eqPart2Interfero4a}) и (\ref{eqPart2Interfero4b})
коэффициенты $r$ и $t$ вещественны и положительны.

В выражении (\ref{eqPart2Interfero4a}) фаза коэффициента отражения
сдвинута на $\frac{\pi}{2}$ относительно фазы коэффициента
прохождения. Во втором варианте (\ref{eqPart2Interfero4b}) все
коэффициенты вещественны, но коэффициенты отражения с разных сторон
имеют разные знаки. Поскольку выбором плоскости отсчета можно общий
случай привести либо к  (\ref{eqPart2Interfero4a}) либо к
(\ref{eqPart2Interfero4b}), то мы в дальнейшем будем пользоваться
одним из этих выражений.



