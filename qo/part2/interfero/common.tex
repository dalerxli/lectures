%% -*- coding:utf-8 -*- 
\section{Общие соображения}
В квантовой оптике, как и в классической оптике, одним из основных
приборов, применяемых для исследования оптических полей (света),
является интерферометр.

В квантовой физике прибор описывается, исходя из представлений
классической физики, а квантовый объект, взаимодействующий с прибором,
описывают квантовыми уравнениями.

Граница между прибором и квантовым объектом несколько условна и может
быть выбрана в зависимости от задачи, которую предстоит решить. 

В квантовой оптике сам интерферометр описывается классическими
уравнениями, так же как в классической оптике. Световое поле
описывается квантовыми уравнениями для операторов поля, при этом
обычно используют представление Гейзенберга.

Важнейшим элементом любого интерферометра является зеркало. В
соответствии со сказанным выше, мы рассматриваем зеркало как часть
прибора и описываем его классически, введя в рассмотрение коэффициенты
отражения $r$, $r'$ и коэффициенты прохождения $t$, $t'$
см. \autoref{figPart2Interfero_1}. На рисунке показаны коэффициенты
$r$, $t$ и $r'$, $t'$ относящиеся к разным сторонам зеркала. Показаны
также операторы уничтожения двух входных мод поля $\hat{a}_0$ и
$\hat{a}_1$ и два оператора выходных мод $\hat{a}_2$ и $\hat{a}_3$

\input ./part2/interfero/fig1.tex

Если зеркало без потерь, то ``матрица рассеяния'' унитарная, как и в
классическом случае:
\[
\hat{S} = 
\begin{bmatrix}
t' & r \\
r' & t \\
\end{bmatrix},
\]
где $r$, $t$ и $r'$, $t'$ - коэффициенты прохождения и отражения с
разных сторон зеркала.

Условие унитарности записывается в следующем виде
\begin{eqnarray}
\hat{S}^{-1} = 
\hat{S}^{+} =
\begin{bmatrix}
t'^\ast & r'^\ast \\
r^\ast & t^\ast \\
\end{bmatrix}, 
\nonumber \\
\hat{S} \hat{S}^{+} =
\begin{bmatrix}
t' & r \\
r' & t \\
\end{bmatrix}
\begin{bmatrix}
t'^\ast & r'^\ast \\
r^\ast & t^\ast \\
\end{bmatrix} =
\nonumber \\
=
\begin{bmatrix}
\left|t'\right|^2 + \left|r\right|^2 & t' r'^\ast + r t^\ast \\
r' t'^\ast + t r^\ast & \left|r'\right|^2 + \left|t\right|^2 \\
\end{bmatrix} = 
\hat{I},
\nonumber \\
\hat{S}^{+} \hat{S} =
\begin{bmatrix}
t'^\ast & r'^\ast \\
r^\ast & t^\ast \\
\end{bmatrix}
\begin{bmatrix}
t' & r \\
r' & t \\
\end{bmatrix}
 =
\nonumber \\
=
\begin{bmatrix}
\left|t'\right|^2 + \left|r'\right|^2 & t'^\ast r + r'^\ast t \\
r^\ast t' + t^\ast r' & \left|r\right|^2 + \left|t\right|^2 \\
\end{bmatrix} = 
\hat{I}
\label{eqPart2Interfero2}
\end{eqnarray}
Для того чтобы выполнялось условие унитарности
\eqref{eqPart2Interfero2}, необходимо выполнение закона сохранения
энергии и теоремы взаимности. Это приводит к некоторым условиям,
которым должны удовлетворять коэффициенты $r$, $r'$, $t$ и $t'$:
\begin{eqnarray}
\left|r\right| = \left|r'\right|, \, \left|t\right| = \left|t'\right|,
\nonumber \\
\left|r\right|^2 + \left|t\right|^2 = 1, 
\nonumber \\
r' t'^\ast + t r^\ast = 0,
\nonumber \\
t' r^\ast + r' t^\ast = 0.
\label{eqPart2Interfero3}
\end{eqnarray}
Стоит отметить, что последнее выражение не является независимым,
действительно, полагая $t \ne 0$ (в противном случае можно рассмотреть
соотношение для коэффициентов отражения)
из соотношения $\left|t\right| = \left|t'\right|$ можно получить
\(
t t^\ast = t' t'^\ast
\)
откуда
\[
t^\ast = \frac{t' t'^\ast}{t}.
\]
Следовательно, привлекая $r' t'^\ast + t r^\ast = 0$, имеем
\begin{eqnarray}
  t' r^\ast + r' t^\ast =
  t' r^\ast + r' \frac{t' t'^\ast}{t} =
  \nonumber \\
  = \frac{t'}{t}\left(t r^\ast + r't'^\ast\right) = 0.
  \nonumber
\end{eqnarray}

Плоскость отсчета фазы можно в небольших пределах смещать относительно
плоскости зеркала, тем более что реальное зеркало не является
бесконечно тонким. Плоскость отсчета в ряде случаев можно перемещать
на расстояние порядка длины волны. Выбором плоскости отсчета можно
перевести матрицу рассеяния к более простому виду, при этом условия 
\eqref{eqPart2Interfero3} автоматически выполняются.
Предполагая $t = t' = T e^{i \theta}$ и $r = r' = R e^{i \phi}$ из
\eqref{eqPart2Interfero3} получим
\[
t r^\ast + r' t^\ast = TR e^{ i \theta} e^{- i \phi} + TR e^{ i \phi}
e^{- i \theta} =
e^{i (\phi-\theta)} TR \left(e^{-2 i (\phi-\theta)} + 1\right) = 0,
\]
т.е. $\phi - \theta = \frac{\pi}{2}$. Полагая $\theta = 0$ и
переназначая $t = T, r = R$ имеем
\begin{equation}
\hat{S} = \left(
\begin{array}{cc}
t & i r \\
i r & t \\
\end{array}
\right).
\label{eqPart2Interfero4a}
\end{equation}

Если принять $t,t',r,r' \in \mathbb{R}$, то очевидно, что необходимые
условия также будут выполнены при $t' = t$ и $r' = -r$:
\begin{equation}
\hat{S} = \left(
\begin{array}{cc}
t & r \\
-r & t \\
\end{array}
\right).
\label{eqPart2Interfero4b}
\end{equation}
В уравнениях \eqref{eqPart2Interfero4a} и \eqref{eqPart2Interfero4b}
коэффициенты $r$ и $t$ вещественны и положительны.

В выражении \eqref{eqPart2Interfero4a} фаза коэффициента отражения
сдвинута на $\frac{\pi}{2}$ относительно фазы коэффициента
прохождения. Во втором варианте \eqref{eqPart2Interfero4b} все
коэффициенты вещественны, но коэффициенты отражения с разных сторон
имеют разные знаки. Поскольку выбором плоскости отсчета можно общий
случай привести либо к  \eqref{eqPart2Interfero4a} либо к
\eqref{eqPart2Interfero4b}, то мы в дальнейшем будем пользоваться
одним из этих выражений.

\begin{example}[Симметричный делитель. Классический случай]
  \input ./part2/interfero/fig1ex.tex
  
  Допустим, что наш делитель симметричный, т. е. $r=r'$ и $t = t'$
  (см. \autoref{figPart2Interfero_1ex})
  Очевидно, что должен выполняться закон сохранения энергии:
  \[
  \left|E_0\right|^2 + \left|E_1\right|^2 =
  \left|E_2\right|^2 + \left|E_3\right|^2, 
  \]
  или же
  \begin{eqnarray}
    \left|E_0\right|^2 + \left|E_1\right|^2 =
    \left|t E_0 + r E_1\right|^2 + \left|r E_0 + t E_1\right|^2 =
    \nonumber \\
    =
    \left(\left|t\right|^2 + \left|r\right|^2\right) 
    \left(\left|E_0\right|^2 + \left|E_1\right|^2\right) +
    \nonumber \\
    + E_0 E_1^\ast \left(t r^\ast + r t^\ast\right)
    + E_1 E_0^\ast \left(t r^\ast + r t^\ast\right)
    \nonumber
  \end{eqnarray}
  Т. е.
  \begin{eqnarray}
    \left|t\right|^2 + \left|r\right|^2 = 1,
    \nonumber \\
    t r^\ast + r t^\ast = 0
    \label{eqPart2InterferoEx1}
  \end{eqnarray}
  Последнее равенство будет справедливо если фаза $t$ и $r$
  различается на $\frac{\pi}{2}$, в частности если принять
  $t = \left|t\right|$, $r = i \left|r\right|$, что соответствует
  выражению \eqref{eqPart2Interfero4a}.

  Кроме того выражения \eqref{eqPart2InterferoEx1} соответствуют
  \eqref{eqPart2Interfero2}, действительно
  \begin{eqnarray}
    \hat{S} \hat{S}^{+} = 
    \left(
    \begin{array}{cc}
      t & r \\
      r & t \\      
    \end{array}
    \right)
    \left(
    \begin{array}{cc}
      t^{*} & r^{*} \\
      r^{*} & t^{*} \\      
    \end{array}
    \right) =
    \nonumber \\
    =
    \left(
    \begin{array}{cc}
      \left|t\right|^2 + \left|r\right|^2 & t r^\ast + r t^\ast \\
      t^\ast r + r^\ast t & \left|t\right|^2 + \left|r\right|^2 \\
    \end{array}
    \right) =
    \left(
    \begin{array}{cc}
      1 & 0 \\
      0 & 1 \\
    \end{array}
    \right) = \hat{I}.
    \nonumber
  \end{eqnarray}
  Таким образом условие унитарности \eqref{eqPart2Interfero2}
  представляет собой закон сохранения энергии.
\end{example}



