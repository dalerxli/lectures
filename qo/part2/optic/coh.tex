%% -*- coding:utf-8 -*- 
\section{Когерентные свойства света}
Когерентные свойства света определяют интерференционные явления в
оптике, которые хорошо описываются классической волновой теорией. В
классической волновой оптике когерентные свойства света описываются
соответствующими функциями когерентности
\cite{bAhmanovNikitinPhysicalOptics2004, bMandel2000, bKilinQuantumOptics2003}.
%\cite{bKaluderSudershan1970}. FIXME!!! check it 
Рассмотрим явление 
интерференции с квантовой точки зрения. Для этого рассмотрим
интерференционный опыт Юнга (рис.\ref{figPart4Ch2_2}). На экране (фотопластинке)
попадание фотона фиксируется темной точкой. Для того чтобы получить
интерференционную картину, сходную с классической, нужно достаточное
время. Тогда отдельные точки сольются в полосы и картина ничем не
будет отличаться от классической (рис.\ref{figPart4Ch2_3}). 

\input ./part2/optic/fig2.tex

\input ./part2/optic/fig3.tex

Операторы поля на экране можно представить в виде суперпозиции полей,
прошедших через отверстия 1, 2:
\begin{equation}
\hat{E}^{(+)}\left(\vec{r}, t\right) = 
\gamma \left[ 
\hat{E}^{(+)}\left(\vec{r}_1, t_1\right) +
\hat{E}^{(+)}\left(\vec{r}_2, t_2\right)
\right],
\label{eqCh4_11}
\end{equation}
где $\vec{r}_1$, $\vec{r}_2$ - координаты отверстий;  $t_1 = t -
\tau_1$,  $t_2 = t - \tau_2$;  $\tau_1$, $\tau_2$ - задержка при
распространении света от отверстий к экрану; коэффициент $\gamma$
характеризует ослабление поля при распространении от отверстий к
экрану. При помощи фотодетектора $D$ исследуем интенсивность света в
различных точках экрана.  

Скорость счета фотонов будет, как мы знаем, пропорциональна
\begin{eqnarray}
W = 
\alpha Sp\left(\hat{\rho}\hat{E}^{(-)}\left(\vec{r},
t\right)\hat{E}^{(+)}\left(\vec{r}, t\right)\right) = 
\nonumber \\
\alpha \left|\gamma\right|^2
Sp \left[\hat{\rho}
\left( 
\hat{E}^{(-)}\left(\vec{r}_1, t_1\right) +
\hat{E}^{(-)}\left(\vec{r}_2, t_2\right)
\right)
\left( 
\hat{E}^{(+)}\left(\vec{r}_1, t_1\right) +
\hat{E}^{(+)}\left(\vec{r}_2, t_2\right)
\right)
\right].
\nonumber
\end{eqnarray}

При почленном перемножении получим:
\begin{eqnarray}
W = 
g Sp \left[\hat{\rho}
\left( 
\hat{E}^{(-)}\left(\vec{r}_1, t_1\right) \hat{E}^{(+)}\left(\vec{r}_1,
t_1\right) +
\hat{E}^{(-)}\left(\vec{r}_2, t_2\right) \hat{E}^{(+)}\left(\vec{r}_2,
t_2\right) +
\right.
\right.
\nonumber \\
\left.
\left.
+
\hat{E}^{(-)}\left(\vec{r}_1, t_1\right) \hat{E}^{(+)}\left(\vec{r}_2,
t_2\right) +
\hat{E}^{(-)}\left(\vec{r}_2, t_2\right) \hat{E}^{(+)}\left(\vec{r}_1,
t_1\right)
\right)
\right],
\label{eqCh4_12}
\end{eqnarray}
где введено следующее обозначение: $g = \alpha \left|\gamma\right|^2$.

Первые два члена дают интенсивность полей, прошедших через первое и
второе отверстия при закрытом другом отверстии. Два последних члена
описывают интерференцию. Так как  
\[
\left(\hat{E}^{(-)}\left(\vec{r}_2, t_2\right) \hat{E}^{(+)}\left(\vec{r}_1,
t_1\right)\right)^{+} = 
\left(\hat{E}^{(-)}\left(\vec{r}_1, t_1\right) \hat{E}^{(+)}\left(\vec{r}_2,
t_2\right)\right)
\]
то  
\(
Sp\left(\hat{\rho}\hat{E}^{(-)}\left(\vec{r}_2, t_2\right) \hat{E}^{(+)}\left(\vec{r}_1,
t_1\right)\right)
\)
и 
\(
Sp\left(\hat{\rho}\hat{E}^{(-)}\left(\vec{r}_1, t_1\right) \hat{E}^{(+)}\left(\vec{r}_2,
t_2\right)\right)
\)
будут комплексно сопряженными величинами. Член 
\[
Sp 
\left[
\hat{\rho}
\left(
\hat{E}^{(-)}\left(\vec{r}_1, t_1\right) \hat{E}^{(+)}\left(\vec{r}_2,
t_2\right) +
\hat{E}^{(-)}\left(\vec{r}_2, t_2\right) \hat{E}^{(+)}\left(\vec{r}_1,
t_1\right)
\right)
\right]
\]
дает
интерференционный осциллирующий член (\autoref{figPart4Ch2_3}). Огибающая
интерференционного члена пропорциональна выражению 
\begin{equation}
\left|
Sp \left[
\hat{\rho}
\hat{E}^{(-)}\left(\vec{r}_1, t_1\right) \hat{E}^{(+)}\left(\vec{r}_2,
t_2\right)
\right]
\right|
\label{eqCh4_13}
\end{equation}
Отсюда следует определение функции когерентности первого порядка
\begin{eqnarray}
G\left(\vec{r}_1, t_1, \vec{r}_2, t_2\right) = 
G_{12}^{1} = 
\nonumber \\
= \frac{\left|
Sp \left[
\hat{\rho}
\hat{E}^{(-)}\left(\vec{r}_1, t_1\right) \hat{E}^{(+)}\left(\vec{r}_2,
t_2\right)
\right]
\right|}
{\sqrt{
Sp \left[
\hat{\rho}
\hat{E}^{(-)}\left(\vec{r}_1, t_1\right) \hat{E}^{(+)}\left(\vec{r}_1,
t_1\right)
\right]
Sp \left[
\hat{\rho}
\hat{E}^{(-)}\left(\vec{r}_2, t_2\right) \hat{E}^{(+)}\left(\vec{r}_2,
t_2\right)
\right]
}}
\label{eqCh4_14}
\end{eqnarray}

Это выражение напоминает классическое определение функции
когерентности, но здесь вместо аналитического сигнала стоят операторы
поля и производится квантовое усреднение при помощи матрицы
плотности. 

Рассмотрим для примера некоторые частные случаи. Начнем с одномодового
поля, находящегося в состояниях  $\left|n\right>$  либо  $\left|\alpha\right>$.  В этом случае имеем: 
\begin{eqnarray}
\hat{\vec{E}}^{(+)}\left(\vec{r}, t\right) = \sqrt{\frac{\hbar \omega_k}{2 \varepsilon_0
    V}} \hat{a}_k \vec{e}_k e^{-i \omega_k t + i \left(\vec{k}\vec{r}
  \right)},
\nonumber \\
\hat{\vec{E}}^{(-)}\left(\vec{r}, t\right) = \sqrt{\frac{\hbar \omega_k}{2 \varepsilon_0
V}} \hat{a}_k^{+} \vec{e}_k^{*} e^{i \omega_k t - i \left(\vec{k}\vec{r}
  \right)} = \left(\hat{\vec{E}}^{(+)}\left(\vec{r}, t\right)\right)^{+}.
\label{eqCh4_15}
\end{eqnarray}

Используя эти выражения для вычисления функции когерентности
(\ref{eqCh4_14}), получим $G_{12}^{1} = 1$,  так как
$\left<n\right|\hat{a}^{+}\hat{a}\left|n\right> = n$   и
$\left|e^{-i x}\right| = 1$ . То же самое получим и для случая
когерентного состояния:  $G_{12}^{1} = 1$,  так как
$\left<\alpha\right|\hat{a}^{+}\hat{a}\left|\alpha\right> =
\left|\alpha\right|^2$. 

Вообще одномодовое поле, возбуждаемое в произвольное чистое состояние,
обладает полной когерентностью первого порядка. Более того,
одномодовое поле, возбуждаемое в произвольное статистическое смешанное
состояние, имеет полную когерентность. В этом случае нужно вычислить
$Sp \left(\hat{\rho} \hat{a}^{+}\hat{a}\right)$,  где  $\hat{\rho} =
\sum_{(m)}\sum_{(l)}\rho_{ml}\left|m\right>\left<n\right|$.  Имеем  
\begin{eqnarray}
Sp \left(\hat{\rho} \hat{a}^{+}\hat{a}\right) = 
\sum_{(n)}\sum_{(m)}\sum_{(l)}\rho_{ml}\left<n\right.\left|m\right>\left<l\right|
\hat{a}^{+}\hat{a}\left|n\right> = 
\nonumber \\
= \sum_{(n)}\sum_{(l)}\rho_{nl}\left<l\right.\left|n\right>n = 
\sum_{(n)}\rho_{nn}n = \bar{n}.
\label{eqCh4_16}
\end{eqnarray}

Аналогичное выражение получаем в знаменателе, следовательно,
$G_{12}^{(1)} = 1$.

Чаще имеют дело с многомодовыми состояниями поля, поэтому рассмотрим
когерентность света в этом случае. Рассмотрим два предельных случая:
свет, находящийся в когерентном состоянии, и полностью хаотический
свет. 

В первом случае вектор состояния равен
\[
\left|\left\{\alpha_{k}\right\}\right> = 
\left|\left\{\alpha_{k_1}\right\}\right>
\left|\left\{\alpha_{k_2}\right\}\right>
\dotsc
\left|\left\{\alpha_{k_s}\right\}\right>
\dots 
\]
Используя свойства когерентных состояний
\[
\hat{E}^{(+)}\left(\vec{r},
t\right)\left|\left\{\alpha_{k}\right\}\right> = 
E\left(\vec{r},
t\right)
\left|\left\{\alpha_{k}\right\}\right>
\]
и
\[
\left<\left\{\alpha_{k}\right\}\right|\hat{E}^{(-)}\left(\vec{r},
t\right) = 
E^{*}\left(\vec{r},
t\right)\left<\left\{\alpha_{k}\right\}\right|,
\]
получим:
\[
\left<\left\{\alpha_{k}\right\}\right|\hat{E}^{(-)}\left(\vec{r}_1,
t_1\right)\hat{E}^{(+)}\left(\vec{r}_2,
t_2\right)\left|\left\{\alpha_{k}\right\}\right> = 
E^{*}\left(\vec{r}_1, t_1\right)E\left(\vec{r}_2,t_2\right) 
\]
где $E$ - собственное значение оператора $\hat{E}^{(+)}$,  являющееся
аналитическим сигналом классического поля. Отсюда имеем 
\begin{equation}
G_{12}^{(1)} = 
\frac{\left|
E^{*}\left(\vec{r}_1, t_1\right)E\left(\vec{r}_2,t_2\right)
\right|}{\sqrt{
E^{*}\left(\vec{r}_1, t_1\right)E\left(\vec{r}_1,t_1\right)
E^{*}\left(\vec{r}_2, t_2\right)E\left(\vec{r}_2,t_2\right)
}} 
= 1
\label{eqCh4_17}
\end{equation}
то есть многомодовое поле, находящееся в когерентном состоянии,
полностью когерентно.  

В случае произвольного состояния многомодового поля имеем только
частичную когерентность. Рассмотрим наиболее часто встречающийся
случай хаотического света (излучение нагретого тела, газового разряда
и т.п.), когда свет излучается множеством независимых источников
(атомов, ионов, молекул). Выражение для матрицы плотности в этом
случае имеет вид (\ref{eqCh1_102}): 
\begin{eqnarray}
\hat{\rho} = 
\sum_{\left\{n_k\right\}} 
 \left|\left\{n_k\right\}\right>\left<\left\{n_k\right\}\right|
\prod_{\left\{n_k\right\}} 
\frac{\bar{n}_k^{n_k}}{\left(1 + \bar{n}_k\right)^{n_k+1}} =
\nonumber \\
= \sum_{\left\{n_k\right\}} \rho_{\left\{n_k\right\},
  \left\{n_k\right\}}
\left|\left\{n_k\right\}\right>\left<\left\{n_k\right\}\right|  
\label{eqCh4_18}
\end{eqnarray}
где
\begin{eqnarray}
\rho_{\left\{n_k\right\},
  \left\{n_k\right\}} =
P_{\left\{n_k\right\}} = 
\prod_k 
\frac{\bar{n}_k^{n_k}}{\left(1 + \bar{n}_k\right)^{n_k+1}} = 
\prod_k P_{n_k},
\nonumber \\
\sum_{\left\{n_k\right\}} \dots = 
\sum_{n_1} \sum_{n_2} \dots \sum_{n_k} \dots. 
\nonumber
\end{eqnarray}
Операторы электрического поля равны
\begin{eqnarray}
\hat{\vec{E}}^{(+)}= \sum_{(k)}\sqrt{\frac{\hbar \omega_k}{2
    \varepsilon_0 V}} \vec{e}_k \hat{a}_k e^{-i \omega_k t + i
    \left(\vec{k} \vec{r}\right)},
\nonumber \\
\hat{\vec{E}}^{(-)}= \sum_{(k)}\sqrt{\frac{\hbar \omega_k}{2
    \varepsilon_0 V}} \vec{e}_k^{*} \hat{a}_k^{+} e^{i \omega_k t - i
    \left(\vec{k} \vec{r}\right)},
\label{eqCh4_19}
\end{eqnarray}
откуда следует:
\begin{eqnarray}
Sp \left(
\hat{\rho}\hat{E}^{(-)}\left(x_1\right)
\hat{E}^{(+)}\left(x_2\right)
\right) = 
\nonumber \\
=\sum_{\left\{n_{k'}\right\}}\sum_{\left\{n_{k}\right\}}
\left<\left\{n_{k'}\right\}\right|\left.\left\{n_{k}\right\}\right>
P_{\left\{n_k\right\}}
\left<\left\{n_{k}\right\}\right|
\hat{E}^{(-)}\left(x_1\right)
\hat{E}^{(+)}\left(x_2\right)
\left|\left\{n_{k'}\right\}\right> =
\nonumber \\
= \sum_{\left\{n_{k}\right\}}
\left<\left\{n_{k}\right\}\right|
\hat{E}^{(-)}\left(x_1\right)
\hat{E}^{(+)}\left(x_2\right)
\left|\left\{n_{k}\right\}\right>
P_{\left\{n_k\right\}}.
\label{eqCh4_20}
\end{eqnarray}
Произведение операторов равно
\begin{equation}
\hat{E}^{(-)}\left(x_1\right)
\hat{E}^{(+)}\left(x_2\right) = \sum_{(k)}\sum_{(k')}
\frac{\hbar \sqrt{\omega_k \omega_{k'}}}{2 \varepsilon_0 V}
\left(\vec{e}_k\vec{e}_{k'}\right)
\hat{a}_k^{+}\hat{a}_{k'}
e^{-i x_2 + i x_1},
\label{eqCh4_21}
\end{equation}
где  $x_1 = \omega_k t_1 - \left(\vec{k}\vec{r}_1\right)$,
$x_2 = \omega_k t_2 - \left(\vec{k}\vec{r}_2\right)$.
Усреднению подлежит оператор $\hat{a}_k^{+}\hat{a}_{k'}$. Поскольку
\[\left|\left\{n_{k}\right\}\right> = 
\left|n_{k_1}\right>
\left|n_{k_2}\right> \dots,
\] 
то 
\[
\left<\left\{n_{k}\right\}\right|
\hat{a}_k^{+}\hat{a}_{k'}
\left|\left\{n_{k}\right\}\right> = 
n_k \delta_{kk'},
\]
и, следовательно, 
\begin{eqnarray}
Sp \left(
\hat{\rho}\hat{E}^{(-)}\left(x_1\right)
\hat{E}^{(+)}\left(x_2\right)
\right) = 
\nonumber \\
=\sum_{k}\sum_{\left\{n_{k}\right\}}
\frac{\hbar \omega_k}{2 \varepsilon_0 V} n_k e^{-i \left(x_2 - x_1
  \right)} 
P_{\left\{n_k\right\}} =
\nonumber \\
= \sum_{k}
\frac{\hbar \omega_k}{2 \varepsilon_0 V}
\sum_{\left\{n_{k}\right\}} n_k
\prod_k P_{n_k} =
\sum_{k} 
\frac{\hbar \omega_k}{2 \varepsilon_0 V}
\bar{n}_k e^{-i \left(x_2 - x_1\right)},
\nonumber
\end{eqnarray}
поскольку
\begin{eqnarray}
\sum_{\left\{n_{k}\right\}} n_k
\prod_kP_{n_k} = 
\sum_{n_1}P_{n_1} 
\sum_{n_2}P_{n_2}
\dots
\sum_{n_k}n_kP_{n_k}
\dots = 
\nonumber \\
=   \sum_{n_k}n_kP_{n_k} = \bar{n}_k,
\nonumber
\end{eqnarray}
т. к.
\[
\sum_{n_k}P_{n_k} = 1.
\]

Отсюда получаем
\begin{equation}
G_{12}^{(1)} = \frac{\left|
\sum_{k} 
\frac{\hbar \omega_k}{2 \varepsilon_0 V}
\bar{n}_k e^{-i \omega_k \left(t_2 - t_1\right) + 
i \left(\vec{k}, \vec{r}_2 - \vec{r}_1\right)}
\right|}{
\sum_{k}
\frac{\hbar \omega_k}{2 \varepsilon_0 V}
\bar{n}_k
}
\label{eqCh4_22}
\end{equation}
Суммирование по $k$ можно заменить интегрированием по
частоте. Использовав выражение для плотности состояний, получим: 
\begin{eqnarray}
\sum_{k}\left(\dots\right) = \frac{L^3}{\left(2 \pi\right)^3}
\int \int \int \left(\dots\right) d k_x d k_y d k_z =
\nonumber \\
= \frac{V}{\left(2 \pi\right)^3}
\int_{\Omega} d \Omega \int \omega^2 \left(\dots\right) d \omega.
\nonumber
\end{eqnarray}
Это приводит нас к выражению
\begin{equation}
G_{12}^{(1)} = \frac{\left|
\int d \Omega \int \omega^3 \bar{n}\left(\omega, \Omega\right) 
e^{-i \omega\left(t_2 - t_1\right) + i \left(\vec{k}, \vec{r}_2 -
  \vec{r}_1 \right)}
d \omega
\right|}{
\int d \Omega \int \omega^3 \bar{n}\left(\omega, \Omega\right) d \omega
}.
\label{eqCh4_23}
\end{equation}
Если линия узкая по сравнению с несущей частотой, то есть 
$\bar{n}\left(\omega, \Omega\right)$ имеет
узкий пик около частоты  $\sim \omega_0$,  функцию $\omega^3$ можно
вынести из-под интеграла и сократить; кроме того, если световой пучок узкий,
то есть область интегрирования по $\Omega$ мала, выражение
(\ref{eqCh4_23}) можно упростить 
\begin{equation}
G_{12}^{(1)} = \frac{\left|
\int \bar{N}\left(\omega\right) 
e^{-i \omega\left(t_2 - t_1\right) + i \left(\vec{k}, \vec{r}_2 -
  \vec{r}_1 \right)}
d \omega
\right|}{
\int \bar{N}\left(\omega\right) d \omega
},
\label{eqCh4_24}
\end{equation}
где $\bar{N}\left(\omega\right) = \int_{\Delta \Omega}
\bar{n}\left(\omega, \Omega\right) d \Omega$.  Например, когда
спектральная линия лоренцовская, имеем:
\[
\bar{N}\left(\omega\right) = \bar{N}_0\frac{\gamma}{\left(\omega_0 -
  \omega\right)^2 + \gamma^2} .
\]  

Рассмотренную задачу можно решить, используя представление когерентных
состояний. Тогда
\begin{equation}
\hat{\rho} = \int \dots \int P\left(\left\{\alpha_k\right\}\right)
\left|\left\{\alpha_k\right\}\right>\left<\left\{\alpha_k\right\}\right|d^2 \left\{\alpha_k\right\},
\nonumber
\end{equation}
где
\begin{equation}
P\left(\left\{\alpha_k\right\}\right) = \prod_k\frac{1}{\pi
  \bar{n}_k}e^{-\frac{\left|\alpha_k\right|^2}{\bar{n}_k}}=
\prod_k P_k\left(\alpha_k\right).
\nonumber
\end{equation}
Имеем
\begin{eqnarray}
Sp \left(
\hat{\rho}\hat{E}^{(-)}\left(x_1\right)
\hat{E}^{(+)}\left(x_2\right)
\right) = 
\nonumber \\
= \int \dots \int
P\left(\left\{\alpha_k\right\}\right)
\left<\left\{\alpha_k\right\}\right|
\hat{E}^{(-)}\left(x_1\right)
\hat{E}^{(+)}\left(x_2\right)
\left|\left\{\alpha_k\right\}\right>
d^2 \left\{\alpha_k\right\}.
\label{eqCh4_coh1_add1}
\end{eqnarray}
Произведение операторов равно
\begin{equation}
\hat{E}^{(-)}\left(x_1\right)
\hat{E}^{(+)}\left(x_2\right) = 
\sum_k \sum_{k'}
\frac{\hbar\sqrt{\omega_{k}\omega_{k'}}}{2 \varepsilon_0 V}
\left(\vec{e}_{k}\vec{e}_{k'}\right)\hat{a}^{+}_{k}\hat{a}_{k'}
e^{-i x_2 + i x_1}.
\nonumber
\end{equation}
Произведение операторов записано у нас в нормальном виде (операторы
поглощения стоят у нас справа от операторов рождения), поэтому
матричный элемент, входящий в (\ref{eqCh4_coh1_add1}), легко записать,
заменив $\hat{a}_{k'} \rightarrow \alpha_{k'}$, $\hat{a}^{+}_{k}
\rightarrow \alpha^{*}_{k}$. Отсюда имеем
\begin{eqnarray}
Sp \left(
\hat{\rho}\hat{E}^{(-)}\left(x_1\right)
\hat{E}^{(+)}\left(x_2\right)
\right) = 
\nonumber \\
= 
\sum_{k \ne k'}
\frac{\hbar\sqrt{\omega_{k}\omega_{k'}}}{2 \varepsilon_0 V}
\left(\vec{e}_{k}\vec{e}_{k'}\right)
e^{-i x_2 + i x_1}
\cdot
\nonumber \\
\cdot
\int 
P\left(\alpha_{k}\right)
\alpha^{*}_{k}
d^2 \alpha_{k} 
\int 
P\left(\alpha_{k'}\right)
\alpha_{k'}
d^2 \alpha_{k'} +
\nonumber \\
+
\sum_k 
\frac{\hbar\omega_{k}}{2 \varepsilon_0 V}
e^{-i x_2 + i x_1}
\int 
\left|
\alpha_{k}
\right|^2
d^2 \alpha_{k}.
\label{eqCh4_coh1_add2}
\end{eqnarray}
При выводе (\ref{eqCh4_coh1_add2}) использовалось следующее
соотношение
\[
\int 
P\left(\alpha_{k}\right)
d^2 \alpha_{k} = 1.
\]
Далее можно воспользоваться следующими соотношениями
\begin{eqnarray}
\int P\left(\alpha_k\right)\alpha_k d^2\alpha_k = 0,
\nonumber \\
\int \left|\alpha_k\right|^2 d^2\alpha_k = \bar{n}_k,
\nonumber
\end{eqnarray}
которые легко доказать, перейдя к полярным координатам:
\begin{equation}
\alpha = \left|\alpha\right|e^{i\theta} = r e^{i\theta}.
\nonumber
\end{equation}
В результате для (\ref{eqCh4_coh1_add2}) получим
\begin{eqnarray}
Sp \left(
\hat{\rho}\hat{E}^{(-)}\left(x_1\right)
\hat{E}^{(+)}\left(x_2\right)
\right) = 
\sum_k 
\frac{\hbar\omega_{k}}{2 \varepsilon_0 V}
e^{-i x_2 + i x_1}
\bar{n}_k.
\nonumber
\end{eqnarray}
Воспользовавшись этими результатами, совпадающими с тем, что было
получено ранее, используя представление чисел заполнения $n$, мы
получим те же окончательные выражения (\ref{eqCh4_23}) и
(\ref{eqCh4_24}). 
