%% -*- coding:utf-8 -*- 
\section{Связь статистики фотонов со статистикой фотоотсчетов}
Начнем с полуклассического рассмотрения, а затем обобщим результаты на
полностью квантовое рассмотрение. Мы видели, что скорость фотоэмиссии
пропорциональна интенсивности светового потока, которая определяется
средним значением оператора $\left<\hat{E}^{(-)}\hat{E}^{(+)}\right>$. 
Поскольку квантово-механическая
интенсивность аналогична в классическом случае усредненной за период 
интенсивности $I\left(t\right)$,  скорость эмиссии электронов можно в
полуклассическом приближении считать пропорциональной
$I\left(t\right)$. Обозначим $P\left(t\right)dt$ - вероятность
появления фотоэлектрона в промежутке времени $t$, $t + dt$.  
$P\left(t\right)dt = \xi I\left(t\right)dt$,  где $\xi$ характеризует
эффективность фотокатода. Обозначим вероятность зафиксировать $m$
фотоотсчетов в интервале времени $t$, $t' + dt'$  через $P_m\left(t,
t' + dt'\right)$. Существуют две возможности получения $m$ отсчетов в
заданном интервале времени, изображенные на рис.\ref{figPart4Ch2_5}: 

\input ./part2/optic/fig5.tex

\begin{enumerate}
\item в интервале $t$, $t'$ произведено $m$ отсчетов, а в интервале
  $dt$ - ни одного
\item в интервале $t$, $t'$ произведено  $m - 1$ отсчетов, а в
интервале $dt'$ - один.
\end{enumerate}
Заметим: интервал настолько мал, что вероятности двух и более отсчетов пренебрежимо малы. Для этих двух случаев можем написать
\begin{eqnarray}
P_m^{(1)}\left(t, t' + dt'\right) = 
P_m\left(t, t'\right)\left(1 - P\left(t'\right)dt'\right),
\nonumber \\
P_m^{(2)}\left(t, t' + dt'\right) = 
P_{m - 1}\left(t, t'\right)P\left(t'\right)dt',
\label{eqCh4_41}
\end{eqnarray}
где $P\left(t\right)dt = \xi I\left(t\right)dt$.  
Полная вероятность будет суммой вероятностей этих двух событий:
\begin{eqnarray}
P_m\left(t, t' + dt'\right) = 
P_m\left(t, t'\right)\left(1 - P\left(t'\right)dt'\right) +
\nonumber \\
+
P_{m - 1}\left(t, t'\right)P\left(t'\right)dt'.
\label{eqCh4_42}
\end{eqnarray}
Отсюда получаем рекуррентную цепочку дифференциальных уравнений
\begin{equation}
\frac{dP_m}{dt'} = \xi I\left(t'\right)\left\{P_{m - 1}\left(t'\right)
- P_m\left(t'\right)\right\},
\label{eqCh4_43}
\end{equation}
которую можно решить, интегрируя последовательно, начиная с $m =
0$. Начальное уравнение имеет вид: 
\begin{equation}
\frac{dP_0}{dt'} = - \xi I\left(t'\right) P_0\left(t'\right),
\label{eqCh4_44}
\end{equation}
при очевидных начальных условиях $P_0\left(t'\right) = 1$ при $t' =
t$.  Решение уравнения при этих начальных условиях равно 
\begin{equation}
P_0\left(t, T\right) = e^{- \xi \int_t^{t + T} I\left(t'\right) d t'}, 
\label{eqCh4_45}
\end{equation}
где $T$ - время счета. Если ввести среднюю за время счета
интенсивность 
\[
\bar{I}\left(t, T\right) = \frac{1}{T}
\int_t^{t + T}I\left(t'\right)dt',
\]
равенство (\ref{eqCh4_45}) можно переписать следующим образом:
\begin{equation}
P_0\left(t, T\right) = e^{- \xi \bar{I} T}
\label{eqCh4_46}
\end{equation}
Остальные вероятности можно последовательно выразить через
$P_0\left(t, T\right)$. Методом индукции легко показать, что  
\begin{equation}
P_m\left(t, T\right) = \frac{\left(\xi \bar{I} T\right)^m}{m!} e^{-
  \xi \bar{I} T} 
\label{eqCh4_47}
\end{equation}
В справедливости этого решения можно убедиться, подставив его в
исходное уравнение (\ref{eqCh4_43}). Вообще же интенсивность
$\bar{I}\left(t, T\right)$
флуктуирует от одного периода счета к другому. Для того чтобы
это учесть, нужно произвести усреднение по ансамблю
измерений. Результатом будет формула Манделя: 
\begin{equation}
P_m\left(T\right) = 
\left<P_m\left(t, T\right)\right> = 
\left<
\frac{\left(\xi \bar{I}\left(t, T\right) T\right)^m}{m!} e^{-
  \xi \bar{I}\left(t, T\right) T} 
\right>.
\label{eqCh4_48}
\end{equation}
Иначе эту формулу можно представить в следующем виде:
\begin{equation}
P_m\left(T\right) = 
\left<P_m\left(t, T\right)\right> = 
\int_0^{\infty}
P\left(\bar{I}\right)
\frac{\left(\xi \bar{I}T\right)^m}{m!} e^{-
  \xi \bar{I} T} 
d \bar{I}.
\label{eqCh4_49}
\end{equation}
где $P\left(\bar{I}\right)$ плотность вероятности для $\bar{I}$. 
