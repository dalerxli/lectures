%% -*- coding:utf-8 -*- 
\section{Когерентность второго порядка}
Когерентность второго порядка может быть введена на основе анализа
различных экспериментов, где измеряется скорость одновременной
регистрации фотонов двумя детекторами. Анализ приводит к
следующему определению степени когерентности второго порядка: 
\begin{eqnarray}
G^{(2)}\left(\vec{r}_1, t_1, \vec{r}_2, t_2\right) = 
G^{(2)}_{12} = 
\nonumber \\
=
\frac{\left<
\hat{E}^{(-)}\left(\vec{r}_2, t_2\right)
\hat{E}^{(-)}\left(\vec{r}_1, t_1\right)
\hat{E}^{(+)}\left(\vec{r}_1, t_1\right)
\hat{E}^{(+)}\left(\vec{r}_2, t_2\right)
\right>}
{\left<
\hat{E}^{(-)}\left(\vec{r}_1, t_1\right)
\hat{E}^{(+)}\left(\vec{r}_1, t_1\right)
\right>
\left<
\hat{E}^{(-)}\left(\vec{r}_2, t_2\right)
\hat{E}^{(+)}\left(\vec{r}_2, t_2\right)
\right>
}.
\label{eqCh4_25}
\end{eqnarray}
Угловые скобки означают квантово-механическое усреднение по ансамблю
при помощи статистического оператора (матрицы плотности) 
\[
\left<\left(\dots\right)\right> = Sp\left\{\hat{\rho}\left(\dots\right)\right\}
\]

Выражение \eqref{eqCh4_25} напоминает классическое выражение,
определяющее когерентность второго порядка, однако в нем на месте
классических полей стоят операторы, а усреднение производится при
помощи матрицы плотности. 

\input ./part2/optic/figadd1.tex

Формулу \eqref{eqCh4_25} можно обосновать следующим
образом. Рассмотрим оператор 
$\hat{E}^{(+)}\left(\vec{r}_1, t_1\right)\hat{E}^{(+)}\left(\vec{r}_2,
t_2\right)$. Он соответствует поглощению одного фотона в точке
$\vec{r}_1$ и в момент времени $t_1 = t$ и второго фотона в точке
$\vec{r}_2$ в момент времени $t_2 = t - \tau$. Реализовать такую
процедуру можно при помощи схемы, изображенной на
\autoref{figPart4Ch2_add1}. Точки  $\vec{r}_1$ и $\vec{r}_2$
определяются положением фотоприемников. Времена $t_1 = t$ и $t_2 = t -
\tau$ определяются совпадением фотоотсчетов. $\tau$ - регулируемая
задержка. Результат с двух детекторов подается на схему совпадений. 

Применим к оператору $\hat{E}^{(+)}\left(\vec{r}_1, t_1\right)\hat{E}^{(+)}\left(\vec{r}_2,
t_2\right)$ процедуру, которую мы применяли при рассмотрении
фотоэффекта. Вероятность регистрации первого фотона во время $t_1 =
t$, а второго во время $t_2 = t - \tau$ в интервале $\Delta t$ равна
\begin{equation}
w\left(\vec{r}_1, \vec{r}_2, t_1, t_2\right) \Delta t= 
\alpha \left| \bra{f}
\hat{E}^{(+)}\left(\vec{r}_1, t_1\right)\hat{E}^{(+)}\left(\vec{r}_2,
t_2\right)
\ket{i}\right|^2 \Delta t,
\nonumber
\end{equation}
где $\ket{i}$ начальное состояние системы, $\ket{f}$
конечное состояние системы, $\alpha$ - величина, зависящая от свойств
фотодетекторов. Тогда скорость счета (число отсчетов в единицу
времени) после суммирования по конечным состояниям определяется
формулой
\begin{equation}
w\left(\vec{r}_1, \vec{r}_2, t_1, t_2\right) = 
\alpha \bra{i}
\hat{E}^{(-)}\left(\vec{r}_2, t_2\right)\hat{E}^{(-)}\left(\vec{r}_1,
t_1\right)
\hat{E}^{(+)}\left(\vec{r}_1, t_1\right)\hat{E}^{(+)}\left(\vec{r}_2,
t_2\right)
\ket{i}.
\nonumber
\end{equation}
Здесь нас интересует только полевая часть, атомная часть выражается
коэффициентом $\alpha$.

Если начальное поле находится в статистически смешанном состоянии, то
усреднение надо проводить при помощи статистического оператора
начального поля. Тогда будем иметь
\begin{equation}
w\left(\vec{r}_1, \vec{r}_2, t_1, t_2\right) = 
\alpha Sp \left\{\hat{\rho}
\hat{E}^{(-)}\left(\vec{r}_2, t_2\right)\hat{E}^{(-)}\left(\vec{r}_1,
t_1\right)
\hat{E}^{(+)}\left(\vec{r}_1, t_1\right)\hat{E}^{(+)}\left(\vec{r}_2,
t_2\right)
\right\}.
\nonumber
\end{equation}
При помощи этого выражения для нормированной степени
когерентности получим формулу \eqref{eqCh4_25}.

Приведем некоторые примеры вычисления
степени когерентности второго порядка. Начнем с простого. Найдем
когерентность второго порядка для одномодового состояния с
определенным числом фотонов. Для этого нужно рассмотреть матричный
элемент  
\[
\bra{n}\hat{a}^{\dag}\hat{a}^{\dag}\hat{a}\hat{a}\ket{n} = n
\left(n - 1\right).
\]
Это выражение будет стоять в числителе. В знаменателе будем иметь 
\[
\left(\bra{n}\hat{a}^{\dag}\hat{a}\ket{n}\right)^2 = 
n^2.
\]

Числовые коэффициенты в числителе и знаменателе сокращаются. Получаем:
\begin{eqnarray}
G^{(2)}_{12} = \frac{n\left(n - 1\right)}{n^2} = \frac{n - 1}{n} 
\mbox{ для } n > 2,
\nonumber \\
G^{(2)}_{12} = 0
\mbox{ для } n = 1,
\nonumber \\
G^{(2)}_{12} 
\mbox{ не определено для } n = 0.
\label{eqCh4_26}
\end{eqnarray}
Стоит отметить очевидный факт в \eqref{eqCh4_26}: если у нас имеется
один фотон то вероятность зарегистрировать 2 фотона равняется 0, что
приводит к $G^{(2)}_{12} = 0$ при $n=1$.

Рассмотрим еще один простой случай: одномодовое поле в когерентном состоянии. Имеем
\begin{eqnarray}
\left<\alpha\right|\hat{a}^{\dag}\hat{a}^{\dag}\hat{a}\hat{a}\left|\alpha\right>
= \left(\alpha^{*}\alpha\right)^2
\mbox{ - в числителе},
\nonumber \\
\left(\left<\alpha\right|\hat{a}^{\dag}\hat{a}\left|\alpha\right>\right)^2
= \left(\alpha^{*}\alpha\right)^2
\mbox{ - в знаменателе}.
\nonumber
\end{eqnarray}

Все вместе дает
\[
G^{(2)}_{12} = 1.
\]

До сих пор мы рассматривали когерентность чистых состояний. Перейдем к
рассмотрению смешанных состояний. Рассмотрим одномодовый хаотический
свет, матрица плотности \index{Матрица плотности} которого равна 
\[
\hat{\rho} = \sum_{n}\frac{\bar{n}^n}{\left(\bar{n} + 1\right)^{n +
    1}} \ket{n}\bra{n} = 
\sum_n\rho_{nn}\ket{n}\bra{n}.
\]
Отсюда получим
\begin{eqnarray}
Sp \left\{\hat{\rho}\hat{a}^{\dag}\hat{a}^{\dag}\hat{a}\hat{a}\right\} = 
\sum_{n}\sum_{m}\bra{n}\ket{m}\bra{m}
\hat{a}^{\dag}\hat{a}^{\dag}\hat{a}\hat{a}
\ket{m} \rho_{mm} = 
\nonumber \\
= \sum_{n}\rho_{nn}\left(n - 1\right)n = \bar{n^2} - \bar{n}.
\label{eqCh4_add1_sp}
\end{eqnarray}

Для хаотического света имеет место соотношение
\begin{equation}
\bar{n^2} = 2
\left(\bar{n}\right)^2 + \bar{n}.
\label{eqCh4_add1_mid_n2}
\end{equation}
Действительно
\begin{eqnarray}
\bar{n^2} = 
\sum_n n^2 \frac{\bar{n}^n}{\left(\bar{n} + 1\right)^{n + 1}} = 
\frac{\bar{n}}{\left(\bar{n} + 1\right)}\sum_n n^2
\frac{\bar{n}^{n-1}}{\left(\bar{n} + 1\right)^{n}} =
\nonumber \\
= 
\frac{\bar{n}}{\left(\bar{n} + 1\right)}\sum_{m = n -1} \left(m +
1\right)^2
\frac{\bar{n}^m}{\left(\bar{n} + 1\right)^{m + 1}} = 
\nonumber \\
= \frac{\bar{n}}{\left(\bar{n} + 1\right)}\sum_m
\left(m^2 + 2 m + 1\right)
\frac{\bar{n}^m}{\left(\bar{n} + 1\right)^{m + 1}} = 
\nonumber \\
=
\frac{\bar{n}}{\left(\bar{n} + 1\right)}\left(\bar{n^2} + 2 \bar{n} +
1\right),
\nonumber
\end{eqnarray}
таким образом получим
\begin{equation}
\bar{n^2}\left(\bar{n} + 1\right) = 
\bar{n}\left(\bar{n^2} + 2 \bar{n} + 1\right)
\nonumber
\end{equation}
откуда и следует искомое выражение \eqref{eqCh4_add1_mid_n2}:
\begin{equation}
\bar{n^2} = 2
\left(\bar{n}\right)^2 + \bar{n}.
\nonumber
\end{equation}

Таким образом из \eqref{eqCh4_add1_sp} и \eqref{eqCh4_add1_mid_n2} имеем
\[
Sp\left\{\dots\right\} = 2\left(\bar{n}\right)^2
\] 
В знаменателе имеем  $\left(\bar{n}\right)^2$. Отсюда получаем, что
для хаотического света 
\begin{equation}
G_{12}^{(2)} = 2,
\label{eqCh4_27}
\end{equation}
из чего следует, что пары фотонов регистрируются чаще, чем это
наблюдается при более упорядоченном свете. Физически это
связано с флуктуациями хаотического света. В связи с этим
иногда говорят о склонности фотонов к группировке. Вообще же
для произвольного одномодового поля когерентность второго
порядка равна 
\begin{equation}
G^{(2)} = \frac{\bar{n^2} - \bar{n}}{\left(\bar{n}\right)^2}.
\label{eqCh4_28}
\end{equation}
Более подробно мы рассмотрим этот вопрос в третьей части данной книги
(см. гл. \ref{chNonClass} Неклассический свет).

На практике чаще имеют дело с многомодовыми полями. Определим степень
когерентности второго порядка в двух предельных случаях:
многомодового когерентного состояния и многомодового
хаотического света. Первый случай рассматривается
просто. Вектор состояния в этом случае можно представить в
виде  
\begin{equation}
\left|\left\{\alpha_k\right\}\right> = 
\left|\left\{\alpha_{k_1}\right\}\right>
\left|\left\{\alpha_{k_2}\right\}\right>
\dotsc
\left|\left\{\alpha_{k_s}\right\}\right>
\dots.
\label{eqCh4_29}
\end{equation}
Используя равенства 
\[
\hat{E}^{(+)}\left(x\right)\left|\left\{\alpha_k\right\}\right> = 
E\left(x\right)\left|\left\{\alpha_k\right\}\right>
\]
и
\[
\left<\left\{\alpha_k\right\}\right|\hat{E}^{(-)}\left(x\right) = 
\left<\left\{\alpha_k\right\}\right|E^{*}\left(x\right),
\]
где $x = \left(t, r\right)$, $E\left(x\right)$ - аналитический сигнал
(частотно-положительная часть) классического поля, получаемого из
оператора $\hat{E}^{(+)}$ заменой $\hat{a}_k \rightarrow
\alpha_k$. Отсюда получим:
\begin{eqnarray}
\left<\left\{\alpha_k\right\}\right|\hat{E}^{(-)}\left(x_2\right)
\hat{E}^{(-)}\left(x_1\right)\hat{E}^{(+)}\left(x_1\right)
\hat{E}^{(+)}\left(x_2\right)\left|\left\{\alpha_k\right\}\right> = 
\nonumber \\
= E^{*}\left(x_2\right)E^{*}\left(x_1\right)
E\left(x_1\right)E\left(x_2\right).
\nonumber
\end{eqnarray}
Таким же образом убеждаемся, что в знаменателе будет стоять величина  
$E^{*}\left(x_1\right)E\left(x_1\right)
E^{*}\left(x_2\right)
E\left(x_2\right)$. Следовательно, в этом случае
\begin{equation}
G_{12}^{(2)} = 1,
\label{eqCh4_30}
\end{equation}

Сложнее рассмотреть случай хаотического многомодового светового
поля. В этом случае статистический оператор будет иметь вид
\eqref{eqCh1_102} 
\begin{eqnarray}
\hat{\rho} = \sum_{\left\{n_k\right\}} P_{\left\{n_k\right\}} \left|\left\{n_k\right\}\right>\left<\left\{n_k\right\}\right| = 
\sum_{\left\{n_k\right\}} 
 \left|\left\{n_k\right\}\right>\left<\left\{n_k\right\}\right|
\prod_{\left\{n_k\right\}} 
\frac{\bar{n}_k^{n_k}}{\left(1 + \bar{n}_k\right)^{n_k+1}} = 
\nonumber \\
= 
\sum_{\left\{n_k\right\}} 
 \left|\left\{n_k\right\}\right>\left<\left\{n_k\right\}\right|
\prod_{\left\{n_k\right\}} P_{\left\{n_k\right\}}.
\label{eqCh4_31}
\end{eqnarray}
При перемножении операторов электрического поля получим четырехкратную
сумму, так как $\hat{E}^{(+)}$ и $\hat{E}^{(-)}$ выражаются в виде
разложений по плоским волнам: 
\begin{eqnarray}
\hat{\vec{E}}^{(+)}\left(\vec{r}, t\right) = \sum_{(k)}
\sqrt{\frac{\hbar \omega_k}{2 \varepsilon_0 V}} \vec{e}_k \hat{a}_k
e^{-i \omega_k t + i \left(\vec{k} \vec{r}\right)},
\nonumber \\
\hat{\vec{E}}^{(-)}\left(\vec{r}, t\right) = \sum_{(k)}
\sqrt{\frac{\hbar \omega_k}{2 \varepsilon_0 V}} \vec{e}_k^{*} \hat{a}_k^{\dag}
e^{i \omega_k t - i \left(\vec{k} \vec{r}\right)}.
\label{eqCh4_32}
\end{eqnarray}
Общий член произведения сумм \eqref{eqCh4_32} будет содержать
произведения операторов вида 
\begin{equation}
\hat{a}^{\dag}_{k^{I}}\hat{a}^{\dag}_{k^{II}}\hat{a}_{k^{III}}\hat{a}_{k^{IV}}.
\label{eqCh4_33}
\end{equation}
Усреднение этого члена при помощи статистического оператора
\eqref{eqCh4_31} приводит к выражению 
\[
Sp\left(\hat{\rho}
\hat{a}^{\dag}_{k^{I}}\hat{a}^{\dag}_{k^{II}}\hat{a}_{k^{III}}\hat{a}_{k^{IV}}
\right) = 
\sum_{\left\{n_k\right\}} P_{\left\{n_k\right\}}
\left<\left\{n_k\right\}\right|
\hat{a}^{\dag}_{k^{I}}\hat{a}^{\dag}_{k^{II}}\hat{a}_{k^{III}}\hat{a}_{k^{IV}}
\left|\left\{n_k\right\}\right>.
\]
Легко увидеть, что члены, в которых все моды разные: 
\[
k^{I} \neq k^{II} \neq k^{III} \neq k^{IV},
\]
будут равны нулю. Отличными от нуля будут только те члены, для которых
выполняются условия:
\[
k^{I} = k^{III} = k_1, \quad k^{II} = k^{IV} = k_2
\]
или
\[
k^{I} = k^{IV} = k_1, \quad k^{II} = k^{III} = k_2
\]
или
\[
k^{I} = k^{II} =  k^{III} = k^{IV} = k.
\]   

Имеем: в первом случае -
\begin{equation}
\sum_{n_{k_1}}\sum_{n_{k_2}} P_{n_{k_1}} P_{n_{k_2}} 
n_{k_1} n_{k_2} = \bar{n}_{k_1} \bar{n}_{k_2},
\label{eqCh4_34}
\end{equation}
во втором -
\begin{equation}
\sum_{n_{k_1}}\sum_{n_{k_2}} P_{n_{k_1}} P_{n_{k_2}} 
n_{k_1} n_{k_2} = \bar{n}_{k_1} \bar{n}_{k_2}.
\label{eqCh4_35}
\end{equation}
Последний случай был нами рассмотрен выше \eqref{eqCh4_28}, что дает
\begin{equation}
\sum_{n_{k}} P_{n_{k}}
\bra{n_k}\hat{a}_k^{\dag}\hat{a}_k^{\dag}\hat{a}_k\hat{a}_k\ket{n_k}
= 2 \left(\bar{n}_k\right)^2.
\label{eqCh4_36}
\end{equation}

Все изложенное выше позволяет написать выражение для функции
когерентности в виде 
\begin{eqnarray}
G_{12}^{(2)} = \frac{\sum_{k_1}\sum_{k_2 \neq k_1} \bar{n}_{k_1}
  \bar{n}_{k_2} \omega_{k_1} \omega_{k_2} e^{i \omega_{k_1} \tau} 
e^{- i \omega_{k_2} \tau} } 
{\left(\sum_{(k)} \bar{n}_k \omega_k\right)^2} + 
\nonumber \\
+
\frac{2 \sum_{k} \bar{n}_{k}^2 \omega_k^2} 
{\left(\sum_{(k)} \bar{n}_k \omega_k\right)^2} + 
\nonumber \\
+
\frac{\sum_{k_1}\sum_{k_2 \neq k_1} \bar{n}_{k_1}
  \bar{n}_{k_2} \omega_{k_1} \omega_{k_2}} 
{\left(\sum_{(k)} \bar{n}_k \omega_k\right)^2}
\label{eqCh4_37}
\end{eqnarray}
где обозначено 
\[
\omega \tau = \omega \left(t_2 - t_1\right) - \left(\vec{k}, \vec{r}_2
- \vec{r}_1\right) = 
\omega \left[
\left(t_2 - t_1\right) - \frac{1}{c}\left(\vec{k}_0, \vec{r}_2
- \vec{r}_1\right)
\right],
\]
если мы имеем узкий световой пучок, в котором все моды
распространяются примерно в одном направлении. 

Половину средней суммы $2 \sum_{k} \bar{n}_{k}^2 \omega_k^2$ можно
объединить с левой суммой, а вторую половину - с правой. Получим: 
\begin{eqnarray}
G_{12}^{(2)} = \frac{\sum_{k_1}\sum_{k_2} \bar{n}_{k_1}
  \bar{n}_{k_2} \omega_{k_1} \omega_{k_2} e^{i \omega_{k_1} \tau} 
e^{- i \omega_{k_2} \tau}} 
{\left(\sum_{(k)} \bar{n}_k \omega_k\right)^2} + 
\nonumber \\
+ \frac{\sum_{k_1}\sum_{k_2} \bar{n}_{k_1}
  \bar{n}_{k_2} \omega_{k_1} \omega_{k_2}} 
{\left(\sum_{(k)} \bar{n}_k \omega_k\right)^2} = 
\nonumber \\
= 
\frac{\sum_{k_1}\bar{n}_{k_1} \omega_{k_1} e^{i \omega_{k_1} \tau}
\sum_{k_2}\bar{n}_{k_2} \omega_{k_2} e^{- i \omega_{k_2} \tau} +
\left(\sum_{(k)} \bar{n}_k \omega_k\right)^2
}
{\left(\sum_{(k)} \bar{n}_k \omega_k\right)^2} = 
\nonumber \\
= \frac{\left|\sum_{(k)}\bar{n}_{k} \omega_{k} e^{i \omega_{k}
  \tau}\right|^2 + \left(\sum_{(k)} \bar{n}_k \omega_k\right)^2}
{\left(\sum_{(k)} \bar{n}_k \omega_k\right)^2} = 
\left(G_{12}^{(1)}\right)^2 + 1,
\label{eqCh4_38}
\end{eqnarray}
где $G_{12}^{(1)}$ - функция когерентности
первого порядка.  

Получен важный результат: когерентность второго порядка для
хаотического света выражается через когерентность первого порядка. На
этом факте основана так называемая интерферометрия
интенсивности. Впервые такое наблюдалось в опытах Хенбери Брауна и
Твисса, схема которых изображена на \autoref{figPart4Ch2_4}.

\input ./part2/optic/fig4.tex

Как видно из \autoref{figPart4Ch2_4}, при помощи 50\% зеркала
световой пучок направляется на два фотоприемника. Задержка
осуществляется перемещением одного из фотодетекторов. Схема совпадений
фиксирует регистрацию двух фотонов с заданной задержкой во времени.

\input ./part2/optic/fig4a.tex

На \autoref{figPart4Ch2_4a} изображена зависимость числа совпадений
от задержки. В классическом случае схема эксперимента остается прежней,
только схема совпадений заменяется коррелятором. Результат
эксперимента и в этом случае имеет вид рисунка \ref{figPart4Ch2_4a}. 

В этих экспериментах измерялась корреляция интенсивностей, то есть 
функция когерентности второго порядка. Используя связь между функциями
корреляций разного порядка, можно вычислить функцию первого порядка, а
затем, используя связь между корреляционной функцией и энергетическим
спектром (теорема Хинчина-Винера), определить спектр излучения. Таким
образом, ``считая фотоны'', можно косвенно производить спектральные
измерения. Этим сейчас широко пользуются. Как будет видно из
дальнейшего, такая процедура позволяет производить измерение спектров
хаотического света с высоким разрешением, дополняя традиционные методы
спектральных измерений (при помощи спектральных приборов типа
дифракционных решеток, призм и т.п.). 

Выражение \eqref{eqCh4_38} можно также получить, используя
представление когерентных состояний. В этом случае
\begin{equation}
\hat{\rho} = \int \dots \int P\left(\left\{\alpha_k\right\}\right)
\left|\left\{\alpha_k\right\}\right>\left<\left\{\alpha_k\right\}\right|d^2 \left\{\alpha_k\right\},
\nonumber
\end{equation}
где
\begin{equation}
P\left(\left\{\alpha_k\right\}\right) = \prod_k\frac{1}{\pi
  \bar{n}_k}e^{-\frac{\left|\alpha_k\right|^2}{\bar{n}_k}}=
\prod_k P_k\left(\alpha_k\right).
\nonumber
\end{equation} 
Усреднению подлежит произведение операторов
\begin{equation}
\hat{a}^{\dag}_{k^{I}}\hat{a}^{\dag}_{k^{II}}\hat{a}_{k^{III}}\hat{a}_{k^{IV}}.
\nonumber
\end{equation}
Это приводит к интегралу
\begin{eqnarray}
Sp \left\{
\hat{\rho}
\hat{a}^{\dag}_{k^{I}}\hat{a}^{\dag}_{k^{II}}\hat{a}_{k^{III}}\hat{a}_{k^{IV}}
\right\} = 
\nonumber \\
= 
\int \dots \int 
d^2 \left\{\alpha_k\right\},
P\left(\left\{\alpha_k\right\}\right)
\left(
\alpha^{*}_{k^{I}}\alpha^{*}_{k^{II}}\alpha_{k^{III}}\alpha_{k^{IV}}
\right).
\nonumber
\end{eqnarray}

По-прежнему отличными от нуля будут члены с парными произведениями,
удовлетворяющими условиям
\[
k^{I} = k^{III} = k_1, \quad k^{II} = k^{IV} = k_2
\]
или
\[
k^{I} = k^{IV} = k_1, \quad k^{II} = k^{III} = k_2
\]
а также
\[
k^{I} = k^{II} =  k^{III} = k^{IV} = k.
\]   
Как легко показать, интегрируя в полярных координатах $\alpha = r
e^{i\theta}$, не равные нулю интегралы будут иметь вид:
\begin{equation}
\int d^2 \alpha_{k_1} P\left(\alpha_{k_1}\right)
\alpha_{k_1}^{*} \alpha_{k_1}
\int d^2 \alpha_{k_2} P\left(\alpha_{k_2}\right)
\alpha_{k_2}^{*} \alpha_{k_2} = 
\bar{n}_{k_1} \bar{n}_{k_2}.
\nonumber
\end{equation}
В третьем случае с учетом 
\[
P\left(\alpha_k\right) = \frac{1}{\pi \bar{n}_k}e^{\frac{r^2}{\bar{n}_k}}
\]
имеем
\begin{eqnarray}
\int d^2\alpha_k P\left(\alpha_k\right)
\left(\alpha_k^{*}\alpha_k\right)^2 = 
\int_0^{\infty}r dr \int_0^{2\pi}P\left(\alpha_k\right) r^4 d \theta = 
\nonumber \\
= \frac{2 \pi}{\pi \bar{n}_k}\int_0^{\infty} r^4 r dr e^{-
  \frac{r^2}{\bar{n}_k}} = 
\bar{n}_k^2 \int_0^{\infty}x^2 d x e^{-x} = 2 \bar{n}_k^2.
\nonumber
\end{eqnarray}
Здесь сделана замена $x = \frac{r^2}{\bar{n}_k}$, $dx = \frac{2 r
  dr}{\bar{n}_k}$.
Таким образом у нас получились те же результаты, которые были получены
ранее. Поэтому окончательный результат \eqref{eqCh4_38} не изменится.
