%% -*- coding:utf-8 -*- 
\section{Эксперименты по счету фотонов. Применение техники счета
  фотонов для спектральных измерений}
Мы видели, что распределение фотоотсчетов зависит от времени счета
$T$. Когда время счета сравнивается со временем корреляции, характер
распределения меняется, на кривой появляется максимум. При дальнейшем
увеличении времени счета характер кривой сохраняется, и при больших
временах счета распределение будет стремиться к распределению
Пуассона. Следовательно, измеряя $P_m\left(T\right)$ при различных
$T$,  можно оценить $\tau_c$ и ширину спектра светового пучка $\sim
1/\tau_c$.  Подобные методы, когда из 
экспериментов по счету фотонов определяют спектральные параметры
света, называют спектроскопией флуктуаций интенсивности. Минимальное
время отсчета определяется разрешением фотодетектора, которое порядка
$10^{-8}\div 10^{-9}$ сек,  что соответствует частоте $10^{8}\div
10^{9}$Гц.  Эта верхняя граница частотных изменений метода. Нижняя
граница определяется максимальным временем счета, которое обычно равно
$1$ сек, что соответствует разрешению  $1$ Гц. 

Таким образом, методом счета фотонов можно исследовать интервал частот
от $1$ Гц до $10^8$ Гц.  Следовательно, этот метод дополняет обычную 
спектроскопию, которая работает в интервале частот от $10^7$ Гц  до
$10^{15}$ Гц.  Более удобно использовать эксперименты по счету фотонов
другого типа. В них измеряется корреляция между числами фотонов $m_1$  и
$m_2$,  то есть $\left<m_1 m_2\right>$,  зарегистрированных в двух
коротких интервалах времени $\Delta t_1 = \Delta t_2 = \Delta t$,
задержанных один относительно другого на время $\tau$.  Оба интервала 
имеют одинаковую длительность $\Delta t$,  меньшую, чем $\tau$,  и
время корреляции $\tau_c$.  В этом случае измеряется когерентность второго порядка,
определяемая по формуле 
\[
G_{12}^{(2)} = \frac{\left<m_1 m_2\right>}{\left(\bar{m}\right)^2} = 
1 + \left(G_{12}^{(1)}\right)^2
\]
где $\bar{m}$ -  среднее число отсчетов за время $\Delta t$. 

Зная $G_{12}^{(2)}$ для хаотического света, можно вычислить функцию
когерентности первого порядка  
\[
G_{12}^{(1)} = \sqrt{G_{12}^{(2)} - 1}
\]
и связанную с ней преобразованием Фурье
форму и ширину спектральной линии светового пучка. Для хаотического света с
лоренцовской спектральной линией имеем: 
\begin{equation}
\left<m_1 m_2\right> = \bar{m}^2\left(e^{-2 \gamma \left|\tau\right|}
+ 1\right),
\label{eqCh4_68}
\end{equation}
а для допплеровской линии -
\begin{equation}
\left<m_1 m_2\right> = \bar{m}^2\left(e^{-\delta^2 \tau^2}
+ 1\right).
\label{eqCh4_69}
\end{equation}
Здесь $\delta$  и $\gamma$ определяют ширину линий в этих
случаях. Измерив эти параметры в эксперименте, можно определить
спектральную ширину линий. 
%Одна из схем, пригодных для таких
%измерений, изображена на рис.7 (схема сдвинутых совпадений).  
%Рис. 7.
