%% -*- coding:utf-8 -*- 
\section{Когерентность высших порядков}
Функции когерентности первого и второго порядков являются частными,
хотя и важнейшими случаями функций когерентности. Функцию
когерентности $n$-ого порядка в квантовом случае определяют
следующим образом 
\begin{eqnarray}
G^{(n)}\left(x_1, x_2, \dots , x_n\right) =  
\nonumber \\
=
\frac{\left<
\hat{E}^{(-)}\left(x_1\right)
\hat{E}^{(-)}\left(x_2\right)
\dots
\hat{E}^{(-)}\left(x_n\right)
\hat{E}^{(+)}\left(x_{n}\right)
\hat{E}^{(+)}\left(x_{n - 1} \right)
\dots
\hat{E}^{(+)}\left(x_{1}\right)
\right>}
{
\left<
\hat{E}^{(-)}\left(x_1\right)
\hat{E}^{(+)}\left(x_1\right)
\right>
\dots
\left<
\hat{E}^{(-)}\left(x_{n}\right)
\hat{E}^{(+)}\left(x_{n}\right)
\right>
}.
\label{eqCh4_39}
\end{eqnarray}
Угловые скобки означают квантово-механическое усреднение при помощи
статистического оператора (матрицы плотности),  $x$ - означает
совокупность переменных $t$, $\vec{r}$. Степень когерентности $n$-ого
порядка определяет скорость счета в эксперименте, где каким-либо
образом регистрируется $n$ фотонов. 
