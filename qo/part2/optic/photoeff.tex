%% -*- coding:utf-8 -*- 
\section{Фотоэффект}
Для регистрации фотонов используется фотоэффект, в котором, как
известно, при поглощении фотона связанный электрон переходит в
свободное состояние и регистрируется соответствующим
образом. Одним из приборов, применяемых для этих целей,
является фотоумножитель, схематически изображенный на
рис.\ref{figPart4Ch2_1}. Принцип действия его хорошо известен. Могут применяться
и полупроводниковые лавинные фотодиоды, снабженные
усилителем. Рассмотрим связь скорости фотоэлектронных
переходов со световым полем. Оператор электрического поля
можно разложить на частотно-положительную и
частотно-отрицательную части (\ref{eqCh1_79}): 
\begin{equation}
\hat{\vec{E}}\left(\vec{r}, t\right) = 
\hat{\vec{E}}^{(+)}\left(\vec{r}, t\right) +
\hat{\vec{E}}^{(-)}\left(\vec{r}, t\right),
\label{eqCh4_1}
\end{equation}
где в частотно-положительную часть входят операторы поглощения, а в
частотно-отрицательную - операторы рождения. 

\input ./part2/optic/fig1.tex

При разложении поля по плоским волнам имеем:
\begin{eqnarray}
\hat{\vec{E}}^{(+)}\left(\vec{r}, t\right) = \sum_{(k)} \sqrt{\frac{\hbar \omega_k}{2 \varepsilon_0
V}} \hat{a}_k \vec{e}_k e^{-i \omega_k t + i \left(\vec{k}\vec{r}
  \right)},
\nonumber \\
\hat{\vec{E}}^{(-)}\left(\vec{r}, t\right) = \sum_{(k)} \sqrt{\frac{\hbar \omega_k}{2 \varepsilon_0
V}} \hat{a}_k^{+} \vec{e}_k^{*} e^{i \omega_k t - i \left(\vec{k}\vec{r}
  \right)}.
\label{eqCh4_2}
\end{eqnarray}

Рассматриваем приближение электродипольного взаимодействия, то есть
считаем, что размеры атома (или другой электронной системы), с
которым взаимодействует свет, много меньше длины
волны. Гамильтониан взаимодействия в этом случае имеет вид: 
\begin{equation}
\hat{\mathcal{H}}_{ED} = - \left(\hat{\vec{p}}\hat{\vec{E}}\right),
\label{eqCh4_3pre}
\end{equation}
где $\hat{\vec{p}}$ - оператор дипольного момента системы. 
Кроме этого мы предполагаем, что учитываются только те процессы,
которые сопровождаются уничтожением фотонов. Таким образом в
(\ref{eqCh4_3pre}) надо оставить только операторы уничтожения
$\hat{a}$, т. е. заменить $\hat{\vec{E}}$ на $\hat{\vec{E}}^{(+)}$:
\begin{equation}
\hat{\mathcal{H}}_{ED} = - \left(\hat{\vec{p}}\hat{\vec{E}}^{(+)}\right).
\label{eqCh4_3}
\end{equation}

Положим, что первоначально атом находится в основном состоянии, а
многомодовое поле содержит $\left\{n_k\right\}$ фотонов, то есть начальным
вектором состояния является 
\begin{equation}
\left|i\right> = \left|\left\{n_k\right\}\right> \left|b\right>
\label{eqCh4_4}
\end{equation}

После взаимодействия система атом-поле будет находится в состоянии
$\left|f\right>$ (детализировать его мы не будем). Вероятность
перехода в единицу времени в первом порядке теории возмущений в
соответствии с золотым правилом Ферми
(см. прил. \ref{addQuantGoldenRuleFermi}) дается 
квадратом модуля матричного элемента перехода: 
\begin{eqnarray}
\left|\left<f\right|\hat{\mathcal{H}}_{ED}\left|i\right>\right|^2 =
\left<f\right|\left(\hat{\vec{p}}\hat{\vec{E}}^{(+)}\right)\left|i\right>
\left<f\right|\left(\hat{\vec{p}}\hat{\vec{E}}^{(+)}\right)\left|i\right>^{*}
= 
\nonumber \\
=
\left<f\right|\left(\hat{\vec{p}}\hat{\vec{E}}^{(+)}\right)\left|i\right>
\left<i\right|\left(\hat{\vec{p}}\hat{\vec{E}}^{(-)}\right)\left|f\right>
=
\left<i\right|\left(\hat{\vec{p}}\hat{\vec{E}}^{(-)}\right)\left|f\right>
\left<f\right|\left(\hat{\vec{p}}\hat{\vec{E}}^{(+)}\right)\left|i\right>
\label{eqCh4_5}
\end{eqnarray}
Конечное состояние не известно. Оно может быть любым, и необходимо
просуммировать (\ref{eqCh4_5}) по всем конечным состояниям (сложить
вероятности). Считаем систему конечных состояний полной, то есть  
$\sum_{(f)} \left|f\right>\left<f\right| = \hat{I}$.
Отсюда имеем: 
\begin{eqnarray}
\sum_{(f)}
\left<i\right|\left(\hat{\vec{p}}\hat{\vec{E}}^{(-)}\right)\left|f\right>
\left<f\right|\left(\hat{\vec{p}}\hat{\vec{E}}^{(+)}\right)\left|i\right>
= 
\left<i\right|\hat{p}_E^2\hat{E}^{(-)}\hat{E}^{(+)}\left|i\right> = 
\nonumber \\
=
\left<\left\{n_k\right\}\right|\hat{E}^{(-)}\hat{E}^{(+)}\left|\left\{n_k\right\}\right>\left<b\right|\hat{p}_E^2\left|b\right>
= 
\alpha \left<\left\{n_k\right\}\right|\hat{E}^{(-)}\hat{E}^{(+)}\left|\left\{n_k\right\}\right>,
\label{eqCh4_6}
\end{eqnarray}
где $\hat{p}_E$ -  проекция дипольного момента на направление поля,
$\alpha$ - некоторая константа взаимодействия, которая от величины
поля не зависит. Обобщим теперь (\ref{eqCh4_6}) на случай смешанного
начального состояния. Допустим, что нам известна вероятность
$P_{\left\{n_k\right\}}$ состояния  $\left|\left\{n_k\right\}\right>$.
Матрица плотности этого состояния может быть представлена в виде 
\begin{equation}
\hat{\rho} = \sum_{\left\{n_k\right\}}P_{\left\{n_k\right\}}
\left|\left\{n_k\right\}\right>\left<\left\{n_k\right\}\right|.
\label{eqCh4_7}
\end{equation}

Для получения результата в случае смешанного состояния нужно усреднить
(\ref{eqCh4_6}) при помощи вероятностей  $P_{\left\{n_k\right\}}$.
Имеем для скорости фотоотсчетов: 
\begin{eqnarray}
W = \alpha \sum_{\left\{n_k\right\}}P_{\left\{n_k\right\}}
\left<\left\{n_k\right\}\right|\hat{E}^{(-)}\hat{E}^{(+)}\left|\left\{n_k\right\}\right> =
\nonumber \\
=
\alpha Sp\left(\hat{\rho}\hat{E}^{(-)}\hat{E}^{(+)}\right)
\label{eqCh4_8}
\end{eqnarray}
- в произвольном представлении, так как $Sp$ не зависит от
представления.

Таким образом, в общем случае статистически смешанного состояния
средняя скорость счета фотонов пропорциональна ожидаемому значению
операторов $\hat{E}^{(-)}\hat{E}^{(+)}$ в начальном состоянии поля,
где $\hat{\rho}$ - статистический оператор начального состояния поля.  

Мы рассмотрели в качестве фотоприемника атом или аналогичную
микросистему. Реальный фотоприемник содержит много атомов. Если
допустить, что приемник содержит $N$ невзаимодействующих между собой
атомов, а размеры его достаточно малы, чтобы можно было с хорошей
точностью считать все атомы находящимися при одинаковых условиях, то в
этом случае можно считать, что вероятность (\ref{eqCh4_8}) можно
просто увеличить в $N$ раз. 

Выясним, какой смысл имеет оператор $\hat{E}^{(-)}\hat{E}^{(+)}$.
Введем оператор 
\begin{equation}
\hat{\vec{J}} = 2 \varepsilon_0
\left(\hat{E}^{(-)}\hat{E}^{(+)}\right) c \vec{k}_0,
\label{eqCh4_9}
\end{equation}
где $\vec{k}_0$ - единичный вектор направления, в котором
распространяется волна; векторный оператор
$\hat{\vec{J}}$ можно рассматривать как 
оператор потока энергии в направлении  $\vec{k}_0$.  Например, для
одномодового поля этот оператор равен 
\begin{equation}
\hat{\vec{J}} = \frac{c \hbar \omega_k}{V} \vec{k}_0 \hat{n}_k,
\label{eqCh4_10}
\end{equation}
$\vec{k}_0$ - единичный вектор направления $\vec{k}$.  Среднее
значение оператора $\hat{\vec{J}}$ в состоянии с определенным числом
фотонов равно  
\[
\left<\hat{\vec{J}}\right> = \frac{c \hbar \omega_k}{V} n_k \vec{k}_0,
\]
В одномодовом смешанном состоянии имеем
\[
\left<\hat{\vec{J}}\right> = \frac{c \hbar \omega_k}{V} \bar{n} \vec{k}_0,   \mbox{ где }   \bar{n} = \sum_{(n_k)}P_{n_k} n_k.
\]

Из приведенных примеров видно, что $\hat{\vec{J}}$ действительно имеет смысл
оператора потока энергии. В одномодовом случае интенсивность не
зависит от времени. В многомодовом случае интенсивность может меняться
во времени и пространстве. Из всего сказанного можно сделать вывод,
что скорость фотоэмиссии пропорциональна среднему потоку фотонов,
т. е. интенсивности света. При полуклассическом рассмотрении этому
соответствует усредненный за период поток энергии. 
