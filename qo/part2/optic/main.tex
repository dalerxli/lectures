%% -*- coding:utf-8 -*- 
\chapter{Оптика фотонов (квантовые явления в оптике)}
\label{chOptic}
Мы будем рассматривать те оптические явления, в которых в той или иной
степени проявляются квантовые свойства света. 

Несмотря на то, что большое число оптических явлений можно
рассматривать с классических позиций, многие явления могут быть до
конца поняты и описаны только в рамках полностью квантового описания. 

Квантовое рассмотрение позволяет более полно понять суть
интерференционных опытов и на этой основе понять связь между
классическим и квантовым описаниями. Кроме того, квантовый подход
позволяет рассматривать эксперименты нового типа, в которых изучается
статистика фотонов в световых пучках и ее связь со спектральными
свойствами света.  

\input ./part2/optic/photoeff.tex
\input ./part2/optic/coh.tex
\input ./part2/optic/coh2.tex
\input ./part2/optic/cohhigh.tex
\input ./part2/optic/calc.tex
\input ./part2/optic/statdep.tex
\input ./part2/optic/destrib.tex
\input ./part2/optic/statdeterm.tex
\input ./part2/optic/quant.tex
\input ./part2/optic/experiment.tex

\section{Упражнения}
\begin{enumerate}
\item Показать, что формула (\ref{eqCh4_47}) действительно дает
  решение задачи. 
\item Получить из формулы (\ref{eqCh4_51}) выражение (\ref{eqCh4_52}).
\item Получить формулу (\ref{eqCh4_66}), используя представление
  когерентных состояний.  
\item Доказать условия ортогональности для полиномов Лагерра
  (\ref{eqCh4_TaskLager1}) и (\ref{eqCh4_TaskLager2}).
\item Разложить $P_T\left(u\right)$ в ряд по полиномам Лагерра:
  (\ref{eqCh4_55})-(\ref{eqCh4_56})
\end{enumerate}

%% \begin{thebibliography}{99}
%% \bibitem{bCh1Optic_MandelVolf} Мандель Л., Вольф Э. Оптическая
%%   когерентность и квантовая оптика. V.: ФМЛ,  2000
%% \bibitem{bCh1Optic_Sudershan} Клаудер Дж., Судершан Э. Основы
%%   квантовой оптики. М.: Мир, 1970. 
%% \bibitem{bCh1Optic_Loudon} Лоудон Р. Квантовая теория света. М.: Мир,
%%   1976. 
%% \bibitem{bCh1Optic_Dvait} Двайт Г.Б. Таблицы интегралов и другие
%%   математические формулы. М.: Наука, 1973. 
%% \end{thebibliography}

