%% -*- coding:utf-8 -*- 
\section{Квантовое описание поляризационных свойств света}
При описании поляризационных свойств однофотонного состояния удобно
\index{вектор Джонса}
использовать вектор Джонса, при этом волновая функция будет иметь вид:
\begin{equation}
\left|\psi\right> = 
\alpha \left|x\right> + 
\beta \left|y\right>,
\label{eqEntangSimpleState}
\end{equation}
где $\left|\alpha\right|^2 + \left|\beta\right|^2 = 1$, а
$\left|x\right> = \left|1\right>_x\otimes\left|0\right>_y$ обозначает
однофотонное состояние, поляризованное по $x$. Аналогично 
$\left|y\right> = \left|0\right>_x\otimes\left|1\right>_y$ -
однофотонное состояние, поляризованное по $y$. 

\index{параметры Стокса!квантовый случай}
Измеряемыми величинами у нас будут параметры Стокса, так что переменные
\eqref{eqEntangStokes} должны быть заменены на операторы. Для этого мы
заменим операторы электрического поля $E_{x,y}$ на операторы
уничтожения $\hat{a}_{x,y}$, в результате получим
\begin{eqnarray}
\hat{S}_0 = \hat{a}_x^{\dag} \hat{a}_x + \hat{a}_y^{+} \hat{a}_y,
\nonumber \\
\hat{S}_1 = \hat{a}_x^{\dag} \hat{a}_x - \hat{a}_y^{+} \hat{a}_y,
\nonumber \\
\hat{S}_2 = \hat{a}_x^{\dag} \hat{a}_y + \hat{a}_x \hat{a}_y^{+},
\nonumber \\
\hat{S}_3 = \frac{\hat{a}_x^{\dag} \hat{a}_y - \hat{a}_x \hat{a}_y^{+}}{i}.
\label{eqEntangStokesOper}
\end{eqnarray}

В дальнейшем нас будут интересовать прежде всего два оператора
$\hat{S}_1$ и $\hat{S}_2$, для которых мы найдем коммутатор,
собственные числа и вектора.

Для коммутатора $\left[\hat{S}_1\hat{S}_2\right]$ имеем:
\begin{eqnarray}
\left[\hat{S}_1\hat{S}_2\right] = \hat{S}_1\hat{S}_2 -
\hat{S}_2\hat{S}_1 = 
\nonumber \\
=
\hat{a}_x^{\dag}\hat{a}_x\hat{a}_x^{+}\hat{a}_y -
\hat{a}_y^{\dag}\hat{a}_y\hat{a}_y\hat{a}_x^{+} + 
\nonumber \\
+ \hat{a}_x^{\dag}\hat{a}_x\hat{a}_x\hat{a}_y^{+} -
\hat{a}_y^{\dag}\hat{a}_y\hat{a}_y^{+}\hat{a}_x -
\nonumber \\
- \hat{a}_x^{\dag}\hat{a}_x^{+}\hat{a}_x\hat{a}_y +
\hat{a}_y\hat{a}_y^{\dag}\hat{a}_y\hat{a}_x^{+} -
\nonumber \\
- \hat{a}_x\hat{a}_x^{\dag}\hat{a}_x\hat{a}_y^{+} +
\hat{a}_y^{\dag}\hat{a}_y^{+}\hat{a}_y\hat{a}_x =
\nonumber \\
= \hat{a}_x^{\dag}\left[\hat{a}_x\hat{a}_x^{+}\right]\hat{a}_y -
\left[\hat{a}_x\hat{a}_x^{\dag}\right]\hat{a}_x\hat{a}_y^{+} +
\nonumber \\
+\left[\hat{a}_y\hat{a}_y^{\dag}\right]\hat{a}_y\hat{a}_x^{+} -
\hat{a}_y^{\dag}\left[\hat{a}_y\hat{a}_y^{+}\right]\hat{a}_x =
\nonumber \\
= \hat{a}_x^{\dag}\hat{a}_y + \hat{a}_y\hat{a}_x^{+} -
\hat{a}_x\hat{a}_y^{\dag} - \hat{a}_y^{+}\hat{a}_x = 
\nonumber \\
= 2 \left(\hat{a}_x^{\dag}\hat{a}_y - \hat{a}_x\hat{a}_y^{+}\right) = 2 i
\hat{S}_3 \ne 0.
\label{eqEntangStokesOperS12Comm}
\end{eqnarray}
При выводе \eqref{eqEntangStokesOperS12Comm} мы воспользовались
коммутационными соотношениями для операторов рождения и уничтожения:
\begin{equation}
\left[\hat{a}_x \hat{a}^{\dag}_x\right] = \left[\hat{a}_y \hat{a}^{+}_y\right] = 1,
\nonumber
\end{equation}
а также тем фактом, что операторы действующие на разные компоненты
поляризации $x$ и $y$, коммутируют между собой.

Для получения собственных чисел и векторов операторов $\hat{S}_1$ и
$\hat{S}_2$ удобно представить их в матричной форме. Для этого будем
использовать базис, образованный векторами $\left|x\right>$ и
$\left|y\right>$:
\begin{eqnarray}
\left|x\right> = \left(
\begin{array}{c}
1 \\
0
\end{array}
\right),
\nonumber \\
\left|y\right> = \left(
\begin{array}{c}
0 \\
1
\end{array}
\right),
\nonumber
\end{eqnarray}

Для оператора $\hat{S}_1$ имеем:
\begin{eqnarray}
\hat{S}_1 \left|x\right> = \hat{a}_x^{\dag} \hat{a}_x
\left|1\right>_x\otimes\left|0\right>_y - \hat{a}_y^{\dag}
\hat{a}_y\left|1\right>_x\otimes\left|0\right>_y =
\nonumber \\
= 
\hat{a}_x^{\dag} \hat{a}_x
\left|1\right>_x\otimes\left|0\right>_y =
\left|1\right>_x\otimes\left|0\right>_y = \left|x\right>,
\nonumber \\
\hat{S}_1 \left|y\right> = \hat{a}_x^{\dag} \hat{a}_x
\left|0\right>_x\otimes\left|1\right>_y - \hat{a}_y^{\dag}
\hat{a}_y\left|0\right>_x\otimes\left|1\right>_y =
\nonumber \\
=
-\hat{a}_y^{\dag}
\hat{a}_y\left|0\right>_x\otimes\left|1\right>_y
=-\left|0\right>_x\otimes\left|1\right>_y = -\left|y\right>,
\label{eqEntangS1MatrixPre}
\end{eqnarray}
откуда получаем следующее матричное представление
\begin{equation}
\hat{S}_1 = 
\left(
\begin{array}{cc}
1 & 0 \\
0 & -1 
\end{array}
\right).
\label{eqEntangS1Matrix}
\end{equation}
Из \eqref{eqEntangS1Matrix} можно написать уравнение для собственных
чисел:
\[
\left(1-s\right)\left(1 + s\right) = 0,
\]
из которого можно найти два собственных значения $s_1 = 1$ и
$s_2 = -1$. Как нетрудно проверить, собственным вектором для $s_1 = 1$
будет $\left|s_1\right> = \left|x\right>$. Действительно, из \eqref{eqEntangS1MatrixPre}
имеем:
\begin{equation}
\hat{S}_1 \left|s_1\right> = \hat{S}_1 \left|x\right> = 1 \cdot \left|s_1\right>.
\label{eq:part2:pol:stocks_s1_1}
\end{equation}
Для второго собственного числа - собственным вектором будет
$\left|s_2\right> = \left|y\right>$:
\begin{equation}
\hat{S}_1 \left|s_2\right>  = - \left|y\right> = -1 \cdot \left|s_2\right>.
\label{eq:part2:pol:stocks_s1_2}
\end{equation}


Для оператора $\hat{S}_2$ имеем:
\begin{eqnarray}
\hat{S}_2 \left|x\right> = \hat{a}_x^{\dag} \hat{a}_y
\left|1\right>_x\otimes\left|0\right>_y + \hat{a}_y^{\dag}
\hat{a}_x\left|1\right>_x\otimes\left|0\right>_y =
\nonumber \\
= 
\hat{a}_y^{\dag}
\hat{a}_x\left|1\right>_x\otimes\left|0\right>_y =
\left|0\right>_x\otimes\left|1\right>_y = \left|y\right>,
\nonumber \\
\hat{S}_2 \left|y\right> = \hat{a}_x^{\dag} \hat{a}_y
\left|0\right>_x\otimes\left|1\right>_y + \hat{a}_y^{\dag}
\hat{a}_x\left|0\right>_x\otimes\left|1\right>_y =
\nonumber \\
=
\hat{a}_x^{\dag} \hat{a}_y
\left|0\right>_x\otimes\left|1\right>_y
=\left|1\right>_x\otimes\left|0\right>_y = \left|x\right>.
\label{eqEntangS2MatrixPre}
\end{eqnarray}
Из \eqref{eqEntangS2MatrixPre} имеем следующее матричное представление
оператора $\hat{S}_2$:
\begin{equation}
\hat{S}_2 = 
\left(
\begin{array}{cc}
0 & 1 \\
1 & 0 
\end{array}
\right).
\label{eqEntangS2Matrix}
\end{equation}
По аналогии с оператором $\hat{S_1}$ из \eqref{eqEntangS2MatrixPre} и
\eqref{eqEntangS2Matrix} можно молучить два собственных числа $s_1 =
1$ и $s_2 = -1$.
Для первого числа собственным вектором будет
\begin{equation}
\left|s_1\right> = \frac{1}{\sqrt{2}}\left(\left|x\right> +
\left|y\right>\right),
\label{eq:part2:pol:stocks_s2_1}
\end{equation}
а для второго
\begin{equation}
\left|s_2\right> = \frac{1}{\sqrt{2}}\left(\left|x\right> - \left|y\right>\right).
\label{eq:part2:pol:stocks_s2_2}
\end{equation}

Для оператора $\hat{S}_3$ имеем:
\begin{eqnarray}
  \hat{S}_3 \left|x\right> = \frac{1}{i}\left(\hat{a}_x^{\dag} \hat{a}_y
\left|1\right>_x\otimes\left|0\right>_y - \hat{a}_y^{\dag}
\hat{a}_x\left|1\right>_x\otimes\left|0\right>_y\right) =
\nonumber \\
= 
-\frac{1}{i}\hat{a}_y^{\dag}
\hat{a}_x\left|1\right>_x\otimes\left|0\right>_y =
-\frac{1}{i}\left|0\right>_x\otimes\left|1\right>_y =
-\frac{\left|y\right>}{i} = i \left|y\right>,
\nonumber \\
\hat{S}_2 \left|y\right> = \frac{1}{i}\left(\hat{a}_x^{\dag} \hat{a}_y
\left|0\right>_x\otimes\left|1\right>_y + \hat{a}_y^{\dag}
\hat{a}_x\left|0\right>_x\otimes\left|1\right>_y\right) =
\nonumber \\
=
\frac{1}{i}\hat{a}_x^{\dag} \hat{a}_y
\left|0\right>_x\otimes\left|1\right>_y
=\frac{1}{i}\left|1\right>_x\otimes\left|0\right>_y =
-i \left|x\right>.
\label{eqEntangS3MatrixPre}
\end{eqnarray}
Таким образом из \eqref{eqEntangS3MatrixPre} получаем следующее
матричное преставление оператора $\hat{S}_3$:
\begin{equation}
\hat{S}_3 = 
\left(
\begin{array}{cc}
0 & i \\
-i & 0 
\end{array}
\right).
\label{eqEntangS3Matrix}
\end{equation}

%% (%i1) A: matrix([0, i],[-i, 0]);  
%%                                   [  0   i ]
%% (%o1)                             [        ]
%%                                   [ - i  0 ]
%% (%i2) [vals, vecs] : eigenvectors (A);
%% (%o2)        [[[- %i i, %i i], [1, 1]], [[[1, - %i]], [[1, %i]]]]
%% (%i3) vecs;
%% (%o3)                      [[[1, - %i]], [[1, %i]]]
%% (%i4) 

Из \eqref{eqEntangS3Matrix} можно получить что собственными числами
являются все те же $s_1 = 1$ и $s_2 = -1$. При этом собственными
состояниями оператора $\hat{S}_3$ являются состояния с левой и правой
круговыми поляризациями:
\begin{eqnarray}
  \left| s_1 \right> = \left| - \right> = \frac{1}{\sqrt{2}}
  \left(
  \left|x\right> - i \left|y\right>
  \right),
  \nonumber \\
  \left| s_2 \right> = \left| + \right> = \frac{1}{\sqrt{2}}
  \left(
  \left|x\right> + i \left|y\right>
  \right).
  \label{eqEntangS3Eigenvec}
\end{eqnarray}


% Подействуем операторами \eqref{eqEntangStokesOper} на состояние 
% \eqref{eqEntangSimpleState}:
% \begin{eqnarray}
% \hat{S}_0\left|\psi\right> = 
% \alpha \left|x\right> + 
% \beta \left|y\right>,
% \nonumber \\
% \hat{S}_1\left|\psi\right> = 
% \alpha \left|x\right> - 
% \beta \left|y\right>,
% \nonumber \\
% \hat{S}_2\left|\psi\right> = 
% \beta \left|x\right> + 
% \alpha \left|y\right>,
% \nonumber \\
% \hat{S}_3\left|\psi\right> = 
% -i \beta \left|x\right> + 
% i \alpha \left|y\right>.
% \label{eqEntangStokesPsi}
% \end{eqnarray}

% Используя \eqref{eqEntangStokesPsi} 
% можно написать значения усредненных параметров Стокса для состояния
% задаваемого \eqref{eqEntangSimpleState}:
% \begin{eqnarray}
% \left<\hat{S}_0\right> = 
% \left<\psi\right|\hat{S}_0\left|\psi\right> = 1,
% \nonumber \\
% \left<\hat{S}_1\right> = 
% \left<\psi\right|\hat{S}_1\left|\psi\right> = 
% \left|\alpha\right|^2 - \left|\beta\right|^2,
% \nonumber \\
% \left<\hat{S}_2\right> = 
% \left<\psi\right|\hat{S}_2\left|\psi\right> =
% \alpha^{*}\beta + \alpha\beta^{*} = 
% 2 Re\left(\alpha^{*}\beta\right),
% \nonumber \\
% \left<\hat{S}_3\right> = 
% \left<\psi\right|\hat{S}_3\left|\psi\right> = 
% \frac{\alpha^{*}\beta - \alpha\beta^{*}}{i} =
% 2 Im\left(\alpha^{*}\beta\right).
% \label{eqEntangStokesMid}
% \end{eqnarray}

% Из \eqref{eqEntangStokesMid} следует что степень поляризации состояния
% $\left|\psi\right>$ равна 1.


