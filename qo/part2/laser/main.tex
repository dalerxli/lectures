%% -*- coding:utf-8 -*- 
\chapter{Квантовая теория лазера}
\label{chLaser}

Полуклассическая теория лазера не может ответить на все вопросы,
возникающие в связи с его работой. По этой теории лазер до достижения
порога  не генерирует вообще, а при превышении порога начинает
генерировать классическое электромагнитное поле (свет). 

В действительности же заметно ниже порога лазер генерирует хаотический
свет, а значительно выше порога его излучение близко к
классическому. На пороге и вблизи него находится переходная область от
хаотического света к упорядоченному излучению. Адекватно описать это
может только полностью квантовая теория. 

Другая задача, также требующая квантования электромагнитного поля, -
это определение предельной (естественной) ширины линии излучения
лазера, когда ширина линии определяется квантовыми флуктуациями поля,
а различные внешние воздействия, принципиально устранимые, не
принимаются во внимание. 

Лазер является открытой системой, в которой активные атомы и поле
резонатора связаны с большими внешними системами, которые мы будем
называть резервуарами, обеспечивающими накачку и потери. 
Отсюда следует, что лазер, как открытую систему, нужно рассматривать
при помощи матрицы плотности.  

\input ./part2/laser/model.tex
\input ./part2/laser/theory.tex
\input ./part2/laser/stat.tex
\input ./part2/laser/theorycoh.tex

\section{Упражнения}
\begin{enumerate}
\item Вывести \eqref{eqCh3_4} из \eqref{eqCh3_3}.
%\item Получить выражение \eqref{eqCh3_8} для $\left<n\right>$.
\item Из \eqref{eqCh3_5} получить выражение для диагональных элементов
  матрицы плотности \eqref{eqCh3_6}.
\item Из \eqref{eqCh3_5} получить выражение для оператора плотности в
  представлении когерентных состояний \eqref{eqCh3_7}.
\item Представить уравнение \eqref{eqCh3_7} в полярных координатах.
\item Получить уравнение \eqref{eqCh3_17} из общего уравнения
  \eqref{eqCh3_12} .
\item Воспользовавшись формулами \eqref{eqCh3_21a} и \eqref{eqCh3_22},
  оценить ''естественную'' ширину линии генерации лазера для $A
  \approx \frac{\omega}{Q}$ (малое превышение порога), 
$\frac{\omega}{Q} = \Delta \omega = 10^6 \mbox{Гц}$, 
$n_{st} = 10^6 \div 10^7$ фотонов в моде. 
\end{enumerate}

%% \begin{thebibliography}{99}
%% \bibitem{bCh1LaserMandel} Л.Мандель, Э.Вольф. Оптическая когерентность и
%%   квантовая оптика. Пер. с англ./Под ред. В.В.Самарцева - М.:
%%   Наука. Физматлит, 2000.- 896с. 
%% \bibitem{bCh1Laser_Lamb} Лекции У.Лэмба//Квантовая оптика и квантовая
%%   радиофизика: Лекции в летней школе. М.: Мир, 1966. 
%% \bibitem{bCh1Laser_Scally} Скалли М. Квантовая теория лазера -
%%   проблема неравновесной статистической механики. Квантовые флуктуации
%%   излучения лазера/Сост. Ф.Арекки, М.Скалли, Г.Хакен, В.Вайдлих. М.:
%%   Мир, 1974. 
%% \bibitem{bCh1Laser_Haken} Г.Хакен. Лазерная светодинамика. М.: Мир,
%%   1988.
%% \bibitem{bCh1LaserSkalliZubari} М. О. Скалли, М. С. Зубайри. Квантовая
%%   оптика. М. Физматлит, 2003.
%% \end{thebibliography}

