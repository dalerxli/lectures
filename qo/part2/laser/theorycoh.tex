%% -*- coding:utf-8 -*- 
\section{Теория лазера. Представление когерентных состояний}
Нужно воспользоваться уравнением для функции квазивероятности
$P\left(\alpha, t\right)$ \eqref{eqCh3_7}:
\begin{equation}
\frac{\partial}{\partial t} P\left(\alpha, t\right) = 
- \frac{1}{2}\left\{ 
\frac{\partial}{\partial \alpha}
\left[
A - \frac{\omega}{Q} - B \left|\alpha\right|^2
\right] \alpha P + \mbox{к. с.}
\right\} + 
A \frac{\partial^2 P}{\partial \alpha \partial \alpha^{*}}.
\nonumber
\end{equation} 

Для наших целей удобно это уравнение представить в полярных
координатах $\alpha = r e^{i\theta}$. Имеем 
\begin{eqnarray}
\frac{\partial}{\partial t} P\left(r, \theta, t\right) =
- \frac{1}{2 r} \cdot \frac{\partial}{\partial r}\left[
r^2\left(A - \frac{\omega}{Q} - B r^2\right)P\left(r, \theta,
t\right) 
\right] +
\nonumber \\
+ \frac{A}{4 r^2}
\left(
r \frac{\partial}{\partial r} r \frac{\partial}{\partial r} + 
\frac{\partial^2}{\partial \theta^2}
\right)
P\left(r, \theta, t\right)
\label{eqCh3_12}
\end{eqnarray}

Допустим, что нас интересует стационарное состояние. Тогда $\frac{\partial
  P}{\partial t} = 0$, а фаза $\theta$ распределена равномерно от $0$ до $2
\pi$. Из сказанного следует, что $\frac{\partial^2 P}{\partial
  \theta^2} = 0$ и квазивероятность $P\left(r, \theta, t\right)$ от $t$ и
$\theta$ не зависит. Уравнение \eqref{eqCh3_12} примет вид   
\begin{equation}
0 = - \frac{\partial}{\partial r}
\left[
r^2\left(A - \frac{\omega}{Q} - B r^2\right)P\left(r\right) 
\right] + 
\frac{1}{2} A \frac{\partial}{\partial r} r \frac{\partial}{\partial
  r} P\left(r\right).
\label{eqCh3_13}
\end{equation}

Это уравнение легко интегрируется. Первое интегрирование дает
\begin{equation}
r\left(A - \frac{\omega}{Q} - B r^2\right)P\left(r\right) 
=
\frac{1}{2} A \frac{\partial}{\partial
  r} P\left(r\right) + C.
\label{eqCh3_14}
\end{equation}
Постоянная  $C = 0$, так как $P$ и $\frac{\partial P}{\partial r}$
должны достаточно быстро стремиться к нулю при  $r \rightarrow
\infty$. 
 
Отсюда следует уравнение
\begin{equation}
\frac{d P}{P} = 
\frac{2}{A} r \left(A - \frac{\omega}{Q} - B r^2\right)d r. 
\label{eqCh3_15}
\end{equation}

Его решением будет
\begin{eqnarray}
P = N exp \left(\frac{1}{A}
\left[
r^2 \left(A - \frac{\omega}{Q}\right)
 - 
\frac{B r^4}{2}
\right]
\right) = 
\nonumber \\
= 
N exp \left(\frac{1}{A}
\left[
r^2 G
 - 
\frac{B r^4}{2}
\right]
\right),
\label{eqCh3_16}
\end{eqnarray}
где $N$ - нормирующий множитель, равный
\[
\frac{1}{N} = 2 \pi \int_0^{\infty}
exp\left(
\frac{1}{A}\left[
r^2 G - \frac{B}{2}r^4
\right]
\right)
r dr.
\]
Распределение \eqref{eqCh3_16}
является функцией от $r^2 = \left|\alpha\right|^2 = \bar{n}$. При
превышении порога $A > \frac{\omega}{Q}$,  то есть при $G > 0$,
$P\left(r\right)$ с ростом $r^2$ сперва растет, достигая максимума при
$r^2 = \frac{A - \omega/Q}{B} = \frac{G}{B}$, а затем убывает.  

Этот же результат мы имели ранее. Полученное значение $r =
\sqrt{\frac{G}{B}}$ соответствует при $G > 0$ наиболее вероятному
значению $\left|\alpha\right|$.
  
При $A < \frac{\omega}{Q}$ , $G < 0$ лазер находится ниже порога. С
ростом $r^2$ кривая монотонно падает. На пороге $A =
\frac{\omega}{Q}$,  $G = 0$  кривая также падает, но при $r^2 = 0$
имеем экстремум.   

\input ./part2/laser/fig6.tex

На \autoref{figPart2Ch1_6} представлены зависимости квазивероятности от $r^2 =
\left|\alpha\right|^2$, при различных значениях параметра $G$,
соответствующих допороговому, пороговому и надпорогогвому режимам. 

На комплексной плоскости $\alpha = r e^{i \theta}$ при значительном
превышении порога область, занятую генерируемым полем, можно наглядно
представить, как это сделано на
\autoref{figPart2Ch1_7}. Ломаная кривая характеризует
распределение амплитуды колебаний, т. е. соотвествует амплитудным 
шумам. Фаза $\theta$ равновероятна в интервале  $0 \div 2\pi$,
поэтому ломаная кривая располагается вблизи окружности $r =
\sqrt{\bar{n}_{st}}$ с радиусом, соответствующим наиболее вероятному
значению $\left|\alpha\right|$. 

\input ./part2/laser/fig7.tex

Из \autoref{figPart2Ch1_7} видно, что амплитуда флуктуирует в узкой области в
окрестности окружности радиуса $r = \sqrt{\bar{n}_{st}}$,  а фаза
свободно диффундирует (блуждает) вдоль этой окружности. 

В лазере диффузия фазы происходит медленно, поэтому, если
рассматривать не слишком большие промежутки времени, можно считать,
что лазер генерирует излучение с хорошо определенной фазой. При
усреднении за большой промежуток времени, когда фаза с равной
вероятностью может принимать любое значение в интервале  $0 \div 2
\pi$,  среднее значение лазерного поля за этот промежуток времени 
будет равно нулю.  

Рассмотрим более подробно процесс диффузии фазы. Предположим, что
распределение амплитуды у нас установлено и $P\left(r\right)$
соответствует стационарному распределению амплитуды. Тогда
\[
P\left(r, \theta, t\right) = P\left(r\right) P\left(\theta, t\right).
\]
В этом случае уравнение \eqref{eqCh3_12} примет вид 
\begin{equation}
\frac{\partial P \left(\theta, t\right)}{\partial t} = 
\frac{A}{4 r^2}
\frac{\partial^2}{\partial \theta^2}
P \left(\theta, t\right)
\label{eqCh3_17}
\end{equation}
Функция $P\left(r\right)$ сокращается, так как она удовлетворяет
стационарному уравнению \eqref{eqCh3_15}. 

Уравнение \eqref{eqCh3_17} можно использовать для получения уравнения,
которому удовлетворяет среднее поле. Среднее электрическое поле моды
лазера, выраженное через $P$, имеет вид  
\[
\left<\hat{E}\right> = E_1 \int P\left(\alpha, t\right) \alpha d^2
\alpha.
\]
В полярных координатах это выглядит так:
\[
\left<\hat{E}\right> = E_1 \int_{0}^{\infty}r dr \int_0^{2 \pi}r e^{i
  \theta} P\left(r, \theta, t\right) d \theta.
\]
Используем теперь уравнение \eqref{eqCh3_17}:
\begin{eqnarray}
\frac{\partial \left<\hat{E}\right>}{\partial t}
= E_1 \frac{\partial}{\partial t} \int_{0}^{\infty}r^2 dr \int_0^{2
  \pi} e^{i
  \theta} P\left(r, \theta, t\right) d \theta = 
\nonumber \\
= E_1 \int_0^{\infty}r^2 dr \int_0^{2\pi}e^{i\theta}\frac{\partial
  P}{\partial t} d \theta
=
\nonumber \\
= \frac{E_1}{\bar{n}_{st}} \frac{A}{4}
\int_{0}^{\infty}r^2 dr \int_0^{2 \pi}e^{i
  \theta} \frac{\partial^2}{\partial \theta^2} P\left(r, \theta,
t\right) d \theta. 
\label{eqCh3_18}
\end{eqnarray}
Здесь мы приближенно заменили $\frac{1}{r^2} \approx
\frac{1}{\bar{n}_{st}}$, считая распределение для амплитуд достаточно
узким. 

Проинтегрируем внутренний интеграл дважды по частям. Имеем
\begin{eqnarray}
\int_0^{2 \pi} e^{i \theta}
\frac{\partial^2 P}{\partial \theta^2} d \theta = 
e^{i \theta} \left.\frac{\partial P}{\partial \theta}\right|_0^{2 \pi}
- i \int_0^{2 \pi} e^{i \theta}
\frac{\partial P}{\partial \theta} d \theta = 
\nonumber \\
= -i e^{i \theta} \left. P \right|_0^{2 \pi} - 
\int_0^{2 \pi} e^{i \theta} P d \theta.
\label{eqCh3_19}
\end{eqnarray}

Первые слагаемые в \eqref{eqCh3_19} равны нулю из-за периодичности $P$,
$\frac{\partial P}{\partial \theta}$.  Учитывая \eqref{eqCh3_19},
уравнению \eqref{eqCh3_18} можно придать вид 
\begin{eqnarray}
  \frac{\partial \left<E\right>}{\partial t} =
  E_1 \int_0^{\infty}r^2 dr \int_0^{2\pi}e^{i\theta}\frac{\partial
    P}{\partial t} d \theta
  =
  \nonumber \\
  = 
- \frac{A E_1}{4 \bar{n}_{st}}
\int_0^{\infty}r dr\int_0^{2 \pi}r e^{i \theta} P d \theta = 
- \frac{A}{4 \bar{n}_{st}} \left<E\right>.
\label{eqCh3_20}
\end{eqnarray}
Решение этого уравнения 
\begin{equation}
\left<E\left(t\right)\right> = 
\left<E\left(0\right)\right> e^{- \frac{D}{2}t},
\label{eqCh3_21}
\end{equation}
где 
\begin{equation}
D = \frac{A}{2 \bar{n}_{st}}.
\label{eqCh3_21a}
\end{equation}

Таким образом мы получили, что среднее поле действительно стремится к
нулю с характерным временем $\tau = \frac{1}{D}$.  В лазерах этот
интервал может быть достаточно большим по сравнению с периодом
колебаний.

Для определения естественной ширины спектра излучения лазера
необходимо найти корреляционную функцию излучения лазера, ее спектр по
\myref{thm:khinchin_wiener}{теореме Хинчина-Винера} будет
энергетическим спектром лазерного излучения. Имеем
\begin{equation}
\left<\hat{E}^{(-)}\left(0\right)\hat{E}^{(+)}\left(t\right)\right>
= E_1^2\int d^2\alpha P\left(\alpha\right) \alpha^{*}\left(0\right)\alpha\left(t\right) e^{-i
  \omega_0 t},
\nonumber
\end{equation}
где $P\left(\alpha\right)$ - квазивероятность состояния лазера,
определяемая формулой \eqref{eqCh3_12}. 
При переходе к полярным координатам
(модуль-фаза) $\alpha = r e^{i \theta}$
мы считаем что амплитуда уже
установилась и не меняется, а фаза менеяется медленно, так что 
\begin{eqnarray}
\alpha^{*}\left(0\right) = r e^{i\theta\left(0\right)} = r e^{i 0} = r,
\nonumber \\
\alpha\left(t\right) = r e^{i\theta\left(t\right)} = r e^{i \theta},
\label{eqCh3_addon1}
\end{eqnarray}
В \eqref{eqCh3_addon1} мы предполагали, что в начальный момент времени 
$\theta\left(0\right) = 0$, а в момент времени $t$ - $\theta\left(t\right) = \theta$.
Таким образом имеем:
\begin{equation}
\left<\hat{E}^{(-)}\left(0\right)\hat{E}^{(+)}\left(t\right)\right>
= E_1^2 e^{-i
  \omega_0 t}\int_0^{\infty}r^3 d r \int_0^{2 \pi}d \theta P\left(r
e^{i \theta}, t\right) e^{i \theta}.
\nonumber
\end{equation}
Выведем уравнение, которому удовлетворяет 
$\left<\hat{E}^{(-)}\left(0\right)\hat{E}^{(+)}\left(t\right)\right>$. 
Имеем уравнение \eqref{eqCh3_17} 
\(
\frac{\partial P \left(\theta, t\right)}{\partial t} = 
\frac{A}{4 \bar{n}_{st}}
\frac{\partial^2}{\partial \theta^2}
P \left(\theta, t\right)
\),
следовательно
\begin{eqnarray}
\frac{d}{dt}\left<\hat{E}^{(-)}\left(0\right)\hat{E}^{(+)}\left(t\right)\right>
= -i \omega_0
\left<\hat{E}^{(-)}\left(0\right)\hat{E}^{(+)}\left(t\right)\right>+
\nonumber \\
+ E_1^2 e^{-i
  \omega_0 t}\int_0^{\infty}r^3 d r \int_0^{2 \pi}d \theta
\frac{\partial}{\partial t}P\left(r
e^{i \theta}, t\right) e^{i \theta}
=
\nonumber \\
=
-i \omega_0
\left<\hat{E}^{(-)}\left(0\right)\hat{E}^{(+)}\left(t\right)\right>+
\nonumber \\ 
+
\frac{E_1^2 A}{4 \bar{n}_{st}} 
\int_0^{\infty}r^3 d r 
\int_0^{2 \pi}
d \theta
\frac{\partial^2}{\partial \theta^2}P\left(r
e^{i \theta}, t\right) e^{i \theta}.
\label{eqPart2Ch1_add84_1}
\end{eqnarray}
Проведем в \eqref{eqPart2Ch1_add84_1} дважды интегрирование по частям
по $\theta$. Получим:
\begin{eqnarray}
\int_0^{2 \pi}
d \theta
\frac{\partial^2}{\partial \theta^2}P e^{i \theta} = 
\left.\frac{\partial}{\partial \theta}P e^{i \theta}\right|_0^{2 \pi}
- 
i \int_0^{2 \pi}
d \theta
\frac{\partial}{\partial \theta}P e^{i \theta} = 
\nonumber \\
= - i \int_0^{2 \pi}
d \theta
\frac{\partial}{\partial \theta}P e^{i \theta}  
= 
- i \left.P e^{i \theta}\right|_0^{2 \pi}
-
\int_0^{2 \pi}
d \theta
P e^{i \theta} = 
- \int_0^{2 \pi}
d \theta
P e^{i \theta}.
\nonumber
\end{eqnarray}
Подставляя получившийся результат в исходное выражение
\eqref{eqPart2Ch1_add84_1}, получим:
\begin{eqnarray}
\frac{d}{dt}\left<\hat{E}^{(-)}\left(0\right)\hat{E}^{(+)}\left(t\right)\right>
=
-i \omega_0
\left<\hat{E}^{(-)}\left(0\right)\hat{E}^{(+)}\left(t\right)\right>-
\nonumber \\
-
\frac{E_1^2 A}{4 \bar{n}_{st}} 
\int_0^{\infty}r^3 d r 
\int_0^{2 \pi}
d \theta
P\left(r
e^{i \theta}, t\right) e^{i \theta} = 
\nonumber \\
=-i \omega_0
\left<\hat{E}^{(-)}\left(0\right)\hat{E}^{(+)}\left(t\right)\right> -
\frac{A}{4 \bar{n}_{st}}\left<\hat{E}^{(-)}\left(0\right)\hat{E}^{(+)}\left(t\right)\right>.
\nonumber
\end{eqnarray}
Решение этого уравнения имеет вид
\begin{equation}
r_{+}\left(t\right) = \left<\hat{E}^{(-)}\left(0\right)\hat{E}^{(+)}\left(t\right)\right> =
\left<\hat{E}^{(-)}\left(0\right)\hat{E}^{(+)}\left(0\right)\right>
e^{-i \omega_0 t - \frac{D}{2}t},
\label{eqPart2LaserCorr1}
\end{equation}
где
\[
\frac{D}{2} = \frac{A}{4 \bar{n}_{st}}.
\]
Выражение \eqref{eqPart2LaserCorr1} определяет корреляционную функцию
$r\left(t\right)$ в области $t \ge 0$. В случае $t \le 0$, с учетом
\eqref{eqAddHinchinStatWide3}, корреляционная функция может быть
записана в виде  
\begin{equation}
r_{-}\left(t\right) = r_{-}\left(-\left|t\right|\right) =
r_{+}^{*}\left(\left|t\right|\right). 
\label{eqPart2LaserCorr2}
\end{equation}
С учетом выражений \eqref{eqPart2LaserCorr1} и
\eqref{eqPart2LaserCorr2} можно написать следующее выражение для фурье
образа корреляционной функции, которая согласно 
\myref{thm:khinchin_wiener}{теореме Хинчина-Винера}
определяет энергетический спектр $S\left(\omega\right)$:
\begin{eqnarray}
S\left(\omega\right) = \tilde{r}\left(\omega\right) = 
\frac{1}{2\pi}
\int_{-\infty}^{+\infty}e^{i \omega t} r\left(t\right) dt =
\nonumber \\
=
\frac{1}{2\pi}
\int_{-\infty}^0e^{i \omega t} r_{-}\left(t\right) dt +
\frac{1}{2\pi}
\int_0^{+\infty}e^{i \omega t} r_{+}\left(t\right) dt = 
\nonumber \\
=
\frac{1}{2\pi}
\int_0^{+\infty}e^{- i \omega t} r_{+}^{*}\left(t\right) dt +
\frac{1}{2\pi}
\int_0^{+\infty}e^{i \omega t} r_{+}\left(t\right) dt = 
\nonumber \\
=
\frac{1}{\pi} Re
\int_0^{+\infty}e^{i \omega t} r_{+}\left(t\right) dt = 
\nonumber \\
= \frac{1}{\pi}Re
\int_0^{+\infty}e^{i \omega t}
\left<\hat{E}^{(-)}\left(0\right)\hat{E}^{(+)}\left(t\right)\right> dt
=
\nonumber \\
=
\frac{\left<\hat{E}^{(-)}\left(0\right)\hat{E}^{(+)}\left(0\right)\right>}{\pi}
\frac{D/2}{\left(\omega - \omega_0\right)^2 + \left(D/2\right)^2}. 
\label{eqCh3_22}
\end{eqnarray}

%% Из \eqref{eqCh3_21} имеем
%% \begin{eqnarray}
%% \left|E\left(\omega\right)\right|^2 = 
%% \left|\int_{-\infty}^{+\infty}e^{i \omega t}\left<E\left(0\right)\right>
%% e^{i \omega_0 t} e^{-\frac{1}{2}D t}
%% dt\right|^2  = 
%% \nonumber \\
%% = 
%% \left|\left<E\left(0\right)\right>\right|^2 
%% \frac{D}{\left(\omega - \omega_0\right)^2 + \left(D/2\right)^2}. 
%% \label{eqCh3_22}
%% \end{eqnarray}

\input ./part2/laser/fig8.tex

Получилось, что линия генерации лоренцовская, а ширина линии равна $D$ 
(\autoref{figPart2Ch1_8}). Поскольку мы рассматривали только ''естественную'',
принципиально не устранимую причину уширения линии генерации,
связанную с квантовыми флуктуациями, и пренебрегали принципиально
устранимыми техническими причинами, полученная ширина получилась
предельно узкой.  

Здесь мы рассмотрели простейшую модель лазера. Более
реалистическая модель трехуровневой системы рассмотрена в
\cite{bHaken1988}. Результаты, полученные для различных моделей
лазера, хорошо согласуются между собой и соответствуют результатам, 
полученным с использованной нами простейшей модели. 
