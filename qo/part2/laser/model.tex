%% -*- coding:utf-8 -*- 
\section{Модель лазера}

Ранее (\ref{eqCh2_rho_final2}) мы рассматривали взаимодействие моды резонатора
(гармонического осциллятора) с резервуаром, находящимся в тепловом
равновесии при температуре  $T$.  Этой же моделью можно воспользоваться
и при рассмотрении работы лазера, но ее нужно дополнить еще одним
резервуаром, обеспечивающим накачку.  
См. рис. \ref{figPart2Ch1_1}: первый резервуар, $R_1$, содержит атомы,
находящиеся при температуре  $T$.  Второй, $R_2$, содержит атомы (другого
сорта), находящиеся в инвертированном состоянии. 

\input ./part2/laser/fig1.tex

В линейном приближении (во втором порядке теории возмущения) мы задачу
о взаимодействии моды резонатора с резервуаром уже решали. Для матрицы
плотности, описывающей состояние поля моды резонатора, были получены
уравнения:  
\begin{eqnarray}
\dot{\hat{\rho}} =
- \frac{1}{2}\frac{\omega}{Q}\left\{
\bar{n}_T\left(\hat{a}\hat{a}^{+}\hat{\rho} - 
\hat{a}^{+}\hat{\rho}\hat{a}
\right)\right.
+ 
\nonumber \\
\left.
+ \left(\bar{n}_T + 1\right)
\left(\hat{a}^{+}\hat{a}\hat{\rho} - 
\hat{a}\hat{\rho}\hat{a}^{+}
\right)
\right\}
 + \mbox{э. с.}
\label{eqCh3_1a}
\end{eqnarray}
или в другой записи (\ref{eqCh2_rho_final2})
\begin{eqnarray}
\dot{\hat{\rho}} =
- \frac{1}{2}R_a
\left(\hat{a}\hat{a}^{+}\hat{\rho} - 
\hat{a}^{+}\hat{\rho}\hat{a}
\right)
- 
\nonumber \\
- \frac{1}{2}R_b
\left(\hat{a}^{+}\hat{a}\hat{\rho} - 
\hat{a}\hat{\rho}\hat{a}^{+}
\right)
 + \mbox{э. с.}
\label{eqCh3_1b}
\end{eqnarray}

Если ограничиться линейным приближением (по полю), уравнение для
матрицы плотности лазера можно написать, используя эти уравнения. Для
описания действия резервуара, вносящего потери, удобно использовать
(\ref{eqCh3_1a}), учтя при этом, что при не слишком высокой
температуре $\bar{n}_T \ll 1$  и $\bar{n}_T$
можно пренебречь по сравнению с  единицей.
 
Для описания резервуара, обеспечивающего накачку, удобно использовать
(\ref{eqCh3_1b}), отбросив второе слагаемое, так как для простоты
предполагается, что все атомы резервуара накачки находятся в верхнем
состоянии. 

Все это приводит к следующему уравнению:
\begin{eqnarray}
\dot{\hat{\rho}} =
- \frac{1}{2}\frac{\omega}{Q}
\left(\hat{a}^{+}\hat{a}\hat{\rho} - 
\hat{a}\hat{\rho}\hat{a}^{+}
\right)
-
\nonumber \\
- \frac{1}{2}A
\left(\hat{a}\hat{a}^{+}\hat{\rho} - 
\hat{a}^{+}\hat{\rho}\hat{a}
\right)
 + \mbox{э. с.}
\label{eqCh3_2}
\end{eqnarray}
где $A = R_a$ определяется интенсивностью накачки.

Уравнение (\ref{eqCh3_2}) получено во втором приближении теории
возмущения. Оно может описать поведение лазера ниже
порога генерации, позволяет определить пороговые условия, но не может
описать лазер выше порога генерации.  

Положим, что потери линейны, и первый член в (\ref{eqCh3_2})
достаточно точно описывает потери. Второй член в (\ref{eqCh3_2}) должен быть
найден в следующем не равном нулю приближении (четвертом приближении
теории возмущения).
 
Процедура нахождения этого приближения аналогична проделанной ранее
при выводе уравнений (\ref{eqCh3_1a} - \ref{eqCh3_1b}). Нужно только
продолжить ее до членов более высокого порядка. Члены третьего порядка
дадут ноль, так как получающаяся матрица имеет нулевые диагональные
члены, и след от нее равен нулю. Следующий отличный от нуля член будет 
четвертого порядка и имеет вид: 
\begin{eqnarray}
Sp_{at}\left\{
\left(-\frac{i}{\hbar}\right)^4
\int_t^{t+\tau}dt_1
\int_t^{t_1}dt_2
\int_t^{t_2}dt_3
\right.
\nonumber \\
\left.
\int_t^{t_3}
\left[\hat{V},
\left[\hat{V},
\left[\hat{V},
\left[\hat{V},
\hat{\rho}_{at}\left(t\right)
\otimes
\hat{\rho}_{f}\left(t\right)
\right]
\right]
\right]
\right]
dt_4
\right\}
\label{eqCh3_3}
\end{eqnarray}
Вычисления, аналогичные проделанным ранее, приводят к выражению
(см. \cite{bMandel2000}):
%FIXME!!! раскрыть в приложении
\begin{eqnarray}
\frac{1}{8}B\left[
\left(\hat{a} \hat{a}^{+}\right)^2\hat{\rho}
+ 3 \hat{a} \hat{a}^{+} \hat{\rho} \hat{a} \hat{a}^{+} -
\right.
\nonumber \\
\left.
-
4 \hat{a}^{+} \hat{a} \hat{a}^{+} \hat{\rho} \hat{a}
\right] + \mbox{э. с.}
\label{eqCh3_4}
\end{eqnarray}
С учетом (\ref{eqCh3_4}) уравнение для матрицы плотности
(статистического оператора) лазерной моды приобретает вид: 
\begin{eqnarray}
\dot{\hat{\rho}}\left(t\right) = 
- \frac{1}{2}\frac{\omega}{Q}
\left(\hat{a}^{+}\hat{a}\hat{\rho} - 
\hat{a}\hat{\rho}\hat{a}^{+}
\right)
- \frac{1}{2}A
\left(\hat{a}\hat{a}^{+}\hat{\rho} - 
\hat{a}^{+}\hat{\rho}\hat{a}
\right) + 
\nonumber \\
+ \frac{1}{8}B\left[
\left(\hat{a} \hat{a}^{+}\right)^2\hat{\rho}
+ 3 \hat{a} \hat{a}^{+} \hat{\rho} \hat{a} \hat{a}^{+} -
4 \hat{a}^{+} \hat{a} \hat{a}^{+} \hat{\rho} \hat{a}
\right] + \mbox{э. с.}
\label{eqCh3_5}
\end{eqnarray}
где $A$ - линейное (ненасыщенное) усиление, 
$B=\frac{g^2 \tau^2 A}{3} = \frac{g}{3}r_ag^4\tau^4$ - параметр
насыщения.

Уравнение (\ref{eqCh3_5}) является уравнением движения матрицы
плотности лазерного поля, взаимодействующего с нелинейной средой,
состоящей из активных атомов, и с линейными потерями. 

Уравнение (\ref{eqCh3_5}) можно записать в различных
представлениях. Мы здесь ограничимся представлением чисел заполнения
(чисел фотонов в моде) и представлением когерентных состояний. 

В первом случае из (\ref{eqCh3_5}) легко получить систему уравнений
для матричных элементов $\left<m\right|\hat{\rho}\left|n\right> =
\rho_{mn}$.  

Для диагональных элементов имеем:
\begin{eqnarray}
\dot{\rho}_{nn}\left(t\right) = 
-\left[A - \left(n + 1\right)B\right]\left(n + 1\right)\rho_{nn} +
\nonumber \\
+ \left(A - n B\right)n \rho_{n - 1, n - 1} 
- \frac{\omega}{Q}n \rho_{nn} + 
\frac{\omega}{Q} \left(n + 1\right)\rho_{n + 1, n + 1}.
\label{eqCh3_6}
\end{eqnarray}
Заметим, что в уравнение входят только диагональные члены. 

Таким же образом получаются уравнения для внедиагональных членов. Они
выглядят несколько сложнее и здесь не приводятся. 

Для того, чтобы записать уравнение (\ref{eqCh3_5}) в представлении
когерентных состояний, можно действовать так же как действовали ранее,
когда рассматривали затухание моды резонатора. Запишем 
\[
\hat{\rho} = \int
P\left(\alpha\right)\left|\alpha\right>\left<\alpha\right| d^2 \alpha 
\]
и подставим в уравнение (\ref{eqCh3_5}). Далее проделаем знакомую
процедуру, использованную при релаксации моды резонатора. Для линейных
членов, определяющих потери и ненасыщенное усиление, можно
воспользоваться полученными тогда результатами. Дополнительно нужно
рассмотреть члены, характеризующие насыщение. Вычисления здесь более
громоздкие, хотя принципиально не отличаются от проделанных
ранее. Получается довольно сложное уравнение, но если вспомнить, что
мы использовали теорию возмущений (малое поле), можно пренебречь
малыми, оставив только главные члены\cite{bMandel2000}. 
%Подробности приведены в приложении (FIX ME!!! add it). 
В итоге всего сказанного получается
уравнение типа уравнения Фоккера-Планка для квазивероятности
$P\left(\alpha, t\right)$:  
\begin{equation}
\frac{\partial}{\partial t} P\left(\alpha, t\right) = 
- \frac{1}{2}\left\{ 
\frac{\partial}{\partial \alpha}
\left[
A - \frac{\omega}{Q} - B \left|\alpha\right|^2
\right] \alpha P + \mbox{к. с.}
\right\} + 
A \frac{\partial^2 P}{\partial \alpha \partial \alpha^{*}}
\label{eqCh3_7}
\end{equation}
то есть по сравнению с формулой (\ref{eqCh2_74}) $-\frac{\omega}{Q}$
заменяется на $A - \frac{\omega}{Q} - B\left|\alpha\right|^2$,  где $A
- \frac{\omega}{Q} = G$ - ненасыщенное усиление минус потери. $B$ - 
характеризует уменьшение усиление из-за насыщения. 
