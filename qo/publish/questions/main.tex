%% -*- coding:utf-8 -*- 
\input preamble.tex

\begin{document}
\Russian
\input title.tex

\section*{Введение}
Вашему вниманию предлагается заявка на издание учебного пособия под
названием ``Квантовая оптика''. Данное учебное пособие составлено на
основе курса лекций по квантовой оптике, читаемого на 
протяжении длительного времени в СПбГПУ (С. Петербургском
Государственном Политехническом Университете) на радиофизическом
факультете для студентов, специализирующихся в области квантовой
радиофизики и квантовой электроники.

Учебное пособие состоит из трех основных разделов
\begin{itemize}
\item Квантовая электродинамика. Квантовая электродинамика служит
  основой квантовой оптики. Раздел 
содержит краткое изложение основ квантовой электродинамики,
достаточное для описания оптических квантовых явлений в диапазоне
частот от инфракрасного до рентгеновского излучения.
\item Квантовая оптика 1 часть. В разделе, на основе квантовой
  электродинамики, рассматривается ряд оптических явлений требующих
  квантового описания. Рассматриваются устоявшиеся результаты
  квантовой оптики.
\item Квантовая оптика 2 часть. В разделе рассматриваются результаты
  квантовой оптики, полученные в последнее время и относящиеся к
  развивающимся в настоящее время направлениям.
\end{itemize}
Все вместе составляет объем \(\approx 15\) п.л. Формул \(\approx
300\), рисунков \(\approx 30\). Текст
набран с использованием пакета \LaTeX. Рисунки выполнены в
форматах {\LaTeX} и PostScript.

Контактная информация для оперативной связи: email:
ivan.murashko@gmail.com (Мурашко Иван Викторович). 

\section{Аннотация}
Пособие соответствует авторскому курсу дисциплины ``Квантовая
оптика''. В нем приводятся сведения по квантовой электродинамике,
минимально необходимые для для изложения ряда вопросов оптики,
требующих учета квантовых свойств света. Главное внимание уделено
вопросам взаимодействия света с атомом, квантовой теории релаксации,
квантовой теории лазера и квантовой теории когерентности. Кроме того
рассматриваются результаты полученные в последнее время, такие как
квантовая теория информации и неклассические свойства света.

Пособие предназначено для студентов старших курсов, специализирующихся
в области квантовой электроники. 

\section{Содержание}
\subsection{Квантовая электродинамика}
\subsubsection{Квантовые свойства электромагнитного поля}
Разложение электромагнитного поля по модам (типам колебаний).
Гамильтонова форма уравнений электромагнитного поля.
Квантование электромагнитного поля.
Разложение поля по плоским волнам в свободном пространстве 
Плотность состояний.
Гамильтонова форма уравнений поля при разложении по плоским
волнам.
Квантование электромагнитного поля при разложении его по
плоским волнам .
Свойства операторов $ \hat a $ и $ \hat a ^+ $ 
Квантовое состояние электромагнитного поля  с определенной
  энергией.
Многомодовые состояния. 
Когерентные состояния.
Смешанные состояния электромагнитного поля.
Представление оператора плотности через когерентные
состояния.
\subsubsection{Взаимодействие квантованного электромагнитного поля с атомом}
Излучение и поглощение атомом света.
Гамильтониан системы атом-поле.
Взаимодействие атома с модой электромагнитного поля.
Взаимодействие атома с многомодовым полем. Вынужденные и
спонтанные переходы.
Спонтанное излучение. Приближение Вайскопфа-Вигнера.
Релаксация динамической системы. Метод матрицы плотности.
Взаимодействие электромагнитного поля резонатора
(гармонического осциллятора) с резервуаром атомов, находящихся при
температуре $T$.
Уравнение для матрицы плотности поля в представлении чисел
заполнения.
Уравнение движения статистического оператора поля моды в
представлении когерентных состояний.
Общая теория взаимодействия динамической системы с
термостатом (диссипативной системой, резервуаром).
Затухание (релаксация) поля и атома в случае простейшего
резервуара, состоящего из гармонических осцилляторов.
Затухание моды резонатора. Приближение Гейзенберга-Ланжевена.

\subsection{Квантовая оптика часть 1}
\subsubsection{Квантовая теория лазера. Метод матрицы плотности}
Модель лазера. Теория лазерной генерации. Статистика лазерных
фотонов. Теория лазера в представлении когерентных состояний.
\subsubsection{Квантовая теория лазера. Метод Гейзенберга-Ланжевена}
Квантовые уравнения Ланжевена. Простой расчет ширины линии излучения
лазера методом Ланжевена.
\subsubsection{Оптика фотонов (квантовые явления в оптике)}
Фотоэффект.
Когерентные свойства света.
Когерентность второго порядка.
Когерентность высших порядков.
Счет и статистика фотонов.
Связь статистики фотонов со статистикой фотоотсчетов.
Распределение фотоотсчетов для когерентного и хаотического
света.
Определение статистики фотонов через распределение
фотоотсчетов.
Квантовое выражение для распределения фотоотсчетов.
Эксперименты по счету фотонов. Применение техники счета
фотонов для спектральных измерений.

\subsection{Квантовая оптика часть 2}
\subsubsection{Неклассический свет}
Критерий классичности квантового состояния. Статистика
фотонов. Пуассоновская, субпуассоновская и суперпуассоновская
статистики. Группировка и антигруппировка фотонов. Экспериментальное
определение статистики фотонов. Квазивероятность и ее связь с
классичностью или неклассичностью состояния. Неклассичность чистых квантовых состояний.
\subsubsection{Квантовое описание оптических интерференционных
  экспериментов}
Классическое описание прибора (интерферометра).
Квантовое описание светового поля и представление Гейзенберга.
Матрица рассеяния света и ее свойства.
Примеры. Интерферометр Майкельсона. Интерферометр Маха-Цендера. 
Квантовое описание. Выход интерферометра. Балансный детектор. Погрешность фазовых измерений.
\subsubsection{Сжатые состояния}
Соотношение неопределенности Гейзенберга. Сжатое состояние. Идеально
сжатое состояние. Операторы квадратурных составляющих
электромагнитного поля. Условие сжатого состояния. Примеры:
когерентное и энергетическое состояния не являются сжатыми. Унитарный
оператор сжатия и его свойства. Действие оператора сжатия на
когерентное состояние. Квадратурно сжатое когерентное
состояние. Параметр сжатия. Сжатый вакуум. Генерация квадратурно
сжатых состояний в параметрическом взаимодействии. Вырожденное
параметрическое рассеяние. Наблюдение сжатого состояния и измерение
степени сжатия. Гомодинный детектор (синхронный детектор) и его
применение для выделения ``сжатой'' квадратуры.  Применение сжатого
состояния для повышения точности интерференционных измерений. Предел
повышения точности. Предел Гейзенберга.
\subsubsection{Перепутанные состояния}
Поляризационные свойства света. Параметры Стокса. Парадокс ЭПР для параметров
  Стокса и перепутанные состояния. Неравенство Белла для параметров
  Стокса. Базисные состояния Белла. Получение и регистрация Белловских
  состояний. Квантовая телепортация.
\subsubsection{Квантовая теория информации}
Информация и энтропия. Передача информации. Классически и квантовый
каналы связи. Кодирование информации. Теорема кодирования
Шенона. Квантовая теорема кодирования. Криптография. Проблемы
классической криптографии. Квантовая криптография.

\section{Сведения об авторах}
\subsection{Петрунькин Всеволод Юрьевич}
Петрунькин Всеволод Юрьевич (5.09.1923 - 7.11.2008). Окончил
Физико-механический факультет Ленинградского политехнического
института (ЛПИ) (1949) по специальности ``инженер-исследователь в
области радиофизики''. Кандидат технических наук (1953). Доктор
технических наук (1965). Трудовая и научная деятельность связаны с
работой в ЛПИ. Аспирант (1949-1952), ассистент (1952-1956). Доцент по
кафедре ``Радиофизика'' (1956-1966), профессор по кафедре ``Радиофизика''
(1966). В 1968 г. основал на радиофизическом факультете ЛПИ кафедру
квантовой электроники, которой заведовал до 1988 г. После этого был
профессором этой кафедры. Научная деятельность связана с работами в
области излучения и распространения электромагнитных волн и антенной
техники. С 1965 г. с квантовой радиофизикой (лазеры, применение
лазеров). Многие годы был членом редколлегии журнала ``Известия
ВУЗов. Радиофизика''. В последнее время был членом редколлегии
журналов ЖТФ (``Журнал технической физики'') и ``Письма в
ЖТФ''. Лауреат Государственной премии за работу в области
радиоэлектроники (1984). Заслуженный деятель науки и техники
Российской Федерации (1994). 
\subsection{Мурашко Иван Викторович} 
Мурашко Иван Викторович (род. 09.04.1975). Окончил радиофизический
факультет С. Петербургского Государственного Технического Университета
(1998). Кандидат физико математических наук (2001). Доцент кафедры
квантовой электроники С. Петербургского Государственного
Политехнического Университета.

Контактная информация (Мурашко И. В.) 
\begin{itemize}
\item дом. адрес: С. Петербург, Шлиссельбургский пр. 17 кор. 2 кв 172
\item email: ivan.murashko@gmail.com,
\item тел. моб. +7 (921) 974 2848 
\item тел. дом. +7 (812) 707 1464
\end{itemize}
\end{document}
