%% -*- coding:utf-8 -*- 
\section{Принцип неопределенности Гейзенберга}
\label{AddHeisenbergUncertaintyPrinciple}
Допустим два оператора $\hat{A}$ и $\hat{B}$ не коммутируют друг с
другом, т.е. 
\begin{equation}
\left[
\hat{A}\hat{B}
\right] = 
\hat{A}\hat{B} - \hat{B}\hat{A} = i \hat{C} \ne 0,
\nonumber
\end{equation}
где $\hat{C}$ некоторый эрмитовый оператор.

Допустим система находится в состоянии $\left|\psi\right>$. Тогда
средние значения операторов выражаются следующими соотношениями:
\begin{eqnarray}
\left<\psi\right|\hat{A}\left|\psi\right> = \left<\hat{A}\right>,
\nonumber \\
\left<\psi\right|\hat{B}\left|\psi\right> = \left<\hat{B}\right>.
\nonumber
\end{eqnarray}

Определим неопределённость измерения величин $A$ и $B$ как их
дисперсии следующим образом
\begin{eqnarray}
\Delta A = \sqrt{\left<\psi\right|
\hat{\mathcal{A}}^2\left|\psi\right>}, 
\nonumber \\
\Delta B = \sqrt{\left<\psi\right|
\hat{\mathcal{B}}^2\left|\psi\right>}, 
\nonumber
\end{eqnarray}
где
\begin{eqnarray}
\hat{\mathcal{A}} = \hat{A}-\left<\hat{A}\right>, 
\nonumber \\
\hat{\mathcal{B}} = \hat{B}-\left<\hat{B}\right>.
\nonumber
\end{eqnarray}

Введем оператор $\hat{D}$ следующим образом
\begin{equation}
\hat{D} = \hat{\mathcal{A}} + i \lambda \hat{\mathcal{B}}.
\nonumber
\end{equation}
Рассмотрим оператор $\hat{D}^{+}\hat{D}$, который является
эрмитовым. Его среднее в состоянии $\left|\psi\right>$:
\begin{eqnarray}
\left<\psi\right|\hat{D}^{+}\hat{D}\left|\psi\right> = 
\left<\phi\right|\left.\phi\right> \ge 0,
\nonumber
\end{eqnarray}
где
$\left|\phi\right> = \hat{D}\left|\psi\right>$. С другой стороны 
\begin{eqnarray}
\left<\psi\right|\hat{D}^{+}\hat{D}\left|\psi\right> = 
\left<\psi\right|\left(\hat{\mathcal{A}} - i \lambda \hat{\mathcal{B}}\right)
\left(\hat{\mathcal{A}} + i \lambda \hat{\mathcal{B}}\right)\left|\psi\right> =
\nonumber \\
=
\left<\psi\right|\hat{\mathcal{A}}^2\left|\psi\right> +
\lambda^2\left<\psi\right|\hat{\mathcal{B}}^2\left|\psi\right> +
i \lambda 
\left<\psi\right|
\left[ \mathcal{\hat{A}}, \mathcal{\hat{B}}\right]
\left|\psi\right>
 = 
\nonumber \\
=
\left(\Delta A\right)^2 + \lambda^2 \left(\Delta B\right)^2 +
i \lambda 
\left<\psi\right|
\left[ \hat{A}, \hat{B}\right]
\left|\psi\right> = 
\nonumber \\
=
\lambda^2 \left(\Delta B\right)^2 - 
\lambda \left<C\right> + \left(\Delta A\right)^2 \ge 0.
\nonumber
\end{eqnarray}
Рассмотрим многочлен 
\[
f\left(\lambda\right) = \lambda^2 \left(\Delta B\right)^2 - 
\lambda \left<C\right> + \left(\Delta A\right)^2.
\]
Имеем $f\left( \pm \infty \right) > 0$ т.о. 
$f\left(\lambda\right) \ge 0$ если этот многочлен имеет не больше
одного вещественного корня, т.е.
\[
\left<C\right>^2 - 4 \left(\Delta A\right)^2 \left(\Delta B\right)^2
\le 0
\]
или
\begin{equation}
  \Delta A \Delta B \ge \frac{\left|\left< C \right>\right|}{2},
  \label{eqAddHeisenbergUncertaintyPrinciple}
\end{equation}
что представляет собой неравенство Гейзенберга. 


