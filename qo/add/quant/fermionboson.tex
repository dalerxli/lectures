%% -*- coding:utf-8 -*- 
\section{Бозоны и фермионы}
\label{AddFermionBoson}

Рассмотрим ансамбль тождественных частиц. Состояние системы состоящей
из $n$ тождественных частиц может быть записано в следующем виде
\begin{equation}
  \left|\psi\right> = \left|x_1, x_2, \dots, x_k, \dots, x_j, \dots,
  x_n\right>.
  \label{eqAddFBAnsamble}
\end{equation}
Для каждой из частиц в выражении \eqref{eqAddFBAnsamble} нас
интересует некоторая физическая характеристика (например координата)
которая обозначена через $x$.

Очевидно, что с точки зрения
классики не должно быть никакой разницы если мы поменяем местами две
частицы. При этом волновая функция которая определена с точностью до
фазы может измениться. Таким образом обозначив через
$\hat{S}_{k,j}$ оператор который меняет местами две частицы с
индексами $k$ и $j$ получим 
\begin{eqnarray}
  \hat{S}_{k, j} \left|\psi\right> = \left|x_1, x_2, \dots, x_j,
  \dots, x_k, \dots,  x_n\right> =
  \nonumber \\
  = e^{i \phi} \left|x_1, x_2, \dots, x_k, \dots, x_j, \dots,
  x_n\right> = e^{i \phi} \left|\psi\right>
  \nonumber
\end{eqnarray}

Если оператор $\hat{S}_{k,j}$ применить два раза то очевидно что
система будет приведена полностью в исходное состояние, т.е.
\begin{eqnarray}
  \hat{S}_{k, j}^2 \left|\psi\right> =
  e^{2 i \phi} \left|\psi\right> = \left|\psi\right>
  \nonumber
\end{eqnarray}
откуда следует, что $\phi$ может принимать два значения $0$ и $\pi$.
Таким образом мы можем разделить все частицы на два класса. Для
первого из них волновая функция не меняется при перестановке двух
частиц ($\psi = 0$), а для второго меняет знак ($\psi = \pi$).

\begin{example}
  Рассмотрим систему состоящую из двух частиц. Допустим что 1-ая частица
  может находится в одном из двух состояний $\left|\psi^{(1)}_1\right>$
  или $\left|\psi^{(1)}_2\right>$. Аналогично 2-ая частица может
  находится в одном из двух состояний $\left|\psi^{(2)}_1\right>$
  или $\left|\psi^{(2)}_2\right>$.

  Предположим общая волновая функция $\left|\psi^{(12)}\right>$ двух
  частиц представима в виде суперпозиции волновых функций одиночных
  частиц, т.е. 
  \begin{equation}
    \left|\psi^{(12)}\right> = \left|\psi^{(1)}_{1,2}\right> \pm
    \left|\psi^{(2)}_{1,2}\right>
    \nonumber
  \end{equation}

  Случай симметричной (при перестановке частиц) волновой функции будет
  иметь вид
  \begin{equation}
    \left|\psi^{(12)}\right> = \left|\psi^{(1)}_{1,2}\right> +
    \left|\psi^{(2)}_{1,2}\right>
    \nonumber
  \end{equation}
  при этом обе частицы могут находиться в одном и том же состоянии,
  т. е. иметь одну и ту же волновую функцию
  \begin{equation}
  \left|\psi\right> = \left|\psi^{(1)}_{1}\right> =
  \left|\psi^{(2)}_{1}\right>.
  \label{eqAddFermionBosonSameState}
  \end{equation}
  В этом случае
  \[
  \left|\psi^{(12)}\right> = 2 \left|\psi\right>, 
  \]
  т. е. частицы могут одновременно находиться в одном и том же
  состоянии.

  Для случая антисимметричной волновой функции имеем
  \begin{equation}
    \left|\psi^{(12)}\right> = \left|\psi^{(1)}_{1,2}\right> -
    \left|\psi^{(2)}_{1,2}\right>
    \nonumber
  \end{equation}
  и если обе частицы находятся в одном и том же состоянии
  \eqref{eqAddFermionBosonSameState}, то для $\psi^{(12)}$ получим
  \[
  \left|\psi^{(12)}\right> = \left|\psi\right> -\left|\psi\right> = 0,
  \]
  т. е. в этом случае частицы не могут одновременно находиться в одном
  и том же состоянии.
  \label{exAddFermionBoson}
\end{example}
Имея в виду пр. \ref{exAddFermionBoson}, можно рассмотреть общий
случай ансамбля тождественных частиц, каждая из которых находится в
одном и том же состоянии или по другому имеют одну и туже волновую
функцию. В этом случае при перемене частиц реальный вид волновой
функции ансамбля не меняется:
\[
\left|\psi^{(12)}\right> = \left|\psi^{(21)}\right>.
\]
Т. е. в антисимметричном случае 
\[
\left|\psi^{(12)}\right> = - \left|\psi^{(21)}\right> =
-\left|\psi^{(12)}\right> = 0.
\] 

Таким образом мы можем сказать, что частицы
состояния ансамбля которых симметричны могут находится одновременно в
одном и том же состоянии. Такие частицы называются бозонами.\index{бозон}

Для антисимметричного случая будем  
предполагать, что такие частицы не могут одновременно находиться в
одном и том же состоянии. Такие частицы называются фермионами.
\index{фермион}

Рассмотрим состояние с определенным числом частиц $n$
$\left|\psi\right> = \left|n\right>$. В случае бозонов \index{бозон}
мы будем иметь 
\begin{equation}
\left|\psi_b\right> = \left|n\right>_b.
\label{eqAddQuantBoson}
\end{equation}
Операторы рождения $\hat{a}_b^{+}$ и уничтожения $\hat{a}_b$ действуют на
состояние \eqref{eqAddQuantBoson} следующим образом
\begin{eqnarray}
\hat{a}_b^{+}\left|n\right>_b = \sqrt{n+1}\left|n+1\right>_b, 
\nonumber \\
\hat{a}_b\left|n\right>_b = \sqrt{n}\left|n\right>_b
\nonumber
\end{eqnarray}

Для фермионов \index{фермион} возможны два состояния: $\left|0\right>_f$ и
$\left|1\right>_f$ при этом соответствующие операторы рождения
$\hat{a}_f^{+}$ и уничтожения $\hat{a}_f$ действуют следующим образом
\begin{eqnarray}
\hat{a}_f^{+}\left|0\right>_f = \left|1\right>_f, 
\nonumber \\
\hat{a}_f^{+}\left|1\right>_f = 0, 
\nonumber \\
\hat{a}_f\left|0\right>_f = 0, 
\nonumber \\
\hat{a}_f\left|1\right>_f = \left|0\right>_f.
\nonumber
\end{eqnarray}
