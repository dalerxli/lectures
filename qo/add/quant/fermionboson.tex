%% -*- coding:utf-8 -*- 
\section{Бозоны и фермионы}

Все элементарные частицы делятся на два класса: бозоны и
фермионы. Рассмотрим состояние с определенным числом частиц $n$
$\left|\psi\right> = \left|n\right>$. В случае бозонов мы будем иметь 
\begin{equation}
\left|\psi_b\right> = \left|n\right>_b.
\label{eqAddQuantBoson}
\end{equation}
Операторы рождения $\hat{a}_b^{+}$ и уничтожения $\hat{a}_b$ действуют на
состояние (\ref{eqAddQuantBoson}) следующим образом
\begin{eqnarray}
\hat{a}_b^{+}\left|n\right>_b = \sqrt{n+1}\left|n+1\right>_b, 
\nonumber \\
\hat{a}_b\left|n\right>_b = \sqrt{n}\left|n\right>_b
\nonumber
\end{eqnarray}

Для фермионов возможны два состояния: $\left|0\right>_f$ и
$\left|1\right>_f$ при этом соотвествующие операторы рождения
$\hat{a}_f^{+}$ и уничтожения $\hat{a}_f$ действуют следующим образом
\begin{eqnarray}
\hat{a}_f^{+}\left|0\right>_f = \left|1\right>_f, 
\nonumber \\
\hat{a}_f^{+}\left|1\right>_f = 0, 
\nonumber \\
\hat{a}_f\left|0\right>_f = 0, 
\nonumber \\
\hat{a}_f\left|1\right>_f = \left|0\right>_f.
\nonumber
\end{eqnarray}


\subsection{Бозоны и распределение Бозе-Эйнштейна}
Системы из нескольких бозонов описываются волновыми функциями, которые
являются симметричными относительно перестановок частиц.

TBD

\subsection{Фермионы. Распределение Ферми-Дирака. Принцип запрета Паули}

TBD
