%% -*- coding:utf-8 -*- 
\section{Бозоны и фермионы}
\label{AddFermionBoson}

Рассмотрим ансамбль тождественных частиц. Состояние системы состоящей
из $n$ тождественных частиц может быть записано в следующем виде
\begin{equation}
  \left|\psi\right> = \left|x_1, x_2, \dots, x_k, x_{k+1}, \dots,
  x_n\right>.
  \label{eqAddFBAnsamble}
\end{equation}
Для каждой из частиц в выражении (\ref{eqAddFBAnsamble}) нас
интересует некоторая физическая характеристика (например координата)
которая обозначена через $x$.

Очевидно, что с точки зрения
классики не должно быть никакой разницы если мы поменяем местами две
частицы. При этом волновая функция которая определена с точностью до
фазы может измениться. Таким образом обозначив через
$\hat{S}_{k,k+1}$ оператор который меняет местами две частицы с
индексами $k$ и $k+1$ получим 
\begin{eqnarray}
  \hat{S}_{k, k+1} \left|\psi\right> = \left|x_1, x_2, \dots, x_{k+1}, x_k, \dots,
  x_n\right> =
  \nonumber \\
  = e^{i \phi} \left|x_1, x_2, \dots, x_k, x_{k+1}, \dots,
  x_n\right> = e^{i \phi} \left|\psi\right>
  \nonumber
\end{eqnarray}

Если оператор $\hat{S}_{k,k+1}$ применить два раза то очевидно что
система будет приведена полностью в исходное состояние, т.е.
\begin{eqnarray}
  \hat{S}_{k, k+1}^2 \left|\psi\right> =
  e^{2 i \phi} \left|\psi\right> = \left|\psi\right>
  \nonumber
\end{eqnarray}
откуда следует, что $\phi$ может принимать два значения $0$ и $\pi$.
Таким образом мы можем разделить все частицы на два класса. Для
первого из них волновая функция не меняется при перестановке двух
частиц ($\phi = 0$), а для второго меняет знак ($\phi = \pi$).


Все элементарные частицы делятся на два класса: бозоны и
фермионы. Рассмотрим состояние с определенным числом частиц $n$
$\left|\psi\right> = \left|n\right>$. В случае бозонов мы будем иметь 
\begin{equation}
\left|\psi_b\right> = \left|n\right>_b.
\label{eqAddQuantBoson}
\end{equation}
Операторы рождения $\hat{a}_b^{+}$ и уничтожения $\hat{a}_b$ действуют на
состояние (\ref{eqAddQuantBoson}) следующим образом
\begin{eqnarray}
\hat{a}_b^{+}\left|n\right>_b = \sqrt{n+1}\left|n+1\right>_b, 
\nonumber \\
\hat{a}_b\left|n\right>_b = \sqrt{n}\left|n\right>_b
\nonumber
\end{eqnarray}

Для фермионов возможны два состояния: $\left|0\right>_f$ и
$\left|1\right>_f$ при этом соотвествующие операторы рождения
$\hat{a}_f^{+}$ и уничтожения $\hat{a}_f$ действуют следующим образом
\begin{eqnarray}
\hat{a}_f^{+}\left|0\right>_f = \left|1\right>_f, 
\nonumber \\
\hat{a}_f^{+}\left|1\right>_f = 0, 
\nonumber \\
\hat{a}_f\left|0\right>_f = 0, 
\nonumber \\
\hat{a}_f\left|1\right>_f = \left|0\right>_f.
\nonumber
\end{eqnarray}


\subsection{Бозоны и распределение Бозе-Эйнштейна}
Системы из нескольких бозонов описываются волновыми функциями, которые
являются симметричными относительно перестановок частиц.

TBD

\subsection{Фермионы. Распределение Ферми-Дирака. Принцип запрета Паули}

TBD
