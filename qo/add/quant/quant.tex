%% -*- coding:utf-8 -*- 
\section{Процедура квантования}
Как мы уже выяснили физическим величинам в квантовой механике
соответствуют самосопряженные операторы. Если нам известна волновая
функция (например с помощью уравнения Шредингера), то зная выражение
для оператора интересующей нас физической величины можно производить
практические расчеты, например средних значений.

Процедура вывода соотношений которым подчиняются операторы называется
квантованием. 

Ниже приведена процедура квантования на примере оператора углового
момента.

\input ./add/quant/figquant.tex
 
\subsection{Классика}
Рассматриваемая система состоит из материальной частицы движущейся по
кругу. Обобщенными координатами, описывающими частицу являются угол
$\theta$ и радиус окружности $r$, который считается постоянным: $r =
\left. const \right|_t$ ( см. \autoref{figAddQuantAngleMoment}).

Гамильтониан системы имеет следующий вид 
\begin{eqnarray}
\mathcal{H} = T + U = \frac{m v^2}{2} + U\left( r \right) = 
\nonumber \\
= \frac{m r^2 \dot{\theta}^2 }{2} + U\left( r \right),
\nonumber
\end{eqnarray}
где $T = \frac{m v^2}{2}$ - кинетическая энергия частицы, 
а $U$ - потенциальная энергия, которая в силу симметрии задачи, не
зависит от угла $\theta$ и зависит только от растояния $r$.

Лангранжин системы
\[
\mathcal{L} = T - U = \frac{m r^2 \dot{\theta}^2 }{2} - U\left( r \right)
\]
Угловой момент (обобщенный импульс соотвествующий обобщенной
координате $\theta$) определяется как 
\begin{equation}
l = \frac{d \mathcal{L}}{d \dot{\theta}} = 
m r^2 \dot{\theta} = I \dot{\theta},
\label{eqAngualrMomentumClass}
\end{equation}
где через $I$ обозначено $I = m r^2$ - момент инерции (moment of
inertia).

Уравнения движения для координаты r:
\[
\frac{ \partial \mathcal{L} }{\partial r} = 
\frac{d}{d t} \frac{\partial \mathcal{L}}{\partial \dot{r}}
\]
 откуда
\begin{eqnarray}
\frac{ \partial \mathcal{L} }{\partial r} = 0,
\nonumber \\
\frac{ \partial T }{\partial r} - \frac{ \partial U }{\partial r} = 0,
\nonumber \\
\frac{\partial U}{\partial r} = m r \dot{\theta}^2,
\nonumber
\end{eqnarray}
 или же
\[
 U = \frac{m r^2 \dot{\theta}^2}{2} = \frac{l^2}{2 I}
\]
 
Таким образом гамильтониан рассматриваемой системы
\begin{eqnarray}
\mathcal{H} = \frac{m r^2 \dot{\theta}^2 }{2} + \frac{l^2}{2 I} = 
\nonumber \\
= \frac{l^2}{2 I} + \frac{l^2}{2 I} = \frac{l^2}{I}
\label{eqHClassical}
\end{eqnarray}

\subsection{Квантование}

Пусть $\left|\psi\right>$ - собственная функция оператора углового
момента $\hat{L}$ отвечающая собственному числу $l$:
\begin{equation}
\hat{L} \left|\psi\right> = l \left|\psi\right>.
\label{eqLPsi}
\end{equation}
Эта же волновая функция должна удовлетворять уравнению Шредингера
\[
i \hbar \frac{\partial \left|\psi\right>}{ \partial t} = 
\hat { \mathcal{H} } \left|\psi\right>
\]
Из \eqref{eqHClassical} имеем
\[
\hat { \mathcal{H} } = \frac{\hat{L}^2}{I}
\]
откуда с учетом \eqref{eqLPsi}
\[
i \hbar \frac{\partial \left|\psi\right>}{ \partial t} = 
\frac{1}{I} \hat{L} \hat{L} \left|\psi\right> = 
\frac{l^2}{I} \left|\psi\right>,
\]
или же
\[
\frac{\partial \left|\psi\right>}{ \partial t} = 
\frac{-i l^2}{\hbar I} \left|\psi\right>
\]
Таким образом
\[
\left|\psi\left(t \right)\right> = C \cdot exp\left\{\frac{-i l^2}{\hbar
    I} t
\right\}
\]
Волновая функция должна удовлетворять условию периодичности
\[
\left|\psi\left(t \right) \right>= \left|\psi\left(t + T \right)\right>
\]
где $T$ - период колебаний. Его можно найти из
\eqref{eqAngualrMomentumClass}:
\begin{eqnarray}
\dot{\theta} = \frac{l}{I},
\nonumber \\
\theta = \theta_0 + \frac{l \cdot t}{I},
\nonumber \\
2 \pi = \frac{l \cdot T}{I},
\nonumber \\
T = \frac{2 \pi I}{l}
\nonumber 
\end{eqnarray}
таким образом
\[
2 i \pi n = \frac{-i l^2}{\hbar I} T = 
\frac{2 \pi I}{l} \frac{-i l^2}{\hbar I}
\]
откуда
\[
l = -\hbar n
\]
