%% -*- coding:utf-8 -*-
\section{Составные системы}
\label{sec:add:quantum:composite}
Системы состоящие из нескольких частей ведут себе принципиально
различным образом для случая классических и квантовых систем.


В качестве примера рассмотрим систему из двух частиц. Положение первой
из них описывается 3 координатами, которые представляют собой точку в
некотором линейном пространстве $L_1$: $(x_1, y_1, z_1) \in L_1$. Для второй
допустим имеем 2 координаты в пространстве $L_2$ полностью задающие
расположение:  
$(x_2, y_2) \in L_2$. Очевидно что для полного описания положения двух
частиц нам необходимо 5 координат: $(x_1, y_1, z_1, x_2, y_2) \in L_1
\times L_2$. Таким образом составные системы, в классическом случае,
описываются точками в пространстве $L^{classical}$, которое является декартовым
произведение исходных: $L = L_1 \times L_2$. Отличительной
особенностью декартова произведения является то, что размерности
соответствующих пространств складываются:
\[
\dim{L^{classical}} = \dim{L_1} + \dim{L_2}.
\] 

В квантовом случае составная система описывается векторами (точками) в
пространстве которое является тензорным произведением исходных: 
\[
L^{quantum} = L_1 \otimes L_2.
\]
В этом пространстве имеется 6 базисных векторов, и соответственно, для
описание положения системы необходимо 6 чисел:
$(x_1 \cdot x_2, y_1 \cdot x_2, z_1 \cdot x_2, 
x_1 \cdot y_2, y_1 \cdot y_2, z_1 \cdot y_2) \in L^{quantum}$. 
Соответствующим образом размерности перемножаются:
\[
\dim{L^{quantum}} = \dim{L_1} \cdot \dim{L_2}.
\] 
