%% -*- coding:utf-8 -*- 
\section{Динамика изменения волновой функции}
\label{AddWaveFunc}
Волновая функция $\left| \phi \right>$ может изменяться посредством
двух механизмов:
\begin{itemize}
\item Редукция волновой функции во время измерения
\item Уравнение Шрёдингера в промежутках между двумя последовательными
  измерениями 
\end{itemize}

\subsection{Уравнение Шрёдингера}
Изменение состояния чистой квантовой системы между двумя
последовательными измерениями описывается следующим уравнением (Шрёдингера)
\begin{equation}
i \hbar \frac{\partial \left| \phi \right>}{\partial t} = \hat{\mathcal{H}}
\left| \phi \right>.
\label{eqAddWaveFunc_Shredinger}
\end{equation}

Уравнение (\ref{eqAddWaveFunc_Shredinger}) является обратимым и,
соответственно, не применимо к описанию изменения волновой функции в
момент измерения.

\subsubsection{Уравнение Шрёдингера в представлении взаимодействия}
\label{AddWaveFuncInter}
Допустим что в гамильтониане можно выделить две части:
\begin{equation}
\hat{\mathcal{H}} = \hat{\mathcal{H}}_0 + \hat{\mathcal{V}}.
\nonumber
\end{equation}

Введем следующее преобразование волновой функции:
\[
\left| \phi \right>_I = 
\exp{\left(\frac{i \hat{\mathcal{H}}_0 t}{\hbar}\right)}
\left| \phi \right>
\]
и посмотрим чему будет равно следующее выражение:
\begin{eqnarray}
i \hbar \frac{\partial \left| \phi \right>_I}{\partial t} = 
i \hbar \frac{i \hat{\mathcal{H}}_0}{\hbar} 
\exp{\left(\frac{i \hat{\mathcal{H}}_0 t}{\hbar}\right)}
\left| \phi \right> +
\exp{\left(\frac{i \hat{\mathcal{H}}_0 t}{\hbar}\right)}
i \hbar \frac{\partial \left| \phi \right>}{\partial t} = 
\nonumber \\
= - \hat{\mathcal{H}}_0 
\exp{\left(\frac{i \hat{\mathcal{H}}_0 t}{\hbar}\right)}
\left| \phi \right> +
\exp{\left(\frac{i \hat{\mathcal{H}}_0 t}{\hbar}\right)}
\left(
\hat{\mathcal{H}}_0 + \hat{\mathcal{V}}
\right)
\left| \phi \right> =
\nonumber \\ 
- \hat{\mathcal{H}}_0 
\exp{\left(\frac{i \hat{\mathcal{H}}_0 t}{\hbar}\right)}
\left| \phi \right> +
\hat{\mathcal{H}}_0 
\exp{\left(\frac{i \hat{\mathcal{H}}_0 t}{\hbar}\right)}
\left| \phi \right>
+
\exp{\left(\frac{i \hat{\mathcal{H}}_0 t}{\hbar}\right)}
 \hat{\mathcal{V}}
\left| \phi \right> =
\nonumber \\
= 
\exp{\left(\frac{i \hat{\mathcal{H}}_0 t}{\hbar}\right)}
 \hat{\mathcal{V}}
\left| \phi \right> = 
\nonumber \\
= 
\exp{\left(\frac{i \hat{\mathcal{H}}_0 t}{\hbar}\right)}
 \hat{\mathcal{V}}
\exp{\left( - \frac{i \hat{\mathcal{H}}_0 t}{\hbar}\right)}
\exp{\left(\frac{i \hat{\mathcal{H}}_0 t}{\hbar}\right)}
\left| \phi \right> = 
\nonumber \\
= 
 \hat{\mathcal{V}}_I \left| \phi \right>_I,
\nonumber
\end{eqnarray}
где 
\begin{equation}
\hat{\mathcal{V}}_I = 
\exp{\left(\frac{i \hat{\mathcal{H}}_0 t}{\hbar}\right)}
 \hat{\mathcal{V}}
\exp{\left( - \frac{i \hat{\mathcal{H}}_0 t}{\hbar}\right)}
\label{eqAddWaveFunc_VInter}
\end{equation} 
гамильтониан взаимодействия в представлении взаимодействия.

Таким образом получаем уравнение Шрёдингера в представлении
взаимодействия:
\begin{equation}
i \hbar \frac{\partial \left| \phi \right>_I}{\partial t} = \hat{\mathcal{V}}_I
\left| \phi \right>_I.
\label{eqAddWaveFunc_ShredingerInter}
\end{equation}


\subsubsection{Уравнение движения матрицы плотности}
Из соотношения (\ref{eqAddWaveFunc_Shredinger}) имеем
\begin{eqnarray}
i \hbar \frac{\partial \left| \phi \right>}{\partial t} = \hat{\mathcal{H}}
\left| \phi \right>,
\nonumber \\
- i \hbar \frac{\partial \left< \phi \right|}{\partial t} = \hat{\mathcal{H}}
\left< \phi \right|,
\nonumber
\end{eqnarray}
таким образом для матрицы плотности 
$\hat{\rho} = \left| \phi \right>\left< \phi \right|$ получаем
\begin{eqnarray}
i \hbar \frac{\partial \hat{\rho} }{\partial t} = 
i \hbar \frac{\partial  \left| \phi \right>\left< \phi \right|
}{\partial t} = 
i \hbar \left( \frac{\partial \left| \phi \right>}{\partial t}\left< \phi
\right| +
\left| \phi \right> \frac{\partial \left< \phi \right|}{\partial t}
\right) =
\nonumber \\
=  \hat{\mathcal{H}} \left| \phi \right>\left< \phi \right| -
\left| \phi \right>\left< \phi \right|\hat{\mathcal{H}} = 
\left[ \hat{\mathcal{H}}, \hat{\rho} \right]
\label{eqAddWaveFunc_Pho}
\end{eqnarray}
Уравнение (\ref{eqAddWaveFunc_Pho}) часто называется квантовым
уравнением Лиувилля и уравнением фон Неймана.

\subsubsection{Оператор эволюции. Представление Гейзенберга и
  представление Шредингера}

Изменение волновой функции по закону (\ref{eqAddWaveFunc_Shredinger})
может быть также описано с помощью некоторого оператора (эволюции) $\hat{U}\left(t,t_0\right)$:
\begin{equation}
\left| \phi\left(t\right) \right> = 
\hat{U}\left(t,t_0\right)\left| \phi\left(t_0\right) \right>.
\label{eqAddWaveFunc_ShredingerU}
\end{equation}

Уравнение (\ref{eqAddWaveFunc_Shredinger}) может быть переписано в
виде
\begin{equation}
\left| \phi\left(t\right) \right> = 
\exp\left( -\frac{i}{\hbar} \hat{\mathcal{H}} \left( t - t_0 \right)  \right)
\left| \phi\left(t_0\right) \right>,
\nonumber
\end{equation}
откуда для оператора эволюции имеем
\begin{equation}
\hat{U}\left(t,t_0\right) = 
\exp\left( -\frac{i}{\hbar} \hat{\mathcal{H}} \left( t - t_0 \right)  \right)
\label{eqAddDiracEvolutionOper}
\end{equation}

Оператор эволюции - унитарный. Действительно:
\begin{eqnarray}
\hat{U}\left(t,t_0\right)\hat{U}^+\left(t,t_0\right) = 
\nonumber \\
= \exp\left( -\frac{i}{\hbar} \hat{\mathcal{H}} \left( t - t_0 \right)
\right)
\exp\left( +\frac{i}{\hbar} \hat{\mathcal{H}} \left( t - t_0 \right)
\right)
= \hat{I}
\nonumber
\end{eqnarray}

Наряду с представлением Шредингера где операторы от времени не зависят
а меняются волновые функции существует представление Гейзенберга где
операторы меняются во времени.

Очевидно средние значения операторов не должны зависеть от
представления:
\begin{eqnarray}
\left< \phi_H\left(t_0\right) \right|\hat{A}_H\left(t\right)\left| 
\phi_H\left(t_0\right) \right> = 
\left< \phi_S\left(t\right) \right|\hat{A}_S\left| 
\phi_S\left(t\right) \right> = 
\nonumber \\
=
\left<
\phi_H\left(t_0\right)\right|\hat{U}^+\left(t,t_0\right)\hat{A}_S\hat{U}\left(t,t_0\right)\left|
\phi_H\left(t_0\right) \right>,
\nonumber
\end{eqnarray}
откуда с учетом $\hat{A}_H\left(t_0\right) = \hat{A}_S\left(t_0\right)$ получаем закон эволюции операторов в представлении Гейзенберга:
\begin{equation}
\hat{A}_H\left(t\right) = \hat{U}^+\left(t,t_0\right)\hat{A}_H\left(t_0\right)\hat{U}\left(t,t_0\right)
\label{eqAddWaveFunc_HeizenbergU}
\end{equation}

При этом уравнение для оператора $\hat{A}_H$ будет выглядеть следующим
образом:
\begin{eqnarray}
  \frac{\partial \hat{A}_H}{\partial t} =
  \frac{i}{\hbar} \hat{\mathcal{H}}
  \hat{U}^+\left(t,t_0\right)\hat{A}_H\left(t_0\right)\hat{U}\left(t,t_0\right)
  -
  \nonumber \\
  - \frac{i}{\hbar}
  \hat{U}^+\left(t,t_0\right)\hat{A}_H\left(t_0\right)\hat{U}\left(t,t_0\right)
  \hat{\mathcal{H}} =
  \frac{i}{\hbar} \left[\hat{\mathcal{H}}, \hat{A}_H \right]
  \label{eqAddWaveFunc_HeizenbergT}
\end{eqnarray}

\subsection{Различия между чистыми и смешанными
  состояниями. Декогеренция}
Особый интерес представляет собой различие между чистыми и смешанными
состояниями, в частности - каким образом происходит переход от чистых
состояний к смешанным.

Рассмотрим двухуровневое состояние
(см. \autoref{figAddDecoherenceModel}). В чистом состоянии оно
описывается 
следующей волновой функцией:
\begin{equation}
\left|\phi\right> = c_a \left|a \right> + c_b \left|b\right>,
\nonumber
\end{equation}
соответствующая матрица плотности имеет вид
\begin{eqnarray}
\hat{\rho} = \left|\phi\right>\left<\phi\right| =
\nonumber \\
= 
\left|c_a\right|^2 \left|a\right>\left<a\right| + 
\left|c_b\right|^2 \left|b\right>\left<b\right| +
\nonumber \\
+
c_a c_b^{\ast}\left|a\right>\left<b\right| +
c_b c_a^{\ast}\left|b\right>\left<a\right|,
\label{eqAddDecoherencePure}
\end{eqnarray}
или в матричном виде
\begin{eqnarray}
\hat{\rho} = 
\begin{pmatrix}
\left|c_a\right|^2 & c_a c_b^{\ast} \\
c_b c_a^{\ast} & \left|c_b\right|^2 \\
\end{pmatrix}.
\nonumber
\end{eqnarray}

Матрица плотности для смешанного состояния имеет только диагональные
элементы:
\begin{eqnarray}
\hat{\rho} = 
\begin{pmatrix}
\left|c_a\right|^2 & 0 \\
0 & \left|c_b\right|^2 \\
\end{pmatrix} = 
\nonumber \\
=
\left|c_a\right|^2 \left|a\right>\left<a\right| + 
\left|c_b\right|^2 \left|b\right>\left<b\right|.
\label{eqAddDecoherenceMix}
\end{eqnarray}

\input add/quant/figdecoherence.tex

Переход от (\ref{eqAddDecoherencePure}) к (\ref{eqAddDecoherenceMix})
называется декогеренцией. В описании процесса декогеренции мы будем следовать
 \cite{bMensky2001}. 

Отличие смешанных состояний от чистых проявляется во влиянии окружения
$\mathcal{E}$. В случае чистых состояний рассматриваемая система и ее
окружение независимы, т. е.
\begin{equation}
\left|\phi\right>_{pure} = \left|\phi\right>_{at} \otimes
\left|\mathcal{E}\right>.
\label{eqAddDecoherencePhiPure}
\end{equation}

В случае смешанных состояний атом и его окружение образуют так
называемое перепутанное состояние в котором состояниям
$\left|a\right>$ и $\left|b\right>$ соответствуют различимые 
состояния окружения $\left|\mathcal{E}_a\right>$ и
$\left|\mathcal{E}_b\right>$.
\begin{equation}
\left|\phi\right>_{mix} = c_a\left|a\right> \left|\mathcal{E}_a\right>
+ c_b\left|b\right> \left|\mathcal{E}_b\right>.
\label{eqAddDecoherencePhiMix}
\end{equation}

Матрица плотности соответствующая (\ref{eqAddDecoherencePhiMix}) имеет вид

\begin{eqnarray}
\hat{\rho}_{mix} = \left|\phi\right>_{mix}\left<\phi\right|_{mix} = 
\nonumber \\
= 
\left|c_a\right|^2 \left|a\right>\left<a\right| \otimes
\left|\mathcal{E}_a\right>\left<\mathcal{E}_a\right| + 
\left|c_b\right|^2 \left|b\right>\left<b\right| \otimes
\left|\mathcal{E}_b\right>\left<\mathcal{E}_b\right| +
\nonumber \\
+
c_a c_b^{\ast}\left|a\right>\left<b\right| \otimes
\left|\mathcal{E}_a\right>\left<\mathcal{E}_b\right| +
c_b c_a^{\ast}\left|b\right>\left<a\right| \otimes
\left|\mathcal{E}_b\right>\left<\mathcal{E}_a\right|.
\label{eqAddDecoherenceRhoMix}
\end{eqnarray}
Если теперь применить к выражению (\ref{eqAddDecoherenceRhoMix})
усреднение по переменным окружения, то получим
\begin{eqnarray}
\left<\hat{\rho}_{mix}\right>_{\mathcal{E}} = 
Sp_{\mathcal{E}}\left(\hat{\rho}\right) = 
\nonumber \\
=
\left<\mathcal{E}_a\right|\hat{\rho}_{mix}\left|\mathcal{E}_a\right> +
\left<\mathcal{E}_b\right|\hat{\rho}_{mix}\left|\mathcal{E}_b\right>
= 
\nonumber \\
= \left|c_a\right|^2 \left|a\right>\left<a\right| + 
\left|c_b\right|^2 \left|b\right>\left<b\right|.
\label{eqAddDecoherenceRhoMixFin}
\end{eqnarray}
Выражение  (\ref{eqAddDecoherenceRhoMixFin}) получено в предположении
ортонормированного базиса $\left\{\left|\mathcal{E}_a\right>,
\left|\mathcal{E}_b\right>\right\}$: 
\begin{eqnarray}
\left<\mathcal{E}_a\right.\left|\mathcal{E}_a\right> = 
\left<\mathcal{E}_b\right.\left|\mathcal{E}_b\right> = 1,
\nonumber \\
\left<\mathcal{E}_a\right.\left|\mathcal{E}_b\right> = 
\left<\mathcal{E}_b\right.\left|\mathcal{E}_a\right> = 0.
\label{eqAddDecoherenceMixECond}
\end{eqnarray}

Условия (\ref{eqAddDecoherenceMixECond}) являются ключевыми для
понимания того почему рассматриваемый базис атомной системы является
выделенным и почему например в случае смешанных состояний не
рассматривают другие базисы такие как базис полученный преобразованием
Адамара по отношению к исходному:
\begin{eqnarray}
\left|\mathcal{A}\right> = \frac{\left|a\right> + \left|b\right>}
              {\sqrt{2}},
\nonumber \\
\left|\mathcal{B}\right> = \frac{\left|a\right> - \left|b\right>}
              {\sqrt{2}}.
\label{eqAddDecoherenceBaseWrong}
\end{eqnarray}
Состояния окружения соответствующие базису
(\ref{eqAddDecoherenceBaseWrong}) не являются ортогональными откуда
следует невозможность использования (\ref{eqAddDecoherenceBaseWrong})
в качестве базисных векторов для смешанных состояний. 

Процесс декогеренции, т. е. перехода от
(\ref{eqAddDecoherencePhiPure}) к (\ref{eqAddDecoherencePhiMix}) может
быть описан с помощью уравнения Шредингера, и следовательно
теоретически является обратимым. Единственное требование -
ортогональность различимых состояний окружения: 
$\left<\mathcal{E}_a\right.\left|\mathcal{E}_b\right> = 0$. Это
требование всегда выполняется для макроскопических систем, где
состояние зависит от очень большого числа переменных. При этом в
случае макроскопических систем стоит отметить, что существует большое
число возможных вариантов конечных состояний
$\left|\mathcal{E}_{a,b}\right>$ 
в силу чего обратный процесс становится практически не реализуемым в
силу того что необходимо контролировать большое число возможных
переменных которыми описывается состояние окружения. В этом смысле
процесс декогеренции имеет туже природу что второй закон термодинамики 
(возрастания энтропии), который описывает
необратимые процессы.
\footnote{Надо быть немного аккуратным здесь
  поскольку второй закон термодинамики применим к закрытым системам а
  сами процессы декогеренции происходят в открытых системах}

%% В качесте примера можно рассмотреть так
%% называемого кота Шредингера, который может находится в двух состояниях
%% $\left|L\right>$ - кот жив и $\left|D\right>$ - кот мертв. 
%% Вместе с этими двумя состояниями согласно принципу суперпозиции 
%% вполне допустимо состояние $\left|\psi\right> = C_L \left|L\right> + C_D
%% \left|D\right>$ - суперпозиции живого и мертвого кота. Этот пример
%% возмущает здравый смысл, который основан на жизненом опыте среди
%% классических систем в которых такие состояния одновременно и живого и мертвого
%% невозможны. Мы наблюдаем в классическом мире так называемые смешанные
%% состояния для которых принцип суперпозиции неприменим. В случае
%% ``классического кота'' можно говорить о том что с вероятностью
%% $\left|C_L\right|^2$ он находится в состоянии $\left|L\right>$ и с
%% вероятностью $\left|C_D\right|^2$ - в состоянии
%% $\left|D\right>$. Такие системы называются смешанными. Очевидно что
%% должно существовать принципиальное отличие между чистыми и смешанными
%% системами. Таким отличием является влияние внешнего окружения. 

\subsection{Редукция волновой функции. Измерение в квантовой механике}
\label{sec:add:reduction}

Процесс выбора (результата измерения) один из самых сложных в
квантовой механике. В отличии от детерминистского изменения волновой
функции, описываемого уравнением Шрёдингера
(\ref{eqAddWaveFunc_Shredinger}), процесс измерения носит случайный
характер и для его описания следует применять другие уравнения. 

\input ./add/quant/figmeasur.tex

Рассмотрим вначале чистые состояния и предположим, что производится
измерение физической наблюдаемой, 
описываемой оператором $\hat{L}$. Собственные числа и собственные
функции этого оператора $\left\{ l_k \right\}$ и 
$\left\{ \left|l_k\right> \right\}$ соответственно. В момент
измерения показания прибора могут принимать значения соответствующие
собственным числам измеряемого оператора
(см. \autoref{figAddMeasur}). Допустим, что показание 
прибора - $l_n$ в этом случае волновая функция должна быть 
$\left|l_n\right>$, т. о. произошло следующее
изменение волновой функции:
\[
\left| \phi \right> \rightarrow \left|l_n\right>,
\] 
которое может быть описано действием оператора проецирования 
$\hat{P}_n = \left|l_n\right> \left<l_n\right|$ (\ref{eqDiracProektor}):
\[
\hat{P}_n \left| \phi \right> = c_n\left|l_n\right>.
\]

Не существует способа предсказать результат который будет получен в
результате единичного измерения. Вместе с тем можно сказать с какой
вероятностью будет получен тот или иной результат.

Действительно в случае смешанного состояния
\begin{equation}
\hat{\rho} = 
\sum_n \left|c_n\right|^2 \left|l_n\right>\left<l_n\right|
\nonumber
\end{equation}
коэффициенты $P_n = \left|c_n\right|^2$
задают вероятности обнаружить систему в состоянии $\left|l_n\right>$. 

Для чистого состояния
\begin{equation}
\left| \phi \right> = 
\sum_n c_n \left|l_n\right>
\nonumber
\end{equation}
мы также имеем, что вероятность обнаружить систему в состоянии
$\left|l_n\right>$ задается числом $P_n = \left|c_n\right|^2$. 

Основное отличие чистых и смешанных состояний с точки зрения измерения
заключается в том, что в первом случае (чистое состояние) в процессе
измерение меняется волновая функция, т. е. само состояние. При этом
если в процессе измерения было получено некоторое конечное состояние
$\left|l_i\right>$, то нельзя сказать, что оно было таким же и до
измерения. Смешанные состояния ведут себя подобно классическим объектам,
т. е. если в процессе измерения было получено состояние
$\left|l_i\right>$, то можно утверждать, что оно было таким же и до
измерения, а само измерение представляет собой выбор
одного состояния из многих возможных.

\begin{example}
\emph{Выбор из урны с шарами двух цветов}
Допустим у нас имеется урна с 4 шарами. С вероятностью $\frac{1}{2}$
будет извлечен либо белый либо черный шар. Допустим что в результате
эксперимента был получен черный шар. Если рассматриваемая система
является квантовой и находится в смешанном состоянии 
(см. \autoref{figAddMixStateExample}), то состояние
извлеченного шара (цвет) не изменилось в результате эксперимента. 

\input add/quant/figmixstateexample.tex
\input add/quant/figpurestateexample.tex

Если рассматриваемая система является
чистой(см. \autoref{figAddPureStateExample}), то состояние каждого
шара описывается суперпозицией двух цветов - черного и белого. Таким
образом в результате эксперимента эта суперпозиция разрушается и шар
приобретает определенный цвет (черный в нашем случае), т. е. можно
сказать что цвет шара меняется.
\end{example}
