%% -*- coding:utf-8 -*- 
\section{Дираковская  формулировка  квантовой  механики}
\label{AddDirac}
В курсе лекций по квантовой оптике мы будем всюду использовать формализм
Дирака \cite{bDiracPrincipleQuantumMechanic}. В обычной формулировке квантовой
механики мы имеем дело с волновыми функциями, например $\psi\left(q,
t\right)$ - волновая функция в 
координатном представлении. Одно и то же состояние системы можно
описать волновыми функциями в различных представлениях, связанных друг
с другом линейными преобразованиями. Например, волновая функция в
импульсном представлении связана с волновой функцией в координатном
представлении равенством 
\begin{equation}
\phi\left(p, t\right) = \frac{1}{2 \pi \hbar} \int_{-\infty}^{+\infty}
\psi \left(q, t\right) e^{-i \frac{p q}{\hbar}} dq
\end{equation}
Главное здесь, что одно и то же состояние можно описывать волновыми
функциями, выраженными через различные переменные. Отсюда следует, что
можно ввести более общее образование, характеризующее состояние
системы независимо от представления. Для такого образования Дирак ввел
понятие волнового вектора, или вектора состояния, обозначаемого: 
\begin{equation}
\left| \dots \right>
\end{equation}
и называемого кет-вектором.

\subsection{Кет-вектор}
$\left| \dots \right>$ общее обозначение кет-вектора;  $\left| a
\right>$,  $\ket{ x }$, $\left| \psi \right>$ и т.д. означают
кет-векторы, описывающие некоторые частные состояния, символы которых
записываются внутри скобок. 

\subsection{Бра-векторы}
Каждому кет-вектору соответствует сопряженный ему
бра-вектор. Бра-вектор обозначается: 
\begin{equation}
\left< \dots \right|, \quad 
\bra{ a }, \quad  
\left< \psi \right|.
\end{equation}

Названия бра- и кет-векторы образованы от первой и второй половины
английского слова  {\itshape bra-cket}  (скобка).

Таким образом, бра-векторам
$\bra{ a }$,  $\bra{ x }$, $\bra{ \psi }$
соответствуют сопряженные им кет-векторы  
$\ket{ a }$,  $\ket{ x }$, $\ket{ \psi }$
и наоборот. Для векторов состояний справедливы те же основные
соотношения, которые справедливы для волновых функций:  
\begin{equation}
\ket{ u } = \ket{ a }  + \ket{ b }, \quad 
\bra{ u } = \bra{ a }  + \bra{ b }, \quad 
\ket{ v } = l \ket{ a }, \quad  
\bra{ v } = l \bra{ a }.
\end{equation}
Бра- и кет-векторы связаны между собой операцией эрмитового 
сопряжения:
\begin{equation}
\ket{ u } = \left( \bra{ u } \right)^{\dag}, \quad 
\bra{ u } = \left( \ket{ u } \right)^{\dag}.
\end{equation}

В известных случаях это сводится к следующим соотношениям:
\[
\left( \psi\left( q \right) \right)^{\dag} = \psi^{*}\left( q \right)
\]
для волновой функции в координатном представлении;
\[
\left(
\begin{array} {c} 
a_1 \\
a_2 \\
\vdots \\
a_n
\end{array}
 \right)^{\dag} = 
\left( a_1^{*}, a_2^{*}, \cdots, a_n^{*}\right)
\]
в матричном представлении.

При помощи бра- и кет-векторов можно определить скалярное 
произведение
\begin{equation}
\bra{ v }\ket{ u } = \bra{ u }\ket{ v }^{*}.
\label{eqAddDirac_swap}
\end{equation}

В конкретных случаях это означает:
\[
\left< \psi \right|\left. \phi \right> = 
\int \psi^{*} \phi dq
\]
в координатном представлении;
\[
\bra{ a }\ket{ b } = 
\left( a_1^{*}, a_2^{*}, \cdots, a_n^{*}\right) 
\left(
\begin{array} {c} 
b_1 \\
b_2 \\
\vdots \\
b_n
\end{array}
 \right) = 
a_1^{*} b_1 +  a_2^{*} b_2 + \cdots + a_n^{*} b_n
\]
в матричном представлении.

Из соотношения \eqref{eqAddDirac_swap} следует, что норма вектора
вещественна. Дополнительно полагаем, что норма вектора положительна
или равна нулю: 
$\bra{ a }\ket{ a } \geq 0$.

\subsection{Операторы}
В квантовой механике используют линейные операторы. Операторы
связывают один вектор состояния с другим: 
\begin{equation}
\ket{ q } = \hat{L}\ket{ p }
\label{eqAddDirac_operator_property1}
\end{equation}
Сопряженное равенство имеет вид
\begin{equation}
\bra{ q } = \bra{ p }  \hat{L}^{\dag}
\label{eqAddDirac_operator_property2}
\end{equation}
где $\hat{L}^{\dag}$ -  оператор, сопряженный оператору $\hat{L}$.

Приведем некоторые соотношения, справедливые для линейных 
операторов:
\begin{eqnarray}
\hat{L}^{++} = \hat{L}, \quad
\left(l \hat{L} \ket{ a } \right)^{\dag} = 
l^{*} \bra{ a } \hat{L}^{\dag}, 
\nonumber \\
\left(\left(\hat{L_1} + \hat{L_2} \right) \ket{ a } \right)^{\dag} = 
\bra{ a } \left(\hat{L_1}^{\dag} + \hat{L_2}^\dag \right), 
\nonumber \\
\left(\left(\hat{L_1} \hat{L_2} \right) \ket{ a } \right)^{\dag} = 
\bra{ a } \left(\hat{L_2}^{\dag} \hat{L_1}^\dag \right),
\nonumber \\
\left(\left(\hat{L_1} \hat{L_2} \hat{L_3}\right) \ket{ a } \right)^{\dag} = 
\bra{ a } \left(\hat{L_3}^{\dag} \hat{L_2}^\dag \hat{L_1}^\dag \right), 
\mbox{ и т.д.}
\label{eqAddDirac_propert}
\end{eqnarray}

Заметим, что алгебра операторов совпадает с алгеброй квадратных
матриц. Матричные элементы операторов обозначаются следующим образом: 
\begin{equation}
\bra{a}\hat{L}\ket{b} = L_{ab}
\end{equation}

Для матричных элементов справедливы равенства
\begin{equation}
\bra{a}\hat{L}\ket{b}^{*} = 
\bra{b}\hat{L}^{\dag}\ket{a}, \quad
\bra{a}\hat{L_1}\hat{L_2}\ket{b}^{*} = 
\bra{b}\hat{L_2}^{\dag}\hat{L_1}^\dag\ket{a}
\end{equation}


\subsection{Собственные  значения  и  собственные  векторы  операторов} 
Собственные значения и собственные векторы операторов определяются равенством
\begin{equation}
\hat{L} \ket{l_n} = l_n \ket{l_n},
\end{equation}
где $l_n$ собственное значение; $\ket{l_n}$ собственный вектор.

Для бра-векторов имеем аналогичные равенства:
\begin{equation}
\bra{d_n} \hat{D}  = d_n \bra{d_n}.
\end{equation}

Если операторы соответствуют наблюдаемым величинам, они должны быть самосопряженными:
\begin{equation}
\hat{L}  = \hat{L}^{\dag}.
\label{eqAddDirac_ermit}
\end{equation}

Собственные значения самосопряженного (эрмитова) оператора
вещественны. Действительно из 
\[
\hat{L} \ket{ l } = l \ket{ l }
\]
следует что 
\[
\bra{ l } \hat{L} \ket{ l } = l \bra{ l }
\ket{ l }.
\]
С другой стороны, вспоминая про \eqref{eqAddDirac_propert}:
$\bra{ l } \hat{L}^{\dag} = l^{*} \bra{ l }$, из
\eqref{eqAddDirac_ermit} имеем
\[
\bra{ l } \hat{L} \ket{ l } = l^{*} \bra{ l }
\ket{ l }.
\] 
Таким образом $l\bra{ l }
\ket{ l } = l^{*}\bra{ l }
\ket{ l }$, т. е. $l  = l^{*}$

Собственные векторы самосопряженного оператора ортогональны. 
Действительно рассмотрим два собственных вектора 
$\ket{ l_1 }$ и $\ket{ l_2 }$:
\[
\hat{L} \ket{ l_1 } = l_1 \ket{ l_1 }, \quad
\hat{L} \ket{ l_2 } = l_2 \ket{ l_2 }
\]
Из второго соотношения получаем
\[
\bra{ l_1 } \hat{L} \ket{ l_2 } = l_2 \bra{ l_1 } \ket{ l_2 }
\]
С учетом вещественности собственных чисел и соотношения
\eqref{eqAddDirac_ermit} для вектора $\ket{ l_1 }$ получим:
\[
\bra{ l_1 } \hat{L} = l_1 \bra{ l_1 }.
\]
Откуда
\[
\bra{ l_1 } \hat{L} \ket{ l_2 } = l_1 \bra{ l_1 } \ket{ l_2 }.
\] 
Таким образом
\[
\left(l_1 - l_2\right) \bra{ l_1 } \ket{ l_2 } = 0, 
\quad \mbox{т. е. } \bra{ l_1 } \ket{ l_2 } = 0,
\mbox{ т. к. } l_1 \neq l_2.
\] 

\subsection{Наблюдаемые  величины.  Разложение  по  собственным  векторам.  
Полнота  системы  собственных  векторов}
Операторы, соответствующие наблюдаемым физическим величинам, являются
самосопряженными операторами. Это обеспечивает действительность
значений наблюдаемой физической величины. Имеем набор собственных
состояний некоторого эрмитового  оператора  
$\ket{ l_n }$,  $\hat{L} \ket{ l_n } = l_n \left| l_n
\right>$.  Если набор собственных состояний полный, согласно принципам
квантовой механики любое состояние можно представить суперпозицией
состояний $\ket{ l_n }$:
\begin{equation}  
\left| \psi \right> = \sum_{(n)} c_n \ket{ l_n }.
\end{equation}  

Отсюда для коэффициентов разложения имеем:  
$c_n = \bra{ l_n } \left. \psi \right>$, и, следовательно,
справедливо равенство 
\begin{equation}  
\left| \psi \right> = \sum_{(n)} \bra{ l_n } \left. \psi
\right> \ket{ l_n } = 
\sum_{(n)} \ket{ l_n } \bra{ l_n } \left. \psi
\right>.
\label{eqAddDirac_full}
\end{equation}  

Из равенства \ref{eqAddDirac_full} следует важное соотношение:
\begin{equation}  
\sum_{(n)} \ket{ l_n } \bra{ l_n } = \hat{I}.
\label{eqAddDiracI}
\end{equation}  
где $\hat{I}$ -  единичный оператор. Это равенство является условием
полноты системы собственных векторов (условием разложимости). 

\subsection{Оператор проектирования}
\label{AddDiracProjector}

Рассмотрим оператор \(\hat{P}_n = \ket{ l_n } \left< l_n
\right|\). 
Результатом действия этого оператора на состояние 
\(\left| \psi \right>\) будет
\begin{equation}
\hat{P}_n \left| \psi \right> = \sum_{(k)} \ket{ l_n } \left<
l_n \right| c_k \ket{ l_k } = c_n \ket{ l_n }.
\label{eqDiracProektor}
\end{equation}
Оператор \(\hat{P}_n = \ket{ l_n } \bra{ l_n }\) называется
оператором проектирования.

Можно написать следующие свойства этого оператора
\begin{equation}  
\sum_{(n)} \hat{P}_n = \hat{I}.
\end{equation}  

\begin{equation}  
\hat{P}_n^2 = \hat{P}_n.
\end{equation}  

\input ./add/quant/figproject.tex
Действие оператора проектирования имеет простую геометрическую
интерпретацию  (см. \autoref{figAddProject}):
\[
\hat{P}_n\left|\psi\right> = \cos{\theta} \ket{l_n},
\]
где $\cos{\theta} = \left<\psi|l_n\right> = c_n$. 

\subsection{След оператора}
\label{AddDiracTrace}
В ортонормированном базисе \(\left\{\ket{l_n}\right\}\) 
величина 
\begin{equation}  
Sp \hat{L} = \sum_n \bra{l_n} \hat{L} \ket{l_n}
\label{eqAddDiracTr}
\end{equation}  
называется следом оператора \(\hat{L}\). При определенных условиях
\cite{bTraceClassOperatorAdd1} ряд \ref{eqAddDiracTr}
абсолютно сходится и не зависит от выбора базиса.

Если использовать матричное представление 
\[
L_{kn} = \bra{l_k} \hat{L} \ket{l_n}, 
\]
то след оператора - сумма диагональных элементов матричного 
представления
\[
Sp \hat{L} = \sum_n L_{nn}
\]

Можно написать следующие свойства следа оператора:
\begin{eqnarray}
Sp\left(l \hat{L} + m \hat{M}\right) = 
l Sp \hat{L} + m Sp \hat{M},
\nonumber \\
Sp\left(\hat{L}\hat{M}\right) = 
Sp\left(\hat{M}\hat{L}\right).
\label{eqAddDiracTrProperty}
\end{eqnarray}

\subsection{Средние  значения  операторов}
Среднее значение оператора $\hat{L}$   в состоянии $\left| \psi
\right>$  дается равенством 
\begin{equation}  
\left< \hat{L} \right>_{\psi} = \bra{\psi}\hat{L}\ket{\psi}
\label{eqAddDiracMid}
\end{equation}  
при условии
\[
\bra{\psi}\ket{\psi} = 1.
\]

Действительно, если принять, что $\ket{\psi}$ раскладывается в
ряд по собственным функциям оператора $\hat{L}$ следующим образом:
\[
\ket{\psi} = \sum_n c_n \ket{l_n},
\]
то $\hat{L}\left|\psi\right>$ можно записать как
\[
\hat{L}\ket{\psi} = \sum_n l_n c_n \ket{l_n},
\]
где $l_n$ собственное число соответствующее собственному состоянию 
$\ket{l_n}$. 
Если теперь подставить два последних выражения в \eqref{eqAddDiracMid}
то получим:
\[
\bra{\psi}\hat{L}\ket{\psi} = \sum_{n,m} 
l_n c_n c_m^{*} \bra{l_m}\ket{l_n}=
\sum_n l_n c_n c_n^{*} = 
\sum_n l_n \left|c_n\right|^2, 
\]
что (при условии $\left<\psi\right.\left|\psi\right> = 1$) доказывает,
что выражение \eqref{eqAddDiracMid} действительно 
представляет собой выражение для среднего значения оператора 
$\hat{L}$   в состоянии $\left|\psi\right>$.
\footnote{Для этого достаточно вспомнить, что $\left|c_n\right|^2$
  задает вероятность получить систему в состоянии $\ket{l_n}$,
  то есть получить показание измерительного прибора в $l_n$}

Если взять некоторый ортонормированный базис $\{\ket{ n }\}$,
образующий полный набор, т. е. подчиняющийся условию
\eqref{eqAddDiracI}: $\sum_n \ket{ n }\bra{ n } =
\hat{I}$, то выражение \eqref{eqAddDiracMid}
может быть переписано следующим образом:
\begin{eqnarray}
\left< \hat{L} \right>_{\psi} = 
\bra{\psi}\hat{L}\ket{\psi} = 
\bra{\psi}\hat{I}\hat{L}\ket{\psi} = 
\nonumber \\
= 
\sum_n \bra{\psi}\ket{n}\bra{n}
\hat{L}\ket{\psi} = 
\sum_n \bra{n}
\hat{L}\ket{\psi}\bra{\psi}\ket{n} = 
Sp \left(\hat{L} \hat{\rho} \right),
\nonumber
\end{eqnarray}
где 
\(
\hat{\rho} = \ket{\psi}\bra{\psi} = \hat{P}_{\psi}
\) - оператор проектирования на состояние 
$\left| \psi \right>$.
С учетом \eqref{eqAddDiracTrProperty} можно записать
\begin{equation}
\left< \hat{L} \right>_{\psi} = Sp \left(\hat{\rho} \hat{L} \right).
\label{eqAddDiracMidViaRho}
\end{equation}

\subsection{Представление  операторов  через  внешние  произведения  
собственных  векторов}
Дважды используя условие полноты \eqref{eqAddDirac_full}, получим
\begin{equation}
\hat{A} = \hat{I} \hat{A} \hat{I} = \sum_{(l)}\sum_{(l')} 
\ket{l}\bra{l} \hat{A} \ket{l'}\bra{l'} = 
\sum_{(l)}\sum_{(l')} 
\ket{l}\bra{l'} A_{ll'},
\end{equation}  
где $A_{ll'} = \bra{l} \hat{A} \ket{l'}$ - матричный
элемент оператора $\hat{A}$  в представлении  $\ket{l}$.
 
Оператор, выраженный через свои же собственные векторы, может быть
представляем разложением \footnote{При условии
 нормировки собственных векторов: $\bra{l}\ket{l} = 1$} 
\begin{equation}
\hat{L} = \sum_{(l)} 
l \ket{l}\bra{l}.
\end{equation}  

Обобщение этого равенства для операторной функции имеет вид
\begin{equation}
F\left(\hat{L}\right) = \sum_{(l)} 
F\left(l\right) \ket{l}\bra{l}.
\label{eqAddDiracFL}
\end{equation}  

\subsection{Волновые  функции  в  координатном  и  импульсном  представлениях}
Переход от вектора состояния к волновой функции осуществляется
посредством скалярного умножения этого вектора состояния на вектор
состояния соответствующей наблюдаемой величины. Например, для волновой
функции в координатном представлении 
\begin{equation}
\phi\left(q\right) = \bra{q}\left.\psi\right>.
\end{equation}  
где $\bra{q}$ - собственный вектор оператора координаты. 
В импульсном представлении получим
\begin{equation}
\phi\left(p\right) = \bra{p}\left.\psi\right>.
\end{equation}  
где $\bra{p}$ - собственный вектор оператора импульса.
