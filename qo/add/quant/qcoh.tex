%% -*- coding:utf-8 -*- 
\section{Координатное представление когерентного состояния}
\label{AddQCoh}
Начнем с вакуумного состояния, которое удовлетворяет уравнению 
\begin{equation}
\hat{a}\ket{0} = 0.
\label{eqAddQCoh1}
\end{equation}
В координатном представлении $\hat{q} = q$, $\hat{p} = -i
\frac{\partial}{\partial q}$, тогда
\begin{eqnarray}
\hat{a} = \frac{1}{\sqrt{2 \hbar \omega}}
\left(
\omega q + \hbar \frac{\partial}{\partial q}
\right),
\nonumber \\
\hat{a}^{\dag} = \frac{1}{\sqrt{2 \hbar \omega}}
\left(
\omega q - \hbar \frac{\partial}{\partial q}
\right).
\label{eqAddQCoh2}
\end{eqnarray}

Подставляя \eqref{eqAddQCoh2} в \eqref{eqAddQCoh1} получим уравнение
для волновой функции вакуумного состояния моды в когерентном
представлении:
\begin{equation}
\left(
\omega q + \hbar \frac{\partial}{\partial q}
\right) \Phi_0\left(q\right) = 0,
\label{eqAddQCoh3}
\end{equation}
где $\Phi_0\left(q\right) = \bra{q}\ket{0}$.
Нормированное решение этого уравнения равно
\begin{equation}
\Phi_0\left(q\right) = \left(\frac{\omega}{\pi
  \hbar}\right)^{\frac{1}{4}} e^{-\frac{\omega q^2}{2 \hbar}}.
\nonumber
\end{equation}
Неопределенность этого решения $\Delta q \Delta p =
\frac{\hbar}{2}$ - минимальна. Плотность вероятности обнаружить
определенную координату $q$ (в нашем случае определенную напряженность
электрического поля) равна:
\begin{equation}
\Phi_0\left(q\right)\Phi_0^{*}\left(q\right) = \left(\frac{\omega}{\pi
  \hbar}\right)^{\frac{1}{2}} e^{-\frac{\omega q^2}{\hbar}}.
\label{eqAddQCoh4}
\end{equation}
Для когерентного состояния справедливо уравнение
\[
\hat{a}\left|\alpha\right> = \alpha \left|\alpha\right>,
\]
которое в координатном представлении будет иметь вид
\begin{equation}
\left(
\omega q + \hbar \frac{\partial}{\partial q}
\right) \Phi_{\alpha}\left(q\right) = 
\alpha \Phi_{\alpha}\left(q\right),
\label{eqAddQCoh5}
\end{equation}
где 
\[
\Phi_{\alpha}\left(q\right) = \bra{q}\left.\alpha\right>.
\]

Сравнивая \eqref{eqAddQCoh3} и \eqref{eqAddQCoh5}, легко увидеть, что
решение уравнения \eqref{eqAddQCoh5} отличается от \eqref{eqAddQCoh4}
сдвигом на величину
$\alpha\sqrt{\frac{2\hbar}{\omega}}$. Действительно, уравнение
\eqref{eqAddQCoh5} можно записать в виде
\begin{equation}
\left(
\omega \left(q - \frac{\alpha}{\omega}\right) + \hbar \frac{\partial}{\partial q}
\right) \Phi_{\alpha}\left(q\right) = 0.
\nonumber
\end{equation}
Решением этого уравнения будет:
\begin{equation}
\Phi_{\alpha}\left(q\right) = \left(\frac{\omega}{\pi
  \hbar}\right)^{\frac{1}{4}} e^{-\frac{\omega}{2
    \hbar}\left(q-\sqrt{\frac{2\hbar}{\omega}}\alpha\right)^2}. 
\label{eqAddQCoh7}
\end{equation}

Когерентное состояние не является стационарным состоянием. Оно должно
удовлетворять нестационарному уравнению Шредингера. От времени зависит
$\alpha = \alpha\left(t\right)$. Подставляя \eqref{eqAddQCoh7} в
уравнение Шредингера, эту зависимость можно определить, но и без этого
ясно, что $\alpha\left(t\right)$ должно колебаться по гармоническому
закону на частоте моды. Примем 
\[
\alpha\left(t\right) = \left|\alpha\right| e^{-i \omega t}.
\]
Тогда плотность вероятности будет иметь вид
\begin{equation}
\Phi_{\alpha}\left(q\right)\Phi_{\alpha}^{*}\left(q\right) = N e^{-\frac{\omega}{2
    \hbar}
\left[\left(q-\sqrt{\frac{2\hbar}{\omega}}\alpha\right)^2
+
\left(q-\sqrt{\frac{2\hbar}{\omega}}\alpha^{*}\right)^2
\right]
}.
\nonumber
\end{equation}
Преобразуем выражение показателя экспоненты
\begin{eqnarray}
\left[\left(q-\sqrt{\frac{2\hbar}{\omega}}\alpha\right)^2
+
\left(q-\sqrt{\frac{2\hbar}{\omega}}\alpha^{*}\right)^2
\right] = 
\nonumber \\
=
q^2 + \frac{2\hbar}{\omega}\alpha^2 - 2 q \alpha
  \sqrt{\frac{2\hbar}{\omega}} +
q^2 + \frac{2\hbar}{\omega}\alpha^{2 *} - 2 q \alpha^{*}
  \sqrt{\frac{2\hbar}{\omega}} =
\nonumber \\
= 2
\left[
q^2 + \frac{2\hbar}{\omega}\frac{\alpha^2 + \alpha^{*2}}{2} -
2 q \sqrt{\frac{2\hbar}{\omega}} \frac{\alpha + \alpha^{*}}{2} +
\frac{2 \hbar}{\omega}\left(\alpha\alpha^{*}\right) -
\frac{2 \hbar}{\omega}\left(\alpha\alpha^{*}\right)
\right] = 
\nonumber \\
= 2
\left[
q^2 + \frac{4\hbar}{\omega}\left(\frac{\alpha + \alpha^{*}}{2}\right)^2 -
2 q \sqrt{\frac{2\hbar}{\omega}} \frac{\alpha + \alpha^{*}}{2}  -
\frac{2 \hbar}{\omega}\left(\alpha\alpha^{*}\right)
\right] = 
\nonumber \\
= 2
\left[
q - \sqrt{\frac{2\hbar}{\omega}} \frac{\alpha + \alpha^{*}}{2}
\right]^2 + const = 
\nonumber \\
= 
2
\left[
q - \sqrt{\frac{2\hbar}{\omega}} \left|\alpha\right|\cos\,\omega t
\right]^2 + const.
\nonumber
\end{eqnarray}
Константа перейдет в нормирующий множитель, в результате мы получим:
\begin{equation}
\Phi_{\alpha}\left(q,t\right)\Phi_{\alpha}^{*}\left(q,t\right) = N
e^{-\frac{\omega}{\hbar}
\left[
q - \sqrt{\frac{2\hbar}{\omega}} \left|\alpha\right|\cos\,\omega t
\right]^2
}.
\label{eqAddQCoh9}
\end{equation}
Из выражения \eqref{eqAddQCoh9} видно, что плотность вероятности для
$q$ (в нашем случае для напряженности электрического поля) для
фиксированного времени такая же, как для вакуумного состояния, но
смещена на величину
$\sqrt{\frac{2\hbar}{\omega}}\alpha\left(t\right)$. Во времени центр
распределения перемещается по гармоническому закону, как это
изображено на \autoref{figQCoh_1} 

\input ./add/figqhoc.tex
% Maxima
%% load(draw);
%% f(q, t):=exp(-1*(q-cos(t))^2);
%% draw3d(explicit(f(q,t), q,-5,5, t,-10,10));

Поскольку ширина распределения не зависит от амплитуды, относительная
неопределенность с ростом амплитуды будет уменьшаться. Следовательно
состояние будет стремиться к классическому.







