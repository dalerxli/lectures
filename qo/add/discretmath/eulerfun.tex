%% -*- coding:utf-8 -*-
\section{Функция Эйлера}

\subsection{Определение}
\begin{definition}[Функция Эйлера]
Функция Эйлера $\phi\left(n\right)$ показывает сколько чисел $k \in
\{1, ... n-1\}$ взаимно просты с $n$, т.е. $\mbox{НОД}\left(k,
n\right) = 1$.
\label{def:add:discretmath:eulerfun}
\end{definition}

\begin{example}[Функция Эйлера]
  Если взять число $n=15$ то имеется 8 чисел взаимно простых с $15$: 
  $1, 2, 4, 7, 8, 11, 13, 14$. Оставшиеся 7 чисел не являются взаимно
  простыми с $n$ поскольку имеют наибольший общийй делитель отличный
  от 1, например $\mbox{НОД}\left(6, 15\right) = 3$.
  Таким образом, $\phi\left(15\right) = 8$. 
\end{example}


\subsection{Свойства}

\begin{property}[Функция Эйлера простого числа]
Если $p$ - простое число, то $\phi(p) = p - 1$
\begin{proof}
Следует из определения \ref{def:add:discretmath:eulerfun}.
\end{proof}
\label{prop:add:discretmath:eulerfun1}
\end{property}


\begin{property}[Функция Эйлера произведения]
Если $\mbox{НОД}\left(n, m\right) = 1$, то
$\phi\left(n \cdot m\right) = \phi\left(n\right) \phi\left( m\right)$
\begin{proof}
TBD
\end{proof}
\label{prop:add:discretmath:eulerfun2}
\end{property}
