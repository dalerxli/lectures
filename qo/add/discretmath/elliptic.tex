%% -*- coding:utf-8 -*-
\section{Эллиптическая криптография}

\subsection{Эллиптические кривые над полем $\mathbb{R}$}

\input ./add/discretmath/figelliptic.tex

В эллиптической криптографии рассматривают определенные наборы
объектов которые образуют группу (см. \autoref{sec:add:group}). В
качестве такого набора мы будем рассматривать точки принадлежащие
некоторой кривой (см. \autoref{fig:add:ellipticR}): 
\[
E: y^2 = x^3 +a x + b,
\]
где коэффициенты $a,b$ должны удовлетворять следующему соотношению
\[
4 a^3 + 27 b^2 \ne 0,
\]
в этом случае кубическое уравнение $x^3 + a x + b = 0$ будет иметь 3
различных действительных корня \cite{Washington:2008:ECN:1388394}. 

\input ./add/discretmath/figellipticsum.tex

В соответствии с определением \ref{def:add:group} для точек на кривой 
задана некоторая бинарная операция которая двум точкам $p, q$ с
координатами $(p_x, p_y)$ и $(q_x, q_y)$, соответственно, сопоставляет
третью точку $r$ с 
координатами $(r_x, r_y)$, данную операцию мы будем называть
сложением:
\[
p + q = r.
\]
Существует простая геометрическая интерпретация операции сложения (см.
\autoref{fig:add:ellipticRsum}). Допустим имеется 2 точки на кривой
которые мы хотим сложить: $p$ и $q$ с координатами $(x_p, y_p), (x_q,
y_q)$ соответственно. Если $x_p \ne x_q$, то через эти точки можно
провести прямую, которая имеет наклон 
\[
m = \frac{y_p - y_q}{x_p - x_q}.
\]
и пересекает кривую в точке $r'$. Если эта точка имеет координаты
$(x_{r'}, y_{r'})$, то в силу того, что на лежит на прямой с наклоном $m$:
\[
m = \frac{y_{r'} - y_p}{x_{r'} - x_p},
\]
следовательно
\[
y_{r'} = y_p + m \left(x_{r'} - x_p\right).
\]
Эта точка должна принадлежать кривой, т.е.
\begin{eqnarray}
y_{r'}^2 = \left(y_p + m \left(x_{r'} - x_p\right)\right)^2 = x_{r'}^3
+ a x_{r'} + b
\nonumber
\end{eqnarray}
которое можно переписать
\begin{eqnarray}
x_{r'}^3 - m^2 x_{r'}^2 + \dots = 0.
\nonumber
\end{eqnarray}
Уравнение $x^3 - m^2 x^2 + \dots = 0$ имеет 3 корня: $x_p, x_q,
x_{r'}$, т.е. его можно также переписать в виде
\begin{eqnarray}
(x - x_{r'})(x - x_p)(x - x_q) = x^3 - (x_{r'} + x_p + x_q) x^2 +
  \dots = 0.
\nonumber
\end{eqnarray}
Т.о. 
\[
x_{r'} + x_p + x_q = m^2.
\]
Следовательно
\begin{eqnarray}
x_{r'} = m^2 - x_p - x_q,
\nonumber \\
y_{r'} = y_p + m \left(x_{r'} - x_p\right),
\nonumber
\end{eqnarray}
Отразив эту точку относительно оси $X$ мы получим финальную точку $r$
которую будем называть суммой исходных точек  (см.
\autoref{fig:add:ellipticRsum}). Координаты этой точки $x_r, y_r$
могут быть получены по следующим формулам
\begin{eqnarray}
x_{r} = m^2 - x_p - x_q,
\nonumber \\
y_{r} = - y_p + m \left(x_p - x_r\right).
\label{eq:add:discretmath:elliptic:add}
\end{eqnarray}

\input ./add/discretmath/figellipticsumeq2.tex
\input ./add/discretmath/figellipticsumeq.tex

В случае $x_p = x_q$ возможны два варианта:
\begin{enumerate}
\item $y_p = y_q$ (см. \autoref{fig:add:ellipticRsumEq2}). В случае
  когда точки на кривой приближаются друг к другу, прямая линия,
  проведенная через них, стремиться к касательной. Коэффициент $m$
  может быть найден по следующей формуле: $m = \frac{dy}{dx}$. С
  учетом $2ydy = 3x^2 dx + a dx$ имеем $m = \frac{dy}{dx} = \frac{3
    x^2 + a}{2y}$. дальнейший расчет ведется по формулам
  \eqref{eq:add:discretmath:elliptic:add}. 
\item $y_p \ne y_q$ (см. \autoref{fig:add:ellipticRsumEq}). В этом
  случае, в силу симметрии кривой относительно оси $X$ возможен только
  один вариант: $y_p = y_q$ и кривая проходящая через эти две точки не
  пересекает кривую в третьей точке. Для этого случая вводят еще одну
  точку $0$ в которую уходит прямая линия проведенная через две точки.
  Таким образом, в этом случае мы имеем $p + q = 0, q = -p$.
\end{enumerate}

Таким образом мы можем определить эллиптическую кривую над полем
вещественных чисел $\mathbb{R}$ как следующее множество точек
\begin{equation}
E\left(\mathbb{R}\right) : \left\{
y^2 = x^3 +ax +b, a,b \in \mathbb{R}
\right\} \cup \{0\}.
\label{eq:add:discretmath:elliptic:er}
\end{equation}

Для данных точек определена бинарная операция, которую мы назвали
сложением. На множестве есть нулевой элемент, так же для каждого
элемента можно определить обратный элемент. Можно доказать, что
введенная операция является ассоциативной: $(a+b) + c = a + (b+c)$
\cite{Washington:2008:ECN:1388394}. Таким образом множество
$E\left(\mathbb{R}\right)$ образует группу относительно операции
сложения. С учетом очевидного равенства $p + q = q + p$, данная группа
будет являться коммутативной, т.е. Абелевой. 

\begin{remark}[Об операции сложения]
Введенная операция сложения для точек на эллиптической кривой выглядит
на первый взгляд неестественно, почему бы не использовать более
очевидные операции сложения точек на плоскости, например правила
сложения векторов. В этом случае, если $p,q \in
E\left(\mathbb{R}\right)$, то вполне может быть что $p + q \not\in
E\left(\mathbb{R}\right)$ и, следовательно, будет нарушено основное
свойство группы - связанность (closure). Вместе с тем для определенной
нами бинарной операции \eqref{eq:add:discretmath:elliptic:add}
справедливы все свойства групп, и соответственно, все вытекающие из
этого свойства, такие например как теорема Лагранжа, которые могут
быть применены к введенным нам объектам.
\end{remark}


\subsection{Эллиптические кривые над полем $\mathbb{F}_p$}
Множество \eqref{eq:add:discretmath:elliptic:er} вместе с операцией
сложения \eqref{eq:add:discretmath:elliptic:add} может определено над
произвольным полем, т.е. не только над $\mathbb{R}$. С точки зрения
криптографии особый интерес представляет поле $\mathbb{F}_p$ (см.
\autoref{sec:add:diskretmath:mod:fp}). Можно определить множество
элементов эллиптической кривой над полем $\mathbb{F}_p$ аналогично
выражению \eqref{sec:add:diskretmath:mod:fp}:
\begin{equation}
E\left(\mathbb{F}_p\right) : \left\{
y^2 \equiv x^3 +ax +b \mod p, a,b \in \mathbb{F}_p
\right\} \cup \{0\}.
\label{eq:add:discretmath:elliptic:fp}
\end{equation}

\input ./add/discretmath/figellipticfp.tex

На \autoref{fig:add:ellipticFp} изображено такое множество для поля
$\mathbb{F}_{19}$, т.е. $p = 19$. Уравнение кривой $y^2 \equiv x^3 -7
x + 10 \mod 19$. 

Для каждой точки $a$ с координатами $x_a, y_a$ определен обратный
элемент $-a$ с координатами $x_{-a} = x_a, y_{-a} \equiv -y_a \mod p$.

Для точек на данной кривой задан следующий закон сложения $a + b = c,
b \ne -a$
\begin{eqnarray}
x_{c} \equiv m^2 - x_a - x_b \mod p,
\nonumber \\
y_{c} \equiv - y_a + m \left(x_a - x_c\right) \mod p,
\label{eq:add:discretmath:elliptic:addfp}
\end{eqnarray}
где $(x_{a,b,c}, y_{a,b,c})$ - координаты точек $a,b$ и $c$
соответственно. Для коэффициента $m$ используются следующие
соотношения: 
\begin{eqnarray}
m = \left(y_a - y_b\right)\left(x_a - x_b\right)^{-1} \mod p, \mbox{ если } x_a \ne x_b
\nonumber \\
m = \left(3x^2 + a\right)\left(2y\right)^{-1} \mod p, \mbox{ если }
x_a = x_b.
\nonumber
\end{eqnarray}
Очевидно что если $b = -a$, то $a + b = a + (-a) = 0$.

\subsection{Скалярное умножение и дискретный логарифм}

Допустим $n$ некоторое натуральное число $n \in \mathbb{N}$ и $a \in
E\left(\mathbb{F}_p\right)$. Определим скалярное произведение
следующим образом
\[
n \cdot a = a + a + \dots a = \sum_{k=1}^n a
\] 

Наивная реализация требует $O\left(n\right)$ операций сложения, но с
помощью парадигмы ``разделяй и властвуй'' (см.
\autoref{addDivideAndConquer}) может быть сведено к $O\left(\log
n\right)$ операций сложения.

%% *Elliptic> c = Curve (-7) 10 19
%% *Elliptic> c
%% y^2 = (x^3 + -7x + 10) mod 19
%% *Elliptic> points c
%% [0,(1,2),(1,17),(2,2),(2,17),(3,4),(3,15),(5,9),(5,10),(7,0),(9,7),(9,12),(10,3),(10,16),(12,1),(12,18),(13,8),(13,11),(16,2),(16,17),(17,4),(17,15),(18,4),(18,15)]
%% *Elliptic> p = Point 13 8 c
%% *Elliptic> 0 .*. p
%% 0
%% *Elliptic> 1 .*. p
%% (13,8)
%% *Elliptic> 2 .*. p
%% (16,17)
%% *Elliptic> 3 .*. p
%% (18,15)
%% *Elliptic> 4 .*. p
%% (12,1)
%% *Elliptic> 5 .*. p
%% (5,10)
%% *Elliptic> 6 .*. p
%% (7,0)
%% *Elliptic> 7 .*. p
%% (5,9)
%% *Elliptic> 8 .*. p
%% (12,18)
%% *Elliptic> 9 .*. p
%% (18,4)
%% *Elliptic> 10 .*. p
%% (16,2)
%% *Elliptic> 11 .*. p
%% (13,11)
%% *Elliptic> 12 .*. p
%% 0


\begin{example}[Скалярное произведение]
Рассмотрим эллиптическую кривую $y^2 \equiv x^3 -7
x + 10 \mod 19$, изображенную на \autoref{fig:add:ellipticFp}. Выберем
в качестве точки $p = (13, 8)$, тогда 
\begin{eqnarray}
0 \cdot p = 0, \nonumber \\
1 \cdot p = p = (13,8), \nonumber \\
2 \cdot p = p + p = (16,7), \nonumber \\
3 \cdot p = 2 \cdot p + p = (18,5), \nonumber \\
4 \cdot p = 3 \cdot p + p = (12,1), \nonumber \\
5 \cdot p = 4 \cdot p + p = (5,10), \nonumber \\
6 \cdot p = 5 \cdot p +p = (7,0), \nonumber \\
7 \cdot p = 6 \cdot p + p = (5,9), \nonumber \\
8 \cdot p = 7 \cdot p +p = (12,18), \nonumber \\
9 \cdot p = 8 \cdot p + p = (18,4), \nonumber \\
10 \cdot p = 9 \cdot p + p = (16,2), \nonumber \\
11 \cdot p = 10 \cdot p + p = (13,11), \nonumber \\
12 \cdot p = 11 \cdot p + p = 0
\nonumber 
\end{eqnarray}
\label{ex:add:discretmath:elliptic:subgroup}
\end{example}
Как видно из прим. \ref{ex:add:discretmath:elliptic:subgroup} каждый
элемент эллиптической кривой является генератором некоторой
циклической подгруппы. Вместе с тем вся группа точек на эллиптической
кривой не обязательно является циклической 
(TBD).
%https://math.stackexchange.com/questions/2323595/under-what-conditions-do-all-the-points-on-an-elliptic-curve-form-a-cyclic-group
С другой стороны для формирования задачи дискретного логарифма нам
требуется именно циклическая группа. Таким образом для заданной
эллиптической кривой вначале вычисляют ее порядок, для этого
существует эффективный алгоритм Шуфа \cite{ReneSchoof:1985}. После
этого находится простой делитель найденного порядка и ищется точка
которая является генератором подгруппы выбранного порядка. Для того
чтобы это сделать используется следующий факт. Для любой точки 
$g \in E$ имеет место соотношение
\[
N g = 0,
\]
где $N = \left|E\right|$ -порядок (число точек) эллиптической кривой.
Допустим $p$ - некоторый простой делитель числа $N$:
\[
N = h p
\]
TBD

Если имеется две точки на кривой $a, b \in E\left(\mathbb{F}_p\right)$
то имеет смысл вопрос о существовании такого $x \in \mathbb{N}$:
\[
x \cdot a = b
\]
данная задача называется дискретным логарифмом на эллиптических
кривых. 

TBD
