%% -*- coding:utf-8 -*-
\section{Эллиптическая криптография}

\subsection{Эллиптические кривые над полем $\mathbb{R}$}

\input ./add/discretmath/figelliptic.tex

В эллиптической криптографии рассматривают определенные наборы
объектов которые образуют группу (см. \autoref{sec:add:group}). В
качестве такого набора мы будем рассматривать точки принадлежащие
некоторой кривой (см. \autoref{fig:add:ellipticR}): 
\[
E: y^2 = x^3 +a x + b,
\]
где коэффициенты $a,b$ должны удовлетворять следующему соотношению
\[
4 a^3 + 27 b^2 \ne 0,
\]
в этом случае кубическое уравнение $x^3 + a x + b = 0$ будет иметь 3
различных действительных корня \cite{Washington:2008:ECN:1388394}. 

\input ./add/discretmath/figellipticsum.tex

В соответствии с определением \ref{def:add:group} для точек на кривой 
задана некоторая бинарная операция которая двум точкам $a_1, a_2$ с
координатами $(x_1, y_1)$ и $(x_2, y_2)$, соответственно, сопоставляет
третью точку $a_3$ с 
координатами $(x_3, y_3)$, данную операцию мы будем называть
сложением:
\[
a_1 + a_2 = a_3.
\]
Существует простая геометрическая интерпретация операции сложения (см.
\autoref{fig:add:ellipticRsum})



\begin{enumerate}
\item Определена бинарная операция, которую мы будем называть сложением
\item Элемент $e$ будет называться нулем
\item Для каждого элемента есть соответствующий обратный элемент
\end{enumerate}
TBD

\subsection{Эллиптические кривые над полем $\mathbb{F}_p$}
TBD
