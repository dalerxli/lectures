%% -*- coding:utf-8 -*-
\section{Протокол Диффи-Хеллмана (Diffie-Hellman, DH)}
\label{sec:add:dm:dh}
Предположим существуют два абонента Алиса и Боб. Им известны два числа
$g$ и $p$, которые не являются секретными.

Алиса выбирает случайное число $a$ и пересылает Бобу следущее значение
\begin{equation}
A \equiv g^a \mod{p}.
\label{eqAddDiscretIndDHA}
\end{equation}

Боб вычисляет следующее число (с помощью секретной случайной величины
$b$)
\begin{equation}
B \equiv g^b \mod{p}.
\label{eqAddDiscretIndDHB}
\end{equation}

Алиса, с помощью только ей известного числа $a$ вычисляет ключ
\begin{equation}
K \equiv B^a\mod{p} \equiv g^{ab} \mod{p}.
\label{eqAddDiscretIndDHKey}
\end{equation}

Боб может получить то же самое значение ключа с помощью своего
секретного числа $b$:
\begin{equation}
K \equiv A^b\mod{p} \equiv g^{ab} \mod{p}.
\label{eqAddDiscretIndDHKeyB}
\end{equation}

Таким образом Алиса и Боб получают один и тот же ключ, который в
дальнейшем может быть использован для переди сообщения с помощью
симметричных алгоритмов шифрования (например AES).

\begin{example}
\emph{(Диффи-Хеллман)}
Исходные данные (открытая информация): $g = 2$, $p = 23$. Алиса
выбирает случайное число $a = 6$ и вычисляет по формуле
\eqref{eqAddDiscretIndDHA} число 
%power_mod(2, 6, 23);
$A = 18$ и отсылает его Бобу.
Боб выбирает случайное число $b=9$ и, с помощью формулы 
\eqref{eqAddDiscretIndDHB}, получает
%power_mod(2, 9, 23);
$B = 6$ и отсылает и отсылает это число Алисе.

Алиса вычисляет ключ 
% power_mod(6, 6, 23);
$K = 12$ по формуле \eqref{eqAddDiscretIndDHKey}. Боб может полчить
тоже значение ключа 
% power_mod(18, 9, 23);
$K = 12$ используя \eqref{eqAddDiscretIndDHKeyB}
\nonumber
\end{example}

Злоумышленнику Еве известны числа $g$, $p$, $A$ и $B$. Для получения
ключа $K$ Еве необходимо получить одно из секретных чисел $a$ или $b$:
\begin{equation}
a \equiv ind_g\left( A \right) \mod{p},
\nonumber
\end{equation}
откуда с помощью \eqref{eqAddDiscretIndDHKey} получается искомое
значение $K$.
