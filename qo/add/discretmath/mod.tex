%% -*- coding:utf-8 -*- 
\section{Сравнение по модулю}
\begin{definition}
Запись 
\begin{equation}
a \equiv b \mod{c}
\label{defAddMod}
\end{equation}
означает, что $a$ и $b$ имеют одинаковые остатки при делении на $c$
или $a$ и $b$ сравнимы по модулю натурального числа $c$. При этом
число $c$ называется модулем сравнения.
\end{definition}

Определение \ref{defAddMod} может также трактоваться как то, что
разность $a - b$ делится на $c$.

\begin{example}
\emph{Сравнение по модулю}
$30 \equiv 8 \mod{11}$, потому что, $30 = 2 \cdot 11 + 8$.
\end{example}

\begin{definition}[Отрицательный элемент]
Если $a < n$, то $n - a$ будет называться отрицательным по отношению к
$a$ элементом и обозначаться $-a \mod n$.
\end{definition}

\begin{example}
\emph{Отрицательный элемент}
\[
-5 \equiv 6 \mod 11,
\]
поскольку $5 < 11, 6 = 11 - 5$.
\end{example}


\subsection{Арифметические операции}

\begin{lemma}[О сложении по модулю]
Если $a_1 \equiv a_2 \mod n, b_1 \equiv b_2 \mod n$, то 
\[
a_1 + b_1 \equiv a_2 + b_2 \mod n
\]
\begin{proof}
Можно записать $a_1 = k_1 n + r_a, a_2 = k_2 n + r_a, b_1 = l_1 n +
r_b, b_2 = l_2 n + r_b$ откуда
\[
a_1 + b_1 = (k_1 + l_1) n + r_a + r_b \equiv r_a + r_b \mod n
\] 
и
\[
a_2 + b_2 = (k_2 + l_2) n + r_a + r_b \equiv r_a + r_b \mod n
\] 
откуда
\[
a_1 + b_1 \equiv a_2 + b_2 \equiv r_a + r_b \mod n
\]
\end{proof}
\end{lemma}

\begin{lemma}[Об умножении по модулю]
Если $a_1 \ equiv a_2 \mod n, b_1 \equiv b_2 \mod n$, то 
\[
a_1 \cdot b_1 \equiv a_2 \cdot b_2 \mod n
\]
\begin{proof}
Если $a_1 \equiv a_2 \mod n, b_1 \equiv b_2 \mod n$, то 
\[
a_1 + b_1 \equiv a_2 + b_2 \mod n
\]
Можно записать $a_1 = k_1 n + r_a, a_2 = k_2 n + r_a, b_1 = l_1 n +
r_b, b_2 = l_2 n + r_b$ откуда
\[
a_1 \cdot b_1 = k_1 l_1 n + l_1 n r_a + k_1 n r_b + r_a r_b \equiv r_a r_b \mod n
\] 
и
\[
a_2 \cdot b_2 = k_2 l_2 n + l_2 n r_a + k_2 n r_b + r_a r_b \equiv r_a r_b \mod n
\] 
откуда
\[
a_1 \cdot b_1 \equiv a_2 \cdot  b_2 \equiv r_a r_b \mod n
\]
\end{proof}
\end{lemma}

\subsection{Решение уравнений}
\label{sec:add:discretmath:mod:equationsolve}
Очень часто в криптографии имеют дело с уравнениями вида
\begin{equation}
a x \equiv b \mod n,
\label{eq:add:sicret:modeq}
\end{equation}
где $a, b, n$ известные целые числа, а $x$ неизвестный параметр,
подлежащий определению.

Очевидно, что если мы найдем такое целое число $a^{-1}$, что 
\[
a a^{-1} \equiv 1 \mod n,
\]
то
\[
x \equiv b a^{-1} \mod n.
\]

Если $\mbox{НОД}\left(a, n\right) = 1$, то в соответствии с
соотношением Безу (см. \myref{thm:besu}{теорему Безу}) 
$\exists x, y: a x + n y = 1$, т.е. 
\[
x \equiv a^{-1} \mod n.
\]
При этом, в соответствии с комментарием \ref{rem:besu}, $a^{-1}$, и
решение уравнения \eqref{eq:add:sicret:modeq}, может быть найдено
достаточно эффективно.

\subsection{Поле $\mathbb{F}_p$}
\label{sec:add:diskretmath:mod:fp}
Как мы видели в модульной арифметике можно складывать, вычитать,
умножать и даже, если сравнение ведется по модулю простого числа,
делить. Таким образом остатки при делении образуют поле которое
называется $\mathbb{F}_p$.
TBD
