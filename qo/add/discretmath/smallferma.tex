%% -*- coding:utf-8 -*- 
\section{Малая теорема Ферма}

\begin{theorem}
\emph{(Малая теорема Ферма)}
\label{addDiscretSmallFerma}
Если $p$ простое число, и $a$ не делится на $p$, то 
\begin{equation}
a^{p-1} \equiv 1 \mod{p}
\label{addDiscretSmallFermaEq}
\end{equation}
\end{theorem}

\begin{proof}
Рассмотрим следующее соотношение
\[
a \cdot k_i \mod{p},
\]
где $k_i \subset \left\{1, \dots, p - 1\right\}$.

Очевидно, что 
\begin{equation}
a \cdot k_i \equiv k_j \mod{p}.
\label{addDiscretSmallFermaInt}
\end{equation}
Действительно
\[
a \cdot k_i \mod{p} \subset \left\{1, \dots, p - 1\right\},
\]
т. к. любой остаток от деления на $p$ принимает значения $0, 1, \dots
p-1$. Нулевой остаток невозможен т. к. $a$ и $p$ взаимно просты.

Кроме этого каждый из остатков $a \cdot k_i \mod{p}$ встречается
только один раз, действительно допустим что $a \cdot k_i \mod{p} = a
\cdot k_j \mod{p}$ или же $a \cdot \left( k_i - k_j \right ) \equiv 0
\mod{p}$, т. е. $a$ делится на $p$, что противоречит условию взаимной
простоты. 

Перемножив все выражения (\ref{addDiscretSmallFermaInt}) получим
\[
a \cdot 2a \cdot 3a \cdot \dots \cdot a\left(p - 1\right) 
\equiv 1 \cdot 2 \cdot 3 \cdot \dots \cdot \left(p - 1\right) \mod{p}.
\]
или же
\[
a^{p-1} \left(p - 1\right)! \equiv \left(p - 1\right)! \mod{p},
\]
откуда получаем требуемое равенство в силу взаимной простоты $p$ и
$\left(p-1\right)!$:
\[
a^{p-1} \equiv 1 \mod{p}
\]
\end{proof}

\subsection{Псевдопростые числа}
Обобщение малой теоремы Ферма не верно, т. е.  если $a$ и $p$ взаимно
простые числа которые удовлетворяют соотношению
(\ref{addDiscretSmallFermaEq}) то $p$ может быть не простым. Например
\[
2 ^ {341 - 1} \equiv 1 \mod{ 341 },
\] 
при том, что $341 = 11 \cdot 31$.

Числа $p$ удовлетворяющие соотношению (\ref{addDiscretSmallFermaEq}),
но при этом не являющиеся простыми называется псевдопростыми числами
по основанию $a$. Например 341 - первое псевдопростое число по
основанию 2.
