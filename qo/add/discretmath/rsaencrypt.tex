%% -*- coding:utf-8 -*-
Допустим надо зашифровать некоторое сообщение $M$. Вначале оно
переводится в целое число(числа) $m$ такое, что 
$0 < m < \phi\left(n\right)$. Далее вычисляется за зашифрованный текст
$c$:
\begin{equation}
c \equiv m^e \mod{n}
\label{eqAddRSACode}
\end{equation}

\begin{example}
\emph{(RSA. Шифрование)}
Допустим у нас есть открытый ключ $\left(n=12, e=5\right)$ (см. прим. \ref{exAddRSAKeyGen}) 
и мы хотим зашифровать следующее сообщение $m = 1101_2 = 11_{10}$. 
%(%i1) power_mod(11,5,21);         
%(%o1)                                  2
Шифротекст вычисляется по формуле \eqref{eqAddRSACode} $c \equiv 11^5 \mod{21} = 2$.
\label{exAddRSACode}
\end{example}
