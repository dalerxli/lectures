%% -*- coding:utf-8 -*-
\begin{itemize}
\item Выбираются два простых числа $p$ и $q$
\item Вычисляется произведение выбранных простых чисел $n = p\cdot q$
\item Вычисляется функция Эйлера \index{функция Эйлера}
\(
\phi\left(n\right)=\left(p - 1 \right)\left(q - 1 \right)
\)
\item Выбирается целое число $e$ такое что 
\(
1 < e < \phi\left(n\right)
\) и $e$ и $\phi\left(n\right)$ взаимно просты,
т. е. 
\(
\mbox{НОД}\left( e, \phi\left(n\right) \right) = 1.
\)
\item вычисляем $d \equiv e^{-1} \mod{\phi\left(n\right)}$, т. е.
$d \cdot e \equiv 1 \mod{\phi\left(n\right)}$.
\end{itemize}

Открытый ключ состоит из двух чисел: модуля $n$ и открытой экспоненты
$e$. Именно эти два числа используются для шифрования исходного
сообщения.

Закрытый ключ состоит тоже из двух чисел: модуля $n$ и закрытой экспоненты
$d$.
