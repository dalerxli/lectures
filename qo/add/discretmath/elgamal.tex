%% -*- coding:utf-8 -*-
\section{Схема Эль-Гамаля (Elgamal)}
\label{sec:add:dm:elgamal}
Одной из вариаций протокола Диффи-Хелмана является схема Эль
Гамаля. Следует различать алгоритм шифрования и алгоритм цифровой
подписи Эль-Гамаля. Цифровая подпись Эль-Гамаля лежит в основе
стандартов цифровой подписи США (DSA) и России (ГОСТ Р 34.10-94).

Ниже мы рассмотрим алгоритм в режиме шифрования.

\subsection{Генерация ключей}

\begin{itemize}
\item Генерируется простое число $p$.
\item Выбирается целое число $g$
\item Выбирается случайное целое число $x: 1 < x < p$.
\item Вычисляется $y = g^x \mod p$
\end{itemize}

Открытым ключом  является тройка $p, g, y$. Закрытым ключом является
число $x$.

\begin{example}[Генерация ключей (Elgamal)]
Выбираем $p = 21, g = 10, x = 3$. $y = 10^3 \mod 21 = 13$.
\label{ex:add:dm:elgamal_gen}
\end{example}

\subsection{Шифрование}
Шифруемое сообщение $M$ должно удовлетворять критерию $0 < M < p$.
\begin{itemize}
\item Выбирается сессионный ключ - случайное число $k: 1 < k < p - 1$.
\item Вычисляется $a = g^k \mod p$
\item Вычисляется $b = y^k M \mod p$
\end{itemize}

Пара чисел $(a, b)$ считается шифротекстом.
\begin{example}[Шифрование (Elgamal)]
Допустим мы хотим зашифровать $M=6$.
Выбираем $k = 7$. $a = 10^7 \mod 21 = 10$, $b = 13^7 \cdot 6 \mod 21 =
15$.
\label{ex:add:dm:elgamal_crypt}
\end{example}

\subsection{Дешифрование}
Зная закрытый ключ $x$ мы можем восстановить исходное сообщение с
помощью
\begin{equation}
M = b\cdot \left(a^x\right)^{-1} \mod p,
\label{eq:add:dm:elgamal_decrypt}
\end{equation}
действительно в силу того что
\[
\left(a^x\right)^{-1} = g^{-kx} \mod p
\]
имеем
\[
b\cdot \left(a^x\right)^{-1} = 
y^k M g^{-kx} = g^{kx} M g^{-kx} = M \mod p.
\]
\begin{example}[Дешифрование (Elgamal)]
Зашифрованное сообщение из прим. \ref{ex:add:dm:elgamal_crypt} 
$C = (a=10, b=15)$. Используя \eqref{eq:add:dm:elgamal_decrypt} имеем
\[
M = 15 \cdot 13 \mod 21 = 6
\]
где было использовано
\[
\left(10^3\right)^{-1} \equiv 13 \mod 21,
\]
т.к. $13 \cdot 10^3 = 1 \mod 21$.
Таким образом мы восстановили зашифрованное в
прим. \ref{ex:add:dm:elgamal_crypt} сообщение $M=6$.
\label{ex:add:dm:elgamal_crypt}
\end{example}
