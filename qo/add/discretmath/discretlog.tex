\section{Дискретный логариф и его применение в криптографии}
\label{AddDiscretLog}
\begin{definition}
Дискретным логарифмом $ind_g\left(a\right) \mod{p}$
\footnote{От слова {\bf ind}ex - альтернативного названия для дискретного логорифма}
называется
минимальное число $x$, которое удовлетворяет следующему уравнению
(если такое число существует): 
\begin{equation}
g^x \equiv a \mod{p}
\nonumber
\end{equation}
\end{definition}

\begin{example}
\emph{($ind_3{13} \mod{17}$)}
Решим задачу методом перебора \cite{bWikiDiscretLog}:
\begin{eqnarray}
3^1 \equiv 3 \mod{17},\: 
3^2 \equiv 9 \mod{17},\: 
3^3 \equiv 10 \mod{17},\:
3^4 \equiv 13 \mod{17}, 
\nonumber \\
3^5 \equiv 5 \mod{17},\: 
3^6 \equiv 15 \mod{17},\: 
3^7 \equiv 11 \mod{17},\: 
3^8 \equiv 16 \mod{17}, 
\nonumber \\
3^9 \equiv 14 \mod{17},\: 
3^{10} \equiv 8 \mod{17},\: 
3^{11} \equiv 7 \mod{17},\: 
3^{12} \equiv 4 \mod{17}, 
\nonumber \\
3^{13} \equiv 12 \mod{17},\: 
3^{14} \equiv 2 \mod{17},\:
3^{15} \equiv 6 \mod{17},\: 
3^{16} \equiv 1 \mod{17},
\nonumber
\end{eqnarray}
т. о. можно видеть, что $ind_3{13} \mod{17} = 4$, 
т. к. $3^4 \equiv 13 \mod{17}$. 
\nonumber
\end{example}

Задача о нахождении дискретного логарифма является сложной
задачей. Самый быстрый из известных алгоритмов
\cite{bGordon93discretelogarithms} решает ее за время порядка 
\(
O\left(c \cdot
exp\left(\log{p}^{\frac{1}{3}}\log{\log{p}}^{\frac{2}{3}}
\right)\right)
\), где $c$ - некоторая константа,
что обуславливает широкое применение алгоритмов использующих
дискретный логарифм в криптографии.

\section{Протокол Диффи-Хеллмана (Diffie-Hellman, DH)}
Предположим существуют два абонента Алиса и Боб. Им известны два числа
$g$ и $p$, которые не являются секретными.

Алиса выбирает случайное число $a$ и пересылает Бобу следущее значение
\begin{equation}
A \equiv g^a \mod{p}.
\label{eqAddDiscretIndDHA}
\end{equation}

Боб вычисляет следующее число (с помощью секретной случайной величины
$b$)
\begin{equation}
B \equiv g^b \mod{p}.
\label{eqAddDiscretIndDHB}
\end{equation}

Алиса, с помощью только ей известного числа $a$ вычисляет ключ
\begin{equation}
K \equiv B^a\mod{p} \equiv g^{ab} \mod{p}.
\label{eqAddDiscretIndDHKey}
\end{equation}

Боб может получить то же самое значение ключа с помощью своего
секретного числа $b$:
\begin{equation}
K \equiv A^b\mod{p} \equiv g^{ab} \mod{p}.
\label{eqAddDiscretIndDHKeyB}
\end{equation}

Таким образом Алиса и Боб получают один и тот же ключ, который в
дальнейшем может быть использован для переди сообщения с помощью
симметричных алгоритмов шифрования (например AES).

\begin{example}
\emph{(Диффи-Хеллман)}
Исходные данные (открытая информация): $g = 2$, $p = 23$. Алиса
выбирает случайное число $a = 6$ и вычисляет по формуле
(\ref{eqAddDiscretIndDHA}) число 
%power_mod(2, 6, 23);
$A = 18$ и отсылает его Бобу.
Боб выбирает случайное число $b=9$ и, с помощью формулы 
(\ref{eqAddDiscretIndDHB}), получает
%power_mod(2, 9, 23);
$B = 6$ и отсылает и отсылает это число Алисе.

Алиса вычисляет ключ 
% power_mod(6, 6, 23);
$K = 12$ по формуле (\ref{eqAddDiscretIndDHKey}). Боб может полчить
тоже значение ключа 
% power_mod(18, 9, 23);
$K = 12$ используя (\ref{eqAddDiscretIndDHKeyB})
\nonumber
\end{example}

Злоумышленнику Еве известны числа $g$, $p$, $A$ и $B$. Для получения
ключа $K$ Еве необходимо получить одно из секретных чисел $a$ или $b$:
\begin{equation}
a \equiv ind_g\left( A \right) \mod{p},
\nonumber
\end{equation}
откуда с помощью (\ref{eqAddDiscretIndDHKey}) получается искомое
значение $K$.

\section{Схема Эль-Гамаля (Elgamal)}
Одной из вариаций протокола Диффи-Хелмана является схема Эль
Гамаля. Следует различать алгоритм шифрования и алгоритм цифровой
подписи Эль-Гамаля. Цифровая подпись Эль-Гамаля лежит в основе
стандартов цифровой подписи США (DSA) и России (ГОСТ Р 34.10-94).

\section{Эллиптическая криптография}
