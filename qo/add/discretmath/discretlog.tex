%% -*- coding:utf-8 -*-
\section{Дискретный логарифм}
\label{AddDiscretLog}

\begin{definition}[Дискретный логарифм]
Рассмотрим некоторую абелеву мультипликативную группу $G$ и уравнение 
\begin{equation}
g^x = a
\label{eq:add:dm:discretlog}
\end{equation}
Решение этого уравнения, т.е. целое неотрицательное число $x$
удовлетворяющее равенству \eqref{eq:add:dm:discretlog} называется
дискретным логарифмом.
\end{definition}

Уравнение \eqref{eq:add:dm:discretlog} в общем случае имеет решение не
для любых значений $a$. Но в том случае, если $g$ - порождающий
элемент $G$, т.е. $G=<g>$, то решение всегда существует и оно
единственно. В дальнейшем говоря о дискретном логарифме мы будем
предполагать, что $g$ выбрано таким образом, что $G=<g>$.

В прикладной криптографии часто имеют дело с особой разновидностью
дискретного логарифма в кольце вычетов по модулю $p$:
\begin{definition}[Дискретный логарифм в кольце вычетов по модулю $p$]
Дискретным логарифмом $ind_g\left(a\right) \mod{p}$
\footnote{От слова {\bf ind}ex - альтернативного названия для дискретного логорифма}
называется
минимальное число $x$, которое удовлетворяет следующему уравнению
(если такое число существует): 
\begin{equation}
g^x \equiv a \mod{p}
\end{equation}
\end{definition}

\begin{example}
\emph{($ind_3{13} \mod{17}$)}
Решим задачу методом перебора \cite{bWikiDiscretLog}:
\begin{eqnarray}
3^1 \equiv 3 \mod{17},\: 
3^2 \equiv 9 \mod{17},\: 
3^3 \equiv 10 \mod{17},\:
3^4 \equiv 13 \mod{17}, 
\nonumber \\
3^5 \equiv 5 \mod{17},\: 
3^6 \equiv 15 \mod{17},\: 
3^7 \equiv 11 \mod{17},\: 
3^8 \equiv 16 \mod{17}, 
\nonumber \\
3^9 \equiv 14 \mod{17},\: 
3^{10} \equiv 8 \mod{17},\: 
3^{11} \equiv 7 \mod{17},\: 
3^{12} \equiv 4 \mod{17}, 
\nonumber \\
3^{13} \equiv 12 \mod{17},\: 
3^{14} \equiv 2 \mod{17},\:
3^{15} \equiv 6 \mod{17},\: 
3^{16} \equiv 1 \mod{17},
\nonumber
\end{eqnarray}
т. о. можно видеть, что $ind_3{13} \mod{17} = 4$, 
т. к. $3^4 \equiv 13 \mod{17}$. 
\label{ex:dm:discretlog}
\end{example}

Задача о нахождении дискретного логарифма является сложной
задачей. Самый быстрый из известных алгоритмов
\cite{bGordon93discretelogarithms} решает ее за время порядка 
\(
O\left(c \cdot
exp\left(\log{p}^{\frac{1}{3}}\log{\log{p}}^{\frac{2}{3}}
\right)\right)
\), где $c$ - некоторая константа,
что обуславливает широкое применение алгоритмов использующих
дискретный логарифм в криптографии.
