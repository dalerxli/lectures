%% -*- coding:utf-8 -*- 
\section{Наибольший общий делитель. Алгоритм Евклида}
\label{AddEuclidean}
\begin{definition}
Наибольшим общим делителем числ $a$ и $b$ ($\mbox{НОД}\left(a,
b\right)$) называется максимальный из их общих делителей.
\label{defAddGCD}
\end{definition}

\begin{theorem}
\emph{(Теорема об общих делителях)}
Допустим, что выполняются следующие неравенства
$a > b > 0$, а число $r$ является остатком от деления $a$ на
$b$. Т. о. можно записать 
\begin{equation}
a = x \cdot b + r,
\label{eqAddGCD_a}
\end{equation}
где $x \ge 1$, $b > r \ge 0$. Если $r=0$, то $b$ является максимальным числом на
которое делятся без остатка и $a$ и $b$. В случае, если $r > 0$, то
\begin{equation}
\mbox{НОД}\left(a, b\right) = \mbox{НОД}\left(b, r\right).
\label{eqAddGCD_ab_br}
\end{equation}
\end{theorem}

\begin{proof}
Для доказательства (\ref{eqAddGCD_ab_br}) покажем, что любой делитель
пары чисел $\left(a,b\right)$  является делителем пары чисел
$\left(b,r\right)$. Пусть $d$ некоторый общий делитель чисел  $a$ и
$b$, т. е. $a = d \cdot x_1$, $b = d \cdot x_2$. Т. о. из (\ref{eqAddGCD_a}) следует
\begin{equation}
r = a - x \cdot b = 
d \cdot \left( x_1 - x \cdot x_2 \right),
\nonumber
\end{equation}
т. е. $d$ является делителем числа $r$.

Теперь докажем, что любой общий делитель чисел $b$ и $r$ будет является общим
делителем чисел $a$ и $b$, действительно пусть $d$ - общий делитель
чисел $b$ и $r$, т. е. $b = y_1 \cdot d$ и  $r = y_2 \cdot d$,
т. о. (\ref{eqAddGCD_a}) переписывается в виде
\begin{equation}
a = x \cdot y_1 \cdot d + y_2 \cdot d = d \cdot \left( x \cdot y_1 +
y_2 \right),
\nonumber
\end{equation}
т. е. $d$ является делителем числа $a$.

Таким образом пары чисел  $\left(a,b\right)$  и $\left(b,r\right)$
имеют общие делители, в том числе и максимальный делитель для одной
пары будет являться таковым и для второй.  
\end{proof}

Соотношение (\ref{eqAddGCD_ab_br}) приводит к следующему алгоритму
вычисления наибольшего общего делителя
\begin{algorithm}
\caption{Алгоритм Евклида}
\begin{algorithmic}
    \STATE $a > b$
    \IF{$b = 0$} 
        \RETURN $a$
    \ENDIF
    \STATE $a \Leftarrow 0$
    \STATE $r \Leftarrow b$
    \STATE $b \Leftarrow a$
    \REPEAT
        \STATE $a \Leftarrow b$
        \STATE $b \Leftarrow r$
        \STATE $r \Leftarrow $ остаток от деления $a$ на $b$
    \UNTIL{ ($r \ne 0$) }
    \RETURN $b$
\end{algorithmic}
\label{algAddEuclid}
\end{algorithm}

Для оценки сложности алгоритма \ref{algAddEuclid} запишем его в
следующем виде 
\begin{eqnarray}
f_k = x_k \cdot f_{k + 1} + f_{k + 2},
\nonumber \\
f_0 = a, \: f_1 = b,
\nonumber \\
x_k \ge 1, \: f_k > f_{k+1} > f_{k + 2},
\nonumber
\end{eqnarray}
т. о. $f_k > 2 \cdot f_{k + 2}$, или же 
$f_0 > 2 \cdot f_2 > \dots > 2^nf_{2n}$ т. е. алгоритм остановится при
$n = log_2\left(f_0\right) = log_2\left(a\right)$
\footnote{Дальнейшие выкладки сделаны в предположении, что
  $log_2\left(a\right)$ - целое число}. Число шагов
алгоритма при этом очевидно равно $2n$ или же 
$2 \cdot log_2\left(a\right)$. Т. о. алгоритмическая сложность
алгоритма Евклида может быть записана как 
$O\left(\log \left(a\right)\right)$.


\begin{example}
\emph{($\mbox{НОД}\left(2345,1456\right)$)}
%(%i5) gcd(2345,1456);
%(%o5)                                  7
%(%i6)      
%(%i8) mod(2345,1456);
%(%o8)                                 889
%(%i9) mod(1456,889);
%(%o9)                                 567
%(%i10) mod(889,567);
%(%o10)                                322
%(%i11) mod(567,322);
%(%o11)                                245
%(%i12) mod(322,245);
%(%o12)                                77
%(%i13) mod(245,77);
%(%o13)                                14
%(%i14) mod(77,14);
%(%o14)                                 7
%(%i15) mod(14,7);
%(%o15)                                 0
%(%i16) 
\begin{eqnarray}
2345 = 1456 + 889,
\nonumber \\
1456 = 889 + 567,
\nonumber \\
889 = 567 + 322,
\nonumber \\
567 = 322 + 245,
\nonumber \\
322 = 245 + 77,
\nonumber \\
245 = 3 \cdot 77 + 14,
\nonumber \\
77 = 5 \cdot 14 + 7,
\nonumber \\
14 = 2 \cdot 7.
\nonumber
\end{eqnarray}
т. о. $\mbox{НОД}\left(2345,1456\right) = 7$. 
Число шагов алгоритма  $8 < 2 \cdot log_2{2345} \approx 2 \cdot 11.2 = 22.4$.
\nonumber
\end{example}
