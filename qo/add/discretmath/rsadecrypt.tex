%% -*- coding:utf-8 -*-
$m$ может быть восстановлено из $c$ по следующей формуле:
\begin{equation}
m \equiv c^d \mod{n}.
\label{eqAddRSADeCode}
\end{equation}
Имея $m$ можно восстановить исходное сообщение $M$.
\begin{example}
\emph{(RSA. Де-шифрование)}
Допустим у нас есть закрытый ключ $\left(n=21, d=5\right)$ (см. прим. \ref{exAddRSAKeyGen}) и шифротекст $c = 2$ из примера \ref{exAddRSACode}.

%(%i2) power_mod(2,5,21);
%(%o2)                                 11
Исходный текст вычисляется по формуле \eqref{eqAddRSADeCode} $m \equiv 2^5 \mod{21} = 11 = 1101_2$.
\label{exAddRSADeCode}
\end{example}
