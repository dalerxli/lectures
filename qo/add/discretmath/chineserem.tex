%% -*- coding:utf-8 -*-
\section{Китайская теорема об остатках}

\begin{theorem}
\label{thm:chineseremainder}
Если имеются взаимно простые целые числа $n_1, n_2, \dots, n_k$, тогда
для любого набора целых чисел $a_1, a_2, \dots a_k$ $\exists x$ такой
что 
\begin{eqnarray}
x \equiv a_1 \mod n_1,
\nonumber \\
x \equiv a_2 \mod n_2,
\nonumber \\
\vdots \nonumber \\
x \equiv a_k \mod n_k,
\end{eqnarray}
При этом для любых $x_1, x_2$ удовлетворяющих этому соостношению имеет
место равенство
\[
x_1 \equiv x_2 \mod N,
\]
где $N = n_1 \cdot n_2 \cdot \dots \cdot n_k$.
\end{theorem}
