%% -*- coding:utf-8 -*-
\section{Кольца, поля}

\begin{definition}[Кольцо]
  Consider a set $R$ with 2 binary operations defined. The first one
  $\oplus$ (addition) and elements of $R$ forms an
  \ref{def:abeliangroup}
  under this operation. The second one is $\odot$ (multiplication) and
  the elements of $R$ forms a \autoref{def:monoid} under 
  the operation. The two binary operations are connected each other
  via the following distributive law
  \begin{itemize}
  \item Left distributivity:
    $\forall a,b,c \in R$:
    $a \odot \left(b \oplus c\right) =
    a \odot b \oplus a \odot c$
  \item Right distributivity:
    $\forall a,b,c \in R$:
    $\left( a \oplus b \right) \odot c =
    a \odot c \oplus b \odot c$
    
  The identity element for $\left(R, \oplus\right)$ is denoted as $0$
  (additive identity).
  The identity element for $\left(R, \odot\right)$ is denoted as $1$
  (multiplicative identity).

  The inverse element to $a$ in $\left(R, \oplus\right)$ is denoted as $-a$
  \end{itemize}

  In this case $\left(R, \oplus, \odot\right)$ is called as ring.
  \label{def:ring}
\end{definition}

The \autoref{def:ring} is a generalization of integer numbers conception.
\begin{example}[Ring of integers $\mathbb{Z}$]
  The set of integer numbers $\mathbb{Z}$ forms a \autoref{def:ring}
  under $+$ and $\cdot$ operations i.e. addition $\oplus$ is
  $+$ and multiplication $\odot$ is $\cdot$. Thus for integer
  numbers we have the following \autoref{def:ring}:
  $\left(\mathbb{Z}, +, \cdot\right)$
  \label{ex:ring}
\end{example}

\begin{definition}[Поле]
  The ring $\left(R, \oplus, \odot\right)$ is called as a field if
  $\left(R \setminus \{0\}, \odot\right)$ is an \autoref{def:abeliangroup}.

  The inverse element to $a$ in
  $\left(R \setminus\{0\}, \odot\right)$ is denoted as $a^{-1}$
  \label{def:field}
\end{definition}

\begin{example}[Field $\mathbb{Q}$]
  Note that $\mathbb{Z}$ is not a field because not for every integer
  number an inverse exists. But if we consider a set of fractions
  $\mathbb{Q} = \left\{a/b \mid a \in \mathbb{Z}, b \in
  \mathbb{Z}\setminus\{0\}\right\}$ when it will be a field.

  The
  inverse element to $a/b$  in
  $\left(\mathbb{Q}\setminus\{0\}, \cdot\right)$  will be $b/a$.
  \label{ex:field_q}
\end{example}

\begin{example}[Поле $\mathbb{R}$]
TBD
  \label{ex:field_r}
\end{example}

\begin{example}[Поле $\mathbb{C}$]
TBD
  \label{ex:field_c}
\end{example}
