%% -*- coding:utf-8 -*-
\section{Классы сложности $P$, $NP$}

\begin{definition}[Размер задачи]
Размером задачи $n$ мы будем называть число символов алфавита машины
Тьюринга, содержащихся на ленте в исходном состоянии, т. е. другими
словами размером задачи мы будем считать длину строки которая
поступает на вход машины Тьюринга.
\end{definition}

\begin{definition}[Класс $P$]
TBD
\end{definition}

\begin{definition}[Класс $NP$]
TBD
\end{definition}

\begin{definition}[Класс $NP$-полный ($NP$-complete)]
TBD
\end{definition}


TBD

\begin{theorem}
\emph{(Теорема Кука-Левина)}
Задача о выполнимости булевой формулы в КНФ\footnote{Конъюнктивная нормальная форма} (SAT) является NP полной
\label{theoremAddAlgoCookTheorem}
\end{theorem}

\begin{proof}
TBD
\end{proof}
