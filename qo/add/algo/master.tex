%% -*- coding:utf-8 -*-
\section{Основная теорема о рекуррентных соотношениях}

\begin{theorem}
\emph{(Основная теорема о рекуррентных соотношениях)}
\label{addAlgoMasterTheorem}
Если имеется следующее рекурентное соотоношение для сложности
некоторого алгоритма
\[
T\left( n \right ) = a T \left( \frac{n}{b} \right) + f\left( n \right ),
\]
то возможно определить асимптотическое поведение функции 
$T\left( n\right ) $ в следующих случаях
\begin{enumerate}
\item Если $f\left(n\right) = O\left( n^{log_ba - \epsilon}\right)$,
  при некоторых $\epsilon > 0$, то 
$T\left(n\right) = \Theta\left(n^{log_ba}\right)$
\item Если 
$f\left(n\right) = \Theta\left( n^{log_ba}log^{k}n\right)$, то 
$T\left(n\right) = \Theta\left(n^{log_ba}log^{k + 1}n\right)$
\item Если $f\left(n\right) = \Omega\left( n^{log_ba + \epsilon}\right)$,
  при некоторых $\epsilon > 0$ и $a f\left(\frac{n}{b}\right) \le c f
  \left( n \right)$ для некоторой константы $c < 1$ и больших $n$, то 
$T\left(n\right) = \Theta\left(f\left(n\right)\right)$
\end{enumerate}
\end{theorem}
