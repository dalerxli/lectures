%% -*- coding:utf-8 -*-
\section{Разрешимые задачи}

Машина Тюринга $M$ (см. рис. \ref{figAddAlgoDecidability}) может
трактоваться как некоторая функция $f$ в качестве 
аргумента которой является набор символов представляющий собой
начальное состояние ленты $x$. В результате обработки $x$ машина $M$
может завершить свою работу, т. е. оказаться в некотором финальном
состоянии $f \in F \subset Q$. В этом случае говорят, что $M$
принимает $x$. 

Машина Тьюринга задает класс функций для которых можно определить
такое понятие как разрешимость.

\input ./add/algo/figdecidability.tex

\begin{theorem}
\emph{(Теорема Райса)}
Задача об определении любого нетривиального свойства машины Тьюринга
является неразрешимой задачей
\end{theorem}

\begin{proof}
TBD
\end{proof}

