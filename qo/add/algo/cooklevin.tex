%% -*- coding:utf-8 -*-
\subsection{Теорема Кука-Левина}

Как только мы ввели класс сложности $NPC$ встает вопрос о задачах
которые в него входят. Ниже будет показано, что задача SAT (о
выполнимости булевой формулы в конъюнктивной нормальной форме)
является $NP$ полной.  

\begin{theorem}
\emph{(Теорема Кука-Левина)}
Задача о выполнимости булевой формулы в КНФ является NP полной,
т. е. принадлежит классу $NPC$.
\label{theoremAddAlgoCookTheorem}
\end{theorem}

\begin{proof}
  Доказательство состоит из двух частей. Во первых мы должны доказать
  задача SAT (см. пример \ref{exAddAlgoSAT}) принадлежит классу
  $NP$. Во вторых мы должны показать, что любая задача из класса $NP$
  может быть сведена к задаче SAT за полиномиальное время.
  
  Принадлежность к классу $NP$ доказывается очевидным образом, -
  т. к. существует возможность проверить истинность решения за число
  шагов $O\left(N\right)$, где $N$ - число булевых операций
  использованных в рассматриваемой булевой формуле.

 
  TBD
\end{proof}

