%% -*- coding:utf-8 -*-
\section{Машина Тюринга}
\label{addTuring}

Машина Тюринга $TM$ - математический объект который представляет собой
совокупность следующих елементов. 
\begin{itemize}
\item{Множество состояний $Q = \left\{q_i\right\}$}
\item{Множество элементов (алфавит) ленты $\Gamma = \left\{\gamma_i\right\}$}
\item{Множество входных элементов $\Sigma \subset \Gamma$}
\item{Таблица переходов $\Delta = \left\{\delta_j\right\}$}
\item{Начальное состояние $q_0 \in Q$}
\item{Множество финальных состояний $F \subset Q$}
\item{Множество направлений движения головки $D = \left\{L, R\right\}$}
\end{itemize}

\input ./add/algo/figturing.tex

Элемент таблицы переходов записывается в следующем виде:
$\delta_j\left(q_{old}, \gamma_{old}\right)$ где $q_{old} \in Q$ текущее состояние,
$\gamma_{old} \in \Gamma$ - выбранный элемент на ленте 
(см. рис. \ref{figAddAlgoTuring}). Значение перехода может быть
неопределено или записано в виде $\left\{q_{new}, \gamma_{new}, d\right\}$, где
  $q_{new} \in Q$ - новое состояние машины Тюринга, $\gamma_{new} \in Gamma$ -
  новое значение элемента на ленте на который указывала головка,
  т. е. $\gamma_{old}$ заменяется на $\gamma_{new}$. $d \in D$ - направление
  движения головки ($L$ налево, $R$ направо). В случае неопределенного
  перехода говорят, что машина Тюринга зависла (halt) в 
  состоянии $q_{old}$.

