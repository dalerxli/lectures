%% -*- coding:utf-8 -*-
\section{Машина Тьюринга}
\label{addTuring}

Машина Тьюринга $TM$ - математический объект состоящий из
управляющего устройства (head) и бесконечной ленты разделенной на
ячейки в каждой из которых может быть записан один символ.

\input ./add/algo/figturing.tex

Управляющее устройство $TM$ может находится в одном из состояний $q_i
\in Q$, где $Q$ 
- множество различных состояний. Состояние в котором $TM$ находилась в
начальный момент времени $q_0 \in Q$ называется начальным состоянием. 

Среди элементов множества $Q$ выбрано подмножество $F \subset Q$
финальных состояний. Если $TM$ оказалась в состоянии $f \in F \subset
Q$, то говорят что машина Тьюринга завершила (terminate) свою работу.

Символы записанные на ленту образуют некоторое множество (алфавит)
$\Gamma = \left\{\gamma_i\right\}$.

Входным алфавитом $\Sigma$ машины Тьюринга называется называется
подмножество элементов $\Gamma$ которыми оперирует управляющее устройство $TM$,
т. е. те символы, которые она может распознать на ленте. Те символы,
которые управляющее устройство не может распознать мы будем обозначать символом
$\beta$ (blank).

В процессе работы $TM$ она выполняет следующие действия:
\begin{itemize}
\item{считывает элемент на который указывает управляющее устройство
  (см. рис. \ref{figAddAlgoTuring})}
\item{изменяет (или оставляет неизменным) этот символ на элемент
  $\sigma \in \Sigma \subset \Gamma$}
\item{изменяет положение управляющего устройства на одну ячейку вправо
  ($R$), влево ($L$) или не меняет его ($S$)}
\end{itemize}

\input ./add/algo/figturingtrans.tex

Изменение состояния $TM$ описывается таблицей переходов
$\Delta$. Элемент таблицы переходов записывается в следующем виде: 
$\delta_j\left(q_{old}, \gamma_{old}\right)$ где $q_{old} \in Q$ текущее состояние,
$\gamma_{old} \in \Gamma$ - выбранный элемент на ленте. Значение
перехода может быть 
не определено или записано в виде $\left\{q_{new}, \gamma_{new},
d\right\}$(см. рис. \ref{figAddAlgoTuringTrans}), где
$q_{new} \in Q$ - новое состояние машины Тьюринга, $\gamma_{new} \in \Gamma$ -
новое значение элемента на ленте на который указывало управляющее устройство,
т. е. $\gamma_{old}$ заменяется на $\gamma_{new}$. $d \in D =
\left\{L, R, S\right\}$ - направление
движения управляющего устройства ($L$ налево, $R$ направо, $S$
оставаться на том же самом месте). В случае неопределенного
перехода говорят, что машина Тьюринга зависла (halt) в состоянии $q_{old}$.

\begin{definition}[Язык машины Тьюринга $L\left(M\right)$]
Исходные символы на ленте могут трактоваться как некоторая строка $x$
которая поступает на вход машины Тьюринга $M$. Все те строки которые
приводят машину $M$ в некоторое финальное состояние $f \in F$
называются языком $L\left(M\right)$ машины $M$
\end{definition}

\begin{definition}[Детерминированная машина Тьюринга]
Если в таблице переходов отображение
\(
\left(q_{old}, \gamma_{old}\right) \rightarrow 
\left\{q_{new}, \gamma_{new}, d\right\}
\)
является биекцией 
\footnote{Биекция - взаимно однозначное отображение, т. е. в нашем
  случае каждому состоянию $\left(q_{old}, \gamma_{old}\right)$
  соответствует только один элемент таблицы переходов 
  $\left\{q_{new}, \gamma_{new}, d\right\}$
},
то соответствующая машина Тьюринга
называется детерминированной машиной Тьюринга
\label{defAlgoDMT}
\end{definition}

\begin{definition}[Не-детерминированная машина Тьюринга]
Если в таблице переходов отображение
\(
\left(q_{old}, \gamma_{old}\right) \rightarrow 
\left\{q_{new}, \gamma_{new}, d\right\}
\)
не является биекцией, т. е. состоянию $\left(q_{old}, \gamma_{old}\right)$
может соответствовать несколько элементов таблицы переходов 
$\left\{q_{new}, \gamma_{new}, d\right\}$,
то соответствующая машина Тьюринга
называется не-детерминированной машиной Тьюринга
\label{defAlgoNDMT}
\end{definition}

