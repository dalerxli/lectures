%% -*- coding:utf-8 -*-
\section{Двумерное преобразование Фурье}
\subsection{Определение}

\begin{definition}[Двумерное преобразование Фурье]
Допустим у нас имеется двумерный сигнал $x\left(k_1, k_2\right)$, где 
$k_1,k_2 \in \{0, \dots, M - 1\}$. Двумерным преобразованием Фурье
называется двумерная функция 
\[
\tilde{X}\left(j_1, j_2\right), j_1,j_2
\in \{0, \dots, M - 1\}
\]
такая, что
\[
\tilde{X}\left(j_1, j_2\right) = 
\frac{1}{M}\sum_{k_1 = 0}^{M-1}\sum_{k_2 = 0}^{M-1}
x\left(k_1, k_2\right)e^{-i \omega\left(k_1 j_1 + k_2 j_2\right)},
\]
где
\[
\omega = \frac{2 \pi}{M}.
\]
\label{def:add:dsp:fourier2d}
\index{двумерное преобразование Фурье!определение}
\end{definition}

