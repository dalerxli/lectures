%% -*- coding:utf-8 -*-
\subsection{Базовые понятия и аксиомы теории вероятности}

Классическая теория вероятностей имеет дело с теорией множеств и
базируется на нескольких простых аксиомах. Данная аксиоматика была
предложена в 30-х годах XX века Колмогоровым
А. Н. \cite{bKolmogorov74basic}. 

Прежде всего несколько определений объектов с которыми мы будем иметь
дело. 

\begin{definition}[Пространство событий $\Omega$]
  Множество $\Omega$ называется пространством событий если
  $\Omega \subset \Omega$, $\emptyset \subset \Omega$ и для любых
  $A,B \subset \Omega$ имеет место
  $A \cup B \subset \Omega$, $A \cap B \subset \Omega$ и
  $A \setminus B \subset \Omega$.  Элемент множества
  $\omega \in \Omega$ называется элементарным событием. Подмножество 
  $A \subset \Omega$ называется событием.
\end{definition}

\begin{definition}[Вероятность события $P$]
  Каждому событию $A \subset \Omega$ ставится в соответствие некоторое
  число $P\left(A\right) \in \mathbb{R}$ которое называется вероятностью.
\end{definition}

\begin{example}[Пространство событий]
\input ./add/probability/figdefinition.tex
На рис. \ref{figAddProbabilityDefinition} изображено множество
$\Omega$ которое мы называем пространством событий. Элементы
множества $\omega_i \in \Omega$ (черные точки на
рис. \ref{figAddProbabilityDefinition}) - элементарные
события. Подмножество $A \subset \Omega$ - событие.
\end{example}

Теперь собственно говоря аксиомы.

\begin{axiom}[Не-отрицательность]
  \label{axProbabilityKolmogorovNonNegativity}
  Вероятность события $A \subset \Omega$ есть неотрицательное
  вещественное число, т. е. $P\left(\right) \ge 0$
\end{axiom}

\begin{axiom}[Нормировка]
  \label{axProbabilityNormalization}
  Вероятность пространства событий $\Omega$ есть $1$, т. е.
  $P\left(\Omega\right) = 1$
\end{axiom}

\begin{axiom}[Аддитивность]
\label{axProbabilityAdditivity}
Если $A_i \cap A_j = \emptyset$ тогда 
$P\left(A_i \cup A_j\right) = P\left(A_i\right) + P\left(A_j\right)$
\end{axiom}

\begin{example}[Аксиомы Колмогоровской теории вероятностей]
\input ./add/probability/figaxioms.tex
Каждому элементарному событию $\omega_i \in \Omega$
на рис. \ref{figAddProbabilityAxioms} поставим в соответствие
неотрицательное число 
$P\left(\omega_i\right) = \frac{1}{12}$. Вероятность события $A$ - 
$P\left(A\right) = \frac{5}{13}$ для события $B$ имеем
$P\left(B\right) = \frac{4}{13}$. Таким образом в силу того, что 
$A \cap B = \emptyset$, используя аксиому
\ref{axProbabilityAdditivity} получим:
\[
P\left(A\cup B\right) = 
P\left(A\right) + P\left(B\right) = 
\frac{5}{12} + \frac{4}{12} = \frac{3}{4}.
\]
Для события $\Omega$ в силу аксиомы \ref{axProbabilityNormalization}
имеем $P\left(\Omega\right) = 1$. С другой стороны 
$\Omega = \cup_i \omega_i$ т.е. 
\[
P\left(\Omega\right) = \sum_i P\left(\omega_i\right) =
12\cdot\frac{1}{12} = 1.
\]
\end{example}


Все известные факты теории вероятности выводятся из этих трех аксиом.  

\input ./add/probability/figcond.tex
