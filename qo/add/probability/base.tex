%% -*- coding:utf-8 -*-
\subsection{Базовые понятия и аксиомы теории вероятности}

Классическая теория вероятностей имеет дело с теорией множеств и
базируется на нескольких простых аксиомах. Данная аксиоматика была
предложена в 30-х годах XX века Колмогоровым
А. Н. \cite{bKolmogorov74basic}. 

Прежде всего несколько определений объектов с которыми мы будем иметь
дело. 

\begin{definition}[Пространство событий $\Omega$]
  Множество $\Omega$ называется пространством событий если
  $\Omega \subset \Omega$ и
  $\emptyset \subset \Omega$. Элемент множества
  $A \subset \Omega$ называется событием.
\end{definition}

\begin{definition}[Вероятность события $P$]
  Каждому событию $A \subset \Omega$ ставится в соответствие некоторое
  число $P\left(A\right) \in \mathbb{R}$ которое называется вероятностью.
\end{definition}

Теперь собственно говоря аксиомы.

\begin{axiom}[Не-отрицательность]
  \label{axProbabilityKolmogorovNonNegativity}
  Вероятность события $A \subset \Omega$ есть неотрицательное
  вещественное число, т. е. $P\left(\right) \ge 0$
\end{axiom}

\begin{axiom}[Нормировка]
  \label{axProbabilityNormalization}
  Вероятность пространства событий $\Omega$ есть $1$, т. е.
  $P\left(\Omega\right) = 1$
\end{axiom}

\begin{axiom}[Аддитивность]
\label{axProbabilityAdditivity}
Если $A_i \cap A_j = \emptyset$ тогда 
$P\left(A_i \cup A_j\right) = P\left(A_i\right) + P\left(A_j\right)$
\end{axiom}

Все известные факты теории вероятности выводятся из этих трех аксиом.  
