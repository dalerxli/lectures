%% -*- coding:utf-8 -*-
\section{Классическая Колмогоровская теория вероятностей}

Классическая теория вероятностей имеет дело с теорией множеств и
базируется на нескольких простых аксиомах. 

Прежде всего мы имеем множество $\Omega$.
\begin{axiom}
\label{axProbabilityKolmogorov1}
Каждому подмножеству $A_i \subset \Omega$ мы ставим в соответствие
некоторое число $0 \ge P_i \le 1$
\end{axiom}

\begin{axiom}
\label{axProbabilityKolmogorov2}
$P\left(\Omega\right) = 1$
\end{axiom}

\begin{axiom}
\label{axProbabilityKolmogorov3}
Если $A_i \cap A_j = \emptyset$ тогда 
$P\left(A_i \cup A_j\right) = P\left(A_i\right) + P\left(A_j\right)$
\end{axiom}

Все известные факты теории вероятности выводятся из этих трех аксиом.  


