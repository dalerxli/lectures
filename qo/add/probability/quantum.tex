%% -*- coding:utf-8 -*-
\section{Квантовая теория вероятностей}
\label{sec:add:quantprobability}
Говоря о квантовой теории вероятности, мы будем опираясь на классическую
(Колмогоровскую), при этом мы постараемся выбрать такой вариант
формулировки который будет применим к обеим случаям. В качестве
модельной будем рассматривать систему с $n$ возможными исходами, т.е.
ту в которой классическое распределение вероятностей задается
следующим распределением:
\begin{equation}
p = \{p_1, p_2, \dots, p_n\}
\label{eq:quantprob_class}
\end{equation}

Квантовое состояние полностью описывается в общем случае (как чистых
так и смешанных состояний) с помощью матрицы плотности
\index{Матрица плотности}
\begin{equation}
\hat{\rho} = \begin{bmatrix}
\rho_{11} & \rho_{12} & \cdots & \rho_{1n} \\
\rho_{21} & \rho_{22} & \cdots & \rho_{2n} \\
\vdots & \vdots & \ddots & \vdots \\
\rho_{n1} & \rho_{n2} & \cdots & \rho_{nn} \\
\end{bmatrix}.
\nonumber
\end{equation}
В классическом случае этому объекту соответствует
распределение вероятностей \eqref{eq:quantprob_class}, которое
также можно представить в виде некоторой матрицы
\begin{equation}
\hat{p} = \begin{bmatrix}
p_1 & 0 & \cdots & 0 \\
0 & p_2 & \cdots & 0 \\
\vdots & \vdots & \ddots & \vdots \\
0 & 0 & \cdots & p_n \\
\end{bmatrix}.
\nonumber
\end{equation}
Стоит отметить, что для обеих случаев выполняется равенство
\[
Sp\left(\hat{\rho}\right) =
Sp\left(\hat{p}\right) = 1,
\]
которое представляет собой переписанный вариант аксиомы
\ref{ax:ProbabilityNormalization}. 

Наблюдаемой $\hat{x}$ соответствует некоторая матрица
\begin{equation}
\hat{x} = \begin{bmatrix}
x_{11} & x_{12} & \cdots & x_{1n} \\
x_{21} & x_{22} & \cdots & x_{2n} \\
\vdots & \vdots & \ddots & \vdots \\
x_{n1} & x_{n2} & \cdots & x_{nn} \\
\end{bmatrix},
\label{eq:quantprob_obser_quant}
\end{equation}
которая в классическом случае является диагональной 
\begin{equation}
\hat{x} = \begin{bmatrix}
x_1 & 0 & \cdots & 0 \\
0 & x_2 & \cdots & 0 \\
\vdots & \vdots & \ddots & \vdots \\
0 & 0 & \cdots & x_n \\
\end{bmatrix}.
\label{eq:quantprob_obser_class}
\end{equation}
При этом стоит отметить, что вычисление среднего при этом приобретает
знакомы классический вид
\begin{equation}
\bar{x} = Sp\left(\hat{p} \hat{x}\right) = \sum_{i=1}^n p_i x_i.
\nonumber
\end{equation}

Отличие квантовой теории вероятностей от классической может быть
показано несколькими способами \cite{bHolevo2003, bHolevo2003add},
один из которых основан на не-коммутативности квантового описания.
Действительно можно отметить, что в общем случае квантовые наблюдаемые
представляемые матрицами вида \eqref{eq:quantprob_obser_quant} не
коммутируют друг с другом. При этом классические наблюдаемые,
записываемые в виде диагональных матриц
\eqref{eq:quantprob_obser_class} всегда коммутируют друг с другом. Из
этого выводится много фактов TBD


