%% -*- coding:utf-8 -*- 
\section{Введение в теорию групп}
\begin{definition}
Группой $\mathcal{G}$ называется множество элементов $g \in
\mathcal{G}$ для которой определена 
операция умножения:
\begin{eqnarray}
\forall g_1,g_2 \in \mathcal{G},
\nonumber \\
g_1 \cdot g_2 \in \mathcal{G}.
\label{addGroupMulDef}
\end{eqnarray}
Операция умножения (\ref{addGroupMulDef}) обладает свойством
ассоциативности:
\begin{equation}
g_1 \cdot \left( g_2 \cdot g_3 \right ) = 
\left( g_1 \cdot  g_2 \right ) \cdot g_3.
\nonumber
\end{equation}
Рассматриваемое множество должно содержать элемент $e$ обладающий
следующим свойством, справедливым для любого элемента множества $g$:
\begin{equation}
g \cdot e = e \cdot g = g.
\nonumber
\end{equation}
Для каждого элемента группы $g$ должен существовать обратный
элемент $g^{-1} \in \mathcal{G}$, обладающий следующим свойством
\begin{equation}
g \cdot g^{-1} = g^{-1} \cdot g = e
\nonumber
\end{equation}
\label{defAddGroup}
\end{definition}

\begin{example}
\emph{Группа $\left(\mathcal{Z}, +\right)$}
Множество целых чисел $\mathcal{Z} = \left\{0, \pm1, \pm2,
\dots\right\}$ представляет собой группу относительно операции сложения.
\nonumber
\end{example}
