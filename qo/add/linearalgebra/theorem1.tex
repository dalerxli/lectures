%% -*- coding:utf-8 -*- 
\subsection{Теорема об операторном разложении}
\label{AddTheorem1}

Докажем тождество:
\begin{equation}
e^{\hat{A}}\hat{B}e^{-\hat{A}} = 
\hat{B} + \left[\hat{A},\hat{B}\right] + 
\frac{1}{2!} \left[\hat{A},\left[\hat{A},\hat{B}\right]\right] + \dots
\label{eqAddTheorem1_Main}
\end{equation}
Для это введем следующую вспомогательную функцию:
\begin{equation}
f\left(\alpha\right) = 
e^{\alpha\hat{A}}\hat{B}e^{-\alpha\hat{A}},
\label{eqAddTheorem1_F}
\end{equation}
которая совпадает с искомым выражением при $\alpha = 1$. Разложим
\eqref{eqAddTheorem1_F} в ряд Тейлора:
\begin{equation}
f\left(\alpha\right) = 
f\left(0\right) + \frac{\alpha}{1!}f'\left(0\right) +
\frac{\alpha^2}{2!}f''\left(0\right) + 
\dots +
\frac{\alpha^k}{k!}f^{(k)}\left(0\right) + \dots.
\label{eqAddTheorem1_T}
\end{equation}
Справедливо следующее рекурентное соотношение:
\begin{eqnarray}
f^{(k)}\left(\alpha\right) = \left[\hat{A},
  f^{(k-1)}\left(\alpha\right)\right] = 
\nonumber \\
=
\underbrace{
[\hat{A}[\dots[\hat{A}}_{\mbox{k
  раз}},
f^{(0)}\left(\alpha\right)
\underbrace{]\dots]]}_{\mbox{k
  раз}},
\label{eqAddTheorem1_rec}
\end{eqnarray}
где 
\[
f^{(0)}\left(\alpha\right) = f\left(\alpha\right).
\]
Действительно по индукции имеем для первой производной:
\begin{eqnarray}
f^{(1)}\left(\alpha\right) = \hat{A}
  e^{\alpha\hat{A}}\hat{B}e^{-\alpha\hat{A}} -
  e^{\alpha\hat{A}}\hat{B}\hat{A}e^{-\alpha\hat{A}} = 
\nonumber \\
= \hat{A}
  e^{\alpha\hat{A}}\hat{B}e^{-\alpha\hat{A}} -
  e^{\alpha\hat{A}}\hat{B}e^{-\alpha\hat{A}}\hat{A} = 
\nonumber \\
=
\left[\hat{A}, e^{\alpha\hat{A}}\hat{B}e^{-\alpha\hat{A}}\right] = 
\left[\hat{A}, f\left(\alpha\right)\right] = 
\left[\hat{A}, f^{(0)}\left(\alpha\right)\right],
\label{eqAddTheorem1_rec_1}
\end{eqnarray}
т. е. \eqref{eqAddTheorem1_rec} справедливо при $k = 1$. При выводе
\eqref{eqAddTheorem1_rec_1} мы учитывали тот факт, что 
\begin{equation}
\hat{A}e^{-\alpha\hat{A}} = e^{-\alpha\hat{A}} \hat{A}.
\nonumber
\end{equation}
Для $k+1$ производной получим:
\begin{eqnarray}
f^{(k + 1)}\left(\alpha\right) = 
\frac{\partial f^{(k)}\left(\alpha\right)}{\partial \alpha} =
\nonumber \\
=\frac{\partial }{\partial \alpha}
\left(
\underbrace{
[\hat{A}[\dots[\hat{A}}_{k},
f^{(0)}\left(\alpha\right)
\underbrace{]\dots]]}_{k} \right) =  
\underbrace{
[\hat{A}[\dots[\hat{A}}_{k},
\frac{\partial }{\partial \alpha}f^{(0)}\left(\alpha\right)
\underbrace{]\dots]]}_{k} =
\nonumber \\
= 
\underbrace{
[\hat{A}[\dots[\hat{A}}_{k},
f^{(1)}\left(\alpha\right)
\underbrace{]\dots]]}_{k} =
\underbrace{
[\hat{A}[\dots[\hat{A}}_{k},
\left[\hat{A}, f^{(0)}\left(\alpha\right)\right]
\underbrace{]\dots]]}_{k} =
\nonumber \\
=
\underbrace{
[\hat{A}[\dots[\hat{A}}_{k + 1},
f^{(0)}\left(\alpha\right)
\underbrace{]\dots]]}_{k + 1},
\nonumber
\end{eqnarray}
что вместе с \eqref{eqAddTheorem1_rec_1} доказывает справедливость 
\eqref{eqAddTheorem1_rec}.

С учетом того, что 
\begin{equation}
f^{(0)}\left(\alpha\right) = \hat{B},
\nonumber
\end{equation}
можно получить из \eqref{eqAddTheorem1_T} искомое тождество
\eqref{eqAddTheorem1_Main} при $\alpha = 1$. 


