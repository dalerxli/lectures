%% -*- coding:utf-8 -*- 
\section{Гамильтонова форма уравнений свободного электромагнитного поля}
Представим поле в объеме  $V$  в виде разложений \eqref{eqCh1_sep1}:
\begin{equation}
\vec{E}\left(r, t\right) = \sum_{(n)}
\frac{q_n\left(t\right) \omega_n}{\sqrt{\varepsilon_0}} \vec{E}_n\left(r\right),
\quad
\vec{H}\left(r, t\right) = \sum_{(n)}
\frac{p_n\left(t\right)}{\sqrt{\mu_0}} \vec{H}_n\left(r\right),
\label{eqCh1_sep1_1}
\end{equation}
причем $p_n = \frac{d q_n}{d t}$, а $q_n$ удовлетворяет уравнению
\[
\frac{d^2 q_n}{d t^2} + \omega_n^2 q_n = 0.
\]
Это уравнение гармонического осциллятора. Его решение:
\begin{equation}
q_n\left(t\right) = q_n\left(0\right) \cos \omega_n t + 
\frac{\dot{q_n}\left(0\right)}{\omega_n} \sin \omega_n t 
\end{equation}
$q_n\left(0\right)$ и 
$\dot{q_n}\left(0\right)$ 
определяются из разложения начального поля по  
$\vec{E_n}$, $\vec{H_n}$.
 
Запишем теперь уравнения электродинамики в гамильтоновой
форме. Функция Гамильтона для электромагнитного поля 
(энергия поля)
\begin{equation}
\mathcal{H} = \frac{1}{2}
\int_{\nu}\left( \varepsilon_0\left(\vec{E}\right)^2 + \mu_0
\left(\vec{H}\right)^2\right) d\nu
\label{eqCh1_hamilton}
\end{equation}
Подставляя в \eqref{eqCh1_hamilton} разложения \eqref{eqCh1_sep1_1} и
используя ортогональность собственных функций, получим 
\begin{eqnarray}
\mathcal{H} = \frac{1}{2} 
\int_{\nu}\left( \sum_{(n)} \sum_{(m)}\varepsilon_0
\frac{q_n q_m \omega_n \omega_m}{\varepsilon_0}
\left( \vec{E_n} \vec{E_m}\right)
\right) d\nu +
\nonumber \\
+  \frac{1}{2} 
\int_{\nu}\left( \sum_{(n)} \sum_{(m)}\mu_0
\frac{p_n p_m}{\mu_0}
\left( \vec{H_n} \vec{H_m}\right)
\right) d\nu =
\nonumber \\
= \frac{1}{2}\sum_{(n)}\left(\omega_n^2 q_n^2 + p_n^2\right).
\end{eqnarray}
Это выражение соответствует функции Гамильтона для совокупности
независимых гармонических осцилляторов. 

Для одной моды функция Гамильтона  
\begin{equation}
\mathcal{H}_n = \frac{1}{2}\left(\omega_n^2 q_n^2 + p_n^2\right).
\label{eqCh1_hamilton_one_mode}
\end{equation}
тогда
\begin{equation}
\mathcal{H} = \sum_{(n)} \mathcal{H}_n
\label{eqCh1_hamilton_sum_mode}
\end{equation}
Уравнения движения получим из функции Гамильтона известным путем:
\begin{equation}
\dot{q}_n = \frac{\partial \mathcal{H}_n}{\partial p_n} = p_n,
\quad
\dot{p}_n = - \frac{\partial \mathcal{H}_n}{\partial q_n} =
- \omega_n^2 q_n = \ddot{q}_n.
\end{equation}
Полученные уравнения совпадают с уравнениями \eqref{eqCh1_after_sep}.  
Функция Гамильтона механического осциллятора
\[
\mathcal{H} = \frac{1}{2}\left(M \Omega^2 x^2 + \frac{p^2}{M}\right)
\]

Следовательно, \eqref{eqCh1_hamilton_one_mode} формально соответствует
осциллятору с единичной массой и собственной частотой $\omega_n$.
Гамильтонова запись уравнений 
электромагнитного поля удобна для процедуры квантования поля. Этот
подход во многих случаях пригоден и для решения различных классических
задач. Подробнее с этим можно познакомиться в книге
\cite{bCh1Quantel_Gin}.  
