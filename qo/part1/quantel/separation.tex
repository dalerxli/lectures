%% -*- coding:utf-8 -*- 
\section{Разложение электромагнитного поля по модам (типам колебаний)}
Электромагнитное поле удовлетворяет системе уравнений Максвелла:
\begin{equation}
rot \vec{H} = \frac{\partial \vec{D}}{\partial t} + \vec{j}, 
\quad
rot \vec{E} = - \frac{\partial \vec{B}}{\partial t}, 
\quad
div \vec{D} = \rho, \quad
div \vec{B} = 0.
\end{equation}
В свободном пространстве  ($\vec{j} = 0$, $\rho = 0$)  имеем более
простую систему: 
\begin{eqnarray}
rot \vec{H} = \varepsilon_0 \frac{\partial \vec{E}}{\partial t}, 
\quad
rot \vec{E} = - \mu_0 \frac{\partial \vec{H}}{\partial t}, 
\nonumber \\
div \vec{H} = 0,
\quad
div \vec{E} = 0, 
\nonumber \\
\vec{B} = \mu_0 \vec{H}, 
\quad 
\vec{D} = \varepsilon_0 \vec{E}, 
\quad
\mu_0 \varepsilon_0 = \frac{1}{c^2}.
\label{eqCh1_EMF_at_FreeSpace}
\end{eqnarray}

Исключив из (\ref{eqCh1_EMF_at_FreeSpace}) величину $\vec{H}$,  получим
уравнение:
\begin{equation}
rot rot \vec{E} = - \mu_0 \frac{\partial}{\partial t} rot \vec{H} = -
\varepsilon_0 \mu_0 \frac{\partial^2 \vec{E}}{\partial t^2} = 
- \frac{1}{c^2} \frac{\partial^2 \vec{E}}{\partial t^2}.
\nonumber
\end{equation}

С учетом $rot rot \vec{E} = grad \, div \vec{E} - \Delta \vec{E} = - \Delta
\vec{E}$ \footnote{при условии $div \vec E = 0$} получим следующую систему

\begin{eqnarray}
\Delta \vec{E} - \frac{1}{c^2} \frac{\partial^2 \vec{E}}{\partial t^2}
= 0;
\nonumber \\
div \vec{E} = 0, 
\quad
\frac{\partial H}{\partial t} = - \frac{1}{\mu_0} rot \vec{E}
\nonumber \\
\vec{B} = \mu_0 \vec{H}, \quad \vec{D} = \varepsilon_0 \vec{E}, \quad \mu_0
\varepsilon_0 = \frac{1}{c^2}.
\label{eqCh1_EMF_at_FreeSpace2}
\end{eqnarray}
которая полностью эквивалентна исходной системе уравнений. 

Для квантования электромагнитного поля его уравнения удобно
представить в так называемом гамильтоновом виде. Суть метода, хорошо
известная радиофизикам, заключается в том, что поле разлагается по
модам, и решение уравнений сводится к решению системы обыкновенных
дифференциальных уравнений для коэффициентов разложения, зависящих от
времени. Для этой цели вводится система ортогональных векторных
функций, являющихся распределением полей, соответствующих собственным
колебаниям электромагнитного поля в некотором объеме  $V$,  ограниченном
идеально проводящей поверхностью  $S$.  Поля должны удовлетворять
системе уравнений (\ref{eqCh1_EMF_at_FreeSpace2}). На поверхности  $S$
должны выполняться некоторые граничные условия, например: 
\begin{equation}
\left. \left[ \vec{n} \vec{E} \right] \right|_S = 0 \,
\mbox{или}
\left. \left( \vec{n} \vec{H} \right) \right|_S = 0,
\label{eqCh1_bound}
\end{equation}
соответствующие идеально проводящей поверхности.

Произвольное в объеме $V$ электромагнитное поле представляется 
разложениями:
\begin{equation}
\vec{E}\left(r, t\right) = \sum_{(n)} Q_n\left(t\right)\vec{E}_n\left(r\right),
\quad
\vec{H}\left(r, t\right) = \sum_{(n)} P_n\left(t\right)\vec{H}_n\left(r\right),
\label{eqCh1_sep0}
\end{equation}
где $Q_n\left(t\right)\vec{E}_n\left(r\right)$ и 
$P_n\left(t\right)\vec{H}_n\left(r\right)$  - 
частные решения уравнений, удовлетворяющие граничным условиям
(\ref{eqCh1_bound}); $\vec{E}_n\left(r\right)$ и  
$\vec{H}_n\left(r\right)$  соответствуют $n$-ому типу 
колебаний в объеме  $V$  (моде).
Из уравнений (\ref{eqCh1_EMF_at_FreeSpace2}) имеем:
\begin{equation}
Q_n\left(t\right) \, \Delta \vec{E}_n\left(r\right) = 
\frac{1}{c^2} \frac{d^2 Q_n}{d t^2}\vec{E}_n\left(r\right). 
\end{equation}

Разделяя переменные, получаем:
\begin{equation}
\frac{d^2 Q_n}{d t^2} + \omega_n^2 Q_n = 0,
\quad
\Delta \vec{E}_n \left(r\right) = - \frac{\omega_n^2}{c^2} 
\vec{E}_n \left(r\right)
\label{eqCh1_after_sep}
\end{equation}
где $\omega_n$ постоянная разделения, являющаяся частотой собственных
колебаний.

Второе из уравнений (\ref{eqCh1_after_sep}) имеет следующий вид:
\begin{equation}
\Delta \vec{E}_n\left(r\right) + k_n^2 \vec{E}_n\left(r\right) = 0,
\quad
k_n^2 = \frac{\omega_n^2}{c^2},
\end{equation}
с граничными условиями  
$\left. \left[ \vec{n} \vec{E}_n \right] \right|_S = 0$.  Эта
задача имеет решение только при определенных значениях  
$k_n\left(\omega_n\right)$.  

Для определения $\vec{H}_n\left(r\right)$ воспользуемся уравнениями 
(\ref{eqCh1_EMF_at_FreeSpace2}):
\begin{equation}
Q_n\left(t\right)  rot
\vec{E}_n\left(r\right) = -\mu_0 \frac{d P_n}{d t}
\vec{H}_n\left(r\right).
\label{eqCh1_separationH_before_podstanovka}
\end{equation}

Прежде всего сделаем следующую подстановку:
\begin{equation}
Q_n\left(t\right) = \frac{\omega_n}{\sqrt{\varepsilon_0}}q_n\left(t\right),
\quad
P_n\left(t\right) = \frac{1}{\sqrt{\mu_0}}p_n\left(t\right).
\label{eqCh1_separation_podstanovka}
\end{equation}
Для новой переменной $q_n$ очевидно имеем из (\ref{eqCh1_after_sep})
\[
\frac{d^2 q_n}{d t^2} + \omega_n^2 q_n = 0.
\]
Разложение (\ref{eqCh1_sep0}) в новых переменных имеет вид:
\begin{equation}
\vec{E}\left(r, t\right) = \sum_{(n)}
\frac{q_n\left(t\right) \omega_n}{\sqrt{\varepsilon_0}} \vec{E}_n\left(r\right),
\quad
\vec{H}\left(r, t\right) = \sum_{(n)}
\frac{p_n\left(t\right)}{\sqrt{\mu_0}} \vec{H}_n\left(r\right),
\label{eqCh1_sep1}
\end{equation}

С учетом подстановки (\ref{eqCh1_separation_podstanovka}) и выражения
для скорости света $c =   \frac{1}{\sqrt{\varepsilon_0 \mu_0}}$
(\ref{eqCh1_EMF_at_FreeSpace}) мы можем переписать 
(\ref{eqCh1_separationH_before_podstanovka}) в следующем виде 
\begin{equation}
q_n\left(t\right) rot \vec{E}_n\left(r\right) = - \frac{1}{c \omega_n}
\frac{d p_n\left(t\right)}{d t} \vec{H}_n\left(r\right)
\label{eqCh1_after_sep2}
\end{equation}
Уравнения (\ref{eqCh1_after_sep2}) будут удовлетворены, если  
\[
p_n = \frac{d q_n}{d t}, \quad \frac{d p_n}{d t} = 
\frac{d^2 q_n}{d t^2} = - \omega_n^2 q_n,
\]
тогда
\begin{eqnarray}
rot \vec{E}_n\left(r\right) = k_n \vec{H}_n\left(r\right),
\nonumber \\
\vec{H}_n\left(r\right) =  \frac{1}{k_n} rot \vec{E}_n\left(r\right)
\end{eqnarray}

Из курса ``Электромагнитные колебания'' известно, что система функций  
$\vec{E}_n\left(r\right)$, $\vec{H}_n\left(r\right)$ ортогональна и
может быть нормирована: 
\begin{equation}
\int_{\nu} \left( \vec{E}_n \vec{E}_m \right) d \nu = \delta_{nm},
\quad
\int_{\nu} \left( \vec{H}_n \vec{H}_m \right) d \nu = \delta_{nm},
\quad
\int_{\nu} \left( \vec{H}_n \vec{E}_m \right) d \nu = 0.
\label{eqCh1_task1}
\end{equation}

Решение исходной электродинамической задачи сводится к нахождению
коэффициентов  $q_n\left(t\right)$, $p_n\left(t\right)$    
в разложениях (\ref{eqCh1_sep1}).
