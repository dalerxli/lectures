%% -*- coding:utf-8 -*- 
\section{Квантование свободного электромагнитного поля}
\label{Ch1_quantumemf}
\index{квантование!свободного электромагнитного поля}
Используя отмеченную ранее аналогию, можно проквантовать
электромагнитное поле подобно тому, как квантуется простой
механический гармонический осциллятор. 

При квантовании $q_n$ и $p_n$ становятся операторами $\hat{q}_n$,
$\hat{p}_n$, удовлетворяющими тем же коммутационным соотношениям, 
что координата и импульс: 
\begin{equation}
\left[\hat{q}_n, \hat{p}_n\right] = \hat{q}_n\hat{p}_n -
\hat{p}_n \hat{q}_n = i\hbar
\label{eqCh1_comut}
\end{equation}
где $\hat{q}_n$ и $\hat{p}_n$ - эрмитовы (самосопряженные)
операторы. В представлении Шредингера они не зависят от
времени, в представлении Гайзенберга - зависят. Удобно через
эти операторы ввести новые, не являющиеся самосопряженными, но
сопряженные друг другу операторы: 
\begin{equation}
\hat{a}_n = \frac{1}{\sqrt{2 \hbar \omega_n}}
\left( \omega_n \hat{q}_n + i \hat{p}_n\right),
\quad
\hat{a}_n^{\dag} = \frac{1}{\sqrt{2 \hbar \omega_n}}
\left( \omega_n \hat{q}_n - i \hat{p}_n\right),
\label{eqCh1_aadef}
\end{equation}
Обратная зависимость дает
\begin{equation}
\hat{q}_n = \sqrt{\frac{\hbar}{2 \omega_n}}
\left(\hat{a}_n + \hat{a}_n^{\dag}\right),
\quad
\hat{p}_n = i \sqrt{\frac{\hbar \omega_n}{2}}
\left(\hat{a}_n^{\dag} - \hat{a}_n\right),
\label{eqCh1_qpdef}
\end{equation}

При помощи выражений \eqref{eqCh1_comut}, 
\eqref{eqCh1_aadef}, \eqref{eqCh1_qpdef} можно получить коммутационные
соотношения для операторов $\hat{a}_n$ и $\hat{a}_n^{\dag}$: 
\begin{eqnarray}
\left[\hat{a}_n, \hat{a}_n^{\dag}\right] = 
\frac{1}{2 \hbar \omega_n}
\left( \omega_n \hat{q}_n + i \hat{p}_n\right) 
\left( \omega_n \hat{q}_n - i \hat{p}_n\right) - 
\nonumber \\
- \frac{1}{2 \hbar \omega_n}
\left( \omega_n \hat{q}_n - i \hat{p}_n\right) 
\left( \omega_n \hat{q}_n + i \hat{p}_n\right) =
\nonumber \\
= \frac{1}{2 \hbar \omega_n}
\left( - 2 i \omega_n 
\left(\hat{q}_n \hat{p}_n - \hat{p}_n \hat{q}_n\right)\right) = 1.
\end{eqnarray}
Таким образом,
\begin{equation}
\left[\hat{a}_n, \hat{a}_n^{\dag}\right] = 1.
\label{eqCh1_aacomutation}
\end{equation}
Кроме того очевидно, что
\begin{equation}
\left[\hat{a}_n, \hat{a}_n\right] = 0,
\quad
\left[\hat{a}_n^{\dag}, \hat{a}_n^{\dag}\right] = 0.
\end{equation}
\index{Гамильтониан}
Гамильтониан, выраженный через новые операторы, принимает вид
\begin{eqnarray}
\hat{\mathcal{H}_n} = 
\frac{1}{2}\left(\omega_n^2 \hat{q}_n^2 + \hat{p}_n^2\right) = 
\nonumber \\
= \frac{1}{2}\frac{\omega_n \hbar}{2}
\left(
\left(\hat{a}_n + \hat{a}_n^{\dag} \right)
\left(\hat{a}_n + \hat{a}_n^{\dag} \right)
-
\left(\hat{a}_n^{\dag} - \hat{a}_n \right)
\left(\hat{a}_n^{\dag} - \hat{a}_n \right)
\right) =
\nonumber \\
= \frac{\omega_n \hbar}{2} 
\left( \hat{a}_n \hat{a}_n^{\dag} + \hat{a}_n^{\dag} \hat{a}_n\right) =
\frac{\omega_n \hbar}{2} 
\left(1 + \hat{a}_n^{\dag} \hat{a}_n + \hat{a}_n^{\dag} \hat{a}_n\right) =
\nonumber \\
= \omega_n \hbar 
\left(\hat{a}_n^{\dag} \hat{a}_n + \frac{1}{2}\right)
\label{eqCh1_quant_stoyachie_volny}
\end{eqnarray}


Здесь использованы связь  $\hat{q}_n$, $\hat{p}_n$ 
с $\hat{a}_n^{\dag}$, $\hat{a}_n$ \eqref{eqCh1_aadef}, а также
коммутационное соотношение \eqref{eqCh1_aacomutation}. 

Энергия $\frac{1}{2}\omega_n \hbar$ соответствует нулевым
колебаниям. Полная энергия нулевых колебаний
\[
\frac{1}{2}\sum_{(n)}\omega_n \hbar \to \infty,
\] 
так как число мод бесконечно. Это не приводит к 
заметным трудностям, так как нас интересуют только разности
энергий и постоянную часть можно отбросить. Тогда гамильтониан
приобретает вид 
\begin{equation}
\hat{\mathcal{H}} = \sum_{(n)}\hat{\mathcal{H}_n} =
\sum_{(n)}\hbar \omega_n \hat{a}_n^{\dag}\hat{a}_n
\end{equation}

Операторы электрического и магнитного поля можно представить в виде
\begin{eqnarray}
\hat{\vec{E}} = \sum_{(n)}\frac{\hat{q}_n
  \omega_n}{\sqrt{\varepsilon_0}} \vec{E}_n\left(r\right) = 
\sum_{(n)}\sqrt{\frac{\hbar \omega_n}{2 \varepsilon_0}}
\left(\hat{a}_n^{\dag} + \hat{a}_n \right)
\vec{E}_n\left(r\right),
\nonumber \\
\hat{\vec{H}} = \sum_{(n)}\frac{\hat{p}_n}
{\sqrt{\mu_0}} \vec{H}_n\left(r\right) = 
i \sum_{(n)}\sqrt{\frac{\hbar \omega_n}{2 \mu_0}}
\left(\hat{a}_n^{\dag} - \hat{a}_n \right)
\vec{H}_n\left(r\right).
\end{eqnarray}

Приведем простой пример. В теории лазера рассматривают резонатор,
образованный двумя параллельными зеркалами. При этом часто
используют приближенное представление поля, считая, что поле
зависит только от продольной координаты (\autoref{figCh1_Res}). В
этом приближении нормированную собственную функцию (моду) можно
представить следующим образом: 
\begin{equation}
E_{nx} = \sqrt{\frac{2}{V}} \sin k_n z,
\quad
H_{ny} = \sqrt{\frac{2}{V}} \cos k_n z,
\quad
k_n = \frac{\pi n}{L},
\quad
\omega_n = \frac{c \pi n}{L},
\end{equation}
где  $V = SL$,  $L$ - длина резонатора,  $S$ - сечение светового пучка.

\input ./part1/quantel/fig1.tex

Оператор электрического поля моды при этом будет иметь вид:
\begin{eqnarray}
\hat{E}_{nx} = 
\sqrt{\frac{\hbar \omega_n}{2 \varepsilon_0}}
\left(\hat{a}_n^{\dag} + \hat{a}_n \right)
\sqrt{\frac{2}{V}} \sin k_n z = 
E_1 \left(\hat{a}_n^{\dag} + \hat{a}_n \right) \sin k_n z, 
\nonumber \\
\hat{H}_{ny} = i E_1 \sqrt{\frac{\varepsilon_0}{\mu_0}}
\left(\hat{a}_n^{\dag} - \hat{a}_n \right) \cos k_n z,
\label{eqCh1_EH_simple}
\end{eqnarray}
где 
$E_1 = \sqrt{\frac{\hbar \omega}{\varepsilon_0 V}}$ - 
электрическое поле, соответствующее одному фотону (кванту) в
моде. 
