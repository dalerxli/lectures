%% -*- coding:utf-8 -*- 
\section{Квантование электромагнитного поля при разложении его по
  плоским волнам}
\index{квантование!электромагнитного поля при разложении его по
  плоским волнам}
Для квантования электромагнитного поля в этом случае обратим внимание
на аналогию \eqref{eqCh1_separation4hamilton} и
\eqref{eqCh1_quant_stoyachie_volny}:
\[
\hat{\mathcal{H}_n} = 
\frac{\omega_n \hbar}{2}
\left(\hat{a}_n \hat{a}_n^{+} + \hat{a}_n^{+} \hat{a}_n\right)
=
\omega_n \hbar 
\left(\hat{a}_n^{+} \hat{a}_n + \frac{1}{2}\right)
\]
и
\[
\mathcal{H} = \varepsilon_0 \sum_{(k)} 
\left(A_k A_k^{*} + A_k^{*} A_k \right).
\]

Из этой аналогии следует, что можно использовать следующую процедуру
квантования:
\[
\sqrt{\varepsilon_0}A_k \rightarrow \sqrt{\frac{\omega_k \hbar}{2}}
\hat{a}_k, \quad
\sqrt{\varepsilon_0}A_k^{*} \rightarrow \sqrt{\frac{\omega_k \hbar}{2}}
\hat{a}_k^{+}.
\]
Такая замена приводит к следующему выражению для гамильтониана:
\[
\hat{\mathcal{H}_k} = \frac{\omega_k \hbar}{2} 
\left(\hat{a}_k \hat{a}_k^{+} + \hat{a}_k^{+} \hat{a}_k\right).
\]

Воспользовавшись коммутационными соотношениями
\[
\left[\hat{a}, \hat{a}^{+} \right] = 1, \quad
\hat{a} \hat{a}^{+} - \hat{a}^{+}\hat{a} = 1, \quad
\hat{a} \hat{a}^{+} = \hat{a}^{+}\hat{a} + 1,
\]
имеем
\(
\hat{\mathcal{H}_k} = \omega_k \hbar 
\left(\hat{a}_k^{+} \hat{a}_k + \frac{1}{2}\right)
\) - 
выражение, полностью совпадающее с \eqref{eqCh1_quant_stoyachie_volny}.
Полный гамильтониан получим суммированием по всем модам:
\begin{equation}
\hat{\mathcal{H}} = \sum_{(k)} \hat{\mathcal{H}}_k = \sum_{(k)} 
\omega_k \hbar \left(\hat{a}_k^{+} \hat{a}_k + \frac{1}{2}\right).
\end{equation}

В гамильтоновом виде можно представить и импульс электромагнитного 
поля. Классический импульс электромагнитного поля, находящегося в
объеме  $V$,  определяется формулой 
\begin{equation}
\vec{G} = \frac{1}{c^2} \int_{(\nu)}
\left[\vec{E} \vec{H} \right] d \nu.
\label{eqCh1_task3_1}
\end{equation}
Используя разложение поля по плоским волнам, учитывая соотношения
ортогональности, получим
\begin{eqnarray}
  \vec{G} = \frac{1}{c^2} \int_{(\nu)}
  \left[\vec{E} \vec{H} \right] d \nu =
  \nonumber \\
  =
  \frac{1}{c^2} \int_{(\nu)}
  \left[
    \left(\sum_{(k)} A_k \vec{E}_k +
    \sum_{(k)} A_k^\ast \vec{E}_k^\ast \right)
    \left(\sum_{(k')} A_{k'} \vec{H}_{k'} +
    \sum_{(k')} A_{k'}^\ast \vec{H}_{k'}^\ast \right)\right]
  d \nu =
  \nonumber \\
  =
  \sum_{(k)}
  \frac{1}{c^2} \int_{(\nu)}
  \left(
  A_k A_k^\ast \left[\vec{E}_k \vec{H}_k^\ast \right]
  +
  A_k^\ast A_k \left[\vec{E}_k^\ast \vec{H}_k \right]
  \right)
  d \nu =
  \nonumber \\
  =
  \frac{1}{c^2} \sum_{(k)} \frac{\vec{k}}{k}
  \sqrt{\frac{\varepsilon_0}{\mu_0}}
  \left(
  A_k A_k^\ast + A_k^\ast A_k
  \right).
  \nonumber
\end{eqnarray}
Переходя к операторам имеем
\begin{eqnarray}
  \hat{\vec{G}} =
  \frac{1}{c^2} \sum_{(k)} \frac{\vec{k}}{k}
  \sqrt{\frac{\varepsilon_0}{\mu_0}}
  \frac{\hbar \omega_k}{2 \varepsilon_0}
  \left(\hat{a}_k^{+} \hat{a}_k + \hat{a}_k \hat{a}_k^{+}\right)
  =
  \nonumber \\
  =
  \sum_{(k)}
  \hbar
  \vec{k}
  \frac{\omega_k}{c k}
  \frac{1}{c \sqrt{\varepsilon_0 \mu_0}}
  \frac{\hat{a}_k^{+} \hat{a}_k + \hat{a}_k \hat{a}_k^{+}}{2}
  =
  \sum_{(k)} \vec{k} \hbar\left( \hat{a}_k^{+} \hat{a}_k +
\frac{1}{2} \right).
\nonumber
\end{eqnarray}
Из симметрии  следует, что $\sum_{(k)} \vec{k} \hbar = 0$ и, следовательно, 
\begin{equation}
\hat{\vec{G}} = \sum_{(k)} \vec{k} \hbar\hat{a}_k^{+} \hat{a}_k.
\label{eqCh1_task3_2}
\end{equation}

Оператор электрического поля выражается теперь следующим образом:
\begin{equation}
\hat{\vec{E}} = \sum_{(k)} \hat{a}_k\sqrt{\frac{\hbar \omega_k}{2 \nu
    \varepsilon_0}} \vec{e}_k e^{i\left(\vec{k}\vec{r}\right)} +
\sum_{(k)} \hat{a}_k^{+}\sqrt{\frac{\hbar \omega_k}{2 \nu
    \varepsilon_0}} \vec{e}_k^{*} e^{-i\left(\vec{k}\vec{r}\right)}.
\end{equation}

В дальнейшем будем пользоваться дираковской формулировкой уравнений
квантовой механики, поэтому в \autoref{AddDirac} дано краткое
изложение дираковского формализма: введено понятие вектора состояния и
изложены способы оперирования с ним.  
