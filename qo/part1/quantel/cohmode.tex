%% -*- coding:utf-8 -*- 
\section{Когерентные состояния}
Состояния $\left|n\right>$  не стремятся при увеличении $n$  к
классическому решению для осциллятора. Этим свойством обладают так
называемые когерентные состояния. Они были введены Э.Шредингером, а
затем развиты в наше время Р.Глаубером
\cite{bQuantumOpticsAndRadioPhisicsLecture1966} и Э.Судершаном 
\cite{bKaluderSudershan1970} применительно к задачам квантовой
оптики. Когерентное   состояние может быть определено различными
путями. Воспользуемся определением в виде суперпозиции состояний
$\left|n\right>$:
\begin{equation}
\left|\alpha\right> = e^{-\frac{1}{2} \left|\alpha\right|^2}
\sum_{(n)} \frac{\alpha^n}{\sqrt{n!}}\left|n\right>,
\quad
\left<\alpha\right| = e^{-\frac{1}{2} \left|\alpha\right|^2}
\sum_{(n)} \frac{\alpha^{*n}}{\sqrt{n!}}\left<n\right|,
\label{eqCh1_coh_def}
\end{equation}
где $\alpha$ -  некоторый комплексный параметр, смысл которого
выяснится в дальнейшем. 

Из определения \eqref{eqCh1_coh_def} следует ряд свойств когерентных
состояний. Прежде  всего,
\begin{equation}
\left<\alpha\right|\left.\alpha\right> = 
e^{-\left|\alpha\right|^2}\sum_{(n)}
\frac{\left(\alpha\alpha^{*}\right)^n}{n!} = 
e^{\left|\alpha\right|^2} e^{-\left|\alpha\right|^2} = 1,
\end{equation}
т.е. когерентные состояния нормированы. Однако эти состояния не
ортогональны: 
\begin{equation}
\left<\alpha\right|\left.\beta\right> = 
e^{-\frac{1}{2}\left(\left|\alpha\right|^2 +
  \left|\beta\right|^2\right)}\sum_{(n)} 
\frac{\left(\alpha^{*}\beta\right)^n}{n!} = 
e^{
-\frac{1}{2} \left|\alpha\right|^2  -\frac{1}{2} \left|\beta\right|^2
+
\alpha^{*} \beta 
}
\label{eqCh1_ortog}
\end{equation}
Отсюда
\begin{equation}
\left|\left<\alpha\right|\left.\beta\right>\right|^2 = 
e^{
-\left|\alpha\right|^2  - \left|\beta\right|^2
+
\alpha^{*} \beta  + \alpha \beta^{*}} = 
e^{-\left|\alpha - \beta\right|^2}, 
\end{equation}
т.е. эти состояния могут считаться приближенно ортогональными, если
$\left|\alpha - \beta\right|$  достаточно велико. 

Неортогональность когерентных состояний является следствием
переполненности системы ($\alpha$ -  любое комплексное число, а  $n$ -
только целые числа). 

Когерентные состояния являются собственными состояниями оператора
уничтожения с собственным числом $\alpha$: 
\begin{equation}
\hat{a}\left|\alpha\right> = e^{-\frac{1}{2} \left|\alpha\right|^2}
\sum_{(n)} \frac{\alpha^n}{\sqrt{n!}}\sqrt{n}\left|n - 1\right> = 
e^{-\frac{1}{2} \left|\alpha\right|^2}
\sum_{(m = n - 1)} \frac{\alpha \alpha^m}{\sqrt{m!}}\left|m\right> = 
\alpha\left|\alpha\right>
\end{equation}

Сопряженное соотношение имеет вид
\begin{equation}
\left<\alpha\right|\hat{a}^{\dag} =  
\alpha^{*}\left<\alpha\right|
\end{equation}

Вероятность обнаружить (измерить) в когерентном состоянии $n$ фотонов
\begin{equation}
\left|\left<n\right|\left.\alpha\right>\right|^2 =
e^{-\left|\alpha\right|^2}
\frac{\left(\left|\alpha\right|^2\right)^n}{n!}
\label{eqCh1_PuassonCoh}
\end{equation}
соответствует распределению Пуассона со средним числом фотонов  
$\left<n\right> = \left|\alpha\right|^2$.  Используя известное нам
соотношение 
\begin{equation}
\left|n\right> = \frac{1}{\sqrt{n!}}
\left(\hat{a}^{\dag}\right)^n\left|0\right>
\label{eqCh1_nstate}
\end{equation}
когерентное состояние можно представить в виде
\begin{equation}
\left|\alpha\right> = e^{-\frac{1}{2} \left|\alpha\right|^2}\sum_{(n)}
\frac{\left(\alpha \hat{a}^{\dag}\right)^n}{n!}\left|0\right> = 
e^{\alpha \hat{a}^{\dag} -
  \frac{1}{2}\left|\alpha\right|^2}\left|0\right>
\label{eqCh1_astate}
\end{equation}

Перепишем \eqref{eqCh1_astate} в ином виде, для этого воспользуемся
следующим соотношением
\[
e^{- \alpha^{*} \hat{a}} \left|0\right> = \left(1 - \alpha^{*} \hat{a} +
\dots\right) \left|0\right> = \left|0\right>,
\]
в результате получим
\begin{equation}
\left|\alpha\right> = 
e^{\alpha \hat{a}^{\dag}} 
e^{-\frac{1}{2}\left|\alpha\right|^2}
e^{- \alpha^{*} \hat{a}} 
\left|0\right>.
\label{eqCh1_astate_add}
\end{equation}
Далее воспользуемся операторным тождеством (формулой Бейкера-Хаусдорфа)
\begin{equation}
e^{\hat{c} + \hat{d}} = e^{- \frac{1}{2}\left[\hat{c},
    \hat{d}\right]}e^{\hat{c}} e^{\hat{d}} 
\label{eqPart1Ch1_BeikerHausdorf}
\end{equation}
справедливым при условии  
$\left[\hat{c},\left[\hat{c}, \hat{d}\right]\right] =
\left[\hat{d},\left[\hat{c}, \hat{d}\right]\right] = 0$.
Если положить $\hat{c} = \alpha\hat{a}^{\dag}$, $\hat{d} = -\alpha^*\hat{a}$
то из \eqref{eqCh1_astate_add}  получим 
\begin{eqnarray}
\left|\alpha\right> =  
e^{\alpha \hat{a}^{\dag}} 
e^{-\frac{1}{2}\left|\alpha\right|^2}
e^{- \alpha^{*} \hat{a}} 
\left|0\right> =
e^{-\frac{1}{2}\left|\alpha\right|^2}
e^{\alpha \hat{a}^{\dag}} 
e^{- \alpha^{*} \hat{a}} 
\left|0\right> = 
\nonumber \\
=
e^{\alpha \hat{a}^{\dag} - \alpha^{*} \hat{a}}\left|0\right>.
\label{eqCh1_astate4squeezed}
\end{eqnarray}

Для того чтобы наглядно представить когерентные состояния, в
\autoref{AddQCoh} дано координатное представление когерентного состояния. 

Приведем еще несколько полезных соотношений, полученных из
\eqref{eqCh1_astate}. Дифференцируя \eqref{eqCh1_astate} по  $\alpha$,
имеем:  
\[
\frac{\partial}{\partial \alpha}\left|\alpha\right> = 
\left( \hat{a}^{\dag} - \frac{1}{2}\alpha^{*}\right)\left|\alpha\right>,
\]
или иначе 
\[
\hat{a}^{\dag}\left|\alpha\right> = \left(\frac{\partial}{\partial
  \alpha} +  \frac{1}{2}\alpha^{*}\right)\left|\alpha\right>.
\]
сопряженное равенство имеет вид
\[
\left<\alpha\right|\hat{a} = \left(\frac{\partial}{\partial
  \alpha^{*}} +  \frac{1}{2}\alpha\right)\left<\alpha\right|
\]

В общем случае оператор положительно частотной части электрического поля имеет вид
\[
\hat{\vec{E}}^{(+)} = \sum_{(k)} \sqrt{\frac{\hbar \omega_k}{2 \varepsilon_0
V}} \hat{a}_k\left(t\right) \vec{e}_k e^{-i \omega_k t + i \left(\vec{k}\vec{r}
  \right)}, 
\]

Положим, что поле (многомодовое) находится в состоянии
\[
\left| \left\{\alpha_s\right\}\right> = 
\left| \alpha_1, \alpha_2, \dots, \alpha_s, \dots\right> = 
\left| \alpha_1\right>
\otimes
\left| \alpha_2\right>
\otimes
\dots
\otimes
\left| \alpha_s\right>
\otimes
\dots,
\]
где $\left\{\alpha_s\right\}= \left\{\alpha_{k_s}\right\}$
обозначает некоторую совокупность параметров $\alpha$ . Если действовать
оператором $\hat{\vec{E}}^{(+)}$ на состояние  $\left|\left\{\alpha_{k_s}\right\}\right>$,  получим 
\begin{eqnarray}
\hat{\vec{E}}^{(+)}\left|\left\{\alpha_{k_s}\right\}\right> = 
\left\{ \sum_{(k_s)} \sqrt{\frac{\hbar \omega_k}{2 \varepsilon_0
V}} \hat{a}_{k_s} \vec{e}_k e^{-i \omega_k t + i \left(\vec{k}\vec{r}
  \right)}\right\}\left|\left\{\alpha_{k_s}\right\}\right> = 
\nonumber \\
= 
\sum_{(k_s)} \sqrt{\frac{\hbar \omega_k}{2 \varepsilon_0
V}} \alpha_{k_s} \vec{e}_k e^{-i \omega_k t + i \left(\vec{k}\vec{r}
  \right)}\left|\left\{\alpha_{k_s}\right\}\right>,
\end{eqnarray}
т.е. $\left|\left\{\alpha_{k_s}\right\}\right>$  является собственным
вектором оператора положительно-частотной части электрического поля, а
собственным значением является классическое поле (аналитический
сигнал), у которого комплексные амплитуды мод равны  
\(
\sqrt{\frac{\hbar \omega_k}{2 \varepsilon_0
V}} \alpha_k.
\)
Отсюда следует, что каждому классическому полю соответствует некоторое
когерентное состояние. 

Среднее значение оператора электрического поля $\hat{\vec{E}} =
\hat{\vec{E}}^{(+)} + \hat{\vec{E}}^{(-)}$   в когерентном
состоянии получим, рассмотрев сначала одну моду: 
\[
\left<\alpha_{k}\right|\hat{\vec{E}}_k\left|\alpha_{k}\right> = 2
\sqrt{\frac{\hbar \omega_k}{2 \varepsilon_0
V}}\left|\alpha_{k}\right|\cos 
\left(\omega_k t - \left(\vec{k} \vec{r} + \theta_k\right) \right),
\]
где $\alpha_k = \left|\alpha_{k}\right| e^{i \theta_k}$, что следует из соотношений
\[
\left<\alpha_k\right|\hat{a}_k\left|\alpha_k\right> = \alpha_k, \quad
\left<\alpha_k\right|\hat{a}^{\dag}_k\left|\alpha_k\right> = \alpha^{*}_k
\]
Обобщая это на многомодовый случай, получим
\begin{equation}
\left<\left\{\alpha_{k}\right\}\right|\hat{\vec{E}}\left|\left\{\alpha_{k}\right\}\right>
= 2 \sum_{(k)} \sqrt{\frac{\hbar \omega_k}{2 \varepsilon_0
V}} \left|\alpha_k\right|\cos \left(\omega_k t -
\left(\vec{k}\vec{r}\right) + \theta_k \right),
\end{equation}
что соответствует классическому многомодовому полю, т.е. среднее
значение $\hat{\vec{E}}$   в когерентном состоянии является
классическим полем с соответствующими амплитудами и фазами мод. 

Когерентные состояния, как мы видели, не ортогональны из-за
переполненности системы собственных векторов, но они удовлетворяют
условию полноты: 
\begin{equation}
\frac{1}{\pi}\int \left|\alpha\right>\left<\alpha\right| d^2 \alpha =
\hat{I},
\label{eqCh1_full4coh}
\end{equation}
Здесь $d^2 \alpha = d\, Re \alpha\, d\, Im \alpha$ , $\hat{I}$ - единичный
оператор, а интегрирование ведется по всей комплексной
плоскости. Докажем это, воспользовавшись полярными координатами,
изображенными на \autoref{figCh1_Coh}.  

\input ./part1/quantel/fig5.tex

Используя разложение $\left|\alpha\right>$ по $\left|n\right>$ в
полярных координатах, получим 
\begin{eqnarray}
\int d^2 \alpha \left|\alpha\right>\left<\alpha\right| =
\nonumber \\
=
\int_0^{\infty} d \left|\alpha\right|\int_0^{2
  \pi}\left|\alpha\right|\sum_{(n)}\sum_{(m)}e^{-i \left(n -
  m\right)\varphi}
e^{-\left|\alpha\right|^2}\left|m\right>\left<n\right|
\frac{\left|\alpha\right|^m \left|\alpha\right|^n}{\sqrt{n! m!}} d\varphi
= 
\nonumber \\
= \pi \sum_{(n)}\frac{1}{n!}\int_0^{\infty} 2
\left|\alpha\right|\left|\alpha\right|^{2n} e^{-\left|\alpha\right|^2} 
\left|n\right>\left<n\right|
d\left|\alpha\right| = 
\nonumber \\
=
\pi
\sum_{(n)}\frac{\left|n\right>\left<n\right|}{n!} \int_0^{\infty}x^n
e^{-x}dx = 
\pi \sum_{(n)}\left|n\right>\left<n\right| = \pi \hat{I},
\end{eqnarray}
где учтено, что
\[
\int_0^{2\pi} e^{-i \left(n - m\right)\varphi} = 2 \pi \delta_{nm},
\]
сделана замена переменных:  
\[
\left|\alpha\right|^2 = x, \quad 2 \left|\alpha\right| d
\left|\alpha\right| = d x,
\]
использовано интегральное представление факториала:
\[
n! = \int_0^{\infty}x^ne^{-x}dx
\]

Таким образом, доказана формула \eqref{eqCh1_full4coh}. При помощи
этого соотношения можно выразить одно когерентное состояние через все
остальные: 
\[
\left|\alpha\right> = \frac{1}{\pi}\int\left|\beta\right>\left<\beta\right|\left.\alpha\right> d^2 \beta,
\]
но в \eqref{eqCh1_ortog} мы имели
\(
\left<\alpha\right|\left.\beta\right> = 
e^{
-\frac{1}{2} \left|\alpha\right|^2  -\frac{1}{2} \left|\beta\right|^2
+
\alpha^{*} \beta
},
\)
следовательно,
\begin{equation}
\left|\alpha\right> = \frac{1}{\pi}\int\left|\beta\right>
e^{
-\frac{1}{2} \left|\alpha\right|^2  -\frac{1}{2} \left|\beta\right|^2
+
\alpha^{*} \beta
} d^2 \beta.
\end{equation}
