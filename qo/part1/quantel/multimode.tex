%% -*- coding:utf-8 -*- 
\section{Многомодовые состояния}
До сих пор мы рассматривали одномодовые поля. Результаты легко
обобщаются на многомодовое состояние. Для электромагнитного поля в
некотором объеме имеем разложение (\ref{eqCh1_sep1}):
\begin{equation}
\vec{E}\left(r, t\right) = \sum_{(s)}
\frac{q_s\left(t\right) \omega_s}{\sqrt{\varepsilon_0}} \vec{E}_s\left(r\right),
\quad
\vec{H}\left(r, t\right) = \sum_{(s)}
\frac{p_s\left(t\right)}{\sqrt{\mu_0}} \vec{H}_s\left(r\right),
\end{equation}
Функция Гамильтона равна сумме функций Гамильтона всех мод, так как
моды соответствуют независимым осцилляторам: 
\begin{equation}
\mathcal{H} = \frac{1}{2} 
\sum_{(s)} \left(\omega_s^2 q_s^2 + p_s^2 \right)
\end{equation}
Квантование сводится к замене $q_s$ и $p_s$   на операторы $\hat{q}_s$
и $\hat{p}_s$ с коммутационными соотношениями: 
\begin{equation}
\left[\hat{q}_s, \hat{p}_{s'}\right] = i\hbar \delta_{ss'},
\quad 
\left[\hat{q}_s, \hat{q}_{s'}\right] = \left[\hat{p}_s,
  \hat{p}_{s'}\right] = 0.
\end{equation}
Функция Гамильтона при этом превратится в оператор Гамильтона:
\begin{equation}
\hat{\mathcal{H}} = \frac{1}{2} \sum_{(s)} \left(\omega_s^2 \hat{q}_s^2 + \hat{p}_s^2 \right)
\end{equation}
Как в разделе \ref{Ch1_quantumemf}, вводим операторы рождения и
уничтожения для каждой моды (\ref{eqCh1_aadef}): 
\begin{equation}
\hat{a}_s = \frac{1}{\sqrt{2 \hbar \omega_s}}
\left( \omega_s \hat{q}_s + i \hat{p}_s\right),
\quad
\hat{a}_s^{+} = \frac{1}{\sqrt{2 \hbar \omega_s}}
\left( \omega_s \hat{q}_s - i \hat{p}_s\right),
\end{equation}
С их помощью гамильтониан можно представить в виде
\begin{equation}
\hat{\mathcal{H}_s} = \omega_n \hbar 
\left(\hat{a}_s^{+} \hat{a}_s + \frac{1}{2}\right)
\nonumber
\end{equation}
Общее состояние, когда в первой моде $n_1$ фотонов, во второй $n_2$
фотонов, в  $s$-й - $n_s$ фотонов и т.д., можно представить в виде
произведения векторов состояний каждой моды: 
\begin{equation}
\left|\left\{n_s\right\}\right> = 
\left|n_1, n_2, \dots, n_s, \dots\right> =
\left|n_1\right> \otimes
\left|n_2\right> \otimes
\dots
\otimes
\left|n_s\right> \otimes
\dots, 
\nonumber
\end{equation}
где $\left\{n_s\right\}$ означает совокупность чисел заполнения мод.

Действие операторов $\hat{a}_s^{+}$ и $\hat{a}_s$,  относящихся к $s$-й
моде, на вектор состояния описывается равенствами 
\begin{eqnarray}
\hat{a}_s \left| n_1, n_2, \dots, n_s, \dots\right> = \sqrt{n_s} \left|
n_1, n_2, \dots, n_s - 1, \dots\right>,
\nonumber \\
\hat{a}_s^{+} \left| n_1, n_2, \dots, n_s, \dots\right> = \sqrt{n_s + 1} \left|
n_1, n_2, \dots, n_s + 1, \dots\right>.
\end{eqnarray} 

Вектор состояния в общем случае можно представить линейной
суперпозицией состояний  $\left| \left\{n_s\right\}\right>$: 
\begin{equation}
\left|\Psi\right> = \sum_{n_1}\sum_{n_2}\dots\sum_{n_s}\dots
C_{n_1, n_2, \dots, n_s, \dots} \left| n_1, n_2, \dots, n_s,
\dots\right>= 
\sum_{\left\{n_s\right\}} C_{\left\{n_s\right\}} \left| \left\{n_s\right\}\right>
\end{equation}

При разложении поля по плоским волнам оператор электрического поля 
имеет вид
\begin{equation}
\hat{\vec{E}} = \sum_{(k)} \sqrt{\frac{\hbar \omega_k}{2 \varepsilon_0
V}} \left( 
\hat{a}_k\left(t\right) \vec{e}_k e^{i \left(\vec{k}\vec{r} \right)} +
\hat{a}_k^{+}\left(t\right) \vec{e}_k^{*} e^{-i \left(\vec{k}\vec{r} \right)}
\right)
\label{eqCh1_multiE}
\end{equation}
Это выражение записано в представлении Гайзенберга (оператор зависит
от времени). В этом представлении 
\[
\hat{a}_k\left(t\right) = \hat{a}_k\left(0\right) e^{-i \omega_k t},
\quad
\hat{a}_k^{+}\left(t\right) = \hat{a}_k^{+}\left(0\right) e^{i \omega_k t},
\]
где $\hat{a}_k\left(0\right)$, $\hat{a}_k^{+}\left(0\right)$ -
операторы в представлении Шредингера. В представлении Шредингера в
выражении (\ref{eqCh1_multiE}) операторы от времени не зависят.
Для каждой моды имеем 
равенства (\ref{eqCh1_E_middle}), (\ref{eqCh1_E2_middle}): 
\[
\left<n_k\right|\hat{a}_k\left|n_k\right> = 
\left<n_k\right|\hat{a}^{+}_k\left|n_k\right> = 0
\]
Отсюда для полного поля получим
\begin{equation}
\left<\left\{n_k\right\}\right|\hat{\vec{E}}\left|\left\{n_k\right\}\right>
= 0, \quad
\left<\left\{n_k\right\}\right|\hat{\vec{E}}^2\left|\left\{n_k\right\}\right>
= \sum_{(k)}\frac{\hbar \omega_k}{\varepsilon_0 V}
\left(n + \frac{1}{2} \right)
\end{equation}

Выражение (\ref{eqCh1_multiE}) можно разбить на два слагаемых:
частотно положительную часть, в которую входят операторы уничтожения,
и частотно отрицательную часть, в которую входят операторы рождения: 
\begin{eqnarray}
\hat{\vec{E}} = \hat{\vec{E}}^{(+)} + \hat{\vec{E}}^{(-)},
\nonumber \\
\hat{\vec{E}}^{(+)} = \sum_{(k)} \sqrt{\frac{\hbar \omega_k}{2 \varepsilon_0
V}} \hat{a}_k\left(t\right) \vec{e}_k e^{i \left(\vec{k}\vec{r}
  \right)}, 
\nonumber \\
\hat{\vec{E}}^{(-)} = \sum_{(k)} \sqrt{\frac{\hbar \omega_k}{2 \varepsilon_0
V}}
\hat{a}_k^{+}\left(t\right) \vec{e}_k^{*} e^{-i \left(\vec{k}\vec{r} \right)}
\label{eqCh1_79}
\end{eqnarray} 
$\hat{\vec{E}}^{(+)}$ соответствует аналитическому сигналу в
классическом случае. 
