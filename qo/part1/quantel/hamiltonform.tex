%% -*- coding:utf-8 -*- 
\section{Гамильтонова форма уравнений поля при разложении по плоским
  волнам}
Собственные функции (\ref{eqCh1_EHmode}) ортогональны, т.е.
\begin{eqnarray}
\int_{\nu} \left( \vec{E}_k \vec{E}_{k'}^{*} \right) d \nu =
C \int_{\nu} e^{i \left(\vec{k}\vec{r}\right)} e^{- i
  \left(\vec{k'}\vec{r}\right)} d \nu =
\nonumber \\
= C \int_0^L d x \int_0^L d y \int_0^L d z
\left[ e^{i\left(
\left(k_x - k'_x\right) x +
\left(k_y - k'_y\right) y +
\left(k_z - k'_z\right) z
\right)}
\right] = 0,
\end{eqnarray}
если справедливо хотя бы одно из неравенств 
$k_x \ne k'_x$,
$k_y \ne k'_y$,
$k_z \ne k'_z$,
(или 
$\vec{k} - \vec{k'} \ne 0$
).  Это следует из периодичности подинтегральных функций. 

Аналогично получим:
\begin{equation}
\int_{\nu} \left( \vec{E}_k \vec{E}_{k'} \right) d \nu = 0,
\quad
\vec{k} \ne - \vec{k'}.
\end{equation}

Собственные функции $\vec{E}_k$ можно нормировать, положив
\begin{equation}
\int_{\nu} \left( \vec{E}_k \vec{E}_{k}^{*} \right) d \nu = 1.
\end{equation}
Тогда собственные функции будут иметь вид
\[
\vec{E}_k\left(r\right) = \frac{1}{\sqrt{\nu}}\vec{e}_k e^{i \left( \vec{k}\vec{r}\right)},
\quad
\vec{H}_k\left(r\right) = 
\frac{1}{k}
\left[\vec{k}\vec{e}_k\left(r\right)\right] 
\sqrt{\frac{\varepsilon_0}{\nu \mu_0}}
e^{i \left( \vec{k}\vec{r}\right)}.
\]
Заметим, что под индексом $k$  мы подразумеваем $k_j$,  где  $j = 1,
2$  соответствуют двум различным поляризациям.  
Функции  $\vec{E}_k$  и  $\vec{H}_k$,  записанные в таком виде,
удовлетворяют следующим условиям ортогональности и нормировки: 
\begin{eqnarray}
\int_{\nu} \left( \vec{E}_k \vec{E}_{k'}^{*} \right) d \nu = 0
\quad
\mbox{при } \vec{k} \ne \vec{k'},
\nonumber \\
\int_{\nu} \left( \vec{E}_k \vec{E}_{k'} \right) d \nu = 0
\quad
\mbox{при } \vec{k} \ne - \vec{k'},
\nonumber \\
\int_{\nu} \left( \vec{E}_k \vec{E}_{k}^{*} \right) d \nu = 1,
\quad
\int_{\nu} \left( \vec{E}_k \vec{E}_{-k} \right) d \nu = 1,
\nonumber \\
\int_{\nu} \left( \vec{H}_k \vec{H}_{k'}^{*} \right) d \nu = 0
\quad
\mbox{при } \vec{k} \ne \vec{k'},
\nonumber \\
\int_{\nu} \left( \vec{H}_k \vec{H}_{k'} \right) d \nu = 0
\quad
\mbox{при } \vec{k} \ne - \vec{k'},
\nonumber \\
\int_{\nu} \left( \vec{H}_k \vec{H}_{k}^{*} \right) d \nu = \frac{\varepsilon_0}{\mu_0},
\quad
\int_{\nu} \left( \vec{H}_k \vec{H}_{-k} \right) d \nu = - \frac{\varepsilon_0}{\mu_0}.
\label{eqCh1_task2}
\end{eqnarray}
Ортогональность мод с одинаковым значением $k$,  но разными
поляризациями следует из равенств:  
\[
\left(\vec{e}_{k_1} \vec{e}_{k_2}^{*}\right) = 0, \quad
\left(\vec{e}_{k_1} \vec{e}_{- k_2}\right) = 0.
\]

Электрическое и магнитное поля выражаются через эти функции при помощи разложения:
\begin{eqnarray}
\vec{E}\left(r, t\right) = 
\sum_{(k)} 
A_k\left(t\right) \vec{E}_k\left(r\right) +
\mbox{к. с.},
\nonumber \\
\vec{H}\left(r, t\right) = 
\sum_{(k)} 
A_k\left(t\right) \vec{H}_k\left(r\right) +
\mbox{к. с.}
\label{eqCh1_separation4six}
\end{eqnarray}
Суммирование по $k = k_j$ подразумевает суммирование по всем значениям
$k$ и двум поляризациям  ($j = 1,2$): 
$\sum_{(k)} \equiv \sum_{j = 1,2} \sum_{(k_j)}$ .
  
Функция Гамильтона магнитного поля имеет вид:
\begin{equation}
\mathcal{H} = \frac{1}{2}
\int_{\nu}\left( \varepsilon_0\left(\vec{E}\left(r,t\right)\right)^2 + \mu_0
\left(\vec{H}\left(r,t\right)\right)^2\right) d\nu
\label{eqCh1_hamilton4six}
\end{equation}
Подставляя в (\ref{eqCh1_hamilton4six}) ряды
(\ref{eqCh1_separation4six}), перемножая их и учитывая условия
ортогональности и нормировки, получим 
\begin{eqnarray}
\mathcal{H} = \frac{1}{2} \varepsilon_0
\int_{(\nu)}\left\{
\sum_{(k)}\sum_{(k')}
\left(A_k\vec{E}_k + A_k^{*}\vec{E}_k^{*}\right)
\left(A_{k'}\vec{E}_{k'} + A_{k'}^{*}\vec{E}_{k'}^{*}\right)
\right\}d \nu +
\nonumber \\
+ 
\frac{1}{2} \mu_0
\int_{(\nu)}\left\{
\sum_{(k)}\sum_{(k')}
\left(A_k\vec{H}_k + A_k^{*}\vec{H}_k^{*}\right)
\left(A_{k'}\vec{H}_{k'} + A_{k'}^{*}\vec{H}_{k'}^{*}\right)
\right\} d \nu = 
\nonumber \\
\frac{\varepsilon_0}{2} \sum_{(k)} 
\left(
A_k A_k^{*} + A_k A_{-k} + A_k^{*} A_{-k}^{*} + A_k^{*} A_k +
\right.
\nonumber \\
\left.
+ A_k A_k^{*} - A_k A_{-k} - A_k^{*} A_{-k}^{*} + A_k^{*} A_k
\right) = \varepsilon_0 \sum_{(k)} 
\left(A_k A_k^{*} + A_k^{*} A_k \right).
\label{eqCh1_separation4hamilton}
\end{eqnarray}
