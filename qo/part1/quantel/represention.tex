%% -*- coding:utf-8 -*- 
\section{Представление оператора плотности через когерентные
  состояния}
Важным для квантовой оптики является описание оптических явлений при
помощи когерентных состояний. Используя дважды условие полноты для  
состояний с различным числом фотонов, для произвольного
статистического оператора получим: 
\[
\hat{\rho} = \hat{I}\hat{\rho}\hat{I} = 
\sum_{(n)}\sum_{(m)}
\ket{n}\bra{n}\hat{\rho}\ket{m}\bra{m} = 
\sum_{(n)}\sum_{(m)} \rho_{mn}\ket{n}\bra{m}
\]
Аналогично для когерентных состояний
\begin{eqnarray}
\hat{\rho} = \hat{I}\hat{\rho}\hat{I} = 
\frac{1}{\pi^2}\int d^2\alpha\int
\left|\alpha\right>\left<\alpha\right|\hat{\rho}\left|\beta\right>\left<\beta\right|
d^2 \beta = 
\nonumber \\
= \frac{1}{\pi^2} \int d^2\alpha \int
R\left(\alpha^{*}, \beta\right) e^{-\frac{1}{2}\left|\alpha\right|^2 -\frac{1}{2}\left|\beta\right|^2}
\left|\alpha\right>\left<\beta\right|d^2 \beta ,
\end{eqnarray}
где
\[
R\left(\alpha^{*}, \beta\right) =
\left<\alpha\right|\hat{\rho}\left|\beta\right>
e^{\frac{1}{2}\left|\alpha\right|^2 +\frac{1}{2}\left|\beta\right|^2}
\]
Если выразить $\left|\alpha\right>$ и $\left|\beta\right>$ через
состояния $\ket{n}$,  получим формулу 
\begin{equation}
R\left(\alpha^{*}, \beta\right) = \sum_{(n)}\sum_{(m)}
\frac{\left(\alpha^{*}\right)^n \left(\beta\right)^m}{\sqrt{n!m!}}\rho_{nm}
\end{equation}
т. е. $R\left(\alpha^{*}, \beta\right)$ легко найти, если известна
матрица плотности 
\index{Матрица плотности}
в представлении чисел заполнения (чисел
фотонов). Как пример рассмотрим тепловое возбуждение моды: 
\begin{eqnarray}
R\left(\alpha^{*}, \beta\right) = \sum_{(n)}\sum_{(m)}
\frac{\left(\alpha^{*}\right)^n \left(\beta\right)^m}{\sqrt{n!m!}}
e^{-\beta n \hbar \omega}\left(1  -  e^{-\beta \hbar \omega}\right)
\delta_{nm} = 
\nonumber \\
= \sum_{(n)}\frac{\left(\alpha^{*} \beta\right)^n }{n!}
e^{-\beta n \hbar \omega}\left(1  -  e^{-\beta \hbar \omega}\right) =
\left(1  -  e^{-\beta \hbar \omega}\right) e^{\alpha^{*}\beta
  e^{-\beta \hbar \omega}}.
\end{eqnarray}

Это представление зависит от двух параметров $\alpha^{*}$ и $\beta$.
Может оказаться более удобным другое представление матрицы плотности,
введенное Р.Глаубером \cite{bQuantumOpticsAndRadioPhisicsLecture1966},
называемое 
диагональным представлением и зависящее от одного параметра. Оно имеет
вид  
\begin{equation}
\hat{\rho} = \int
P\left(\alpha\right)\left|\alpha\right>\left<\alpha\right| d^2 \alpha,
\label{eqCh1_rhorepresent}
\end{equation}
где  $P\left(\alpha\right)$ -  действительная функция,
удовлетворяющая условию 
\[
\int
P\left(\alpha\right) d^2 \alpha = 1.
\]
Кроме того, $P\left(\alpha\right)$ - вещественная функция комплексного
аргумента. Это все следует из условий  $Sp\left(\hat{\rho}\right) = 1$
и  $\hat{\rho} = \hat{\rho}^{\dag}$.
  
Такое разложение возможно из-за переполненности системы когерентных
состояний. Если   $P\left(\alpha\right)$  положительная функция, ее
можно интерпретировать 
как распределение вероятностей. Это относится к некоторым наиболее
интересным с практической точки зрения состояниям поля, например, к
полностью хаотическому состоянию, но это не верно в общем
случае. Иногда   $P\left(\alpha\right)$  может быть отрицательной в
некоторой ограниченной области значений $\alpha$,  тогда ее нельзя интерпретировать как
распределение вероятности.   $P\left(\alpha\right)$  может быть и обобщенной функцией
(примером является $\delta$-функция). 

Строгое обоснование возможности представления \eqref{eqCh1_rhorepresent}
содержится в литературе \cite{bQuantumOpticsAndRadioPhisicsLecture1966},
\cite{bKaluderSudershan1970}. 
Обосновать возможность введения диагонального представления
\eqref{eqCh1_rhorepresent} можно следующим образом. Положим что
статистический оператор $\hat{\rho}$, как и любой другой оператор,
действующий на электромагнитное поле, может быть представлен в виде
функции операторов рождения и уничтожения:
\[
\hat{\rho} = \bar{f}\left(\hat{a}^{\dag}, \hat{a}\right).
\]
Тогда оператор может быть представлен в упорядоченном виде: нормальном
и антинормальном. В первом случае операторы рождения располагаются
справа от операторов уничтожения, а во втором случае наоборот - слева.
Например, 
\[
\left(\hat{a}^{\dag}\right)^m\left(\hat{a}\right)^n
\]
нормально упорядоченный оператор. В антинормально упорядоченном
операторе порядок обратный: операторы рождения стоят справа от
оператора уничтожения. Например, оператор 
\[
\left(\hat{a}\right)^n\left(\hat{a}^{\dag}\right)^m
\]
антинормально упорядочен. Упорядочение можно осуществить многократным
применением коммутационного соотношения. Например, оператор  
\[
\left(\hat{a}^{\dag}\hat{a}\right)^2 =
\hat{a}^{\dag}\hat{a}\hat{a}^{\dag}\hat{a} 
\]
не является ни нормальным, ни антинормальным. Используя условие 
$\left[\hat{a},\hat{a}^{\dag}\right] = 1$,
приведем его к нормальному виду: 
\[
\left(\hat{a}^{\dag}\hat{a}\right)^2 = \hat{a}^{\dag}\left(1 +
\hat{a}^{\dag}\hat{a}\right)\hat{a} = 
\hat{a}^{\dag}\hat{a} + \left(\hat{a}^{\dag}\right)^2\left(\hat{a}\right)^2
\]
можно этот оператор представить в антинормальном виде:
\begin{eqnarray}
\left(\hat{a}^{\dag}\hat{a}\right)^2 = 
\left(\hat{a}\hat{a}^{\dag} - 1\right) \left(\hat{a}\hat{a}^{\dag} -
1\right) = \hat{a}\hat{a}^{\dag} \hat{a}\hat{a}^{\dag} - 2
\hat{a}\hat{a}^{\dag} + 1 =
\nonumber \\
= \hat{a}\left(\hat{a}\hat{a}^{\dag} - 1\right)\hat{a}^{\dag} -
2\hat{a}\hat{a}^{\dag} + 1 = 
\left(\hat{a}\right)^2\left(\hat{a}^{\dag}\right)^2 - 3
\hat{a}\hat{a}^{\dag} + 1.
\nonumber
\end{eqnarray}

Существуют и другие, более эффективные методы упорядочения операторов
\cite{bLuisell1972}.


Представим оператор плотности в
антинормальном виде (рассматриваем одномодовый случай): 
\begin{equation}
\hat{\rho}^{\left(a\right)}\left(\hat{a},\hat{a}^{\dag}\right) = 
\sum_{(n)}\sum_{(m)}C^{\left(a\right)}_{nm}\left(\hat{a}\right)^n\left(\hat{a}^{\dag}\right)^m
\end{equation}
Воспользуемся теперь разложением \eqref{eqCh1_full4coh} для единичного
оператора $\hat{I}$
\[
\hat{I} = \frac{1}{\pi}\int \left|\alpha\right>\left<\alpha\right| d^2 \alpha
\]
в результате получим
\begin{eqnarray}
\hat{\rho}^{\left(a\right)}\left(\hat{a},\hat{a}^{\dag}\right) = 
\sum_{(n)}\sum_{(m)}C^{\left(a\right)}_{nm}\left(\hat{a}\right)^n\hat{I}\left(\hat{a}^{\dag}\right)^m
= 
\nonumber \\
= \frac{1}{\pi}\sum_{(n)}\sum_{(m)}C^{\left(a\right)}_{nm}\int d^2 \alpha
\left(\hat{a}\right)^n
\left|\alpha\right>\left<\alpha\right|
\left(\hat{a}^{\dag}\right)^m = 
\nonumber \\
= \frac{1}{\pi}\sum_{(n)}\sum_{(m)}\int d^2 \alpha
C^{\left(a\right)}_{nm}
\alpha^n
\alpha^{*m}
\left|\alpha\right>\left<\alpha\right|.
\end{eqnarray}
Обозначив теперь 
\[
P\left(\alpha, \alpha^{*}\right) = \frac{1}{\pi}\sum_{(n)}\sum_{(m)}
C^{\left(a\right)}_{nm}
\alpha^n
\alpha^{*m}
\]
получим что $\hat{\rho}$ можно представить в виде
\begin{equation}
\hat{\rho} = \int d^2 \alpha P\left(\alpha, \alpha^{*}\right) 
\left|\alpha\right>\left<\alpha\right|.
\label{eqCh1_Rho_in_alpha}
\end{equation}

Среднее значение оператора при использовании
диагонального представления 
\begin{eqnarray}
\left<\hat{O}\right> = Sp \left(\hat{\rho}\hat{O}\right) =
\nonumber \\
= \sum_{(n)}\bra{n}
\int d^2 \alpha P\left(\alpha, \alpha^{*}\right) 
\left|\alpha\right>\left<\alpha\right|
\hat{O}\ket{n} =
\nonumber \\
= \sum_{(n)} \int d^2 \alpha P\left(\alpha, \alpha^{*}\right)
\left<\alpha\right|\hat{O}\ket{n}
\bra{n}\left.\alpha\right> = 
\nonumber \\
=  
\int d^2 \alpha P\left(\alpha, \alpha^{*}\right)
\left<\alpha\right|\hat{O}\left|\alpha\right>,
\label{eqCh1_middleO}
\end{eqnarray}
т. к. $\sum_{(n)}\ket{n}\bra{n} = \hat{I}$. 

Выражение $\left<\alpha\right|\hat{O}\left|\alpha\right>$ легко
вычисляется, если оператор $\hat{O}$ представлен в нормальном виде:
\begin{equation}
\hat{O} = \hat{O}^{(n)} = \sum_{(n)}\sum_{(m)} d_{nm}
\left(\hat{a}^{\dag}\right)^n
\left(\hat{a}\right)^m,
\label{eqCh1_normalO}
\end{equation}
тогда
\begin{equation}
\left<\alpha\right|\hat{O}\left|\alpha\right> = 
\sum_{(n)}\sum_{(m)} d_{nm}
\left<\alpha\right|
\left(\hat{a}^{\dag}\right)^n
\left(\hat{a}\right)^m
\left|\alpha\right> = 
\sum_{(n)}\sum_{(m)} d_{nm}
\alpha^{*n}\alpha^{m}.
\end{equation}
Отсюда следует простое правило: чтобы получить необходимый матричный
элемент, нужно в операторе, представленном в нормально упорядоченном
виде, произвести замену $\hat{a}\rightarrow\alpha$, 
$\hat{a}^{\dag}\rightarrow\alpha^{*}$.

В качестве примера покажем, что для теплового возбуждения моды 
\begin{equation}
P\left(\alpha, \alpha^{*}\right) = \frac{1}{\pi \bar{n}}
e^{- \frac{\left|\alpha\right|^2}{\bar{n}}}
\label{eqCh1_123a}
\end{equation}
и, следовательно
\[
\hat{\rho} = \frac{1}{\pi \bar{n}}
\int d^2 \alpha 
e^{- \frac{\left|\alpha\right|^2}{\bar{n}}}
\left|\alpha\right>
\left<\alpha\right|.
\]
Найдем теперь матричные элементы оператора $\hat{\rho}$ в
представлении чисел заполнения (числа фотонов)
\begin{eqnarray}
\bra{n}
\hat{\rho}
\ket{n} = 
\frac{1}{\pi \bar{n}}
\int d^2 \alpha 
e^{- \frac{\left|\alpha\right|^2}{\bar{n}}}
\bra{n}\left.\alpha\right>
\left<\alpha\right|
\ket{n} = 
\nonumber \\
= 
\frac{1}{\pi \bar{n}}
\int d^2 \alpha 
e^{- \frac{\left|\alpha\right|^2}{\bar{n}}}
\frac{\left|\alpha\right|^{2n}}{n!} 
e^{- \left|\alpha\right|^2} 
\nonumber 
\end{eqnarray}
Получившийся интеграл можно вычислить записав в полярной системе координат:
\[
\alpha = r e^{i \varphi}, \quad 
d^2 \alpha = r d r d \varphi , \quad 
\mbox{т. е.} 
\left|\alpha\right| = r, \quad arg\,\alpha = \varphi,
\]
откуда
\begin{eqnarray}
\frac{1}{\pi \bar{n}}
\int d^2 \alpha 
e^{- \frac{\left|\alpha\right|^2}{\bar{n}}}
\frac{\left|\alpha\right|^{2n}}{n!} 
e^{- \left|\alpha\right|^2} =
\nonumber \\
=
\frac{1}{\pi \bar{n}}
\int_0^{2 \pi} d \varphi 
\int_0^{\infty}
r dr \frac{exp \left(- r^2\frac{\bar{n} + 1}{\bar{n}}\right)}{n!} r^{2n}= 
\nonumber \\
= 
\frac{1}{\bar{n}}
\int_0^{\infty}
2 r dr \frac{exp \left(- r^2\frac{\bar{n} + 1}{\bar{n}}\right)}{n!}  r^{2n} = 
\nonumber \\
= 
\frac{1}{\bar{n}}
\int_0^{\infty}
\frac{e^{-x}}{n!}\frac{\bar{n}^{n + 1}}
{\left(\bar{n} + 1\right)^{n + 1}}x^n dx = 
\nonumber \\
=
\frac{\bar{n}^{n}}
{\left(\bar{n} + 1\right)^{n + 1}}
\frac{1}{n!}
\int_0^{\infty}
e^{-x}x^n dx = 
\frac{\bar{n}^{n}}
{\left(\bar{n} + 1\right)^{n + 1}},
\label{eqCh1_matrelemRho}
\end{eqnarray}
где сделана замена $x = r^2\frac{\bar{n} + 1}{\bar{n}}$.

Таким образом, мы получили известный нам матричный элемент
статистического оператора хаотического поля в представлении чисел
заполнения. Отсюда следует, что выражение \eqref{eqCh1_123a}
действительно является статистическим оператором хаотического поля в
преставлении когерентных состояний. Следовательно для теплового
возбуждения фотонов в моду имеем
\begin{equation}
P\left(\alpha\right) = \frac{1}{\pi \bar{n}}
e^{-\frac{\left|\alpha\right|^2}{\bar{n}}},
\mbox{где }
\bar{n} = \frac{1}{e^{\hbar \omega \beta} - 1}
\label{eqCh1_task4}
\end{equation}
  
Обобщение полученных выражений на многомодовый случай очевидно: 
\begin{equation}
\left<\hat{O}\right> = \idotsint\limits_{\left\{\alpha_k\right\}}
P^{(a)}
\left(
\left\{\alpha_k\right\}
\right)
O^{(n)}
\left(
\left\{\alpha_k\right\}
\right)
\prod_k d^2 \alpha_k,
\label{eqCh1_113}
\end{equation}
\[
P^{(a)}
\left(
\left\{\alpha_k\right\}
\right)
=
\prod_k \frac{1}{\pi \bar{n}_k}
e^{-\frac{\left|\alpha_k\right|^2}{\bar{n}_k}}. 
\]
