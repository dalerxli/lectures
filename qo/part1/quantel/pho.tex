%% -*- coding:utf-8 -*- 
\section{Плотность состояний}
Если объем квантования $L^3$ велик, плотность состояний (число мод,
приходящихся на единичный интервал частот) будет весьма
большой. Волновой вектор моды определяется соотношением
(\ref{eqCh1_period})
\begin{equation}
\vec{k}_{n_x, n_y, n_z} = \frac{2 \pi}{L}\left(n_x \vec{x}_0
+ n_y \vec{y}_0
+ n_z \vec{z}_0
\right)
\end{equation}
Каждому набору целых чисел соответствуют две волны, отличающиеся
поляризацией. Моды можно представить наглядно точками в декартовой
системе координат, по осям которой отложены  $n_x$, $n_y$  ,$n_z$
(\autoref{figCh1_pic3}). Число колебаний в объеме  
\(
\Delta n_x \Delta n_y \Delta n_z
\)
, очевидно, составит  
\(
\Delta N = 2 \Delta n_x \Delta n_y \Delta n_z,
\)
или, с учетом связи  
между  $k$  и  $n$  (\ref{eqCh1_period}),
\begin{equation}
\Delta N = 2 \left(\frac{L}{2 \pi} \right)^3 \Delta k_x \Delta k_y \Delta k_z.
\label{eqCh1_modenumber}
\end{equation}

\input ./part1/quantel/fig3.tex

\input ./part1/quantel/fig4.tex

Двойка в формуле (\ref{eqCh1_modenumber}) появляется из-за того, что
одному значению  $k$  соответствуют две моды с различными
поляризациями. При большом $L$ ($L\rightarrow \infty$)  распределение
квазинепрерывно и суммирование по модам,
которое почти всегда требуется при решении задач квантовой оптики,
можно заменить интегрированием
по $k$: 
\begin{equation}
2 \sum_{(n)} \left( \dots \right) = 2 \left(\frac{L}{2 \pi} \right)^3
\iiint\limits_{-\infty}^{+\infty} \left( \dots \right) d k_x d k_y d k_z.
\label{eqCh1_modenumber_kvazy_contig}
\end{equation}
Переход от прямоугольных координат  $k_x$,  $k_y$, $k_z$   к сферическим координатам  
$k$, $\theta$, $\varphi$    (\autoref{figCh1_pic4}), дает
\begin{eqnarray}
k_x = k \sin \theta \cos \varphi,
\quad 
k_y = k \sin \theta \sin \varphi,
\quad 
k_z  = k \cos \theta,
\nonumber \\
d k_x d k_y d k_z = k^2 \sin \theta d k d \theta d \varphi = k^2 d k
d \Omega,
\end{eqnarray}
где $d \Omega = \sin \theta d \theta d \varphi$ элемент телесного угла в
направлении  $\vec{k}$.  Таким образом, получим
\begin{equation}
d N = 2 \left(\frac{L}{2 \pi} \right)^3 k^2 d k d \Omega
\label{eqCh1_modenumber_1pre}
\end{equation}
или, учитывая, что $k^2 c^2 = \omega^2$ т. е. $k^2 d k =
\frac{\omega^2 d \omega}{c^3}$: 
\begin{equation}
d N = 2 \left(\frac{L}{2 \pi c} \right)^3 \omega^2 d \omega d \Omega
\label{eqCh1_modenumber_1}
\end{equation}
т.е. число состояний, приходящихся на единицу объема, единицу частоты и единицу телесного угла
\begin{equation}
g\left(\omega\right)  = \frac{2 \omega^2}{\left(2 \pi c\right)^3}
\end{equation}
(спектральная плотность собственных состояний). 
