%% -*- coding:utf-8 -*- 
\section{Смешанные состояния электромагнитного поля}
До сих пор мы рассматривали только чистые состояния электромагнитного
поля, описываемые векторами состояний. 
Одним из наиболее отличительных свойств чистых состояний является
принцип суперпозиции. 

Чистыми системами могут быть только изолированные системы, тогда как
смешанные системы не изолированны от своего окружения. Для описания
смешанных состояний используют аппарат матрицы плотности. Оператор
плотности можно ввести следующим образом.  

Рассмотрим смешанное состояние, 
\index{Смешанное состояние}
для которого известны вероятности
$P_M$
нахождения поля в каждом из состояний 
$\ket{M}$.  Вычислим среднее значение
оператора $\hat{O}$,  соответствующего наблюдаемой $O$ в этом статистически
смешанном состоянии. Среднее для статистической смеси состояний  
\begin{equation}
\left<\hat{O}\right> = \sum_{(M)} P_M\bra{M}\hat{O}\ket{M},
\end{equation}
где, очевидно, 
\[
\sum_{(M)} P_M = 1.
\]

Это выражение можно представить в другом виде. Используя некоторый
полный набор состояний $\ket{S}$   и условие его полноты 
\[
\sum_{(S)}\ket{S}\bra{S} = \hat{I},
\]
получим
\begin{equation}
\left<\hat{O}\right> = \sum_{(M)}
P_M\sum_{(S)}\bra{M}\hat{O}\ket{S}\bra{S}\ket{M}
= \sum_{(M)}\sum_{(S)}P_M\bra{S}\ket{M}\bra{M}\hat{O}\ket{S}.
\label{eqCh1_middle}
\end{equation}

Исходя из \eqref{eqCh1_middle}, можно ввести статистический оператор $\hat{\rho}$  при помощи выражения
\begin{equation}
\hat{\rho} = \sum_{(M)}
P_M\ket{M}\bra{M}.
\end{equation}

Тогда равенство \eqref{eqCh1_middle} можно представить в виде
\begin{equation}
\left<\hat{O}\right> = \sum_{(S)}
\bra{S}\hat{\rho}\hat{O}\ket{S} = Sp \left(\hat{\rho}\hat{O}\right),
\end{equation}
Это соотношение не зависит от выбора  $\ket{S}$,  так как след
оператора не зависит от представления. Заметим, что 
\begin{equation}
Sp \left(\hat{\rho}\right) = \sum_{(M)}
\bra{M}\hat{\rho}\ket{M} = \sum_{(M)} P_M = 1
\label{eqCh1_spequal1}
\end{equation}

В качестве примера рассмотрим тепловое возбуждение фотонов в одну
моду:
\begin{equation}
\hat{\rho} = \sum_{(n)}
P_n\ket{n}\bra{n},
\label{eqCh1_teplovvozb}
\end{equation}
где $P_n$ определяется распределением Больцмана
\[
P_n = e^{-\beta n \hbar \omega}\left(1  -  e^{-\beta \hbar \omega}\right),
\]
здесь  $\beta = \frac{1}{k_b T}$, а $\left(1  -  e^{-\beta \hbar
  \omega}\right)$ - нормирующий множитель. 

Среднее число фотонов в моде при тепловом возбуждении:
\begin{eqnarray}
\bar{n} = \left<\hat{n}\right> =  Sp \left(\hat{\rho}\hat{n}\right) = 
Sp \left(\hat{\rho}\hat{a}^{\dag}\hat{a}\right) = 
\nonumber \\
=\sum_{(m)}\sum_{(n)}
P_n\bra{m}\ket{n}\bra{n}\hat{a}^{\dag}\hat{a}\ket{m}
= 
\nonumber \\
= \sum_{(n)}
P_n\bra{n}\hat{n}\ket{n} = \sum_{(n)} n
e^{-\beta n \hbar \omega}\left(1  -  e^{-\beta \hbar \omega}\right) = 
\frac{1}{e^{\beta \hbar \omega} - 1}
\label{eqCh1_plank}
\end{eqnarray}
Это известное выражение, полученное Планком. При этом мы
воспользовались следующим выражением
\[
\sum_{n=0}^{\infty} n r^{n -1} = \frac{d}{d r} \sum_{n=0}^{\infty}
r^{n} = \frac{1}{\left(1 - r\right)^2},
\]
справедливым при $\left|r\right| < 1$.

Равенства \eqref{eqCh1_teplovvozb}, \eqref{eqCh1_plank} можно записать в
другом виде. Из \eqref{eqCh1_plank} получим 
\[
e^{-\beta \hbar \omega} = \frac{\bar{n}}{1 + \bar{n}},
\]
следовательно, 
\begin{equation}
P_n = e^{-\beta \hbar \omega n} \left(1  -  e^{-\beta \hbar
  \omega}\right) = \frac{\bar{n}^n}{\left(1 + \bar{n}\right)^{n+1}}.
\label{eqCh1_plank2}
\end{equation}
При этом матрица плотности \index{Матрица плотности}
приобретает вид:
\[
\hat{\rho} = \sum_{(n)}\frac{\bar{n}^n}{\left(1 + \bar{n}\right)^{n+1}}\ket{n}\bra{n}.
\]
Эта формула описывает матрицу плотности хаотического света через
$\bar{n}$ -  среднее значение числа фотонов в моде. При этом
$\bar{n}$ не обязательно зависит от частоты в соответствии с формулой
Планка \eqref{eqCh1_plank2}. Зависимость $\bar{n}$ от частоты
определяет форму линии хаотического света (различным частотам
соответствуют различные моды). Обобщая полученный результат, можем
написать выражение для статистического оператора хаотического света:  
\begin{equation}
\hat{\rho} = \sum_{\left\{n_k\right\}} P_{\left\{n_k\right\}} \left|\left\{n_k\right\}\right>\left<\left\{n_k\right\}\right| = 
\sum_{\left\{n_k\right\}} 
 \left|\left\{n_k\right\}\right>\left<\left\{n_k\right\}\right|
\prod_{\left\{k\right\}} 
\frac{\bar{n}_k^{n_k}}{\left(1 + \bar{n}_k\right)^{n_k+1}},
\label{eqCh1_102}
\end{equation}
где $\bar{n}_k$  зависят от частоты. Например, для хаотического света,
имеющего лоренцовскую линию, 
\[
\bar{n}_k = \frac{I S}{\hbar \omega_{k_0}}
\frac{\gamma}{\left(\omega_{k_0} - \omega_{k}\right)^2 + \gamma^2},
\]
где $\frac{I S}{\hbar \omega_{k_0}}$   -  среднее число фотонов в
пучке;  $I$ -  интенсивность пучка (поток энергии);  $S$ -  сечение
пучка.  
