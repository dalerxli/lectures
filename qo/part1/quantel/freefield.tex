%% -*- coding:utf-8 -*- 
\section{Разложение поля по плоским волнам в свободном пространстве}
До сих пор мы рассматривали электромагнитное поле в физически
выделенном объеме - в резонаторе. Если мы имеем дело со свободным
пространством, можно искусственно выделить достаточно большой объем,
содержащий интересующую нас область пространства, определить для него
моды (типы колебаний) при подходящих граничных условиях, а дальше
действовать уже рассмотренным способом. При необходимости объем в
конечном результате можно устремить в бесконечность. Обычно используют
достаточно большой кубический объем со стороной  $L$  
(\autoref{figCh1_Vfree}). В этом случае принято использовать
периодические граничные условия:  
\begin{eqnarray}
\vec{E}\left(0, y, z \right) = \vec{E}\left(L, y, z \right),
\nonumber \\
\vec{E}\left(x, 0, z \right) = \vec{E}\left(x, L, z \right),
\nonumber \\
\vec{E}\left(x, y, 0 \right) = \vec{E}\left(x, y, L \right).
\label{eqCh1_period_def}
\end{eqnarray}

\input ./part1/quantel/fig2.tex

Удобно вести разложение по плоским волнам. Как известно из курса
электромагнитных колебаний, решение, соответствующее плоской волне,
имеет вид 
\begin{equation}
\vec{E}_k\left(r, t\right) = 
A_k\left(t\right)\vec{e}_k e^{i\left(\vec{k}\vec{r}\right)} +
\mbox{(к. с.)},
\label{eqCh1_Emode}
\end{equation}
где $\vec{e}_k$ - единичный вектор поляризации волны;  
$k$ - волновое число; $\vec{k}$ - волновой вектор;  
$A_k\left(t\right) = A_k\left(0\right) e^{-i \omega_k t}$.

Магнитное поле связано с электрическим полем соотношением
\begin{equation}
\vec{H}_k\left(r, t\right) =
\sqrt{\frac{\varepsilon_0}{\mu_0}}
\frac{1}{k}\left[\vec{k} \vec{e}_k\right] A_k\left(t\right) 
e^{i\left(\vec{k}\vec{r}\right)} + \mbox{(к. с.)}
.
\label{eqCh1_Hmode}
\end{equation}

Справедливы следующие равенства:
\begin{equation}
\left(\vec{k}\vec{E}_k\right) = 
\left(\vec{k}\vec{H}_k\right) = 
\left(\vec{E}_k\vec{H}_k\right) = 0,
\quad
k^2 = \left(\vec{k}\vec{k}\right) = 
\frac{\omega_k^2}{c^2} 
\end{equation}
указывающие, что 
$\vec{E}_k$
и $\vec{H}_k$ перпендикулярны направлению распространения волны и друг
к другу (волна поперечная). 

Условие периодичности \eqref{eqCh1_period_def} будет удовлетворено,
если 
\begin{equation}
\vec{k} = \frac{2 \pi}{L}\left(n_x \vec{x}_0
+ n_y \vec{y}_0
+ n_z \vec{z}_0
\right),
\quad
n_x, n_y, n_z = 0, \pm 1, \pm 2, \dots .
\label{eqCh1_period}
\end{equation}
Тогда получим
\begin{eqnarray}
\left(\vec{k}\vec{r}\right) = \frac{2 \pi}{L}\left(n_x x
+ n_y y
+ n_z z
\right),
\nonumber \\
\left.\left(\vec{k}\vec{r}\right)\right|_{x = L} = 2 \pi n_x + \frac{2 \pi}{L}\left(n_y y
+ n_z z
\right) = 
2 \pi n_x + \left.\left(\vec{k}\vec{r}\right)\right|_{x = 0}.
\end{eqnarray}
Следовательно,    
$e^{i\left(\vec{k}\vec{r}\right)}$
периодично по $x$.  Таким же образом показывается
периодичность по  $y$  и  $z$.
 
В дальнейшем вместо вещественных функций \eqref{eqCh1_Emode}, 
\eqref{eqCh1_Hmode} будем пользоваться
комплексными собственными функциями 
\begin{equation}
\vec{E}_k\left(r\right) = \vec{e}_k e^{i \left( \vec{k}\vec{r}\right)},
\quad
\vec{H}_k\left(r\right) = \sqrt{\frac{\varepsilon_0}{\mu_0}}\frac{1}{k}
\left[\vec{k}\vec{E}_k\left(r\right)\right].
\label{eqCh1_EHmode}
\end{equation}
Справедливы соотношения
\[
\left(\vec{k}\vec{E}_k\right) = 
\left(\vec{k}\vec{H}_k\right) = 
\left(\vec{E}_k\vec{H}_k\right) = 0,
\quad
k^2 = \left(\vec{k}\vec{k}\right) = 
\frac{\omega_k^2}{c^2}.
\]
Собственные частоты $\omega_k$ определяются уравнением
\eqref{eqCh1_period}. Действительно из \eqref{eqCh1_period} с учетом
$k^2 = \frac{\omega_k^2}{c^2}$ имеем 
\begin{equation}
\omega_k = c \sqrt{k_x^2 + k_y^2 + k_z^2} = 
\frac{2 \pi c}{L} \sqrt{n_x^2 + n_y^2 + n_z^2}.
\end{equation}

При линейной поляризации векторы поляризации $\vec{e}_k$ равны:
\[
\vec{e}_{k_1} = \vec{\xi}_0,
\quad
\vec{e}_{k_2} = \vec{\eta}_0,
\]
где $\vec{\xi}_0$, $\vec{\eta}_0$, - орты направлений, составляющих с
вектором $\vec{k}$ прямоугольную систему направлений, следовательно, 
\[
\left(\vec{\xi}_0\vec{k}\right) =
\left(\vec{\eta}_0\vec{k}\right) =
\left(\vec{e}_{k_1}\vec{e}_{k_2}\right) = 0.
\]

Для круговой поляризации
\[
\vec{e}_{k_1} = \frac{\vec{\xi}_0 + i \vec{\eta}_0}{\sqrt{2}},
\quad
\vec{e}_{k_2} = \frac{\vec{\xi}_0 - i \vec{\eta}_0}{\sqrt{2}}.
\]

В этом случае выполняются условие ортогональности:
$\left(\vec{e}_{k_1} \vec{e}_{k_2}^{*}\right)$
и условия нормировки:
$\left(\vec{e}_{k_1} \vec{e}_{k_1}^{*}\right) = \left(\vec{e}_{k_2}
\vec{e}_{k_2}^{*}\right) = 1$.

Произвольное электромагнитное поле при помощи комплексных функций 
\eqref{eqCh1_EHmode} можно представить разложениями 
\begin{eqnarray}
\vec{E}\left(r, t\right) = 
\sum_{(k)} 
A_k\left(t\right) \vec{E}_k\left(r\right) + \mbox{(к. с.)},
\nonumber \\
\vec{H}\left(r, t\right) = 
\sum_{(k)} 
A_k\left(t\right) \vec{H}_k\left(r\right) +
\mbox{(к. с.)}
\end{eqnarray}
