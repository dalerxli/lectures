%% -*- coding:utf-8 -*- 
\section{Общая теория взаимодействия динамической системы с
  термостатом}

До сих пор мы имели дело с частной моделью взаимодействия моды
резонатора (динамической системы) с атомным пучком, находящимся в
равновесии при
температуре  $T$  (с резервуаром). Здесь мы рассмотрим более общий
подход к задаче о взаимодействии динамической системы с резервуаром
\cite{bLaks1974}. Имеем дело с двумя взаимодействующими
системами:  $A$ - динамическая система и  $B$ - резервуар
(диссипативная система). Системы $A$ и $B$ слабо связаны между собой,
то есть энергия связи мала по сравнению с энергией динамической
системы. Гамильтониан взаимодействия обозначим $\hat{V}$.
Диссипативная система  $B$ - обширная система, находящаяся в
термодинамическом равновесии при температуре $T$.  Сперва будем
рассматривать системы  $A$ и $B$ как равноправные. Обозначим
статистический оператор полной системы $\hat{\rho}_{AB}$. 
Введем статистические операторы $\hat{\rho}_{A}$ и $\hat{\rho}_{B}$,
относящиеся соответственно только к системам $A$, $B$.   
С оператором $\hat{\rho}_{AB}$ они связаны через соотношения 
\begin{equation}
\hat{\rho}_{A} = Sp_B\left\{\hat{\rho}_{AB}\right\}, \quad
\hat{\rho}_{B} = Sp_A\left\{\hat{\rho}_{AB}\right\}, \quad
\label{eqCh2_78}
\end{equation}
Уравнение движения статистического оператора всей объединенной системы
имеет в представлении взаимодействия вид 
\begin{equation}
i\hbar\dot{\hat{\rho}}_{AB}
 = \left[
\hat{V}, \hat{\rho}_{AB}
\right].
\label{eqCh2_79}
\end{equation}
Уравнения движения для свернутых операторов очевидно равны
\begin{eqnarray}
i\hbar\dot{\hat{\rho}}_{A} = Sp_B
\left[
\hat{V}, \hat{\rho}_{AB}
\right],
\nonumber \\
i\hbar\dot{\hat{\rho}}_{B} = Sp_A
\left[
\hat{V}, \hat{\rho}_{AB}
\right].
\label{eqCh2_80}
\end{eqnarray}
Статистический оператор $\hat{\rho}_{AB}$ представим в следующем виде
\begin{equation}
\hat{\rho}_{AB} = \hat{\rho}_{A} \otimes \hat{\rho}_{B} + \hat{\rho}_{C},
\label{eqCh2_81}
\end{equation}
где $\hat{\rho}_{C}$ является следствием взаимодействия между
системами, $\otimes$ - знак внешнего (тензорного) произведения. 
\index{Тензорное произведение}
В начальный момент $t_0$
системы предполагаются  невзаимодействующими, тогда полный
статистический оператор равен внешнему произведению 
\[
\left.\hat{\rho}_{AB}\right|_{t = t_0} = \hat{\rho}_{A} \otimes \hat{\rho}_{B},
\] 
а 
\[
\left.\hat{\rho}_{C}\right|_{t = t_0} = 0.
\] 
Преобразуем теперь член 
$\left[\hat{V}, \hat{\rho}_{A} \otimes \hat{\rho}_{B}\right]$,
входящий в уравнение \eqref{eqCh2_80}, и получим  
\begin{eqnarray}
\left[\hat{V}, \hat{\rho}_{A} \otimes \hat{\rho}_{B}\right] = 
\hat{V}  \hat{\rho}_{A} \otimes \hat{\rho}_{B} - 
\hat{\rho}_{A} \otimes \hat{\rho}_{B} \hat{V} = 
\nonumber \\
= \hat{V}  \hat{\rho}_{B} \otimes \hat{\rho}_{A} - 
\hat{\rho}_{A} \otimes \hat{\rho}_{B} \hat{V}.
\label{eqCh2_82}
\end{eqnarray}
так как 
\(
\hat{\rho}_{A} \otimes \hat{\rho}_{B} = 
\hat{\rho}_{B} \otimes \hat{\rho}_{A}
\)   из-за того, что  $\hat{\rho}_{A}$ и $\hat{\rho}_{B}$ действуют на
разные  переменные. Отсюда, производя операцию $Sp_B$, имеем 
\begin{eqnarray}
Sp_B \left[\hat{V}, \hat{\rho}_{A} \otimes \hat{\rho}_{B}\right] = 
Sp_B\left(\hat{V} \hat{\rho}_{B}\right) \hat{\rho}_{A} - 
\nonumber \\
-
\hat{\rho}_{A} Sp_B \left(\hat{\rho}_{B} \hat{V}\right) = 
\left[\hat{V}_A, \hat{\rho}_A\right],
\label{eqCh2_83}
\end{eqnarray}
где $\hat{V}_A = Sp_B\left(\hat{V}\hat{\rho}_B\right)$. Таким же
образом получим 
\begin{equation}
Sp_A \left[\hat{V}, \hat{\rho}_{A} \otimes \hat{\rho}_{B}\right] =
\left[\hat{V}_B, \hat{\rho}_B\right],
\label{eqCh2_84}
\end{equation}
где $\hat{V}_B= Sp_A\left(\hat{V}\hat{\rho}_A\right)$.  
Учитывая равенства \eqref{eqCh2_81}, \eqref{eqCh2_82},
\eqref{eqCh2_83}, \eqref{eqCh2_84}, уравнения движения 
$\hat{\rho}_{A}$ и $\hat{\rho}_{B}$ можно представить в виде  
\begin{eqnarray}
i\hbar\frac{d}{d t} \hat{\rho}_A\left(t\right) = 
\left[\hat{V}_A\left(t\right), \hat{\rho}_A\right] +
Sp_B \left[\hat{V}, \hat{\rho}_C\right],
\nonumber \\
i\hbar\frac{d}{d t} \hat{\rho}_B\left(t\right) = 
\left[\hat{V}_B\left(t\right), \hat{\rho}_B\right] +
Sp_A \left[\hat{V}, \hat{\rho}_C\right].
\label{eqCh2_85}
\end{eqnarray}
Из вида \eqref{eqCh2_85} понятно, что $\hat{V}_A$ и $\hat{V}_B$ имеют
смысл изменения энергии одной системы связанной с ее взаимодействием с
другой, например с взаимодействием динамической системы с
термостатом. Эти изменения малы и в нашем случае их можно не
учитывать. Таким образом \eqref{eqCh2_85} преобразуется в 
\begin{eqnarray}
i\hbar\frac{d}{d t} \hat{\rho}_A\left(t\right) = 
Sp_B \left[\hat{V}, \hat{\rho}_C\right],
\nonumber \\
i\hbar\frac{d}{d t} \hat{\rho}_B\left(t\right) = 
Sp_A \left[\hat{V}, \hat{\rho}_C\right].
\label{eqCh2_85a}
\end{eqnarray}

Для оператора $\hat{\rho}_C$,  используя то, что  
\[
\hat{\rho}_C = \hat{\rho}_{AB} - \hat{\rho}_A \otimes \hat{\rho}_B
\]
можем написать уравнение движения следующим образом
\begin{eqnarray}
i \hbar \frac{d \hat{\rho}_C}{d t} = i \hbar \frac{d
  \hat{\rho}_{AB}}{d t} - i \hbar 
\frac{d}{d t} \left(\hat{\rho}_A \otimes \hat{\rho}_B\right) = 
\nonumber \\
= i \hbar
\left(
\frac{d \hat{\rho}_{AB}}{d t} -
\frac{d \hat{\rho}_{A}}{d t} \otimes \hat{\rho}_{B} - 
\hat{\rho}_{A} \otimes \frac{d \hat{\rho}_{B}}{d t}
\right).
\label{eqCh2_86}
\end{eqnarray}
Используя уравнения движения $\hat{\rho}_{AB}$, $\hat{\rho}_{A}$,
$\hat{\rho}_{B}$  \eqref{eqCh2_85a},  получим для \eqref{eqCh2_86} 
\begin{eqnarray}
i \hbar \frac{d \hat{\rho}_C}{d t} = 
\left[\hat{V}, \left(\hat{\rho}_A \otimes \hat{\rho}_B +
  \hat{\rho}_{C}\right)\right] - 
\nonumber \\
- 
\left(Sp_B\left[\hat{V}, \hat{\rho}_C\right]\right)\hat{\rho}_B -
\hat{\rho}_A \left(Sp_A\left[\hat{V}, \hat{\rho}_C\right]\right)
\label{eqCh2_89}
\end{eqnarray}
Пренебрегая членами второго порядка малости, получим простое уравнение:
\begin{equation}
i \hbar \frac{d \hat{\rho}_C}{d t} = 
\left[\hat{V}, \left(\hat{\rho}_A \otimes \hat{\rho}_B\right)\right]. 
\label{eqCh2_90}
\end{equation}
Членами первого порядка малости мы считаем $\hat{V}$, $\hat{\rho}_C$.
Это следует из предположения, что взаимодействие между динамической
системой и термостатом малое. Члены, в которые входят произведения
типа  $\hat{V}\hat{\rho}_C$, считаем малым второго порядка. Формальное
решение уравнения \eqref{eqCh2_90} имеет вид  
\begin{equation}
\hat{\rho}_C\left(t\right) = - \frac{i}{\hbar}\int_{t_0}^t
\left[\hat{V}\left(t'\right), \left(\hat{\rho}_A \otimes
  \hat{\rho}_B\right)\right] dt' 
\label{eqCh2_91}
\end{equation}
при начальных условиях $\hat{\rho}_C\left(t_0\right) = 0$. 

До сих пор мы считали, что системы  $A$  и  $B$  равноправны. Теперь
учтем, что динамическая система  $A$ - малая, а диссипативная система
$B$ - настолько большая, что взаимодействие ее с динамической системой
не может в заметной степени нарушить ее равновесие. Учтем это, положив
$\frac{d \hat{\rho}_B}{d t} = 0$: 
\begin{equation}
\hat{\rho}_B\left(t'\right) =
\hat{\rho}_B\left(t_0\right).
\label{eqCh2_92}
\end{equation}
Подставив в уравнение для $\hat{\rho}_A$ \eqref{eqCh2_85} равенства
\eqref{eqCh2_89} и \eqref{eqCh2_91}, получим 
\begin{eqnarray}
\frac{d \hat{\rho}_A}{d t} = - \frac{1}{\hbar^2}
\int_{t_0}^t Sp_B\left[
\hat{V}\left(t\right),
\left[
\hat{V}\left(t'\right),
\hat{\rho}_A\left(t'\right) \otimes
\hat{\rho}_B\left(t_0\right)
\right]
\right]dt' = 
\nonumber \\ 
=
- \frac{1}{\hbar^2}
\int_{t_0}^t d t' Sp_B\left\{
\left[
\hat{V}\left(t\right)
\hat{V}\left(t'\right)
\hat{\rho}_A\left(t'\right) \otimes
\hat{\rho}_B\left(t_0\right) -
\right.
\right.
\nonumber \\
-
\left.
\left.
\hat{V}\left(t\right)
\hat{\rho}_A\left(t'\right) \otimes
\hat{\rho}_B\left(t_0\right)
\hat{V}\left(t'\right)
\right] - 
\right.
\nonumber \\
-
\left.
\left[
\hat{V}\left(t'\right)
\hat{\rho}_A\left(t'\right) \otimes
\hat{\rho}_B\left(t_0\right)
\hat{V}\left(t\right)
-
\hat{\rho}_A\left(t'\right) \otimes
\hat{\rho}_B\left(t_0\right)
\hat{V}\left(t'\right)\hat{V}\left(t\right) 
\right]
\right\}.
\label{eqCh2_93}
\end{eqnarray}

Это уравнение для оператора плотности динамической системы  $A$,
взаимодействующей с диссипативной системой $B$.  При обширной системе 
В  время корреляции $\hat{V}\left(t\right)$ и $\hat{V}\left(t'\right)$
очень мало, значительно меньше 
характерного времени изменения $\hat{\rho}_A$.  Следовательно,
существенная область интегрирования находится в окрестности $t' =
t$. Тогда во многих случаях при интегрировании можно положить
\[
\hat{\rho}_A\left(t'\right) =
\hat{\rho}_A\left(t\right).
\]
Более полно с общей теорией взаимодействия динамической системы с
термостатом можно познакомиться в \cite{bLaks1974}. Применим
общую теорию к конкретным случаям.  
 
