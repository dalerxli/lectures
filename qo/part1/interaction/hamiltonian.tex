%% -*- coding:utf-8 -*- 
\section{Гамильтониан системы атом-поле}
Гамильтониан системы атом-поле можно представить в виде
\begin{equation}
\hat{\mathcal{H}} = \hat{\mathcal{H}}^{(A)} +
\hat{\mathcal{H}}^{(F)} + \hat{V}
\nonumber
\end{equation}
где
$\hat{\mathcal{H}}^{(A)}$ - гамильтониан двухуровнего атома,
$\hat{\mathcal{H}}^{(F)}$ - гамильтониан электромагнитного поля и
$\hat{V}$ - гамильтониан взаимодействия.

Для гамильтониана электромагнитного поля мы имеем выражение
(\ref{eqCh1_quant_stoyachie_volny}):
\begin{equation}
\hat{\mathcal{H}}^{(F)} = \hbar \omega 
\left(\hat{a}^{+}\hat{a} + \frac{1}{2}\right).
\nonumber
\end{equation}

В случае гамильтониана двухуровненго атома воспользуемся следующими
соотношениями:
\begin{eqnarray}
\hat{\mathcal{H}}^{(A)}\left|a\right> = \hbar \omega_a \left|a\right>,
\nonumber \\
\hat{\mathcal{H}}^{(A)}\left|b\right> = \hbar \omega_b \left|b\right>,
\nonumber \\
\left|a\right>\left<a\right| + \left|b\right>\left<b\right| = \hat{I},
\nonumber
\end{eqnarray}
откуда получим
\begin{eqnarray}
\hat{\mathcal{H}}^{(A)} = \hat{\mathcal{H}}^{(A)}\hat{I} = 
\hat{\mathcal{H}}^{(A)}
\left|a\right>\left<a\right| + \hat{\mathcal{H}}^{(A)}
\left|b\right>\left<b\right| =
\nonumber \\
 = 
\hbar\omega_a
\left|a\right>\left<a\right| + \hbar\omega_b
\left|b\right>\left<b\right| =
\nonumber \\
= \hbar \omega_a \hat{\sigma}_a +
\hbar \omega_b \hat{\sigma}_b,
\label{eqCh2HamiltonianAOperator}
\end{eqnarray}
где введены следующие обозначения
\begin{equation}
\hat{\sigma}_a = \left|a\right>\left<a\right|,
\quad
\hat{\sigma}_b = \left|b\right>\left<b\right|.
\nonumber
\end{equation}
Выражение (\ref{eqCh2HamiltonianAOperator}) может быть переписано в
матричной форме:
\begin{equation}
\hat{\mathcal{H}}^{(A)} = \hbar 
\left(
\begin{array} {cc}
\omega_a & 0  
\\
0 & \omega_b 
\end{array}
\right).
\label{eqCh2HamiltonianA}
\end{equation}

Для гамильтониана взаимодействия $\hat{V}$ при полуклассическом
подходе в дипольном приближении имеем:
\begin{equation}
\hat{V} = - e \left(\hat{\vec{r}} \vec{E}\left(t\right)\right).
\label{eqCh2HamiltonianVsemiclass}
\end{equation}
Применив к (\ref{eqCh2HamiltonianVsemiclass}) дважды условие
полноты получим
\begin{eqnarray}
\hat{V} = \hat{I}\hat{V}\hat{I} = - e \left(\vec{E}\left(t\right)
\hat{I}\hat{\vec{r}}\hat{I}\right) = 
\nonumber \\
= - e \left(\vec{E}\left(t\right) 
\left(\left|a\right>\left<a\right| +
  \left|b\right>\left<b\right|\right)\hat{\vec{r}}
\left(\left|a\right>\left<a\right| +
  \left|b\right>\left<b\right|\right)\right) = 
\nonumber \\
= - \left(\vec{E}\left(t\right)
  \left(\vec{P}_{ab}\left|a\right>\left<b\right| +
    \vec{P}_{ba}\left|b\right>\left<a\right|\right)\right),
\label{eqCh2HamiltonianVsemiclass2}
\end{eqnarray}
где 
\[
\vec{P}_{ab} = \vec{P}_{ba}^{*} = e \left<a\right|\hat{\vec{r}}\left|b\right>
\]
матричный элемент электрического дипольного момента. 
Для простоты будем считать, что $\vec{P}_{ab} = \vec{P}_{ba} =
\vec{p}$ - вещественная величина. В результате выражение
(\ref{eqCh2HamiltonianVsemiclass2}) может быть записано в виде
\begin{eqnarray}
\hat{V} 
= - \left(\vec{E}\left(t\right)\vec{p}\right)
\left(
  \left|a\right>\left<b\right| +
  \left|b\right>\left<a\right|\right) = 
\nonumber \\
= - \left(\vec{E}\left(t\right)\vec{p}\right) \left(\hat{\sigma}^{+} + \hat{\sigma}\right),
\label{eqCh2HamiltonianVsemiclass3}
\end{eqnarray}
где  $\hat{\sigma}^{+} = \left|a\right>\left<b\right|$ и 
$\hat{\sigma} = \left|b\right>\left<a\right|$ 
-  операторы перехода. 
$\hat{\sigma}^{+}$ -  оператор перехода из нижнего состояния в верхнее (повышающий
оператор):
\[
\hat{\sigma}^{+} \left|b\right> = 
\left|a\right>\left<b\right|\left.b\right> = 
\left|a\right>,
\]
$\hat{\sigma}$
- оператор перехода из верхнего состояния в нижнее
(понижающий оператор):
\[
\hat{\sigma}
\left|a\right> = 
\left|b\right>\left<a\right|\left.a\right> = 
\left|b\right>.
\]
Кроме этого выполняются равенства
\[
\hat{\sigma}^{+} \left|a\right> = 
\left|a\right>\left<b\right|\left.a\right> = 
0
\]
и
\[
\hat{\sigma}
\left|b\right> = 
\left|b\right>\left<a\right|\left.b\right> = 
0.
\]
Для операторов $\hat{\sigma}$ и $\hat{\sigma}^{+}$ может быть записано
следующее соотношение: 
\begin{eqnarray}
\hat{\sigma}\hat{\sigma}^{+} + \hat{\sigma}^{+}\hat{\sigma} =
\nonumber \\
= \left|b\right>\left<a\right|\left.a\right>\left<b\right| + 
 \left|a\right>\left<b\right|\left.b\right>\left<a\right| = 
\nonumber \\
= \left|b\right>\left<b\right| + \left|a\right>\left<a\right| = \hat{I}, 
\label{eqCh2_task1}
\end{eqnarray}
где $\hat{I}$ - единичный оператор.

Выражение (\ref{eqCh2HamiltonianVsemiclass3}) может быть также
записано в матричной форме 
\begin{equation}
\hat{V} = - \left(\vec{p} \vec{E}\right)
\left(
\begin{array} {cc}
0 & 1  
\\
1 & 0 
\end{array}
\right).
\label{eqCh2_8}
\end{equation}

Переход от полуклассического гамильтониана взаимодействия к полностью
квантовому выражению осуществляется заменой классического
электрического поля на оператор электрического поля. Таким
образом, получаем 
\begin{equation}
\hat{V} = - p E_1 \sin k z \left(\hat{a} + \hat{a}^{+}\right)
\left(\hat{\sigma} + \hat{\sigma}^{+}\right)
\label{eqCh2_8_add1}
\end{equation}
Здесь использовано простейшее выражение для оператора электрического
поля моды (\ref{eqCh1_EH_simple}). 

Полный гамильтониан системы атом-поле имеет вид
\begin{equation}
\hat{\mathcal{H}}_{AF} = 
\hbar \omega_a \sigma_a + \hbar \omega_b \sigma_b +
\hbar \omega 
\left(\hat{a}^{+}\hat{a} + \frac{1}{2}\right)
+ \hbar g \left(\hat{a} + \hat{a}^{+}\right)
\left(\hat{\sigma} + \hat{\sigma}^{+}\right)
\nonumber
\end{equation}
где $g = -\frac{p}{\hbar}E_1 \sin k z$ - константа взаимодействия,
зависящая от распределения поля моды в пространстве.  

\input ./part1/interaction/fig2_1.tex

\input ./part1/interaction/fig2_2.tex

\input ./part1/interaction/fig2_3.tex

\input ./part1/interaction/fig2_4.tex


Не все члены, входящие в гамильтониан взаимодействия
(\ref{eqCh2_8_add1}), равноправны. При перемножении скобок получаем
члены: 
\begin{enumerate}
\item $\hat{a}\hat{\sigma}$ - этот член соответствует поглощению
  фотона и переходу атома из верхнего состояния в нижнее. Условно этот
  процесс изображен на рис. \ref{figPart1Ch2_2_1};  
\item $\hat{a}\hat{\sigma}^{+}$ - этот член соответствует поглощению
  фотона и переходу атома из нижнего состояния в верхнее (см. рис. \ref{figPart1Ch2_2_2});  
\item $\hat{a}^{+}\hat{\sigma}$ - этот член соответствует излучению
  фотона и переходу атома из верхнего состояния в нижнее (см. рис. \ref{figPart1Ch2_2_3}); 
\item $\hat{a}^{+}\hat{\sigma}^{+}$ - этот член соответствует
  излучению фотона и переходу атома из нижнего состояния в верхнее (см. рис. \ref{figPart1Ch2_2_4}). 
\end{enumerate}


Случаи 1 и 4 относятся к процессам (см. рис. \ref{figPart1Ch2_2_1} и
\ref{figPart1Ch2_2_4}) при которых нарушается закон сохранения энергии.
Эти процессы можно отнести к так называемым виртуальным процессам. Их
вероятность с ростом времени стремится к 0 и
их вкладом в гамильтониан взаимодействия можно
пренебречь. Остаются члены 2 и 3 (см. рис. \ref{figPart1Ch2_2_2} и
\ref{figPart1Ch2_2_3}), соответствующие процессам,
идущим без нарушения закона сохранения энергии. С учетом всего
сказанного гамильтониан взаимодействия принимает вид: 
\begin{equation}
\hat{V} = g \hbar \left(
\hat{a}\hat{\sigma}^{+} + 
\hat{a}^{+}\hat{\sigma}
\right)
\end{equation}

Удобно перейти к представлению взаимодействия(см. прил. \ref{AddWaveFuncInter}),
используя известную 
формулу (\ref{eqAddWaveFunc_VInter})
\begin{equation}
\hat{V}_I = 
e^{i \frac{\hat{\mathcal{H}}_0 t}{\hbar}}
\hat{V}
e^{- i \frac{\hat{\mathcal{H}}_0 t}{\hbar}}
\label{eqCh2_task21}
\end{equation}
где 
\(
\hat{\mathcal{H}}_0 = 
\hat{\mathcal{H}}^{\left(A\right)} +
\hat{\mathcal{H}}^{\left(F\right)}
\)
- невозмущенный гамильтониан системы атом-поле. Можно показать, что в
представлении взаимодействия гамильтониан взаимодействия имеет вид 
\begin{equation}
\hat{V}_I = 
g \hbar \left(
\hat{a}^{+}\hat{\sigma} e^{i \left(\omega - \omega_{ab}\right)t} +
\hat{\sigma}^{+} \hat{a} e^{-i \left(\omega - \omega_{ab}\right)t}
\right).
\label{eqCh2_task22}
\end{equation}
Это проще всего сделать, вычислив матричные элементы 
$\left<n +
1\right|\left<b\right|\hat{V}_I\left|a\right>\left|n\right>$ и
$\left<n\right|\left<a\right|\hat{V}_I\left|b\right>\left|n +
1\right>$ для оператора взаимодействия $\hat{V}_I$, представимого в
виде (\ref{eqCh2_task21}) и (\ref{eqCh2_task22}), и убедиться, что они
одинаковы в обоих случаях.
