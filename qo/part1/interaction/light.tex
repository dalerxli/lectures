%% -*- coding:utf-8 -*- 
\section{Излучение и поглощение атомом света}
Рассмотрим простейшую задачу о взаимодействии одномодового излучения с
двухуровневым атомом. Упрощения оправданы тем, что лазерное излучение
часто можно рассматривать как одномодовое, и при резонансном
взаимодействии можно пренебречь всеми уровнями атома, кроме двух,
частота перехода между которыми близка к частоте взаимодействующей с
атомом моды. Эта простейшая ситуация изображена на рис. \ref{figPart1Ch2_1}

\input ./part1/interaction/fig1.tex

Положим, что атом и поле первоначально находятся в состояниях
\begin{equation}
\left|\psi_A\right> = C_a\left|a\right> + C_b\left|b\right>, \,
\left|\psi_F\right> = \sum_{(n)}C_n\left|n\right>
\end{equation}
где:  $\left|a\right>$, $\left|b\right>$ - векторы соответственно
верхнего и нижнего состояний атома, $\left|n\right>$ вектор состояния
моды, содержащей  $n$  фотонов.
  
Полный вектор состояния системы атом-поле равен
\begin{equation}
\left|\psi_{AF}\right> = \sum_{(n)} 
\left(
C_{an}\left|a\right>\left|n\right> + 
C_{bn}\left|b\right>\left|n\right>
\right).
\end{equation}
Здесь $C_{an}$, $C_{bn}$ - соответствующие амплитуды
вероятностей. Из-за взаимодействия между атомом и полем
амплитуды вероятностей меняются во времени. Например, если
атом первоначально находился на верхнем уровне, а поле моды
содержало  $n$  фотонов, то есть  
$\left|\psi_{AF}\right> = \left|a\right>\left|n\right>$,
то в последующие моменты времени поле будет находиться в состоянии  
\begin{equation}
\left|\psi_{AF}\right> =
C_{an}\left|a\right>\left|n\right> + 
C_{b,n + 1}\left|b\right>\left|n + 1\right>.
\end{equation}

