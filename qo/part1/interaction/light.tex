%% -*- coding:utf-8 -*- 
\section{Излучение и поглощение атомом света}
Рассмотрим простейшую задачу о взаимодействии одномодового излучения с
двухуровневым атомом. Упрощения оправданы тем, что лазерное излучение
часто можно рассматривать как одномодовое, и при резонансном
взаимодействии можно пренебречь всеми уровнями атома, кроме двух,
частота перехода между которыми близка к частоте взаимодействующей с
атомом моды. Эта простейшая ситуация изображена на \autoref{figPart1Ch2_1}

\input ./part1/interaction/fig1.tex

Положим, что атом и поле первоначально находятся в состояниях
\begin{equation}
\left|\psi_A\right> = C_a\ket{a} + C_b\ket{b}, \,
\left|\psi_F\right> = \sum_{(n)}C_n\ket{n}
\end{equation}
где:  $\ket{a}$, $\ket{b}$ - векторы соответственно
верхнего и нижнего состояний атома, $\ket{n}$ вектор состояния
моды, содержащей  $n$  фотонов.
  
Полный вектор состояния системы атом-поле равен
\begin{equation}
\left|\psi_{AF}\right> = \sum_{(n)} 
\left(
C_{an}\ket{a}\ket{n} + 
C_{bn}\ket{b}\ket{n}
\right).
\end{equation}
Здесь $C_{an}$, $C_{bn}$ - соответствующие амплитуды
вероятностей. Из-за взаимодействия между атомом и полем
амплитуды вероятностей меняются во времени. Например, если
атом первоначально находился на верхнем уровне, а поле моды
содержало  $n$  фотонов, то есть  
$\left|\psi_{AF}\right> = \ket{a}\ket{n}$,
то в последующие моменты времени поле будет находиться в состоянии  
\begin{equation}
\left|\psi_{AF}\right> =
C_{an}\ket{a}\ket{n} + 
C_{b,n + 1}\ket{b}\ket{n + 1}.
\label{eq:part1:rabi_solution}
\end{equation}

