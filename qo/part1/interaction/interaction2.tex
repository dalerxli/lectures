%% -*- coding:utf-8 -*- 
\section{Взаимодействие атома с многомодовым полем. Вынужденные и
  спонтанные переходы}
Выше мы установили, что вероятность перехода атома из нижнего в
верхнее состояние при взаимодействии с одной модой поля, находящейся в
состоянии с $n$ фотонами, равна (\ref{eqCh2_prob_C_an}) 
\begin{equation}
\left|C_{a,n}\left(t\right)\right|^2 = 4 g^2 n
  \frac{sin^2\left(\left(\omega - 
  \omega_{ab}\right)t/2\right)} {\left(\omega - 
  \omega_{ab}\right)^2}
\label{eqCh2_prob_C_an_2}
\end{equation}

Для получения вероятности перехода, учитывая взаимодействие со всеми
модами, нужно сложить вероятности (\ref{eqCh2_prob_C_an_2}),
соответствующие каждой моде. Отличие формулы (\ref{eqCh2_prob_C_an_2})
от формулы (\ref{eqCh2_prob_C_an}) связано с тем, что в первом 
случае мы полагали в моде $n + 1$  фотонов, а здесь  $n$  фотонов.   

В случае атома, взаимодействующего с полем свободного пространства
(разложение поля по плоским волнам), константа взаимодействия равна
\cite{bLuisell1972}: 
\begin{equation}
g = - \frac{\left(\vec{p}\vec{e}\right)}{\hbar}
\sqrt{\frac{\hbar \omega}{2 \varepsilon_0 V}}
\end{equation}
где $\vec{p}$ - матричный элемент оператора дипольного момента,
$\vec{e}$ - вектор поляризации поля. 

Если в выражении (\ref{eqCh2_prob_C_an_2}) время $t$ не очень мало (но
существенно меньше, чем характерное время изменения вероятностей),
зависимость $\left|C_{a,n}\left(t\right)\right|^2$ от частоты будет
иметь резкий пик при $\omega = \omega_{ab}$,  что указывает на
сохранение энергии в элементарном акте (энергия поглощенного фотона
равна изменению энергии атома). Положим, что падающее в направлении
$\vec{k}$ в телесном угле $d \Omega$ поле имеет спектр, мало
меняющийся вблизи от частоты $\omega_{ab}$. 
 
Число мод, приходящихся на интервал частот $d \omega$  вблизи
$\omega_{ab}$ и телесный угол $d \Omega$  вокруг направления
$\vec{k}$,  как известно (\ref{eqCh1_modenumber_1}), равно
\begin{equation}
d N = 2 \left(\frac{L}{2 \pi c} \right)^3 \omega^2 d \omega d \Omega
\end{equation}

Вероятность поглощения фотона при взаимодействии атома с одной модой
дается формулой (\ref{eqCh2_prob_C_an_2}). Полная вероятность может
быть получена суммированием по всем модам. Считая спектр мод
квазинепрерывным (см. (\ref{eqCh1_modenumber_kvazy_contig})),
суммирование можно заменить интегрированием
\begin{equation}
W_{b \rightarrow a} = \int_{\Omega} 8 \left|g\right|^2 n(\vec{k})
\left(\frac{L}{2 \pi c}\right)^3 \omega_{ab}^2 d \Omega 
\int_{-\infty}^{+\infty} 
\frac{sin^2\left(\left(\omega - 
  \omega_{ab}\right)t/2\right)} {\left(\omega - 
  \omega_{ab}\right)^2} d \omega.
\label{eqCh2Wba_1}
\end{equation}
Здесь учтено, что $\frac{\sin^2 x}{x^2}$   имеет узкий пик вблизи $x =
0$, при этом в выражении (\ref{eqCh2Wba_1}) мы предполагаем, что $t$ мало по
сравнению с характерным временем изменения $C$, однако, достаточно
велико, чтобы проявились фильтрующие свойства интеграла.
Все
медленно меняющиеся члены взяты при $\omega = \omega_{ab}$ и вынесены
из-под интеграла. Известно, что $\int_{-\infty}^{+\infty} \frac{\sin^2
x}{x^2} dx = \pi$.  Отсюда легко получить  
\[
\int_{-\infty}^{+\infty} \frac{sin^2\left(\left(\omega - 
  \omega_{ab}\right)t/2\right)} {\left(\omega - 
  \omega_{ab}\right)^2} d \omega = \frac{\pi t }{2}.
\]
Таким образом, имеем   
\begin{equation}
W_{b \rightarrow a} = t \frac{\omega^2 2 \pi \omega}
{\hbar \varepsilon_0 \left(2 \pi c\right)^3}\int_{\Omega} 
n(\vec{k})\left|\left(\vec{p} \vec{e}\right)\right|^2
d \Omega.
\label{eqCh2_Wab} 
\end{equation}
Заметим, что в окончательное выражение объем квантования не
вошел, а число фотонов $n$ зависит от направления, т. е. телесного
угла $\Omega$. Формула (\ref{eqCh2_Wab} ) показывает, что вероятность перехода 
пропорциональна времени. Это позволяет ввести понятие скорости
перехода, т.е. вероятности перехода в единицу времени 
\begin{equation}
w_{b \rightarrow a} = \frac{W_{b \rightarrow a}}{t} = \frac{2 \pi
  \omega^3 }  
{\hbar \varepsilon_0 \left(2 \pi c\right)^3}\int_{\Omega} 
n(\vec{k}) \left|\left(\vec{p} \vec{e}\right)\right|^2
d \Omega.
\label{eqCh2_wab} 
\end{equation}
Скорость перехода или скорость поглощения фотона можно выразить
\cite{bLuisell1972} через поток энергии (поток фотонов),
распространяющихся в направлении $\vec{k}$ в  
телесном угле $d \Omega$.  Поток энергии определяется как энергия,
перенесенная через единичную площадь $dS$ в единицу времени $dt$ т. е.
\[
d I = \frac{dH}{dS dt}.
\]
\input ./part1/interaction/fig_add1.tex

Из рис. \ref{figPart1Ch2_add1} имеем, что через площадь $dS$ за единицу
времени $dt$ пройдет столько фотонов, сколько их содержится в цилиндре
изображенном на рис. \ref{figPart1Ch2_add1}, т. е. перенесенная
энергия может быть записана как
\[
dH = n \hbar \omega d N ,
\]
где $dN$ -  число мод, приходящихся на интервал $d \omega d \Omega$ в
 рассматриваемом объеме $c \cdot dS  dt$ (\ref{eqCh1_modenumber_1}):
\[
d N = 2 \left(\frac{1}{2 \pi c} \right)^3 \omega^2 
c \cdot dS  dt
d \omega d \Omega,
\]
откуда имеем 
\begin{equation}
d I = I\left(\omega, \vec{k}\right) d \omega d \Omega = 
\frac{2 n \hbar \omega c}{\left(2 \pi c\right)^3}
\omega^2 d \omega d \Omega,
\label{eqCh2_dI}
\end{equation}
где $I\left(\omega, \vec{k}\right)$ - 
поток энергии фотонов в направлении $\vec{k}$,  приходящийся на единичный
интервал частот и единичный телесный угол. Таким образом из
(\ref{eqCh2_dI}) получим 
\begin{equation}
I\left(\omega, \vec{k}\right) = 
\frac{2 n \hbar \omega^3 c}{\left(2 \pi c\right)^3}
\nonumber
\end{equation}
Подставляя это в выражение (\ref{eqCh2_wab}), получим:
\begin{equation}
w_{ab} = \frac{\pi}{\hbar^2}\sqrt{\frac{\mu_o}{\varepsilon_0}}
\int_{\Omega}I\left(\omega, \vec{k}\right)
\left|\left(\vec{p} \vec{e}\right)\right|^2
d \Omega.
\label{eqCh2_WWab}
\end{equation}
Здесь использовано соотношение $\mu_0 \varepsilon_0 = \frac{1}{c^2}$.

\input ./part1/interaction/fig6.tex

Для того чтобы найти полную скорость поглощения, нужно
проинтегрировать (\ref{eqCh2_WWab}) по всем направлениям
распространения волн. Кроме того, падающее излучение обычно бывает 
неполяризовано. Чтобы это учесть, нужно произвести усреднение
по всем направлениям поляризации. Воспользуемся системой
координат, изображенной на рис. \ref{figPart1Ch2_6}. В качестве полярной оси $z$
выбрано направление $\vec{k}$.  Векторы поляризации $\vec{e}_1$ и $\vec{e}_2$ направлены по $x$ и $y$ соответственно.
Углы $\varphi$ и $\theta$   определяют направление $\vec{p}$. 
Угол $\varphi'$ определяет направление поляризации падающей
волны. Из рисунка видно, что
\[
\left|\left(\vec{p} \vec{e}\right)\right|^2 = 
\left|p\right|^2 \sin^2 \theta \cos^2 \varphi'. 
\]
Усреднение по всем поляризациям дает: 
\begin{equation}
\frac{\left|p\right|^2}{2 \pi} \int_0^{2 \pi}
\cos^2 \varphi' d \varphi' = \frac{\left|p\right|^2}{2}.
\label{eqCh2_PolyarMedian}
\end{equation}

Суммирование по всем направлениям прихода волн приводит к выражению
\begin{equation}
w_{ab} = \frac{\pi}{2 \hbar^2}\sqrt{\frac{\mu_o}{\varepsilon_0}}
\left|p\right|^2
\int_{\Omega}I\left(\omega, \vec{k}\right)
\sin^3 \theta d \theta d \varphi.
\end{equation}
Считая, что излучение приходит со всех направлений и изотропно, 
получим: 
\begin{equation}
w_{ab} = \frac{\pi}{2 \hbar^2}\sqrt{\frac{\mu_o}{\varepsilon_0}}
\left|p\right|^2 I\left(\omega\right)
\int_{0}^{2 \pi}d \varphi \int_0^{\pi}
\sin^3 \theta d \theta = 
\frac{\pi}{ \hbar^2}\sqrt{\frac{\mu_o}{\varepsilon_0}}
\frac{\left|p\right|^2}{3}I_0,
\label{eqCh2_Wab_1}
\end{equation}
где $I_0 = 4 \pi I\left(\omega\right)$ - полный поток энергии,
падающий на атом. При выводе  (\ref{eqCh2_Wab_1})
мы воспользовались следующим соотношением:
\[
 \int_0^{\pi}
\sin^3 \theta d \theta = \frac{4}{3}
\]

Если же излучение приходит из малой области
$\Delta \Omega$ направлений вблизи углов $\theta_0$, $\varphi_0$, то 
можем написать:  
\begin{eqnarray}
w_{ab} = \frac{\pi}{2 \hbar^2}\sqrt{\frac{\mu_o}{\varepsilon_0}}
\left|p\right|^2 \sin^2 \theta_0
\int_{\Delta \Omega} I\left(\theta, \varphi\right)
\delta \Omega = 
\nonumber \\
= 
\frac{\pi}{2 \hbar^2}\sqrt{\frac{\mu_o}{\varepsilon_0}}
\left|p\right|^2  I_0 \sin^2 \theta_0,
\label{eqCh2_Wab_2}
\end{eqnarray}
где $I_0 = \int_{\Delta \Omega} I\left(\theta, \varphi\right)
d \Omega$ -  полный поток, облучающий атом.

Рассмотрим теперь другую задачу. Определим вероятность излучения
фотона или перехода возбужденного атома в нижнее состояние.  

Для вероятности излучения атомом фотона в одну моду мы имели выражение
(\ref{eqCh2_prob_C_bn}):
\[
\left|C_{b, n + 1}\left(t\right)\right|^2 = 4 g^2 \left(n + 1\right)
\frac{sin^2\left(\left(\omega - \omega_{ab}\right)t/2\right)}
{\left(\omega - \omega_{ab}\right)^2}.
\]
Здесь можно рассматривать два слагаемых: с
множителем $n$ и с 
множителем $1$: первое слагаемое
\[
4 g^2 n
\frac{sin^2\left(\left(\omega - \omega_{ab}\right)t/2\right)}
{\left(\omega - \omega_{ab}\right)^2}
\]
соответствует вынужденному излучению, 
второе 
\[
4 g^2 
\frac{sin^2\left(\left(\omega - \omega_{ab}\right)t/2\right)}
{\left(\omega - \omega_{ab}\right)^2}
\]
спонтанному. Член, соответствующий вынужденному
излучению, рассматривается таким же образом, как и в случае поглощения
фотона. Результат будет, очевидно, тем же самым. Соответствующие
формулы будут совпадать с (\ref{eqCh2_Wab_1}), 
(\ref{eqCh2_Wab_2}). Отсюда следует, что вероятности
вынужденных процессов поглощения и излучения равны между
собой. Вероятность спонтанного излучения в единицу времени, очевидно, 
будет равна 
\begin{equation}
w_{\mbox{спон.}} = 
\frac{2 \pi \omega^3}
{\hbar \varepsilon_0 \left(2 \pi c\right)^3}
\int_{\Omega}
\left|\left(\vec{p} \vec{e}\right)\right|^2
d \Omega
\label{eqCh2_Wspon}
\end{equation}
Для того чтобы получить полную вероятность, нужно (\ref{eqCh2_Wspon})
проинтегрировать по всем направлениям, так как спонтанное излучение
может происходить в любую моду. Для поляризаций - воспользуемся
средним значением (\ref{eqCh2_PolyarMedian}).Таким образом вся
процедура сведется к вычислению интеграла 
\begin{equation}
w_{\mbox{спон.}} = 
\frac{\left|p\right|^2 \omega^3}
{\varepsilon_0 \hbar \left(2 \pi\right)^2 c^3}
\int_{0}^{2 \pi}d \varphi \int_0^{\pi}
\sin^3 \theta d \theta
= 
\frac{4 \pi}{3}\frac{\left|p\right|^2 \omega^3}
{\hbar \varepsilon_0 \left(2 \pi\right)^2 c^3}
\end{equation}
Окончательно получаем
\begin{equation}
w_{\mbox{спон.}} = 
\frac{\left|p\right|^2 \omega^3}
{3 \pi c^2 \hbar}
\sqrt{\frac{\mu_0}{\varepsilon_0}}
\label{eqCh2_Wspon_final}
\end{equation}
Здесь использовано соотношение  
\[
\frac{1}{c} = \sqrt{\mu_0 \varepsilon_0}
\]
Формула (\ref{eqCh2_Wspon_final}) указывает на сильную
зависимость вероятности спонтанного перехода от частоты, как  
$\omega^3$.  Таким
образом, исходя из уравнений квантовой электродинамики, мы
непосредственно, не привлекая посторонних соображений, получили
выражения для вероятностей вынужденных и спонтанных переходов атомов в
единицу времени. Недостатком рассмотрения является использование
теории возмущения и, следовательно, необходимость ограничиваться
малыми временами. Решая, например, задачу о времени жизни
возбужденного атома, нельзя уже ограничиваться малыми временами. В
этом случае применяется другое приближение, называемое приближением
Вайскопфа-Вигнера \cite{bLuisell1972}. Интересно, что результаты,
полученные в этих двух случаях, не противоречат друг другу.  

