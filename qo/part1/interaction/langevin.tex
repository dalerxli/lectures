%% -*- coding:utf-8 -*- 
\section{Затухание моды резонатора. Уравнение Гейзенберга - Ланжевена.}

\subsection{Постановка задачи.}
Задачу о релаксации поля (моды) резонатора мы рассматриваем, используя
метод матрицы плотности, т. е. представление Шредингера. Существует
другой подход, принадлежащий Ланжевену, в котором уравнения
динамической системы для учета влияния диссипативной системы
дополняются случайными силами, статистические свойства которых
известны. В квантовом случае динамическими уравнениями являются
уравнения Гейзенберга для операторов наблюдаемых величин, а случайные
воздействия, связанные с термостатом, учитываются добавлением некого
шумового оператора. Получаемые уравнения носят названия уравнений
Гейзенберга-Ланжевена.

Рассмотрим таким методом задачу о релаксации моды резонатора
(гармонического осциллятора), взаимодействующей с термостатом. В
представлении Гейзенберга операторы зависят от времени и удовлетворяют
уравнениям Гейзенберга:
\begin{equation}
\frac{d \hat{O}}{dt} = \frac{i}{\hbar}
\left[\hat{\mathcal{H}}, \hat{O}\right],
\nonumber
\end{equation}
\index{Гамильтониан}
где $\hat{\mathcal{H}}$ - гамильтониан рассматриваемой
системы. Например, для для свободного поля ранее мы получили уравнения
\eqref{eqCh1_54}:
\begin{equation}
\frac{d \hat{a}}{d t} = \frac{i}{\hbar}\left[\hat{\mathcal{H}},
  \hat{a}\right] = -i \omega \hat{a}, \quad
\frac{d \hat{a}^{\dag}}{dt} = \frac{i}{\hbar}\left[\hat{\mathcal{H}},
  \hat{a}^{\dag}\right] = i \omega \hat{a}^{+},
\nonumber
\end{equation}
где $\omega$ - частота моды, а $\hat{a}$ и $\hat{a}^{\dag}$ операторы
уничтожения и рождения.

\input ./part1/interaction/fig_add3.tex

Рассматриваемая задача схематически изображена на
\autoref{figPart1Ch2_add3}. Мода электромагнитного поля,
рассматриваемая как динамическая система, взаимодействует с обширным
резервуаром, состоящим из большого числа гармонических осцилляторов
(фононов) и находящимся в равновесии при температуре $T$. Гамильтониан
такой системы можно представить в виде
\begin{equation}
\hat{\mathcal{H}} = \hat{\mathcal{H}}_0 + \hat{V},
\nonumber
\end{equation}
где
\begin{equation}
 \hat{\mathcal{H}}_0 = \hbar\omega \hat{a}^{\dag}\hat{a} +
\sum_{(k)} \hbar\omega_k \hat{b}_k^{\dag} \hat{b}_k
\nonumber
\end{equation}
гамильтониан системы без учета взаимодействия между ее частями,
\begin{equation}
 \hat{V} = \hbar \sum_{(k)}g_k\left(
\hat{b}_k^{\dag}\hat{a} + \hat{a}^{+}\hat{b}_k
\right)
\label{eqPart1Ch2_LanzgevenV}
\end{equation}
гамильтониан взаимодействия между динамической системой и
термостатом. Через $g_k$ в \eqref{eqPart1Ch2_LanzgevenV} обозначена
постоянная взаимодействия для $k$-ой моды резервуара. $\hat{a}$,
$\hat{a}^{\dag}$ - операторы уничтожения и рождения для поля моды,
$\hat{b}_k$, $\hat{b}_{k}^{\dag}$ - операторы уничтожения и рождения для
$k$-ой моды резервуара (фононов). Вакуумные члены в гамильтониане не
учтены, т. к. в ходе вывода уравнений они взаимно сокращаются.

\subsection{Уравнения движения Гейзенберга для операторов.}
Уравнения движения Гейзенберга для $\hat{a}$ и $\hat{b}_{k}$
\begin{eqnarray}
\frac{d \hat{a}}{d t} = \frac{i}{\hbar}\left[
 \hat{\mathcal{H}}, \hat{a}
\right],
\nonumber \\
\frac{d \hat{b}_k}{d t} = \frac{i}{\hbar}\left[
 \hat{\mathcal{H}}, \hat{b}_k
\right]
\nonumber
\end{eqnarray}
при помощи коммутационных соотношений 
$\left[\hat{a}, \hat{a}^{\dag}\right] = 1$ и 
$\left[\hat{b}_k, \hat{b}_{k}^{\dag}\right] = 1$ приводятся к виду
\begin{eqnarray}
\frac{d \hat{a}\left(t\right)}{d t} = -i \omega \hat{a}\left(t\right) - i\sum_{(k)}g_k
\hat{b}_k\left(t\right),
\nonumber \\
\frac{d \hat{b}_k\left(t\right)}{d t} = -i \omega_k
\hat{b}_{k}\left(t\right) - i g_k \hat{a}\left(t\right).
\label{eqPart1Ch2_LanzgevenAB}
\end{eqnarray}
Из системы \eqref{eqPart1Ch2_LanzgevenAB} можно исключить $\hat{b}_k$,
проинтегрировав второе уравнение и подставив результат в первое
уравнение. Интегрирование проще всего выполнить, используя операционное
исчисление. Получаем
\begin{equation}
\hat{b}_k\left(t\right) = 
\hat{b}_k\left(0\right) e^{-i \omega_k t} 
- i g_k \int_0^t d t' \hat{a}\left(t'\right)e^{-i \omega_k\left(t - t'\right)}
\nonumber
\end{equation}
и далее
\begin{eqnarray}
\frac{d \hat{a}\left(t\right)}{dt} = 
- i \omega \hat{a}\left(t\right) - \sum_{(k)} g_k^2 \int_0^t
d t'  \hat{a}\left(t'\right)e^{-i \omega_k\left(t - t'\right)}
+ \hat{f}\left(t\right),
\nonumber \\
\hat{f}\left(t\right) = -i \sum_{(k)} g_k \hat{b}_k\left(0\right)
e^{-i \omega_k t}.
\label{eqPart1Ch2_LanzgevenA}
\end{eqnarray}
Оператор $\hat{f}\left(t\right)$ зависит от переменных термостата и
является шумовым оператором, описывающим воздействие резервуара на
динамическую систему.

Уравнение \eqref{eqPart1Ch2_LanzgevenA} интегродифференциальное. при
некоторых приближениях его можно преобразовать в чисто
дифференциальное. Сперва перейдем к медленным переменным, записав
\begin{eqnarray}
\hat{a}\left(t\right) = \hat{A}\left(t\right)e^{-i \omega t},
\nonumber \\
\hat{f}\left(t\right) = \hat{F}\left(t\right)e^{-i \omega t},
\nonumber
\end{eqnarray}
где $\omega$ - частота моды. Отсюда имеем
\begin{equation}
\frac{d \hat{A}\left(t\right)}{dt} = 
- \sum_{(k)} g_k^2 \int_0^t
d t'  \hat{A}\left(t'\right)e^{i \left(\omega -
  \omega_k\right)\left(t - t'\right)} 
+ \hat{F}\left(t\right).
\label{eqPart1Ch2_LanzgevenAA}
\end{equation}
% где 
% $\hat{F}\left(t\right) = \hat{f}\left(t\right) e^{i \omega t}$. 
Первый член выражения \eqref{eqPart1Ch2_LanzgevenAA} имеет вид, с которым мы
сталкивались рассматривая спонтанное излучение методом
Вайскопфа-Вигнера. Примененный там подход можно применить и здесь. Из
выражения \eqref{eqPart1Ch2_LanzgevenAA} видно, что основной вклад
дадут члены с частотами, близкими к частоте моды: $\omega_k \approx
\omega$. По этой причине все медленно меняющиеся члены можно взять при
частоте $\omega$ и вынести их из под интеграла. Суммирование по $k$,
считая спектр фононов квазинепрерывным, можно заменить интегрированием
по частоте:
\begin{eqnarray}
\sum_{(k)}g_k^2\int_0^t d t' \hat{A}\left(t'\right) e^{i\left(\omega -
  \omega_k\right)\left(t - t'\right)} = 
\nonumber \\
=
\int_0^t d t'
  \hat{A}\left(t'\right)\int_0^{\infty}g^2\left(\omega\right) 
D\left(\omega\right) e^{i\left(\omega -
  \omega'\right)\left(t - t'\right)} d \omega' = 
\nonumber \\
= g^2\left(\omega\right) 
D\left(\omega\right)
\int_0^t d t'
  \hat{A}\left(t'\right)
\int_0^{\infty}e^{i\left(\omega -
  \omega'\right)\left(t - t'\right)} d \omega',
\nonumber
\end{eqnarray}
где $D\left(\omega\right)$ частотная плотность состояний.
$D\left(\omega\right)$ и $g^2\left(\omega\right)$ взяты при частоте
моды поля и вынесены из под знака интеграла.

Рассмотрим интеграл по частоте. Действуя так же как при рассмотрении
спонтанного излучения методом Вайскопфа-Вигнера, получим
\begin{equation}
\int_0^{\infty}e^{- i\left(\omega' -
  \omega\right)\left(t - t'\right)} d \omega' = 
\int_{-\omega}^{\infty} e^{-i \nu \left(t - t'\right)} d \nu \approx 
\int_{-\infty}^{\infty} e^{-i \nu \left(t - t'\right)} d \nu = 
2 \pi \delta\left(t - t'\right).
\nonumber
\end{equation}
Следовательно можно написать
\begin{eqnarray}
\frac{d \hat{A}\left(t\right)}{d t} = 
- 2 \pi g^2\left(\omega\right) 
D\left(\omega\right)
\int_0^t d t'
  \hat{A}\left(t'\right)
\delta\left(t - t'\right)  + \hat{F}\left(t\right) = 
\nonumber \\
=
- 2 \pi g^2\left(\omega\right) 
D\left(\omega\right) \hat{A}\left(t\right)
 + \hat{F}\left(t\right).
\nonumber
\end{eqnarray}
 Обозначим 
\(
\left.2 \pi D g^2\right|_{\omega} = \frac{\gamma}{2}
\) - коэффициент затухания поля. Получим
\begin{equation}
\frac{d \hat{A}\left(t\right)}{d t} = 
- \frac{\gamma}{2} \hat{A}\left(t\right) + 
\hat{F}\left(t\right),
\label{eqPart1Ch2_LanzgevenAeq}
\end{equation}
где $\hat{F}\left(t\right)$ случайный шумовой оператор, свойства
которого необходимо определить. Как будет видно из дальнейшего,
шумовой оператор обеспечивает сохранение коммутационных соотношений для
оператора моды поля.

Изучим теперь статистические свойства, характеризуемые корреляционными
функциями $\hat{F}$. Предположим, что резервуар находится в тепловом
равновесии при температуре $T$. Тогда имеем:
\begin{eqnarray}
\left<\hat{b}_k\left(0\right)\right>_R = 
\left<\hat{b}_k^{\dag}\left(0\right)\right>_R = 0, 
\nonumber \\
\left<\hat{b}_k^{\dag}\left(0\right)\hat{b}_{k'}\left(0\right)\right>_R = 
\bar{n}_k \delta_{k k'},
\nonumber \\
\left<\hat{b}_k\left(0\right)\hat{b}_{k'}^{\dag}\left(0\right)\right>_R = 
\left(\bar{n}_k + 1\right)\delta_{k k'},
\nonumber \\
\left<\hat{b}_k\left(0\right)\hat{b}_{k'}\left(0\right)\right>_R = 
\left<\hat{b}_k^{\dag}\left(0\right)\hat{b}_{k'}^{+}\left(0\right)\right>_R
= 0,
\label{eqPart1Ch2_LanzgevenPropertyF}
\end{eqnarray}
где $\left<\dots\right>_R$ означает усреднение по переменным
термостата, а $\bar{n}_k$ - среднее число фононов в моде $k$. Шумовой
оператор был определен формулой 
\begin{equation}
\hat{F}\left(t\right) = -i \sum_{(k)}g_k 
\hat{b}_k\left(0\right)
e^{-i \left(\omega_k - \omega\right)t}.
\label{eqPart1Ch2_LanzgevenDefenitionF}
\end{equation}
Используя \eqref{eqPart1Ch2_LanzgevenPropertyF}, легко видеть
\begin{eqnarray}
\left<\hat{F}\left(t\right)\right>_R =
\left<\hat{F}^{\dag}\left(t\right)\right>_R  = 0,
\nonumber \\
\left<\hat{F}^{\dag}\left(t\right)\hat{F}\left(t'\right)\right>_R = 
\nonumber \\
= \sum_{k}\sum_{k'}g_k g_{k'}
\left<\hat{b}_k^{\dag}\left(0\right)\hat{b}_{k'}\left(0\right)\right>_R 
e^{i\left(\omega_k - \omega\right)t} 
 e^{-i\left(\omega_{k'} - \omega\right)t'} =
\nonumber \\
= \sum_{k}g_k^2 \bar{n}_k 
e^{i\left(\omega_k - \omega\right)\left(t - t'\right)} = 
\nonumber \\
=
\int_0^\infty
g^2\left(\omega\right)D\left(\omega\right)\bar{n}_T
e^{i\left(\omega' - \omega\right)\left(t - t'\right)}d \omega' = 
\nonumber \\
= 2 \pi
g^2\left(\omega\right)D\left(\omega\right)\bar{n}_T
\delta\left(t - t'\right) =
\frac{\gamma \bar{n}_{T}}{2} \delta\left(t - t'\right),
\label{eqPart1Ch2_LanzgevenCorrelations}
\end{eqnarray}
где $\bar{n}_T$ среднее число фононов в моде резервуара при
температуре $T$, определяемое формулой Планка,
$\gamma = 4 \pi g^2\left(\omega\right)D\left(\omega\right)$.
Если согласно Ланжевену ввести коэффициент диффузии $\mathcal{D} =
\frac{\gamma \bar{n}_{T}}{2}$, то последнее уравнение системы
\eqref{eqPart1Ch2_LanzgevenCorrelations} можно записать в виде
\begin{equation}
\left<\hat{F}^{\dag}\left(t\right)\hat{F}\left(t'\right)\right>_R = 
\mathcal{D} \delta\left(t - t'\right). 
\nonumber
\end{equation}
Заметим что $\left<\dots\right>_R$ означает усреднение по переменным
термостата, следовательно результатом усреднения будет корреляционная
функция, зависящая только от переменных динамической системы (поля).

Подобным образом доказывается
\begin{equation}
\left<\hat{F}\left(t\right)\hat{F}^{\dag}\left(t'\right)\right>_R = 
\frac{\gamma\left(\bar{n}_{T} + 1\right)}{2} \delta\left(t - t'\right)
\label{eqPart1Ch2_Lanzgeven_Task1}
\end{equation}
и
\begin{equation}
\left<\hat{F}\left(t\right)\hat{F}\left(t'\right)\right>_R = 
\left<\hat{F}^{\dag}\left(t\right)\hat{F}^{+}\left(t'\right)\right>_R = 0.
\label{eqPart1Ch2_Lanzgeven_Task2}
\end{equation}
Таким образом, в нашем приближении, шум является
$\delta$-коррелированым.

Нам еще понадобится корреляционная функция вида
\(
\left<\hat{F}^{\dag}\left(t\right)\hat{A}\left(t'\right)\right>_R 
\) 
- необходимая для вывода уравнения, которому удовлетворяет 
\(
\left<\hat{A}^{\dag}\left(t\right)\hat{A}\left(t'\right)\right>_R 
\). Для этого проинтегрируем уравнение для $\hat{A}\left(t\right)$
\eqref{eqPart1Ch2_LanzgevenAeq} и в результате при помощи
операционного исчисления получаем
\begin{equation} 
\hat{A}\left(t\right) = \hat{A}\left(0\right)e^{-\frac{\gamma}{2}t} +
\int_0^t d t' e^{-\frac{\gamma}{2}\left(t - t'\right)} \hat{F}\left(t'\right).
\nonumber
\end{equation} 
Помножим это выражение на $\hat{F}^{\dag}\left(t\right)$ и усредним по
термостату. Получим
\begin{eqnarray} 
\left<\hat{F}^{\dag}\left(t\right)\hat{A}\left(t\right)\right>_R = 
\int_0^t  
\left<\hat{F}^{\dag}\left(t\right)\hat{F}\left(t'\right)\right>_R 
e^{-\frac{\gamma}{2}\left(t - t'\right)}
d t' +
\nonumber \\
+ \left<\hat{F}^{\dag}\left(t\right)\hat{A}\left(0\right)\right>_R e^{-\frac{\gamma}{2}t}.
\label{eqPart1Ch2_LanzgevenFACorr}
\end{eqnarray} 
Последний член в \eqref{eqPart1Ch2_LanzgevenFACorr} равен $0$, т. к. 
$\hat{F}^{\dag}\left(t\right)$ и $\hat{A}\left(0\right)$ независимы,
%\footnote{$\hat{F}^{\dag}\left(t\right)$ в более поздний момент времени
% не может влиять на $\hat{A}\left(0\right)$ в начальный момент
% времени} 
 а их средние равны $0$. Принимая во внимание второе уравнение системы 
\eqref{eqPart1Ch2_LanzgevenCorrelations} т. е.
\(
\left<\hat{F}^{\dag}\left(t\right)\hat{F}\left(t'\right)\right>_R = 
\frac{\gamma \bar{n}_{T}}{2} \delta\left(t - t'\right),
\)
получим
\begin{equation}
\left<\hat{F}^{\dag}\left(t\right)\hat{A}\left(t\right)\right>_R = 
\frac{\gamma \bar{n}_{T}}{2}.
\nonumber
\end{equation}
Теперь можно рассмотреть корреляционную функцию 
\(
\left<\hat{A}^{\dag}\left(t\right)\hat{A}\left(t\right)\right>_R
\). Дифференцируя эту функцию по времени, получим
\begin{equation}
\frac{d \left<\hat{A}^{\dag}\left(t\right)\hat{A}\left(t\right)\right>_R}{d
t}=
\left<\frac{d \hat{A}^{\dag}\left(t\right)\hat{A}\left(t\right)}{d
t}\right>_R =
\left<\frac{d \hat{A}^{\dag}\left(t\right)}{d
t} \hat{A}\left(t\right) \right>_R + 
\left<\hat{A}^{\dag}\left(t\right)\frac{d \hat{A}\left(t\right)}{d
t}  \right>_R.
\label{eqPart1Ch2_Lanzgeven_dAA}
\end{equation}
Далее воспользуемся уравнением \eqref{eqPart1Ch2_LanzgevenAeq} и
сопряженным ему, в результате получим
\begin{equation}
\frac{d \left<\hat{A}^{\dag}\left(t\right)\hat{A}\left(t\right)\right>_R}{d
t}=
- \gamma \left<\hat{A}^{\dag}\left(t\right)\hat{A}\left(t\right)\right>_R
+ \left<\hat{F}^{\dag}\left(t\right)\hat{A}\left(t\right)\right>_R + 
\left<\hat{A}^{\dag}\left(t\right)\hat{F}\left(t\right)\right>_R.
\nonumber
\end{equation}
Учитывая выражения 
\begin{equation}
\left<\hat{F}^{\dag}\left(t\right)\hat{A}\left(t\right)\right>_R =
\left<\hat{A}^{\dag}\left(t\right)\hat{F}\left(t\right)\right>_R = 
\frac{\gamma \bar{n}_T}{2},
\nonumber
\end{equation}
окончательно имеем
\begin{equation}
\frac{d \left<\hat{A}^{\dag}\left(t\right)\hat{A}\left(t\right)\right>_R}{d
t}=
- \gamma \left<\hat{A}^{\dag}\left(t\right)\hat{A}\left(t\right)\right>_R
+ \gamma \bar{n}_T.
\nonumber
\end{equation}
Подобным образом можно получить
\begin{equation}
\frac{d \left<\hat{A}\left(t\right)\hat{A}^{\dag}\left(t\right)\right>_R}{d
t}=
- \gamma \left<\hat{A}\left(t\right)\hat{A}^{\dag}\left(t\right)\right>_R
+ \gamma \left(\bar{n}_T + 1\right).
\label{eqPart1Ch2_Lanzgeven_Task3}
\end{equation}
Вычитая эти выражения одно из другого, получим для коммутатора
\begin{equation}
\frac{d \left<\left[\hat{A}\left(t\right),\hat{A}^{\dag}\left(t\right)\right]\right>_R}{d
t}=
\gamma \left( 1 - \left<\left[\hat{A}\left(t\right),\hat{A}^{\dag}\left(t\right)\right]\right>_R
\right).
\label{eqPart1Ch2_Lanzgeven_CommutatorAA}
\end{equation}
Из уравнения \eqref{eqPart1Ch2_Lanzgeven_CommutatorAA} следует, что
если в начальный момент времени 
\[
\left<\left[\hat{A}\left(0\right),\hat{A}^{\dag}\left(0\right)\right]\right>_R
= 1,
\]
т. е. коммутационные соотношения выполняются, то эти соотношения будут
выполняться всегда. Другими словами, коммутационные соотношения,
усредненные по переменным термостата, сохраняются со временем. 

Картина была бы совершенно иная, если бы мы не принимали во внимание
квантовые шумы. В этом случае уравнение
\eqref{eqPart1Ch2_Lanzgeven_CommutatorAA} приняло бы вид
\begin{equation}
\frac{d \left<\left[\hat{A}\left(t\right),\hat{A}^{\dag}\left(t\right)\right]\right>_R}{d
t}=
- \gamma \left<\left[\hat{A}\left(t\right),\hat{A}^{\dag}\left(t\right)\right]\right>_R,
\nonumber
\end{equation}
решение которого имеет вид
\begin{equation}
\left<\left[\hat{A}\left(t\right),\hat{A}^{\dag}\left(t\right)\right]\right>_R
= 
\left<\left[\hat{A}\left(0\right),\hat{A}^{\dag}\left(0\right)\right]\right>_R
e^{- \gamma t}= e^{- \gamma t}.
\nonumber
\end{equation}
Это означает, что при достаточно большом времени операторы $\hat{A}$ и
$\hat{A}^{\dag}$ коммутируют, чего на самом деле не наблюдается. По этой
причине учет квантового шумового члена обязателен.


Уравнение движения усредненного по переменным термостата поля легко
получить из \eqref{eqPart1Ch2_LanzgevenAeq}. Для этого вначале запишем
усредненное поле
\begin{eqnarray}
\left<\hat{E}\left(t\right)\right>_R 
= \left< E_0 sin\,kz \left( \hat{a}\left(t\right) +
\hat{a}^{\dag}\left(t\right)\right)\right>_R  = 
\nonumber \\
= E_0 sin\,kz
\left(\left<\hat{A}\left(t\right)\right>_R e^{-i \omega t}
+
\left<\hat{A}^{\dag}\left(t\right)\right>_R e^{i \omega t}
\right),
\nonumber
\end{eqnarray}
где учтено $\hat{a} = \hat{A}e^{-i \omega t}$. Из
\eqref{eqPart1Ch2_LanzgevenAeq} имеем 
\begin{equation}
\frac{d \left<\hat{A}\left(t\right)\right>_R}{d t} = 
- \frac{\gamma}{2} \left<\hat{A}\left(t\right)\right>_R,
\nonumber
\end{equation}
т. к. $\left<\hat{F}\left(t\right)\right>_R = 0$. Таким образом со
временем $\left<\hat{A}\left(t\right)\right>_R$ и
$\left<\hat{A}^{\dag}\left(t\right)\right>_R$ а вместе с ними и  
$\left<\hat{E}\left(t\right)\right>_R$ стремятся к нулю:
\begin{equation}
\left<\hat{E}\left(t\right)\right>_R \rightarrow 0.
\nonumber
\end{equation}

\subsection{Флуктационно-диссипационная формула}
Нами было получено выражение \eqref{eqPart1Ch2_LanzgevenCorrelations}:
\begin{equation}
\left<\hat{F}^{\dag}\left(t\right)\hat{F}\left(t'\right)\right>_R = 
\frac{\gamma \bar{n}_{T}}{2} \delta\left(t - t'\right).
\nonumber
\end{equation}
Интегрируя обе части по $d t'$, получим
\begin{equation}
\gamma =
\frac{2}{\bar{n}_T}\int_0^{\infty}\left<\hat{F}^{\dag}\left(t\right)\hat{F}\left(t'\right)\right>_R
d t',
\label{eqPart1Ch2_LanzgevenGamma}
\end{equation}
т. е. скорость затухания и шумовые флуктуации связаны между собой
соотношением \eqref{eqPart1Ch2_LanzgevenGamma}. Большему затуханию
соответствуют большие шумы.
