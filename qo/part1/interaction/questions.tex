%% -*- coding:utf-8 -*-
\section{Упражнения}
\begin{enumerate}
\item Исходя из \eqref{eqCh2_task21}, получить выражение
  \eqref{eqCh2_task22} для оператора $\hat{V}$ в представлении
  взаимодействия. 
\item Вывести систему уравнений \eqref{eqCh2_task3}.
\item Получить формулу \eqref{eqCh2_task4}.
\item Вывести из \eqref{eqCh2_rho_final2} уравнение для матрицы
  плотности в представлении чисел заполнения \eqref{eqCh2_task5}. 
\item Привести формулу \eqref{eqCh2_66} к виду \eqref{eqCh2_69}.
\item Интегрируя по частям, получить выражения \eqref{eqCh2_70},
  \eqref{eqCh2_71}.
\item Записать общее уравнение \eqref{eqCh2_73} в полярной системе
  координат \eqref{eqCh2_77}. 
\item Записать общее уравнение \eqref{eqCh2_73} в прямоугольной
  системе координат \eqref{eqCh2_77a}. 
%\item Получить из \eqref{eqCh2_75} уравнение \eqref{eqCh2_76}.
\item Вывести соотношения \eqref{eqCh2_96} и \eqref{eqCh2_96_add}.
\item Вывести уравнение \eqref{eqCh2_97}.
\item Доказать \eqref{eqPart1Ch2_Lanzgeven_Task1} и \eqref{eqPart1Ch2_Lanzgeven_Task2}.
\item Получить выражение \eqref{eqPart1Ch2_Lanzgeven_Task3}.
\item \label{qInteractionFreq} Какова частота перехода $f_0 = \frac{\omega_o}{2 \pi}$ между двумя
  состояниями атома лития если в результате экспериментов были
  получены данные, изображенные на
  \autoref{figPart1InteractionQuestionFreq} \footnote{Взято из
    заданий \cite{courseIntroQuantumOpticsCoursera}} 
\end{enumerate}

\input ./part1/interaction/figfreq.tex 
