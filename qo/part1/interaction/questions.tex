%% -*- coding:utf-8 -*-
\section{Упражнения}
\begin{enumerate}
\item Исходя из (\ref{eqCh2_task21}), получить выражение
  (\ref{eqCh2_task22}) для оператора $\hat{V}$ в представлении
  взаимодействия. 
\item Вывести систему уравнений (\ref{eqCh2_task3}).
\item Получить формулу (\ref{eqCh2_task4}).
\item Вывести из (\ref{eqCh2_rho_final2}) уравнение для матрицы
  плотности в представлении чисел заполнения (\ref{eqCh2_task5}). 
\item Привести формулу (\ref{eqCh2_66}) к виду (\ref{eqCh2_69}).
\item Интегрируя по частям, получить выражения (\ref{eqCh2_70}),
  (\ref{eqCh2_71}).
\item Записать общее уравнение (\ref{eqCh2_73}) в полярной системе
  координат (\ref{eqCh2_77}). 
\item Записать общее уравнение (\ref{eqCh2_73}) в прямоугольной
  системе координат (\ref{eqCh2_77a}). 
%\item Получить из (\ref{eqCh2_75}) уравнение (\ref{eqCh2_76}).
\item Вывести соотношения (\ref{eqCh2_96}) и (\ref{eqCh2_96_add}).
\item Вывести уравнение (\ref{eqCh2_97}).
\item Доказать (\ref{eqPart1Ch2_Lanzgeven_Task1}) и (\ref{eqPart1Ch2_Lanzgeven_Task2}).
\item Получить выражение (\ref{eqPart1Ch2_Lanzgeven_Task3}).
\item Вы хотите измерить частоту перехода $f_0 = \frac{\omega_o}{2
  \pi}$ между двумя состояниями атома лития посредством оценки
  осцилляций Раби. Вы приготавливаете холодное облако атомов лития в
  состоянии $\left|1\right>$ (так называемое $\left|F=1/2\right>$
  состояние) и вы можете применять электромагнитный импульс
  фиксированной интенсивности и изменяемой длительности
  $\tau$. Окончательно вы измеряете число атомов в состоянии
  $\left|2\right>$ (так называемое $\left|F=3/2\right>$
  состояние).

  Вы знаете, что разность частот между этими двумя уровнями близка к
  $2 \pi \cdot 228.5 \mbox{МГц}$, следовательно вы применяете
  электромагнитный импульс с частотой близкой к этой величине, вначале
  с частотой $\omega_1 = 2 \pi \cdot 228.5 \mbox{МГц}$, потом вы повторяете
  измерения с частотой $\omega_1 = 2 \pi \cdot 228.4 \mbox{МГц}$. Результат
  изображен на рис. \ref{figPart1InteractionQuestionFreq}. Какова
  частота перехода $f_0 = \frac{\omega_o}{2 \pi}$ между двумя
  состояниями атома лития. 
\end{enumerate}

\input ./part1/interaction/figfreq.tex
