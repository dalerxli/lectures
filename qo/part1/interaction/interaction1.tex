%% -*- coding:utf-8 -*- 
\section{Взаимодействие атома с модой электромагнитного поля}
Уравнение Шредингера в представлении взаимодействия имеет вид
\begin{equation}
i \hbar \frac{\partial}{\partial t} \left|\psi\left(r, t\right)\right>_I
= \hat{V}_I \left|\psi\left(r, t\right)\right>_I,
\label{eqCh2_Shredinger_inter}
\end{equation}
 где через 
\(
\left|\psi\left(r, t\right)\right>_I
\)
записана волновая функция в представлении взаимодействия:
\begin{equation}
\left|\psi\left(r, t\right)\right>_I = e^{i \frac{\hat{\mathcal{H}}_0
    t}{\hbar}}
\left|\psi\left(r, t\right)\right>.
\label{eqCh2_Psi_inter}
\end{equation}
Решим это уравнение для начального состояния 
$\left|\psi\left(r, 0\right)\right>_I = \left|b, n + 1\right>_I$ , т. е. 
будем считать, что в начальный момент атом находится в нижнем
состоянии, а поле содержит  $n + 1$  фотонов. Другой случай
соответствует начальному состоянию системы  
$\left|a, n\right>_I$,  т.е. атом находится в верхнем
состоянии, а поле моды содержит  $n$  фотонов. Через некоторое время в 
результате взаимодействия атома и поля система будет находиться в
суперпозиционном состоянии  
\begin{equation}
\left|\psi\left(r, t\right)\right>_I = 
C_{a,n}\left(t\right)_I\left|a, n\right>_I +
C_{b,n+1}\left(t\right)_I\left|b, n + 1\right>_I.
\label{eqCh2_OurPsi_inter}
\end{equation}

Воспользовавшись (\ref{eqCh2_Psi_inter}) в левой части выражения
(\ref{eqCh2_OurPsi_inter}) получим 
\begin{eqnarray}
C_{a,n}\left(t\right)_I\left|a, n\right>_I +
C_{b,n+1}\left(t\right)_I\left|b, n + 1\right>_I =
\nonumber \\
= C_{a,n}\left(t\right)_I e^{i \frac{\hat{\mathcal{H}}_0
    t}{\hbar}} \left|a, n\right> +
C_{b,n+1}\left(t\right)_I e^{i \frac{\hat{\mathcal{H}}_0
    t}{\hbar}} \left|b, n + 1\right> = 
\nonumber \\
=
C_{a,n}\left(t\right)_I exp\left\{i \left(
 \omega_a +  \omega 
\left( n + \frac{1}{2}\right) 
\right) t
\right\} \left|a, n\right> +
\nonumber \\
+
C_{b,n+1}\left(t\right)_I exp\left\{i 
\left(
\omega_b +  \omega 
\left( n + 1 + \frac{1}{2}\right) 
\right)
    t \right\} \left|b, n + 1\right>
\nonumber \\
=
C_{a,n}\left(t\right) \left|a, n\right> +
C_{b,n+1}\left(t\right) \left|b, n + 1\right>,
\nonumber
\end{eqnarray}
где через $C_{a,n}\left(t\right)$ и $C_{b,n+1}\left(t\right)$
обозначено 
\begin{eqnarray}
C_{a,n}\left(t\right) = 
C_{a,n}\left(t\right)_I exp\left\{i 
\left(
\omega_a +  \omega 
\left( n + \frac{1}{2}\right) 
\right) 
    t \right\},
\nonumber \\
C_{b,n+1}\left(t\right) =
C_{b,n+1}\left(t\right)_I exp\left\{i \left(
\omega_b + \omega 
\left( n + 1 + \frac{1}{2}\right) 
\right)
    t\right\}.
\nonumber
\end{eqnarray}

Таким образом уравнение (\ref{eqCh2_Shredinger_inter}) принимает
следующий вид   
\begin{eqnarray}
i \left( 
{\dot C}_{a,n}\left(t\right)\left|a, n\right> +
{\dot C}_{b,n+1}\left(t\right)\left|b, n + 1\right>
\right) =
\nonumber \\
= g \left(
\hat{a}\hat{\sigma}^{+} e^{-i \left(\omega - \omega_{ab}\right)t} +
\hat{a}^{+}\hat{\sigma} e^{i \left(\omega - \omega_{ab}\right)t}
\right) 
\left( 
C_{a,n}\left|a, n\right> +
C_{b,n+1}\left|b, n + 1\right>
\right) = 
\nonumber \\
= g \sqrt{n + 1} \left(
C_{b, n+1} e^{-i \left(\omega - \omega_{ab}\right)t} \left|a, n\right> + 
C_{a, n} e^{i \left(\omega - \omega_{ab}\right)t} \left|b, n + 1\right>
\right).
\end{eqnarray}
Помножим это уравнение слева соответственно на 
$\left<a, n\right|$ и $\left<b, n + 1\right|$.  Учитывая свойства
операторов  $\hat{a}$, $\hat{a}^{+}$,  $\hat{\sigma}$,
$\hat{\sigma}^{+}$ и ортогональность векторов состояния, получим  
систему уравнений для амплитуд вероятностей 
\begin{eqnarray}
{\dot C}_{a,n}\left(t\right) = -i g \sqrt{n + 1}
e^{-i \left(\omega - \omega_{ab}\right)t} 
C_{b, n + 1}\left(t\right),
\nonumber \\
{\dot C}_{b, n + 1}\left(t\right) = -i g \sqrt{n + 1}
e^{i \left(\omega - \omega_{ab}\right)t} 
C_{a, n}\left(t\right).
\label{eqCh2_task3}
\end{eqnarray}

\input ./part1/interaction/fig3.tex

Для случая, когда атом находится в нижнем состоянии (случай поглощения
фотона) имеем в начальный момент  
$C_{a, n}\left(0\right) = 0$, $C_{b, n + 1}\left(0\right) = 1$
(рис. \ref{figPart1Ch2_3}). Формальное интегрирование (\ref{eqCh2_task3}) дает  
\begin{equation}
C_{a,n}\left(t\right) = -i g \sqrt{n + 1}
\int_0^t e^{-i \left(\omega - \omega_{ab}\right)t'} 
C_{b, n + 1}\left(t'\right) dt'
\label{eqCh2_ampl_prob_int}
\end{equation}
В первом приближении для малых времен можно в
(\ref{eqCh2_ampl_prob_int}) под интегралом 
положить $C_{b, n + 1}\left(t'\right) = 1$.  Тогда, интегрируя,
получим  
\begin{equation}
C_{a,n}\left(t\right) = -i g \sqrt{n + 1}
\frac{e^{-i \left(\omega - \omega_{ab}\right)t} - 1}
{-i \left(\omega - \omega_{ab}\right)}
\end{equation}
откуда для вероятности возбуждения атома и поглощения фотона имеем
\begin{equation}
\left|C_{a,n}\left(t\right)\right|^2 = g^2 \left(n + 1\right) t^2
\frac{sin^2\left(\left(\omega - \omega_{ab}\right)t/2\right)}
{\left(\omega - \omega_{ab}\right)^2\left(t/2\right)^2}. 
\label{eqCh2_prob_C_an}
\end{equation}

\input ./part1/interaction/fig4.tex

В таком приближении можно решить и вторую задачу (рис. \ref{figPart1Ch2_4}) 
$C_{a, n}\left(0\right) = 1$, $C_{b, n + 1}\left(0\right) = 0$. 
Получим для вероятности перехода атома в нижнее состояние и излучения
фотона аналогичное (\ref{eqCh2_prob_C_an}) выражение  
\begin{equation}
\left|C_{b, n + 1}\left(t\right)\right|^2 = g^2 \left(n + 1\right) t^2
\frac{sin^2\left(\left(\omega - \omega_{ab}\right)t/2\right)}
{\left(\omega - \omega_{ab}\right)^2\left(t/2\right)^2}.
\label{eqCh2_prob_C_bn}
\end{equation}

В случае, когда поле не имеет ни одного фотона  ($n = 0$),
получаем отличную от нуля вероятность 
$\left|C_{b, 1}\left(t\right)\right|^2 \cong g^2 t^2$
($\frac{sin^2 x}{x^2} \cong 1$,  если
$x \ll 1$), откуда видно, что возбужденный атом даже при
отсутствии фотонов может перейти в нижнее состояние. Такой процесс
носит название спонтанного излучения или спонтанного перехода. Он
связан с нулевыми колебаниями поля. Заметим, что при полуклассическом
рассмотрении спонтанное излучение не следует из полуклассических
уравнений и вводится из дополнительных соображений. Из квантовых
уравнений спонтанное излучение следует естественным образом. 
Систему уравнений (\ref{eqCh2_task3}) можно решить точно. Для простоты
будем рассматривать резонансный случай $\omega = \omega_{ab}$.
Исключим из уравнений $C_{b, n + 1}\left(t\right)$:   
\[
{\ddot C}_{a,n}\left(t\right) = -i g \sqrt{n + 1}
{\dot C}_{b, n + 1}\left(t\right) = -g^2 \left(n + 1\right)
C_{a,n}\left(t\right). 
\]
Решение полученного уравнения:
\begin{equation}
C_{a,n}\left(t\right) = A \sin\left(g \sqrt{n + 1} t\right) +
B \cos\left(g \sqrt{n + 1} t\right)
\nonumber
\end{equation}

При поглощении начальные условия ($C_{a,n}\left(0\right) = 0$, $C_{b,n+1}\left(0\right)
= 1$) дают $A = i$, $B = 0$, следовательно,  
\begin{equation}
C_{a,n}\left(t\right) = -i \sin\left(\frac{\omega_R t}{2}\right), \quad
C_{b, n + 1}\left(t\right) = \cos\left(\frac{\omega_R t}{2}\right),
\label{eqPart1RabiAbsorbtion}
\end{equation}
где $\omega_R = 2g\sqrt{n + 1}$ - частота Раби в квантовом случае. Для случая
излучения из начальных условий ($C_{a,n}\left(0\right) = 1$,
$C_{b,n+1}\left(0\right) = 0$) имеем $A = 0$, $B = 1$, следовательно, 
\begin{equation}
C_{a,n}\left(t\right) = \cos\left(\frac{\omega_R t}{2}\right), \quad
C_{b, n + 1}\left(t\right) = -i \sin\left(\frac{\omega_R t}{2}\right),
\label{eqPart1RabiEmission}
\end{equation}
На рис. \ref{figPart1Ch2_5} изображены зависимости вероятностей 
$\left|C_{a,n}\left(t\right)\right|^2$ и 
$\left|C_{b, n + 1}\left(t\right)\right|^2$  от времени для случая
излучения фотона.

\input ./part1/interaction/fig5.tex

В полуклассическом выражении частота Раби $\omega_R$ равна 
$\frac{\left|p E\right|}{\hbar}$,  в квантовом - $2g\sqrt{n + 1}$,
т. е. $E$  соответствует $E_1\sqrt{n + 1}$,  где $E_1$ - поле,
соответствующее одному фотону в моде. При больших $n$ эти два
выражения практически совпадают. Но при малых $n$ различие может быть 
значительным. Из квантового  выражения следует, что даже когда в моде
нет фотонов, вероятности осциллируют с частотой  $g$.  На самом деле 
известно, что осцилляции на частоте Раби затухают. Расхождение связано
с тем, что атом взаимодействует со всеми модами пространства, мы же
учитывали только взаимодействие с одной модой.  

Попробуем теперь найти общее решение системы уравнений
(\ref{eqCh2_task3}). Рассматриваемые соотношения можно записать в
следующем виде
\begin{eqnarray}
  {\dot C}_{a,n}\left(t\right) = -i \frac{\omega_R}{2} e^{-i \delta t} 
C_{b, n + 1}\left(t\right),
\nonumber \\
{\dot C}_{b, n + 1}\left(t\right) = -i \frac{\omega_R}{2} e^{i \delta t} 
C_{a, n}\left(t\right),
  \nonumber
\end{eqnarray}
где $\delta = \omega - \omega_{ab}$ - расстройка частоты.
Задачу будем решать только для случая излучения, т. е. начальные
условия имеют вид
\begin{eqnarray}
  C_{a,n}\left(0\right) = 1,
  \nonumber \\
  C_{b,n+1}\left(0\right) = 0.
  \nonumber
\end{eqnarray}
При этом интересовать нас будет прежде всего поведение вероятностей
$\left|C_{b, n + 1}\left(t\right)\right|^2$. Т. о.
обозначив $x\left(t\right) = C_{b,n+1}\left(t\right)$ получим
\begin{eqnarray}
  \ddot{x} =  -i \frac{\omega_R}{2} \frac{d \left(e^{i \delta t} 
    C_{a, n}\left(t\right)\right) }{d t} =
  \nonumber \\
  = i \delta \left(- i \frac{\omega_R}{2} e^{ i\delta t} 
  C_{a, n}\left(t\right) \right) -i \frac{\omega_R}{2} e^{i \delta t}
  \frac{d C_{a, n}\left(t\right)}{dt} =
  \nonumber \\
  = i \delta \dot{x} - \frac{\omega_R^2}{4} e^{-i \delta t} e^{i \delta t} x =
  \nonumber \\
  = i \delta \dot{x} - \frac{\omega_R^2}{4} x,
  \nonumber
\end{eqnarray}
т. е.
\begin{equation}
  \ddot{x} - i \delta \dot{x} + \frac{\omega_R^2}{4} x = 0.
  \label{eqPart1RabiCommon}
\end{equation}
Уравнение (\ref{eqPart1RabiCommon}) должно быть решено при следующих
начальных условиях:
\begin{eqnarray}
  \left.x\right|_{t=0} = C_{b,n+1}\left(0\right) = 0,
  \nonumber \\
  \left.\dot{x}\right|_{t=0} = -i \frac{\omega_R}{2} C_{a,n}\left(0\right) =
  -i \frac{\omega_R}{2}.
  \nonumber
\end{eqnarray}
%% (%i1) eq: 'diff(x(t),t,2) - 2*%i*d*'diff(x(t),t) + w^2*x(t)/4 = 0;
%% (%i2) atvalue(x(t),t=0,0);
%% (%i3) atvalue('diff(x(t),t),t=0,-%i*w/2);
%% (%i4) desolve(eq, x(t));
%% positive;
%% (%i5) 
Результат:
\begin{equation}
  x\left(t\right) = -  i e^{\frac{ i \delta t}{2}}
  \frac{\omega_R}{\sqrt{\omega_R^2 + \delta^2}}
  \sin{\left(\frac{\sqrt{\omega_R^2 + \delta^2}}{2}t\right)},
  \nonumber
\end{equation}
обозначив через $\Omega_R = \sqrt{\omega_R^2 + \delta^2}$ -
обобщенную частоту Раби - для искомой вероятности
$\left|C_{b,n+1}\right|^2$ получим см. рис. \ref{figPart1InteractionRabiDelta}
\begin{eqnarray}
  \left|C_{b,n+1}\right|^2 =  \frac{\omega_R^2}{\Omega_R^2}
  \sin^2{\frac{\Omega_R}{2} t}.
  \label{eqPart1InteractionRabiProbability}
\end{eqnarray}
%% TODO FIXME wrong result - should be
%%  \frac{\omega_R^2}{\Omega_R^2}
%%  \sin^2{\frac{\Omega_R t}{2}}

\input ./part1/interaction/figrabi.tex

\begin{remark}[Переменный Штарк - эффект]
  Стоит отметить, что в случае $\delta \ne 0$ состояния, которые мы
  использовали в выражении \ref{eq:part1:rabi_solution} не
  удовлетворяют закону сохранения энергии. Действительно если система
  изначально находилась в состоянии $\left|a\right>\left|n\right>$ с
  энергией $E_a + \hbar n \omega$, то через некоторое время она,
  с ненулевой вероятностью,
  может
  быть зарегистрирована в состоянии
  $\left|b\right>\left|n+1\right>$ с энергией
  \[
  E_b + \hbar (n+1)
  \omega \ne E_a + \hbar n \omega.
  \]
  Т.е. получается нарушение закона
  сохранения энергии. Указанное противоречие разрешается следующим
  образом. В силу переменного Штарк эффекта
  (в англоязычной литературе этот эффект еще называют AC-Stark effect или
  Autler-Townes effect)\cite{wiki:autler_townes_effect},
  энергетические уровни атома смещаются, 
  что обеспечивает выполнение закона сохранения энергии.

  Например,
  при малых расстройках $\delta \approx 0$, это смещение составляет
  \cite{wiki:autler_townes_effect} 
  \[
  \Delta E \approx \pm \frac{\hbar \omega_R}{2}.
  \]
\end{remark}
