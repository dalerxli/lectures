%% -*- coding:utf-8 -*- 
\section{Взаимодействие электромагнитного поля резонатора
  с резервуаром атомов, находящихся при
  температуре $T$}
\label{ch2_6}
Применим общий подход, изложенный выше, к системе, состоящей из
гармонического осциллятора (электромагнитного поля моды),
взаимодействующего с резервуаром в виде атомного пучка двухуровневых
атомов, находящихся в равновесии при температуре $T$.  Как мы увидим
в дальнейшем, конкретная модель резервуара для окончательного
результата значения не имеет.\footnote{Если взаимодействие слабое,
  т. е. энергия связи мала по сравнению с энергией динамической системы.} 
Рассматриваемая модель изображена на
рис. \ref{figPart1Ch2_7}.  

\input ./part1/interaction/fig7.tex

Начальная атомная матрица плотности соответствует распределению
Больцмана при температуре $T$:  
\begin{equation}
\hat{\rho}_{at} = 
\left(
\begin{array} {cc}
\rho_{aa} & 0  
\\
0 & \rho_{bb} 
\end{array}
\right)
=
z^{-1}
\left(
\begin{array} {cc}
e^{-\frac{\hbar \omega_a}{k_B T}} & 0  
\\
0 & e^{-\frac{\hbar \omega_b}{k_B T}} 
\end{array}
\right).
\end{equation}
Здесь $\hbar \omega_a = E_a$ - энергия верхнего уровня;  
$\hbar \omega_b = E_b$ - энергия нижнего уровня; 
$z = e^{-\frac{\hbar \omega_a}{k_B T}} + e^{-\frac{\hbar \omega_b}{k_B
    T}}$ статистическая сумма; $T$ - температура резервуара. 
Гамильтониан взаимодействия возьмем в виде
(\ref{eqCh2_task22}). Положим для простоты, 
что $\omega_{ab} = \omega$,  т. е. частота перехода равна частоте 
моды. Тогда  
\begin{equation}
\hat{V} = \hbar g \left(\hat{\sigma}\hat{a}^{+} + 
\hat{\sigma}^{+}\hat{a} \right)= 
\hbar g 
\left(
\begin{array} {cc}
0 & \hat{a}  
\\
\hat{a}^{+} & 0 
\end{array}
\right)
\end{equation}
так как  
\(
\hat{\sigma} = 
\left(
\begin{array} {cc}
0 & 0  
\\
1 & 0 
\end{array}
\right)
\);
\(
\hat{\sigma}^{+} = 
\left(
\begin{array} {cc}
0 & 1  
\\
0 & 0 
\end{array}
\right)
\); $g$ - константа взаимодействия. 

Сохраним в итерационном ряде (\ref{eqCh2_rho_sequance}) члены до
второго порядка включительно. В нашем случае после интегрирования
получается простое выражение  
\begin{eqnarray}
\hat{\rho}_{f}\left(t\right) =
\hat{\rho}_{f}\left(t_0 + \tau\right) = 
\nonumber \\
= \hat{\rho}_{f}\left(t_0\right) +
Sp_{at}
\left\{
- \frac{i}{\hbar}\tau\left[\hat{V}, \hat{\rho}_{at, f}\right]
- \frac{1}{2} \left(\frac{\tau}{\hbar}\right)^2
\left[\hat{V},\left[\hat{V}, \hat{\rho}_{at, f}
\right]\right]
\right\}, 
\label{eqCh2_task4}
\end{eqnarray}
где $\hat{\rho}_{at, f}$ -  статистический оператор системы атом-поле;
$\hat{\rho}_{f}$ статистический оператор поля, свернутый по
атомным переменным.
  
При получении (\ref{eqCh2_task4}) предполагалось, что атом пучка
взаимодействует с полем за время $\tau$ (время пролета через
резонатор). Из начальных условий (во время $t_0$) имеем:  
\begin{equation}
\hat{\rho}_{at, f} = \hat{\rho}_{f}
\left(
\begin{array} {cc}
\rho_{aa} & 0  
\\
0 & \rho_{bb} 
\end{array}
\right) = 
\left(
\begin{array} {cc}
\rho_{aa}\hat{\rho}_{f} & 0  
\\
0 & \rho_{bb}\hat{\rho}_{f}
\end{array}
\right)
\end{equation}
Член первого порядка в (\ref{eqCh2_task4}) дает матрицу с нулевыми
диагональными элементами по переменным резервуара. След матрицы по
этим переменным очевидно равен нулю  
\begin{eqnarray}
Sp_{at}\left[\hat{V}, \hat{\rho}_{at, f}\right] = 
\hbar g Sp_{at}
\left\{
\left(
\begin{array} {cc}
0 & \hat{a} \rho_{bb}\hat{\rho}_{f}  
\\
\hat{a}^{+} \rho_{aa}\hat{\rho}_{f}   & 0
\end{array}
\right)
\right.
-
\nonumber \\
-
\left.
\left(
\begin{array} {cc}
0 & \rho_{aa} \hat{\rho}_{f} \hat{a}   
\\
\rho_{bb} \hat{\rho}_{f} \hat{a}^{+}   & 0
\end{array}
\right)
\right\}
= 0.
\label{eqCh2_sp_1}
\end{eqnarray}
Матрица, входящая в член второго порядка, имеет отличные от нуля
диагональные элементы, след от нее по переменным резервуара отличен от 
0  и может быть вычислен: 
\begin{eqnarray}
Sp_{at}\left[\hat{V},\left[\hat{V}, \hat{\rho}_{at, f}\right] \right]
= \hbar^2 g^2 \cdot
\nonumber \\
\cdot Sp_{at}
\left(
\begin{array} {cc}
\hat{a} \hat{a}^{+}\rho_{aa} \hat{\rho}_{f} -
\hat{a}\rho_{bb} \hat{\rho}_{f}\hat{a}^{+}  & 0
\\
0 & 
\hat{a}^{+} \hat{a}\rho_{bb} \hat{\rho}_{f} -
\hat{a}^{+}\rho_{aa} \hat{\rho}_{f}\hat{a} 
\end{array}
\right) + \mbox{э. с.} 
\label{eqCh2_sp_2}
\end{eqnarray}
Отсюда, используя уравнения (\ref{eqCh2_task4}),
(\ref{eqCh2_sp_1}), (\ref{eqCh2_sp_2}), получаем окончательное
уравнение для статистического оператора поля  $\hat{\rho}$:  
\begin{eqnarray}
\hat{\rho}_{f}\left(t_0 + \tau\right) =
\hat{\rho}_{f}\left(t_0\right) - \frac{1}{2}g^2 \tau^2
\left\{
\left(\hat{a}\hat{a}^{+}\hat{\rho}_{f} - 
\hat{a}^{+}\hat{\rho}_{f}\hat{a}
\right)\rho_{aa} +
\right.
\nonumber \\
\left.
+
\left(\hat{a}^{+}\hat{a}\hat{\rho}_{f} - 
\hat{a}\hat{\rho}_{f}\hat{a}^{+}
\right)\rho_{bb}
\right\} + \mbox{э. с.}
\label{eqCh2_rho_final}
\end{eqnarray}
Таким образом, статистический оператор моды из-за взаимодействия с
одним атомом изменится на величину 
\(
\delta\hat{\rho}_{f}=
\hat{\rho}_{f}\left(t_0 + \tau\right) -
\hat{\rho}_{f}\left(t_0\right)
\), определяемую (\ref{eqCh2_rho_final}). Если скорость инжекции
атомов в резонатор $r$  атомов в секунду, за время  $\Delta t$
провзаимодействуют  $r \Delta t$  атомов. Изменение оператора
плотности за это  время будет равно 
$\Delta \hat{\rho}_{f} = \delta\hat{\rho}_{f}r \Delta t$. Определим сглаженную
производную как 
\[
\dot{\hat{\rho}}_{f} = \frac{\Delta \hat{\rho}_{f}}{\Delta t} = 
\delta\hat{\rho}_{f} r = r 
\left(
\hat{\rho}_{f}\left(t_0 + \tau\right) -
\hat{\rho}_{f}\left(t_0\right)
\right).
\]
Обозначим: $r_a = r \rho_{aa}$ - скорость инжекции атомов, находящихся
в верхнем состоянии; $r_b = r \rho_{bb}$ - скорость инжекции атомов,
находящихся в нижнем состоянии. Это приводит нас к окончательному виду
уравнения для статистического оператора поля моды резонатора 
\begin{eqnarray}
\dot{\hat{\rho}}_{f} =
- \frac{1}{2}R_a
\left(\hat{a}\hat{a}^{+}\hat{\rho}_{f} - 
\hat{a}^{+}\hat{\rho}_{f}\hat{a}
\right)
- 
\nonumber \\
- \frac{1}{2}R_b
\left(\hat{a}^{+}\hat{a}\hat{\rho}_{f} - 
\hat{a}\hat{\rho}_{f}\hat{a}^{+}
\right)
 + \mbox{э. с.}
\label{eqCh2_rho_final2}
\end{eqnarray}
Здесь обозначено 
\begin{eqnarray}
R_a=r_a g^2\tau^2,
\nonumber \\
R_b=r_b g^2\tau^2.
\label{eqCh2_RaRbDefenition}
\end{eqnarray}
Уравнение
(\ref{eqCh2_rho_final2}) написано в операторном виде. Его можно
написать в различных представлениях.   
