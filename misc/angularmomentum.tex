\input preamble.tex

\title{Квантование углового момента}

\begin{document}
\Russian
\maketitle

\section{Классика}
Здесь кое что про квантование углового момента. Гамильтониан системы
имеет следующий вид 
\begin{eqnarray}
\mathcal{H} = \frac{m v^2}{2} + U\left( r \right) = 
\nonumber \\
= \frac{m r^2 \dot{\theta}^2 }{2} + U\left( r \right)
\nonumber
\end{eqnarray}
мы предпологаем что потенциальная энергия не зависит от угла и зависит
только от растояния (симметрия задачи)

Лангранжин системы
\[
\mathcal{L} = \frac{m r^2 \dot{\theta}^2 }{2} - U\left( r \right)
\]
Угловой момент определяется как 
\begin{equation}
l = \frac{d \mathcal{L}}{d \dot{\theta}^2} = 
m r^2 \dot{\theta} = I \dot{\theta},
\label{eqAngualrMomentumClass}
\end{equation}
где через $I$ обозначено $I = m r^2$ - момент инерции (moment of
inertia).

Уравнения движения для координаты r:
\[
\frac{ \partial \mathcal{L} }{\partial r} = 
\frac{d}{d t} \frac{\partial \mathcal{L}}{\partial \dot{r}}
\]
 откуда
\[
\frac{\partial U}{\partial r} = m r \dot{\theta}^2
\]
 или же
\[
 U = \frac{m r^2 \dot{\theta}^2}{2} = \frac{l^2}{2 I}
\]
 
Таким образом гамильтониан рассматриваемой системы
\begin{eqnarray}
\mathcal{H} = \frac{m r^2 \dot{\theta}^2 }{2} + \frac{l^2}{2 I} = 
\nonumber \\
= \frac{l^2}{2 I} + \frac{l^2}{2 I} = \frac{l^2}{I}
\label{eqHClassical}
\end{eqnarray}

\section{Квантование}

Пусть $\left|\psi\right>$ - собственная функция оператора углового
момента $\hat{L}$ отвечающая собственному числу $l$:
\begin{equation}
\hat{L} \left|\psi\right> = l \left|\psi\right>.
\label{eqLPsi}
\end{equation}
Эта же волновая функция должна удовлетворять уравнению Шредингера
\[
i \hbar \frac{\partial \left|\psi\right>}{ \partial t} = 
\hat { \mathcal{H} } \left|\psi\right>
\]
Из (\ref{eqHClassical}) имеем
\[
\hat { \mathcal{H} } = \frac{\hat{L}^2}{I}
\]
откуда с учетом (\ref{eqLPsi})
\[
i \hbar \frac{\partial \left|\psi\right>}{ \partial t} = 
\frac{1}{I} \hat{L} \hat{L} \left|\psi\right> = 
\frac{l^2}{I} \left|\psi\right> = 
\frac{-i l^2}{\hbar I} \left|\psi\right>.
\]
Таким образом
\[
\left|\psi\right>\left(t \right) = C \dot exp\left\{\frac{-i l^2}{\hbar
    I} t
\right\}
\]
Волновая функция должна удовлетворять условию периодичности
\[
\left|\psi\right>\left(t \right) = \left|\psi\right>\left(t + T \right)
\]
где $T$ - период колебаний. Его можно найти из
(\ref{eqAngualrMomentumClass}):
\[
T = \frac{2 \pi I}{l}
\]
таким образом
\[
2 i \pi n = \frac{-i l^2}{\hbar I} T = 
\frac{2 \pi I}{l} \frac{-i l^2}{\hbar I}
\]
откуда
\[
l = -\hbar n
\]

\end{document}
